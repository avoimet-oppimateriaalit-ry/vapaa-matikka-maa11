\renewcommand{\kurssinTunnus}{MAA11}
\renewcommand{\kurssinNimi}{\maaXI}
\renewcommand{\sitaatti}{''How often have I said to you that when you have eliminated the impossible, whatever remains,
\textit{however improbable}, must be the truth?'' -- Sherlock Holmes}
\renewcommand{\sitaatinLahde}{The Sign of the Four (1890), Chap. 6, p. 111}
%\varitfalse
\mikrofalse
\versiofalse

\renewcommand{\metasivu}{
\vspace*{\fill}
\begin{flushleft}
    \sffamily
    Jos olet kiinnostunut Vapaa matikka -sarjan kirjoittamisesta,
    voit tehdä pull requestin muutoksillesi GitHub-palvelussa tai
    osallistua yhdistyksen toimintaan. \\
    \vspace*{25pt}
    Lisätietoja Vapaa matikka -kirjasarjasta työryhmältä,
    \href{mailto:vapaamatikka@avoimetoppimateriaalit.fi}{vapaamatikka@avoimetoppimateriaalit.fi}. \\
    \vspace*{25pt}
    Erityisesti tätä kirjaa koskevaa palautetta voi lähettää Antti Rasilalle,
    \href{mailto:antti.rasila@iki.fi}{antti.rasila@iki.fi}. \\
    \vspace*{25pt}
    draft \today
\end{flushleft}
}

% kirjaspesifejä komentoja
\newcommand{\lequiv}{\leftrightarrow}
\newcommand{\compl}{\complement}
\DeclareMathOperator{\xor}{xor}
\DeclareMathOperator{\syt}{syt}
\DeclareMathOperator{\pyj}{pyj}
\DeclareMathOperator{\lcm}{lcm}
\DeclareMathOperator{\pym}{pym}
\renewcommand{\div}{\mathop{\mathrm{div}}}
\renewcommand{\mod}{\mathop{\mathrm{mod}}}
