% Niko

\begin{tehtavasivu}

\begin{tehtava}
	Osoita, että looginen algebra toteuttaa Boolen algebran ehdot 1, 2, 5--8 ja 10.
\end{tehtava}

\begin{tehtava}Osoita, että seuraavat laskusäännöt ovat voimassa loogisessa algebrassa.
	\alakohdat{
		§ $x + x = x$
		§ $x \cdot x = x$
		§ $x + 1 = 1$
		§ $x \cdot 0 = 0$
		§ $x + x \cdot y = x$
		§ $x \cdot (x + y) = x$
		§ $x + (-x) \cdot y = x + y$
		§ $x \cdot (-x + y) = x \cdot y$
		§ $-(-x) = x$
		§ $-(x + y) = (-x) \cdot (-y)$
		§ $-(x \cdot y) = (-x) + (-y)$
	}
\end{tehtava}

\begin{tehtava}
	Tulkitse, mitä Boolen algebran tulokset
	\alakohdat{
		§ $x \cdot x = x$
		§ $x + 1 = 1$
		§ $x \cdot 0 = 0$
		§ $-(-x) = x$
		§ $x + x \cdot y = x$
		§ $x \cdot (x + y) = x$
	}
	tarkoittavat joukko-opissa. Perustele tulokset myös käyttäen Venn-diagrammeja.
	\begin{vastaus}
		Olkoon $x, y \subseteq z$
		\alakohdat{
			§ $x\cap x = x$
			§ $x\cup z = z$
			§ $x\cap \empty = \empty$
			§ $z\setminus (z\setminus x) = x$
			§ $x\cup (x\cap y) = x$
			§ $x\cap (x\cup y) = x$
		}
	\end{vastaus}
\end{tehtava}

\begin{tehtava}
	Sievennä lausekkeet Boolen algebran määritelmän ja tehtävän 2 laskusääntöjen avulla.
	\alakohdat{
		§ $x \cdot x + x \cdot x$
		§ $x \cdot x \cdot x \cdot y \cdot y \cdot y$
		§ $(a + b) \cdot (a + c)$
		§ $a + b \cdot a + b$
		§ $(a + (-b)) \cdot b$
		§ $a + b + (-(a \cdot b))$
	}
	\begin{vastaus}
		\alakohdat{
			§ $x$
			§ $x\cdot y$
			§ $a+(b\cdot c)$
			§ $a+b$
			§ $a\cdot b$
			§ $1$
		}
	\end{vastaus}
\end{tehtava}

\begin{tehtava}
	Piirrä Boolen algebran lauseketta
	\alakohdat{
		§ $(x + y) \cdot (x + z)$
		§ $(x + y) \cdot (x + (-y))$
		§ $x + (-x) \cdot y$
	}
	vastaava looginen piiri. Sievennä lauseke Boolen algebran säännöillä ja piirrä sievennettyä muotoa vastaava piiri. Loogisia piirejä on käsitelty kappaleessa 2.1 tehtävissä 11 ja 12.
	\begin{vastaus}
		\alakohdat{
			§ $x+(y\cdot z)$
			§ $x$
			§ $x+y$
		}
	\end{vastaus}
\end{tehtava}

\begin{tehtava}
	* Osoita, että tehtävän 2 säännöt ovat voimassa kaikissa Boolen algebroissa.
	%{\bf Vain todellisille matemaatikoille.}
\end{tehtava}

\begin{tehtava}
	* Reaalilukujen joukossa $\rr$ määritellään yhteenlaskua muistuttava laskutoimitus $x \circ y$ seuraavasti: $x \circ y = x + y - 2$ kaikilla $x, y \in \rr$. Osoita, että laskutoimitus toteuttaa seuraavat ehdot: 
	\begin{description}
	\item[i]
	$(x \circ y) \circ z = x \circ (y \circ z)$ kaikilla $x, y, z \in \rr$. 
	\item[ii]
	$x \circ y = y \circ x$ kaikilla $x, y \in \rr$. 
	\item[iii]
	On olemassa sellainen luku $\omega \in \rr$, että $x \circ \omega = \omega \circ x = x$ kaikilla $x \in \rr$. Mikä $\omega$ on? 
	\item[iv]
	Jokaisella $x \in \rr$ on vasta-alkio $x^*$, jolle $x \circ x^* = x^* \circ x = \omega$. 
	\end{description}
	[YO syksy 1997 tehtävä 9b]
\end{tehtava}

\end{tehtavasivu}
