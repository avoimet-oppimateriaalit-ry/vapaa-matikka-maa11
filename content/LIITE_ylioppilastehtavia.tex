\chapter{Ylioppilastehtäviä}

\begin{tehtava}
(k2008:11)\alakohdat{
§ Määritä lukujen $154$ ja $126$ suurin yhteinen tekijä. 
§ Ratkaise Diofantoksen yhtälö $154x+126y=56$
}
\begin{vastaus}
\end{vastaus}
\end{tehtava}


\begin{tehtava}
(k2009:11) Määritä kaikki positiiviset kokonaisluvut $n$, joille $\frac{9n^2+117n+34}{3n+5}$ on myös positiivinen kokonaisluku. 
\begin{vastaus}
\end{vastaus}
\end{tehtava}


\begin{tehtava}
(s2009:12) Parilliset luonnolliset luvut voidaan esittää muodossa $2p$, $p=0,1,2,3,\ldots$, ja parittomat muodossa $2p+1$, $p=0,1,2,3,\ldots$ Osoita tämän perusteella, että
\alakohdat{
§ kahden parittoman luvun summa on parillinen
§ kahden parittoman luvun tulo on pariton. 
}
\begin{vastaus}
\end{vastaus}
\end{tehtava}


\begin{tehtava}
(k2010:12) Osoita, että muotoa $p^2-1$ oleva luku on jaollinen luvulla $12$, kun $p$ on alkuluku ja suurempi kuin $3$. 
\begin{vastaus}
\end{vastaus}
\end{tehtava}

%siirretään MAA12?
\begin{tehtava}
(s2010:12) Määritä $a$ siten, että polynomi $P(x)=2x^4-3x^3-7x^2+a$ on jaollinen binomilla $2x-1$. Määritä tätä $a$:n arvoa vastaavat yhtälön $P(x)$ juuret. 
\begin{vastaus}
\end{vastaus}
\end{tehtava}


\begin{tehtava}
(k2011:12) Tutki, onko luku $46^{78}$ jaollinen luvulla $89^{67}$.
\begin{vastaus}
\end{vastaus}
\end{tehtava}


\begin{tehtava}
(s2011:13) Osoita epäsuoraa todistusta käyttämällä, että $lg50$ ei ole rationaaliluku. ($lg=log_{10}$) 
\begin{vastaus}
\end{vastaus}
\end{tehtava}



\begin{tehtava}
(s2013:13) Osoita epäsuoraa todistusta käyttämällä, että $3\sqrt{2}$ ei ole rationaaliluku. 
\begin{vastaus}
\end{vastaus}
\end{tehtava}






