\begin{tehtavasivu}

%Harj.Tehtävät Luku 1, vastausten tekijä Valtteri Vistiaho 9.11.2013
\begin{tehtava}
    Onko päättely loogisesti pätevä? Perustele.
    \alakohdat{
        § Kaikki ihmiset ovat kuolevaisia. Lasse-kissa on kuolevainen. Siis Lasse-kissa on ihminen.
        § Kaikki koirat osaavat haukkua. Halli on koira. Siis Halli osaa haukkua.
    }

    \begin{vastaus}
        \alakohdat{
            § Ei.
            § On.
        }
    \end{vastaus}
    
\end{tehtava}

\begin{tehtava}
    Ovatko seuraavat päättelyt loogisesti päteviä? Perustele.
    \alakohdat{
        § \kolmepaattely{Kaikki tetraedrit ovat pyramideja.}{Jotkut kartiot ovat tetraedrejä.}{Jotkut kartiot ovat pyramideja.}
        § \kolmepaattely{Kaikki sylinterit ovat lieriöitä.}{Mikään lieriö ei ole kartio.}{Jotkut sylinterit ovat kartioita.}
        § \kolmepaattely{Luku $345$ päättyy numeroon $5$.}{Nollaan päättyvä luku on viidellä jaollinen.}{Luku $345$ on viidellä jaollinen.}	        
    }

    \begin{vastaus}
        \alakohdat{
            § On.
            § Ei.
            § Ei.
        }
    \end{vastaus}
    
\end{tehtava}

\begin{tehtava}
    Onko seuraava päättely loogisesti pätevä? Perustele.
    \alakohdat{
        § Jos ulkona on pakkanen, menen hiihtämään. Ulkona ei ole pakkanen. Siis en mene hiihtämään.
        § Jos ulkona on pakkanen, menen hiihtämään. En mene hiihtämään. Siis ulkona ei ole pakkanen.
        § Jos tiedän nukkuvani, niin nukun. Jos tiedän nukkuvani, niin en nuku. Siis en tiedä nukkuvani.
    }

    \begin{vastaus}
        \alakohdat{
            § On.
            § Ei.
            § Ei. % tark. Niko
        }
    \end{vastaus}
    
\end{tehtava}

\begin{tehtava}
    Tutkitaan polynomia
        \[ P(x) = x^5 -10x^4+35x^3 -50 x^2 +25x. \]
    \alakohdat{
        § Laske $P(0)$, $P(1)$, $P(2)$, $P(3)$ ja $P(4)$.
        § Mitä voit sanoa luvuista $P(n)$, kun $n$ on luonnollinen luku?
        § Testaa päätelmääsi kokeilemalla myös muilla luonnollisilla luvuilla esimerkiksi laskinta käyttäen.
    }

    \begin{vastaus}
        \alakohdat{
            § $P(0)=0$, $P(1)=1$, $P(2)=2$, $P(3)=3$, $P(4)=4$
            § $P(n)=n$
            § $P(5)=125$, päätelmä ei päde.
        }
    \end{vastaus}
    
\end{tehtava}

\begin{tehtava}
    Arkiajattelussa käytetään usein ajattelumalleja, jotka eivät ole loogisesti perusteltavissa.
        Mikä virhe on seuraavissa päätelmissä?
    \alakohdat{
        § Ilta-Sanomien kyselyssä $66~\%$ vastaajista uskoo maan ulkopuoliseen elämään. Maan ulkopuolista elämää on olemassa.
        § The Sunday Times -lehden haastattelussa kuuluisa tiedemies Stephen Hawking totesi pitävänsä lähes varmana, että avaruudessa on maan ulkopuolista älykästä elämää. Maan ulkopuolista elämää on olemassa.
        § Tiedemiehistä 90~\% väittää, että nykyinen ilmastonmuutos on ihmisen aiheuttamaa eikä johdu maapallon lämpötilan luontaisesta jaksollisuudesta. Siis nykyinen ilmastonmuutos on ihmisen aiheuttamaa.
        § Televisiouutisissa kerrottiin, että toisen maailmansodan aikainen holokausti oli vain liittoutuneiden propagandaa. Siis holokaustia ei tapahtunut toisen maailmansodan aikana.
    }

    \begin{vastaus}
        \alakohdat{
            § Johtopäätös tehty vastaajien uskomuksesta.%suomen kielessä ei vedetä vaan tehdään johtopäätöksiä
            § Johtopäätös tehty Stephen Hawkingin uskomuksesta.
            § Johtopäätös perustuu tiedemiehien väitteisiin, joista ei ole varmaa totuutta.
            § Johtopäätös perustuu televisiouutisen väitteeseen, josta ei ole selviä todisteita.
        }
    \end{vastaus}
    
\end{tehtava}

\begin{tehtava}
    Ovatko seuraavat päättelyt loogisesti päteviä? Perustele.
    \alakohdat{
        § \kolmepaattely{Kaikilla $x$ toteutuu $y$.}{Joillakin $z$ toteutuu $x$.}{Joillakin $z$ toteutuu $y$.}
        § \kolmepaattely{Kaikilla $A$ toteutuu $B$.}{$C$ toteuttaa $B$:n.}{$C$ toteuttaa $A$:n.}
        § \kolmepaattely{Kaikilla $A$ toteutuu $B$.}{Millään $C$ ei toteudu $B$.}{Millään $C$ ei toteudu $A$.}
    }

    \begin{vastaus}
        \alakohdat{
            § On.
            § Ei. Jos A toteuttaa B:n, B ei välttämättä toteuta A:ta.
            § On. Vastaoletuksesta ''Jollakin $C$ toteutuu $A$.'' on johdettavissa ristiriita. % /Niko
        }
    \end{vastaus}
    
\end{tehtava}

\end{tehtavasivu}

% -----

\begin{kotitehtavasivu}

%Kotitehtävät Luku 1, vastausten tekijä Valtteri Vistiaho 9.11.2013
\begin{tehtava}
    Ovatko seuraavat päättelyt päteviä? Perustele.
    \alakohdat{
        § \kolmepaattely{Kaikki kissat osaavat kehrätä.}{Tämä eläin osaa kehrätä.}{Tämä eläin on kissa.}
        § \kolmepaattely{Kukaan laiska opiskelija ei selviä kokeesta.}{On opiskelijoita, jotka selviävät kokeesta.}{On opiskelijoita, jotka eivät ole laiskoja.}
    }

    \begin{vastaus}
        \alakohdat{
            § Ei.
            § On.
        }
    \end{vastaus}
    
\end{tehtava}

\begin{tehtava}
    Ovatko seuraavat päättelyt päteviä? Perustele.
    \alakohdat{
        § \kolmepaattely{Neljäkkään lävistäjät ovat kohtisuorassa toisiaan vastaan.}{Neljäkäs on suunnikas.}{Suunnikkaan lävistäjät ovat kohtisuorassa toisiaan vastaan.}
        § \kolmepaattely{Kaikki suorakulmiot ovat suunnikkaita.}{Jotkut nelikulmiot ovat suorakulmioita.}{Jotkut nelikulmiot ovat suunnikkaita.}
        § \kolmepaattely{Kolmio $ABC$ on tasakylkinen.}{Tasasivuiset kolmiot ovat tasakylkisiä.}{Kolmio $ABC$ on tasasivuinen.}
    }

    \begin{vastaus}
        \alakohdat{
            § Ei.
            § On.
            § Ei.
        }
    \end{vastaus}
    
\end{tehtava}

\begin{tehtava}
    Ovatko seuraavat päättelyt päteviä? Perustele.
    \alakohdat{
        § \kaksipaattely{Kaikki lammasfarmin lampaat ovat joko mustia tai valkoisia.}{Kaikki lampaat ovat mustia tai valkoisia.}
        § \neljapaattely{Tavallisessa korttipakassa kortti on aina joko pata,\\& risti, hertta tai ruutu.}{Pata- ja risti-kortit ovat mustia.}{Hertta- ja ruutu-kortit ovat punaisia.}{Kaikki tavallisten korttipakkojen kortit ovat\\ & joko mustia tai punaisia.}
    }

    \begin{vastaus}
        \alakohdat{
            § Ei.
            § On.
        }
    \end{vastaus}
    
\end{tehtava}

    %§ Onko seuraava päättely pätevä? Perustele.
    %    \neljapaattely{Millään $x$ ei toteudu $y$.}{Kaikilla $x$ toteutuu $z$.}{Jokin $x$ on olemassa.}
    %    {Joillakin $z$ ei toteudu $y$.}

\begin{tehtava}
    Jäämaa on kokonaan Merimaan itäpuolella.
        Kummallakin on etelärajaa Aurinkomaan kanssa.
        Merimaalla ja Aurinkomaalla on länsiraja Lumimaan kanssa.
        Kukkamaa on kokonaan Jäämaan ja Aurinkomaan itäpuolella.
    \alakohdat{
        § Onko Merimaalla ja Kukkamaalla yhteistä rajaa?
        § Voiko Lumimaalla ja Kukkamaalla olla yhteistä rajaa?
    }

    \begin{vastaus}
        \alakohdat{
            § Ei.
            § Voi, koska Merimaalla ja Aurinkomaalla voi olla Lumimaan kanssa muutakin rajaa kuin länsiraja. Kuvan piirtäminen saattaa auttaa. % /Niko
        }
    \end{vastaus}
    
\end{tehtava}

\begin{tehtava}
    Ovatko seuraavat päättelyt päteviä? Perustele.
    \alakohdat{
        § \kolmepaattely{Millään $x$ ei toteudu $y$.}{Kaikilla $z$ toteutuu $x$.}{Millään $z$ ei toteudu $y$.}
        § \kolmepaattely{Kaikilla $A$ toteutuu $B$.}{Joillakin $C$ toteutuu $B$.}{Joillakin $A$ toteutuu $C$.}
    }

    \begin{vastaus}
        \alakohdat{
            § On.
            § Ei.
        }
    \end{vastaus}
    
\end{tehtava}

\end{kotitehtavasivu}
