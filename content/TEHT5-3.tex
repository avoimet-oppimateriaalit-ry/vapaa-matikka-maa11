\setcounter{tehtava}{0}

\begin{tehtavasivu}

\begin{tehtava}
  Määritä jakojäännös, kun
  \begin{alakohdat}
  \alakohta{luku $80 \cdot 30 - 352$ jaetaan luvulla $7$}
  \alakohta{luku $2^{140}$ jaetaan luvulla $3$}
  \alakohta{luku $0+1+2+3+ \ldots + 79 + 80$ jaetaan luvulla $4$.}
  \end{alakohdat}
  \begin{vastaus}
  \begin{alakohdat}
  \alakohta{$4$}
  \alakohta{$1$}
  \alakohta{$0$}
  \end{alakohdat}
  \end{vastaus}
\end{tehtava}


\begin{tehtava}
  Määritä jakojäännös, kun
  \begin{alakohdat}
  \alakohta{luku $5 \cdot 11^{99} + 20$ jaetaan luvulla $6$}
  \alakohta{luku $6^{120} \cdot 4^{301}$ jaetaan luvulla $5$}
  \alakohta{luku $3^{60} + 55$ jaetaan luvulla $8$}
  \alakohta{luku $2^{151}$ jaetaan luvulla $14$.}
  \end{alakohdat}
  \begin{vastaus}
  \begin{alakohdat}
  \alakohta{3}
  \alakohta{4}
  \alakohta{0}
  \alakohta{2}
  \end{alakohdat}
  \end{vastaus}
\end{tehtava}

\begin{tehtava}
  Näytä, että luku $7^{2502} + 2^{1573}$ on jaollinen kolmella. [YO kevät 1999 10b kohta 2]
  \begin{vastaus}
  Vihje: $7 \equiv 1 (\mod{3})$ ja $2 \equiv -1 (\mod{3})$
  %Vihje: $7^{2502}+2^{1573}\equiv \left(7-2\cdot 3\right)^{2502}+\left(2-3\right)^{1573}$
  \end{vastaus}
\end{tehtava}

\begin{tehtava}
  Määritä jakojäännös, kun summa $(1^3 + 2^3 + 3^3 + \ldots + 105^3)$ jaetaan luvulla $3$.
  \begin{vastaus}
  $0$
  \end{vastaus}
\end{tehtava}

\begin{tehtava}
  Mikä on luvun a) $2^{77}$ b) $3^{33}$ viimeinen numero?
  \begin{vastaus}
  \begin{alakohdat}
  \alakohta{2}
  \alakohta{3}
  \end{alakohdat}
  \end{vastaus}
\end{tehtava}

\begin{tehtava}
  Olkoon $n$ luonnollinen luku. Osoita, että luku $4\cdot 7^{n+1}+2$ on jaollinen luvulla $3$.
  \begin{vastaus}
      Vihje: $4 \cdot 7^{n+1} \equiv 4 \cdot (7-2 \cdot 3)^{n+1}+2 $
  \end{vastaus}
\end{tehtava}

\begin{tehtava}
  Olkoon $n$ luonnollinen luku. Osoita, että luku $13^{2n} + 2^{3n+4} - 10$ on jaollinen luvulla $7$.
  \begin{vastaus}
  Vihje: $13 \equiv -1 (\mod{7})$ ja $8 \equiv 1 (\mod{7})$
  \end{vastaus}
\end{tehtava}

\begin{tehtava}
  Tutki, onko luku jaollinen luvulla $3$.
  \begin{alakohdat}
  \alakohta{$123\, 456\, 789$}
  \alakohta{$987\, 654\, 321\, 233$}
  \end{alakohdat}
  \begin{vastaus}
  \begin{alakohdat}
  \alakohta{On.}
  \alakohta{Ei.}
  \end{alakohdat}
  \end{vastaus}
\end{tehtava}

\begin{tehtava}
  Määritä jakojäännös, kun luku jaetaan luvulla $9$.
  \begin{alakohdat}
  \alakohta{$999\, 000\, 000\, 800$}
  \alakohta{$111\, 111\, 111\, 111$}
  \end{alakohdat}
  \begin{vastaus}
  \begin{alakohdat}
  \alakohta{8}
  \alakohta{3}
  \end{alakohdat}
  \end{vastaus}  
\end{tehtava}

\begin{tehtava}
  Todista, että $(n+1)$-numeroinen kokonaisluku $a_na_{n-1}a_{n-2}\ldots a_2a_1a_0$ on kongruentti numeroidensa summan kanssa modulo 9. Ohje: Esitä luku kymmenen potenssien avulla:
  \begin{multline*}
  a_na_{n-1}a_{n-2}\ldots a_2a_1a_0\\ = a_n \cdot 10^n + a_{n-1}\cdot 10^{n-1} + a_{n-2} \cdot 10^{n-2} + \\
  \ldots + a_2 \cdot 10^2 + a_1 \cdot 10 + a_0.
  \end{multline*}
  \begin{vastaus}
  Vihje: $10 \equiv 1 (\mod{9})$
  \end{vastaus}  
\end{tehtava}

\begin{tehtava}
  \begin{alakohdat}
  \alakohta{Ajattele jotakin kolminumeroista lukua ja kirjoita se paperille. Vaihda lukusi numeroiden järjestystä ja kirjoita uusi luku paperille. Laske lukujen erotus, jaa erotus luvulla $3$ ja kirjoita jakojäännös paperille. Lisää jakojäännökseen luku $7$, kerro näin saatu luku luvulla $2$ ja vähennä tuloksesta luku $4$. Tulos on $10$, eikö vain? Miten se voidaan tietää?}
  \alakohta{Keksi oma algoritmisi, jonka tulos tiedetään etukäteen.}
  \end{alakohdat}
  \begin{vastaus}
    \begin{alakohdat}
      \alakohta{Merkitään alkuperäistä lukua kymmenjärjestelmässä $ABC$. Kun järjestys käännetään, saadaan $CBA$. Näiden erotus on $100(A-C)+(C-A)=100(A-C)-(A-C)=99(A-C)$, joka on aina jaollinen kolmella. Näin jakojäännökseksi saadaan aina $0$. Algoritmin loppuosa on obfuskaatiota, sillä se ei riipu alkuperäisestä luvusta.}
    \end{alakohdat}
  \end{vastaus}
\end{tehtava}

\end{tehtavasivu}

% -----
\setcounter{tehtava}{0}

\begin{kotitehtavasivu}

\begin{tehtava}
	Määritä jakojäännös, kun
	\begin{alakohdat}
	\alakohta{luku $700 - 22 \cdot 185$ jaetaan luvulla $6$}
	\alakohta{luku $2799^{2799}$ jaetaan luvulla $7$}
	\alakohta{luku $1+2+3+4+ \ldots + 36 + 37$ jaetaan luvulla $8$.}
	\end{alakohdat}
\end{tehtava}

\begin{tehtava}
	Määritä jakojäännös, kun
	\begin{alakohdat}
	\alakohta{luku $2^{256}$ jaetaan luvulla $5$}
	\alakohta{luku $12^{10} \cdot 13^{3} - 114 \cdot 552$ jaetaan luvulla $11$}
	\alakohta{luku $34^{3} - 966$ jaetaan luvulla $9$}
	\alakohta{luku $6^{57}$ jaetaan luvulla $15$.}
	\end{alakohdat}
\end{tehtava}

\begin{tehtava}
	Tutki, onko luku $46^{78} + 89^{67}$ jaollinen viidellä. (Yo-koe, kevät 2011)
\end{tehtava}

\begin{tehtava}
	Määritä jakojäännös, kun summa $(2^0 + 2^1 + 2^2 + 2^3 + 2^4 + 2^5 + \ldots + 2^{244})$ jaetaan luvulla $5$.
\end{tehtava}

\begin{tehtava}
	Mikä on luvun a) $2^{103}$  b) $4^{111}$ viimeinen numero?
\end{tehtava}

\begin{tehtava}
	Mitkä ovat luvun $5^{30}$ kaksi viimeistä numeroa?
\end{tehtava}

\begin{tehtava}
	Osoita, että
	\begin{alakohdat}
	\alakohta{luku $10^n - 1$ on jaollinen luvulla $9$ kaikilla positiivisilla kokonaisluvuilla $n$}
	\alakohta{luku $11^n + 1$ on jaollinen luvulla $12$ kaikilla parittomilla positiivisilla kokonaisluvuilla $n$.}
	\end{alakohdat}
\end{tehtava}

\begin{tehtava}
	Olkoon $n$ luonnollinen luku. Määritä jakojäännös, kun luku $501 \cdot 4^{2n} + 2^{2n}$ jaetaan luvulla $5$.
\end{tehtava}

\begin{tehtava}
	Tutki, onko luku jaollinen luvulla $9$.
	\begin{alakohdat}
	\alakohta{$182\, 736\, 451$}
	\alakohta{$18\, 273\, 645\, 144$}
	\end{alakohdat}
\end{tehtava}

\begin{tehtava}
	Määritä jakojäännös, kun luku jaetaan luvulla $3$.
	\begin{alakohdat}
	\alakohta{$222\, 333\, 445$}
	\alakohta{$101\, 010\, 101\, 111$}
	\end{alakohdat}
\end{tehtava}

\begin{tehtava}
	Osoita luvun $11$ jaollisuussääntö: Kokonaisluku on jaollinen luvulla $11$, jos ja vain jos sen numeroiden vuorotteleva summa on jaollinen luvulla $11$. Vuorotteleva summa lasketaan siten, että luvun viimeisestä numerosta vähennetään toiseksi viimeinen numero, lisätään kolmanneksi viimeinen numero, vähennetään neljänneksi viimeinen numero jne. Vihje: $10\equiv -1\quad(\mod 11)$.
\end{tehtava}

\begin{tehtava}
	Tutki, onko luku jaollinen luvulla $11$.
	\begin{alakohdat}
	\alakohta{$24\, 927$}
	\alakohta{$928\, 072\, 618$}
	\alakohta{$7\, 000\, 000\, 000\, 000\, 007$}
	\alakohta{$7\, 000\, 000\, 000\, 000\, 070$}
	\end{alakohdat}
\end{tehtava}

\begin{tehtava}
	Todista, että $a^3b-b^3a$ on tasan jaollinen kolmella, jos $a$ ja $b$ ovat kokonaislukuja ja $a>b$. 
	[YO 1896 tehtävä 3]
\end{tehtava}

\end{kotitehtavasivu}
