\begin{enumerate}

\item Määritä jakojäännös, kun
\begin{enumerate}[a)]
\item luku $80 \cdot 30 - 352$ jaetaan luvulla $7$
\item luku $2^{140}$ jaetaan luvulla $3$
\item luku $0+1+2+3+ \ldots + 79 + 80$ jaetaan luvulla $4$.
\end{enumerate}

\item Määritä jakojäännös, kun
\begin{enumerate}[a)]
\item luku $5 \cdot 11^{99} + 20$ jaetaan luvulla $6$
\item luku $6^{120} \cdot 4^{301}$ jaetaan luvulla $5$
\item luku $3^{60} + 55$ jaetaan luvulla $8$
\item luku $2^{151}$ jaetaan luvulla $14$.
\end{enumerate}

\item
Näytä, että luku $7^{2502} + 2^{1573}$ on jaollinen kolmella. [YO kevät 1999 10b kohta 2]

\item Määritä jakojäännös, kun summa $(1^3 + 2^3 + 3^3 + \ldots + 105^3)$ jaetaan luvulla $3$.

\item Mikä on luvun a) $2^{77}$ b) $3^{33}$ viimeinen numero?

\item Olkoon $n$ luonnollinen luku. Osoita, että luku $4\cdot 7^{n+1}+2$ on jaollinen luvulla $3$.

\item Olkoon $n$ luonnollinen luku. Osoita, että luku $13^{2n} + 2^{3n+4} - 10$ on jaollinen luvulla $7$.

\item
Tutki, onko luku jaollinen luvulla $3$.
\begin{enumerate}[a)]
\item $123\, 456\, 789$
\item $987\, 654\, 321\, 233$
\end{enumerate}

\item
Määritä jakojäännös, kun luku jaetaan luvulla $9$.
\begin{enumerate}[a)]
\item $999\, 000\, 000\, 800$
\item $111\, 111\, 111\, 111$
\end{enumerate}

\item
Todista, että $(n+1)$-numeroinen kokonaisluku $a_na_{n-1}a_{n-2}\ldots a_2a_1a_0$ on kongruentti numeroidensa summan kanssa modulo 9. Ohje: Esitä luku kymmenen potenssien avulla:
\begin{multline*}
a_na_{n-1}a_{n-2}\ldots a_2a_1a_0\\ = a_n \cdot 10^n + a_{n-1}\cdot 10^{n-1} + a_{n-2} \cdot 10^{n-2} + \\
\ldots + a_2 \cdot 10^2 + a_1 \cdot 10 + a_0.
\end{multline*}

\item
\begin{enumerate}[a)]
\item Ajattele jotakin kolminumeroista lukua ja kirjoita se paperille. Vaihda lukusi numeroiden järjestystä ja kirjoita uusi luku paperille. Laske lukujen erotus, jaa erotus luvulla $3$ ja kirjoita jakojäännös paperille. Lisää jakojäännökseen luku $7$, kerro näin saatu luku luvulla $2$ ja vähennä tuloksesta luku $4$. Tulos on $10$, eikö vain? Miten se voidaan tietää?
\item Keksi oma algoritmisi, jonka tulos tiedetään etukäteen.
\end{enumerate}

\end{enumerate}

\subsection*{Kotitehtäviä}

\begin{enumerate}

\item Määritä jakojäännös, kun
\begin{enumerate}[a)]
\item luku $700 - 22 \cdot 185$ jaetaan luvulla $6$
\item luku $2799^{2799}$ jaetaan luvulla $7$
\item luku $1+2+3+4+ \ldots + 36 + 37$ jaetaan luvulla $8$.
\end{enumerate}

\item Määritä jakojäännös, kun
\begin{enumerate}[a)]
\item luku $2^{256}$ jaetaan luvulla $5$
\item luku $12^{10} \cdot 13^{3} - 114 \cdot 552$ jaetaan luvulla $11$
\item luku $34^{3} - 966$ jaetaan luvulla $9$
\item luku $6^{57}$ jaetaan luvulla $15$.
\end{enumerate}


\item
Tutki, onko luku $46^{78} + 89^{67}$ jaollinen viidellä. (Yo-koe, kevät 2011)

\item Määritä jakojäännös, kun summa $(2^0 + 2^1 + 2^2 + 2^3 + 2^4 + 2^5 + \ldots + 2^{244})$ jaetaan luvulla $5$.

\item Mikä on luvun a) $2^{103}$  b) $4^{111}$ viimeinen numero?

\item Mitkä ovat luvun $5^{30}$ kaksi viimeistä numeroa?

\item Osoita, että
\begin{enumerate}[a)]
\item luku $10^n - 1$ on jaollinen luvulla $9$ kaikilla positiivisilla kokonaisluvuilla $n$
\item luku $11^n + 1$ on jaollinen luvulla $12$ kaikilla parittomilla positiivisilla kokonaisluvuilla $n$.
\end{enumerate}

\item Olkoon $n$ luonnollinen luku. Määritä jakojäännös, kun luku $501 \cdot 4^{2n} + 2^{2n}$ jaetaan luvulla $5$.

\item
Tutki, onko luku jaollinen luvulla $9$.
\begin{enumerate}[a)]
\item $182\, 736\, 451$
\item $18\, 273\, 645\, 144$
\end{enumerate}

\item
Määritä jakojäännös, kun luku jaetaan luvulla $3$.
\begin{enumerate}[a)]
\item $222\, 333\, 445$
\item $101\, 010\, 101\, 111$
\end{enumerate}

\item
Osoita luvun $11$ jaollisuussääntö: Kokonaisluku on jaollinen luvulla $11$, jos ja vain jos sen numeroiden vuorotteleva summa on jaollinen luvulla $11$. Vuorotteleva summa lasketaan siten, että luvun viimeisestä numerosta vähennetään toiseksi viimeinen numero, lisätään kolmanneksi viimeinen numero, vähennetään neljänneksi viimeinen numero jne. Vihje: $10\equiv -1\quad(\mod 11)$.

\item Tutki, onko luku jaollinen luvulla $11$.
\begin{enumerate}[a)]
\item $24\, 927$
\item $928\, 072\, 618$
\item $7\, 000\, 000\, 000\, 000\, 007$
\item $7\, 000\, 000\, 000\, 000\, 070$
\end{enumerate}

\item Todista, että $a^3b-b^3a$ on tasan jaollinen kolmella, jos $a$ ja $b$ ovat kokonaislukuja ja $a>b$. 
[YO 1896 tehtävä 3]

\end{enumerate}

\newpage
