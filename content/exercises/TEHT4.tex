\begin{tehtavasivu}

	\begin{tehtava}
		\begin{alakohdat}
			\alakohta{Laske neljän peräkkäisen kokonaisluvun summa.
			Toista lasku useilla neljän peräkkäisen kokonaisluvun
			joukoilla. Mitä havaitset summasta?}
			\alakohta{Jos neljän peräkkäisen kokonaisluvun joukon pienin
			luku on $m$, niin millainen esitysmuoto on kolmella
			muulla luvulla?}
			\alakohta{Todista a-kohdan tulos matemaattisesti käyttäen
			hyväksi b-kohdan esitysmuotoja.}
		\end{alakohdat}
		\begin{vastaus}
			\begin{alakohdat}
				\alakohta{$1+2+3+4=10$, $2+3+4+5=14$, $3+4+5+6=18$}
				\alakohta{$\{m, m+1, m+2, m+3\}$}
				\alakohta{$m+(m+1)+(m+2)+(m+3) = m+m+m+m+1+2+3 = 4m+6 = 2(2m+3)$}
			\end{alakohdat}
		\end{vastaus}
	\end{tehtava}

	\begin{tehtava}
		Todista, että parillisen ja parittoman kokonaisluvun summa on pariton.
		\begin{vastaus}
			Parillinen luku voidaan esittää muodossa $2k$ ja pariton luku muodossa $2l+1$, missä $k, l \in \zz$.
			Tällöin summaksi saadaan $2k+2l+1 = 2(k+l)+1$, joka on pariton.
		\end{vastaus}
	\end{tehtava}

	\begin{tehtava}
		Todista, että kahden parillisen kokonaisluvun tulo on parillinen.
		\begin{vastaus}
			Parilliset luvut voidaan esittää muodoissa $2k$ ja $2l$, missä $k, l \in \zz$
			Tällöin tuloksi saadaan $2k \cdot 2l = 2(2kl)$, joka on parillinen.
		\end{vastaus}
	\end{tehtava}

	\begin{tehtava}
		Todista, että kahden rationaaliluvun tulo on rationaalinen.
		\begin{vastaus}
			Rationaaliluvut voidaan esittää muodoissa
		\end{vastaus}
	\end{tehtava}

	\begin{tehtava}
		Olkoot $a$ ja $b$ kokonaislukuja. Todista, että luku
		$a + b$ on parillinen silloin ja vain silloin, kun luku
		$a - b$ on parillinen.
		\begin{vastaus}
		\end{vastaus}
	\end{tehtava}

	\begin{tehtava}
		Olkoon $m$ sellainen kokonaisluku, että $m^2$ on
		pariton. Todista, että tällöin $m$ on pariton.
		\begin{vastaus}
		\end{vastaus}
	\end{tehtava}

	\begin{tehtava}
		Todista väite todeksi tai epätodeksi: Kahden
		irrationaaliluvun summa on aina irrationaaliluku.
		\begin{vastaus}
		\end{vastaus}
	\end{tehtava}

	\begin{tehtava}
		Luku $\sqrt{3}$ on irrationaaliluku. Todista,
		että $\sqrt{3}+1/2$ on irrationaaliluku. Vihje: Käytä
		epäsuoraa todistusta.
		\begin{vastaus}
		\end{vastaus}
	\end{tehtava}

	\begin{tehtava}
		Todista, että irrationaaliluvun ja nollasta poikkeavan rationaaliluvun tulo on irrationaaliluku.
		\begin{vastaus}
		\end{vastaus}
	\end{tehtava}

	\begin{tehtava}
		Todista, että luku $\sqrt{6}$ on irrationaaliluku.
		\begin{vastaus}
		\end{vastaus}
	\end{tehtava}

	\begin{tehtava}
		Todista: Jos luku $a$ on irrationaaliluku, niin
		myös luku
		\[
			\frac{2a-5}{3a-11}
		\]
		on irrationaaliluku. Vihje: Käytä symbolisen laskimen
		\texttt{solve}-toimintoa.
		\begin{vastaus}
		\end{vastaus}
	\end{tehtava}

	\begin{tehtava}
		Tarkastellaan suorakulmaista kolmiota, jonka kateettien
		pituudet ovat $a$ ja $b$ ja hypotenuusan pituus $c$.
		Todista väite todeksi tai epätodeksi.
		\begin{alakohdat}
			\alakohta{On olemassa sellainen suorakulmainen kolmio,
				jonka sivujen pituudet $a$, $b$ ja $c$ ovat kaikki parillisia kokonaislukuja.}
			\alakohta{On olemassa sellainen suorakulmainen kolmio,
				jonka sivujen pituudet $a$, $b$ ja $c$ ovat kaikki 
				parittomia kokonaislukuja.}
		\end{alakohdat}
		\begin{vastaus}
		\end{vastaus}
	\end{tehtava}

	\begin{tehtava}
		Kokonaisluvut $0, 1, 2, \ldots, 12$ asetetaan
		mielivaltaiseen järjestykseen ympyrän kehälle. Todista,
		että joidenkin neljän peräkkäisen luvun summa on
		vähintään 26.
		\begin{vastaus}
		\end{vastaus}
	\end{tehtava}

\end{tehtavasivu}

\begin{tehtavasivu} % kotitehtäviä

	\begin{tehtava}
		\begin{alakohdat}
		\alakohta{Laske luonnollisten lukujen $1, 2, 3,\ldots, 10$ neliöt.}
		\alakohta{Esitä väite luonnollisten lukujen neliöiden parillisuudesta tai parittomuudesta.}
		\alakohta{Todista väitteesi matemaattisesti.}
		\end{alakohdat}
		\begin{vastaus}
		\end{vastaus}
	\end{tehtava}

	\begin{tehtava}
		Todista, että kolmen parittoman kokonaisluvun summa on pariton.
		\begin{vastaus}
		\end{vastaus}
	\end{tehtava}

	\begin{tehtava}
		Todista, että kolmen peräkkäisen kokonaisluvun summa on jaollinen luvulla 3.
		\begin{vastaus}
		\end{vastaus}
	\end{tehtava}

	\begin{tehtava}
		Todista, että kahden parittoman kokonaisluvun tulo on pariton.
		\begin{vastaus}
		\end{vastaus}
	\end{tehtava}

	\begin{tehtava}
		Olkoot $a$ ja $b$ reaalilukuja. Todista, että tulo $ab = 0$, jos ja vain jos $a=0$ tai $b=0$.
		\begin{vastaus}
		\end{vastaus}
	\end{tehtava}

	\begin{tehtava}
		Olkoon $n$ kokonaisluku. Todista, että luku $n^{2} + 3n + 1$ on aina pariton. Käytä a) suoraa b) epäsuoraa
		todistusta.
		\begin{vastaus}
		\end{vastaus}
	\end{tehtava}

	\begin{tehtava}
		Todista, että ei ole olemassa sellaisia positiivisia kokonaislukuja $x$ ja $y$, jotka toteuttavat
		yhtälön $4^{x} = 7^{y}$.
		\begin{vastaus}
		\end{vastaus}
	\end{tehtava}

	\begin{tehtava}
		Todista väite todeksi tai epätodeksi: Kahden
		irrationaaliluvun tulo voi olla rationaaliluku.
		\begin{vastaus}
		\end{vastaus}
	\end{tehtava}

	\begin{tehtava}
		Todista, että luku $\sqrt{5}$ on irrationaaliluku.
		Tarvitset seuraavaa lisätietoa: jos kahden kokonaisluvun
		tulo on jaollinen luvulla 5, niin ainakin toinen tulon
		tekijöistä on jaollinen luvulla 5.
		\begin{vastaus}
		\end{vastaus}
	\end{tehtava}

	\begin{tehtava}
		Tarkastellaan paritonta määrää parittomia kokonaislukuja.
		Todista, että lukujen keskiarvo ei voi olla nolla.
		\begin{vastaus}
		\end{vastaus}
	\end{tehtava}

	\begin{tehtava}
		Tarkastellaan kolmiota, jonka sivujen pituudet ovat
		$a$, $b$ ja $c$. Niin sanotun \termi{Heronin kaava}{Heronin kaavan} mukaan
		kolmion pinta-alalle pätee \[A = \sqrt{p(p-a)(p-b)(p-c)},\]
		missä $p = \frac{1}{2}(a+b+c)$. Todista, että jos kolmion
		sivujen pituudet ovat luvulla 4 jaollisia kokonaislukuja,
		niin kolmion pinta-alan neliö on jaollinen luvulla 16.
		\begin{vastaus}
		\end{vastaus}
	\end{tehtava}

	\begin{tehtava}
		Niin sanotun \termi{Fermat'n suuri lause}{Fermat'n suuren lauseen} mukaan ei ole
		olemassa sellaisia positiivisia kokonaislukuja $x$, $y$
		ja $z$, jotka toteuttaisivat yhtälön $x^{n} + y^{n} = z^{n}$,
		missä $n$ on lukua 2 suurempi kokonaisluku.
		Erityisesti siis yhtälöllä $x^{3} + y^{3} = z^{3}$ ei
		ole ratkaisua, jos $x$, $y$ ja $z$ ovat positiivisia
		kokonaislukuja. Käytä tätä tulosta hyväksesi a- ja b-kohtien ratkaisemisessa.
            \begin{alakohdat}
			\alakohta{Todista, että ei ole olemassa sellaisia
			positiivisia rationaalilukuja $x$, $y$ ja $z$, jotka
			toteuttavat yhtälön $x^{3} + y^{3} = z^{3}$.}
			\alakohta{Todista väite todeksi tai epätodeksi: ei ole
			olemassa sellaisia keskenään eri suuria kokonaislukuja
			$x$, $y$ ja $z$, jotka toteuttavat yhtälön $x^{3} + y^{3} = z^{3}$.}
		\end{alakohdat}
		\begin{vastaus}
		\end{vastaus}
	\end{tehtava}

	\begin{tehtava}
		Olkoot $a$ ja $b$ reaalilukuja, joille pätee $0 \le a \le 1$ ja $0 \le b \le 1$.
		Todista, että tällöin $0 \le \frac{a + b}{1 + ab} \le 1$.
		\begin{vastaus}
		\end{vastaus}
	\end{tehtava}

\end{tehtavasivu}
