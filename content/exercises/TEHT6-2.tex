\begin{tehtavasivu}

\begin{tehtava}
    Tutki, onko Diofantoksen yhtälöllä ratkaisua.
    
    \begin{alakohdat}
        \alakohta{$7x + 5y = 3$}
        \alakohta{$5x + 85y = 42$}
        \alakohta{$6x + 51y = 100$}
    \end{alakohdat}

    \begin{vastaus}
        \begin{alakohdat}
            \alakohta{On ratkaisu.}
            \alakohta{Ei ole ratkaisua.}
            \alakohta{Ei ole ratkaisua.}
        \end{alakohdat}
    \end{vastaus}
    
\end{tehtava}

\begin{tehtava}
    Leirille osallistui $364$ nuorta. Oliko mahdollista majoittaa osallistujat $24$ ja $16$ hengen parakkeihin siten, että yksikään parakki ei jäänyt vajaaksi?
    
    \begin{vastaus}
        Ei ole mahdollista.
    \end{vastaus}
    
\end{tehtava}

\begin{tehtava}
    Määritä Diofantoksen yhtälön jokin ratkaisu.
    
    \begin{alakohdat}
        \alakohta{$14x + 49y = \syt(14, 49)$}
        \alakohta{$56x + 72y = \syt(56, 72)$}
    \end{alakohdat}

    \begin{vastaus}
        \begin{alakohdat}
            \alakohta{Esimerkiksi $x = -3$ ja $y = 1$.}
            \alakohta{Esimerkiksi $x = -5$ ja $y = 4$.}
        \end{alakohdat}
    \end{vastaus}
    
\end{tehtava}

\begin{tehtava}
    Määritä Diofantoksen yhtälön jokin ratkaisu.
    
    \begin{alakohdat}
        \alakohta{$56x + 72y = 40$}
        \alakohta{$24x + 138y = -24$}
    \end{alakohdat}

    \begin{vastaus}
        \begin{alakohdat}
            \alakohta{Esimerkiksi $x = -7$ ja $y = 6$.}
            \alakohta{Esimerkiksi $x = -1$ ja $y = 0$.}
        \end{alakohdat}
    \end{vastaus}
    
\end{tehtava}

\begin{tehtava}
    Tutki, onko suoralla
    \begin{alakohdat}
        \alakohta{$26x + 91y + 10 = 0$}
        \alakohta{$529x + 621y - 92 = 0$}
    \end{alakohdat}
    pisteitä, joiden molemmat koordinaatit ovat kokonaislukuja.

    \begin{vastaus}
        \begin{alakohdat}
            \alakohta{Ei ole.}
            \alakohta{On, esimerkiksi piste $(-1, 1)$.}
        \end{alakohdat}
    \end{vastaus}
    
\end{tehtava}

\begin{tehtava}
    Määritä Diofantoksen yhtälön kaikki ratkaisut.
    
    \begin{alakohdat}
        \alakohta{$2x + 3y = 1$}
        \alakohta{$2x + 3y = 7$}
    \end{alakohdat}

    \begin{vastaus}
        \begin{alakohdat}
            \alakohta{$x = -1 + 3n$ ja $y = 1 - 2n$, $n\in\zz$.}
            \alakohta{$x = 2 + 3n$ ja $y = 1 - 2n$, $n\in\zz$.}
        \end{alakohdat}
    \end{vastaus}
    
\end{tehtava}

\begin{tehtava}
    Määritä Diofantoksen yhtälön $45x + 21y = -6$ kaikki ratkaisut.

    \begin{vastaus}
        $x = -2 + 7n$ ja $y = 4 - 15n$, $n\in\zz$.
    \end{vastaus}
    
\end{tehtava}

\begin{tehtava}
    Määritä Diofantoksen yhtälön $13509x + 10203y = 228$ kaikki ratkaisut.
    
    \begin{vastaus}
        $x = 105 + 179n$ ja $y = -139 - 237n$, $n\in\zz$.
    \end{vastaus}
    
\end{tehtava}

\begin{tehtava}
    Keksi Diofantoksen yhtälö, jolla
    \begin{alakohdat}
        \alakohta{ei ole ratkaisua}
        \alakohta{on äärettömän monta ratkaisua.}
    \end{alakohdat}

    \begin{vastaus}
        \begin{alakohdat}
            \alakohta{Esimerkiksi $6x + 8y = 1$}
            \alakohta{Esimerkiksi $13x + 24y = 47$}
        \end{alakohdat}
    \end{vastaus}
    
\end{tehtava}

\begin{tehtava}
    Määritä Diofantoksen yhtälön $63x + 279y = 450$ kaikki ratkaisut. Mitkä niistä toteuttavat ehdon $|x| + |y| < 25$?
    
    \begin{vastaus}
        Ratkaisut ovat $x = 16 + 31n$ ja $y = -2 - 7n$, $n\in\zz$. Näistä ehdon $|x| + |y| < 25$ toteuttaa ratkaisu $x = -15$ ja $y = 5$ sekä ratkaisut $x = 16$ ja $y = -2$.
    \end{vastaus}
    
\end{tehtava}

\begin{tehtava}
    Käytettävissä on 8 gramman ja 12 gramman punnuksia. Kuinka monta kummankinlaista punnusta tarvitaan, jotta punnusten kokonaismassaksi tulisi 100 grammaa? Selvitä kaikki vaihtoehdot.
    
    \begin{vastaus}
        Vaihtoehdot kun 8 gramman punnuksien määrä on $x$ ja 12 gramman punnuksien määrä on $y$:
        \begin{itemize}
            \item $x = 2$ ja $y = 7$
            \item $x = 5$ ja $y = 5$
            \item $x = 8$ ja $y = 3$
            \item $x = 11$ ja $y = 1$.
        \end{itemize}
    \end{vastaus}
    
\end{tehtava}

\begin{tehtava}
    Ruhtinas jakoi $63$ yhtä suurta kekoa hedelmiä sekä $7$ erillistä hedelmää tasan $23$ matkalaiselle. Kuinka monta hedelmää kussakin keossa oli? Vihje: Tutki yhtälöä $63x + 7 = 23y$. (Mahavira, v. 850)
    
    \begin{vastaus}
        Mikä tahansa hedelmien määrä muotoa $5 + 23n$, missä $n\in\zz$ ja $n\geq 0$, on mahdollinen.
    \end{vastaus}
    
\end{tehtava}

\begin{tehtava}
    % Tarkistettu (Topi Talvitie, 9.11.2013)
    Olkoot $a$, $b$, $c$ ja $d$ positiivisia kokonaislukuja. Osoita, että yhtälöllä $ax+by+cz=d$ on kokonaislukuratkaisu, jos ja vain jos luku $d$ on jaollinen lukujen $a, b$ ja $c$ suurimmalla yhteisellä tekijällä.
\end{tehtava}

\end{tehtavasivu}
\begin{kotitehtavasivu}

\begin{tehtava}
    Tutki, onko Diofantoksen yhtälöllä ratkaisua.
    
    \begin{alakohdat}
        \alakohta{$9x + 6y = 72$}
        \alakohta{$12x + 10y = 323$}
        \alakohta{$14x + 35y = -91$}
    \end{alakohdat}

    \begin{vastaus}
        \begin{alakohdat}
            \alakohta{On ratkaisu.}
            \alakohta{Ei ole ratkaisua.}
            \alakohta{On ratkaisu.}
        \end{alakohdat}
    \end{vastaus}
    
\end{tehtava}

\begin{tehtava}
    Emilia sai valmistujaislahjaksi $200$ euron lahjakortin erääseen keramiikkapajaan. Hän osti pajasta $27$ euron hintaisia kynttilänjalkoja ja $15$ euron hintaisia jälkiruokalautasia. Hän maksoi ostoksensa lahjakortilla ja sai rahaa takaisin $12$ euroa. Laskiko myyjä oikein?
    
    \begin{vastaus}
        Ei laskenut.
    \end{vastaus}
    
\end{tehtava}

\begin{tehtava}
    Määritä Diofantoksen yhtälön jokin ratkaisu.

    \begin{alakohdat}
        \alakohta{$59x + 12y = \syt(59, 12)$}
        \alakohta{$119x + 272y = \syt(119, 272)$}
    \end{alakohdat}
    
    \begin{vastaus}
        \begin{alakohdat}
            \alakohta{Esimerkiksi $x = 1$ ja $y = -4$.}
            \alakohta{Esimerkiksi $x = -9$ ja $y = 4$.}
        \end{alakohdat}
    \end{vastaus}
    
\end{tehtava}

\begin{tehtava}
    Määritä Diofantoksen yhtälön jokin ratkaisu.

    \begin{alakohdat}
        \alakohta{$36x + 16y = 28$}
        \alakohta{$221x + 35y = 2$}
    \end{alakohdat}
    
    \begin{vastaus}
        \begin{alakohdat}
            \alakohta{Esimerkiksi $x = -1$ ja $y = 4$.}
            \alakohta{Esimerkiksi $x = -3$ ja $y = 19$.}
        \end{alakohdat}
    \end{vastaus}
    
\end{tehtava}

\begin{tehtava}
    Määritä Diofantoksen yhtälön kaikki ratkaisut.
    \begin{alakohdat}
        \alakohta{$2x + 6y = 2$}
        \alakohta{$2x + 6y = -10$}
    \end{alakohdat}

    \begin{vastaus}
        \begin{alakohdat}
            \alakohta{$x = 1 + 3n$ ja $y = -n$, $n\in\zz$}
            \alakohta{$x = -5 + 3n$ ja $y = -n$, $n\in\zz$}
        \end{alakohdat}
    \end{vastaus}
    
\end{tehtava}

\begin{tehtava}
    Määritä Diofantoksen yhtälön $35x + 84y = 14$ kaikki ratkaisut.
    
    \begin{vastaus}
        $x = -2 + 12n$ ja $y = 1 - 5n$, $n\in\zz$
    \end{vastaus}
    
\end{tehtava}

\begin{tehtava}
    Määritä Diofantoksen yhtälön $11925x + 3843y = -117$ kaikki ratkaisut.
    
    \begin{vastaus}
        $x = -10 + 427n$ ja $y = 31 - 1325n$, $n\in\zz$
    \end{vastaus}
    
\end{tehtava}

\begin{tehtava}
    Määritä Diofantoksen yhtälön $168x + 204y = 24$ kaikki ratkaisut. Mille ratkaisuille pätee $-50 \le x \le 0$ ja $y > 10$?
    
    \begin{vastaus}
        Kaikki ratkaisut ovat $x = 5 + 17n$ ja $y = -4 - 14n$, missä $n\in\zz$. Niistä ehdot $-50 \le x \le 0$ ja $y > 10$ toteuttaa ratkaisut $x = -29$ ja $y = 24$ sekä $x = -46$ ja $y = 38$.
    \end{vastaus}
    
\end{tehtava}

\begin{tehtava}
    Esitä luku $100$ kahden positiivisen kokonaisluvun summana niin, että toinen luvuista on jaollinen luvulla $7$ ja toinen luvulla $11$. (Euler, v. 1770)
    
    \begin{vastaus}
        $100 = 56 + 44$, missä $56 = 8 \cdot 7$ ja $44 = 4 \cdot 11$.
    \end{vastaus}
    
\end{tehtava}

\begin{tehtava}
    % Miksi tämä on lineaaristen Diofantoksen yhtälöiden luvussa?
    Yhtiön kauppavoitto 150 mk on jaettava tasan osakkaille. Jos osakkaita olisi ollut 5 enemmän, olisi jokainen saanut 5 mk vähemmän. Montako osakasta oli yhtiössä?  [YO 1874 tehtävä 6]
    
    \begin{vastaus}
        10 osakasta
    \end{vastaus}
    
\end{tehtava}

\begin{tehtava}
    Sata lyhdettä viljaa jaetaan sadalle henkilölle niin, että kukin mies saa $3$ lyhdettä, nainen $2$ lyhdettä ja lapsi puoli lyhdettä. Kuinka monta miestä, naista ja lasta on? (Alcuin Yorkilainen, v. 775)
    
    \begin{vastaus}
        Merkitään miesten lukumäärää kirjaimella $a$, naisten lukumäärää kirjaimella $b$ ja lasten lukumäärää kirjaimella $c$. Mahdolliset vastaukset ovat
        \begin{itemize}
            \item $a = 20$, $b = 0$ ja $c = 80$
            \item $a = 17$, $b = 5$ ja $c = 78$
            \item $a = 14$, $b = 10$ ja $c = 76$
            \item $a = 11$, $b = 15$ ja $c = 74$
            \item $a = 8$, $b = 20$ ja $c = 72$
            \item $a = 5$, $b = 25$ ja $c = 70$
            \item $a = 2$, $b = 30$ ja $c = 68$.
        \end{itemize}
    \end{vastaus}
    
\end{tehtava}

\begin{tehtava} %(Lisämateriaalia.)
    % Tarkistettu (Topi Talvitie, 10.11.2013)
    * Osoita suoralla sijoituksella, että $a=2$, $b=4$, $c=3$ ja $d=1$ on yksi Kexleruksen viiniongelman ratkaisuista.
\end{tehtava}

\begin{tehtava} %(Lisämateriaalia.)
    * Ratkaise Kexleruksen viiniongelma, kun asetetaan $b=0$ ja $d=0$.

    \begin{vastaus}
        $a = 4$ ja $c = 6$
    \end{vastaus}
    
\end{tehtava}

\begin{tehtava} %(Lisämateriaalia.)
    % Tarkistettu (Topi Talvitie, 10.11.2013)
    * Osoita, että Kexleruksen viiniongelmalla ei ole ratkaisuja, jos asetetaan $a=0$ ja $d=0$.
\end{tehtava}

\end{kotitehtavasivu}
