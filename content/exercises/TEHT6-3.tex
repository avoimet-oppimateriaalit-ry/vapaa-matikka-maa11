\begin{tehtavasivu}

\begin{tehtava}
    Mitkä ovat kymmenen pienintä alkulukua?
    
    \begin{vastaus}
        $2$, $3$, $5$, $7$, $11$, $13$, $17$, $19$, $23$, $29$
    \end{vastaus}
    
\end{tehtava}

\begin{tehtava}
    Onko luku
    \begin{alakohdat}
        \alakohta{$37$}
        \alakohta{$77$}
        \alakohta{$87$}
        \alakohta{$101$}
    \end{alakohdat}
    alkuluku?
    
    \begin{vastaus}
        \begin{alakohdat}
            \alakohta{On.}
            \alakohta{Ei ole.}
            \alakohta{Ei ole.}
            \alakohta{On.}
        \end{alakohdat}
    \end{vastaus}
    
\end{tehtava}

\begin{tehtava}
    Määritä Eratostheneen seulan avulla kaikki välillä $75,\ldots, 100$ olevat alkuluvut.

    \begin{vastaus}
        $79$, $83$, $89$, $97$
    \end{vastaus}
    
\end{tehtava}

\begin{tehtava}
    Pitääkö väite paikkansa? Perustele.
     
    \begin{alakohdat}
        \alakohta{Jos tulo $mn$ on jaollinen luvulla $97$, niin ainakin toinen kokonaisluvuista $m$ ja $n$ on jaollinen luvulla $97$.}
        \alakohta{Jos tulo $mn$ on jaollinen luvulla $55$, niin ainakin toinen kokonaisluvuista $m$ ja $n$ on jaollinen luvulla $55$.}
    \end{alakohdat}

    \begin{vastaus}
        \begin{alakohdat}
            \alakohta{Väite pätee Eukleideen lemman nojalla, sillä $97$ on alkuluku.}
            \alakohta{Väite ei päde. Vastaesimerkki: $m = 11$ ja $n = 5$.}
        \end{alakohdat}
    \end{vastaus}
    
\end{tehtava}

\begin{tehtava}
    Määritä luvun alkutekijähajotelma.
    
    \begin{alakohdat}
        \alakohta{$42$}
        \alakohta{$600$}
        \alakohta{$4410$}
    \end{alakohdat}

    \begin{vastaus}
        \begin{alakohdat}
            \alakohta{$2\cdot 3\cdot 7$}
            \alakohta{$2^3\cdot 3\cdot 5^2$}
            \alakohta{$2\cdot 3^2\cdot 5\cdot 7^2$}
        \end{alakohdat}
    \end{vastaus}
    
\end{tehtava}

\begin{tehtava}
    Määritä luvun $8!$ alkutekijähajotelma.
    
    \begin{vastaus}
        $2^7\cdot 3^2\cdot 5\cdot 7$
    \end{vastaus}
    
\end{tehtava}

\begin{tehtava}
    Millä luvuilla on luvun $241$ jaollisuus vähintään tutkittava, jotta saadaan selville, onko se alkuluku?
    
    \begin{vastaus}
        Riittää tarkistaa jaollisuus alkuluvuilla $\sqrt{241}$ asti, eli siis alkuluvuilla $2$, $3$, $5$, $7$, $11$, $13$.
    \end{vastaus}
\end{tehtava}

\begin{tehtava}
    Onko luku
    \begin{alakohdat}
        \alakohta{$187$}
        \alakohta{$197$}
        \alakohta{$299$}
        \alakohta{$601$}
    \end{alakohdat}
    alkuluku?

    \begin{vastaus}
        \begin{alakohdat}
            \alakohta{Ei ole.}
            \alakohta{On.}
            \alakohta{Ei ole.}
            \alakohta{On.}
        \end{alakohdat}
    \end{vastaus}
    
\end{tehtava}

\begin{tehtava}
    % Tarkistettu (Topi Talvitie, 10.11.2013)
    Luvut $2$ ja $3$ ovat kaksi peräkkäistä kokonaislukua, jotka molemmat ovat alkulukuja. Osoita, että muita peräkkäisiä alkulukuja ei ole olemassa.
\end{tehtava}

\begin{tehtava}
    Mikä on lukujen suurin yhteinen tekijä?
    
    \begin{alakohdat}
        \alakohta{$2^3 \cdot 3^2 \cdot 5^5 \cdot 7^5$ ja $2^5 \cdot 3^2
\cdot 5^2$}
        \alakohta{$2 \cdot 3 \cdot 5 \cdot 11$ ja $2 \cdot 3 \cdot 5
\cdot 11$}
    \end{alakohdat}
    
    \begin{vastaus}
        \begin{alakohdat}
            \alakohta{$1800 = 2^3\cdot 3^2\cdot 5^2$}
            \alakohta{$330 = 2 \cdot 3 \cdot 5 \cdot 11$}
        \end{alakohdat}
    \end{vastaus}
    
\end{tehtava}

\begin{tehtava}
    Mikä on lukujen pienin yhteinen jaettava?
    
    \begin{alakohdat}
        \alakohta{$3^7 \cdot 5^3 \cdot 7^3$ ja $2^9 \cdot 3^6 \cdot 5^9$}
        \alakohta{$19^{31}$ ja $19^{17}$}
    \end{alakohdat}
    
    \begin{vastaus}
        \begin{alakohdat}
            \alakohta{$2^9\cdot 3^7\cdot 5^9\cdot 7^3$}
            \alakohta{$19^{31}$}
        \end{alakohdat}
    \end{vastaus}
    
\end{tehtava}

\begin{tehtava}
    Määritä alkutekijähajotelman avulla lukujen
    \begin{alakohdat}
        \alakohta{$15$ ja $42$}
        \alakohta{$20$ ja $75$}
        \alakohta{$2250$ ja $30800$}
    \end{alakohdat}
    suurin yhteinen tekijä ja pienin yhteinen jaettava.
    
    \begin{vastaus}
        Suurin yhteinen tekijä:
        \begin{alakohdatrivi}
            \alakohta{$3$}
            \alakohta{$5$}
            \alakohta{$50$.}
        \end{alakohdatrivi}
        Pienin yhteinen jaettava:
        \begin{alakohdatrivi}
            \alakohta{$210$}
            \alakohta{$300$}
            \alakohta{$1386000$.}
        \end{alakohdatrivi}
    \end{vastaus}
    
\end{tehtava}

\begin{tehtava}
    Jokamiesluokan kilpa-auto A kiertää radan 55 sekunnissa ja B 65 sekunnissa. Kuinka pitkän ajan kuluttua lähdöstä autot ovat uudestaan yhtä aikaa lähtölinjalla?
    
    \begin{vastaus}
        715 sekunnin kuluttua.
    \end{vastaus}
\end{tehtava}

\begin{tehtava}
    Olkoot $x$ ja $y$ kokonaislukuja. Ratkaise yhtälö $(3x+2y)(x+y)=21$.
    
    \begin{vastaus}
        Ratkaisut ovat:
        \begin{itemize}
            \item $x = -41$ ja $y = 62$
            \item $x = -19$ ja $y = 18$
            \item $x = -11$ ja $y = 18$
            \item $x = -1$ ja $y = -2$
            \item $x = 1$ ja $y = 2$
            \item $x = 11$ ja $y = -18$
            \item $x = 19$ ja $y = -18$
            \item $x = 41$ ja $y = -62$.
        \end{itemize}
    \end{vastaus}
    
\end{tehtava}

\begin{tehtava}
    Perustele väite todeksi tai epätodeksi.
    
    \begin{alakohdat}
        \alakohta{Jos luku on jaollinen luvuilla $6$ ja $25$, niin se on
jaollinen myös luvulla $150$.}
        \alakohta{Jos luku on jaollinen luvuilla $6$ ja $9$, niin se on
jaollinen myös luvulla $54$.}
    \end{alakohdat}
    
    \begin{vastaus}
    
        \begin{alakohdat}
            \alakohta{Väite pätee, sillä $\syt(6, 25) = 1$ ja $150 = 6\cdot 25$.}
            \alakohta{Väite ei päde. Vastaesimerkki: luku $18$ on jaollinen luvuilla $6$ ja $9$ mutta ei luvulla $54$.}
        \end{alakohdat}
    \end{vastaus}
    
\end{tehtava}

\begin{tehtava}
    % Tarkistettu (Topi Talvitie, 10.11.2013)
    Olkoon $n$ kokonaisluku. Osoita, että luku $n(n+1)(n+2)$ on jaollinen luvulla $6$.
\end{tehtava}

\begin{tehtava}
    % Tarkistettu (Topi Talvitie, 10.11.2013)
    Osoita, että jos $n$ on kokonaisluku, niin luku $7n^3 - 7n$ on jaollinen luvulla $42$.
\end{tehtava}

\begin{tehtava}
    % Tarkistettu (Topi Talvitie, 10.11.2013)
    Osoita, että jos $n$ on kokonaisluku, niin luku $n^4 - n^2$ on jaollinen luvulla $12$.
\end{tehtava}

\begin{tehtava}
    Määritä jakojäännös, kun luku $9^{13}$ jaetaan luvulla $13$.
    
    \begin{vastaus}
        $9$
    \end{vastaus}
    
\end{tehtava}

\begin{tehtava}
    % Tarkistettu (Topi Talvitie, 10.11.2013)
    Toisinaan Fermat'n pieni lause esitetään seuraavassa
    muodossa:
    jos $p$ on alkuluku ja $a$ on kokonaisluku, joka ei ole
    luvun $p$ monikerta, niin tällöin
    \[
    a^{p-1}\equiv 1\quad (\mod p).
    \]
    Osoita lauseen tämän muodon avulla, että $n^{19}\equiv n \quad
    (\mod 19)$ kaikilla luonnollisilla luvuilla $n$.
\end{tehtava}

\begin{tehtava}
    Muotoa $F_n = 2^{2^n}+1$, $n=0, 1, 2, \ldots$ olevia lukuja sanotaan \termi{Fermat'n luvut}{Fermat'n luvuiksi}. Euler tutki 1700-luvulla, ovatko kaikki Fermat'n luvut alkulukuja. Ratkaise tämä Eulerin ongelma laskinta käyttäen.
    
    \begin{vastaus}
        Kaikki Fermat'n luvut eivät ole alkulukuja, sillä $F_5$ ei ole alkuluku. Tämä voidaan todeta laskimella käymällä Fermat'n lukuja läpi.
    \end{vastaus}
\end{tehtava}

\begin{tehtava}
    % Tarkistettu (Topi Talvitie, 10.11.2013)
    % Tämä spoilaa edellisen tehtävän, pitäisikö järjestellä uudestaan?
    % Voisi esimerkiksi laittaa kotitehtäviin?
    Fermat'n luvut $F_0, F_1, F_2, F_3, F_4$ ovat alkulukuja. Tämän perusteella Fermat oletti, että loputkin luvut $F_n$, $n=5, 6, \ldots$, ovat alkulukuja. Vuonna 1732 Euler kuitenkin havaitsi, että $5^4 + 2^4 = 641$
ja $1 + 5\cdot 2^7 = 641$ ja onnistui näiden yhtälöiden avulla todistamaan, että $641 | F_5$. Todista tämä tulos.

Vihje: Eulerin havaitsemista yhtälöistä seuraa, että
\[
2^4 \equiv -5^4\ (\mod\ 641)\quad\text{ ja }\quad5\cdot 2^7 \equiv -1\ (\mod\ 641).
\]
\end{tehtava}

\begin{tehtava}
    % Tarkistettu (Topi Talvitie, 10.11.2013)
    Osoita, että kaikilla $n\in\zz_{+}=1, 2, 3, \ldots$ pätee
\[
F_{0}F_{1}\cdot \ldots \cdot F_{n-1} = F_{n} - 2.
\]
Esitä tämän yhtälön perusteella uusi todistus sille,
että alkulukuja on äärettömän monta.
\end{tehtava}

\begin{tehtava}
    Päättele, kuinka monta nollaa on luvun $50!$ lopussa.

    \begin{vastaus}
        12
    \end{vastaus}
    
\end{tehtava}

\end{tehtavasivu}
\begin{kotitehtavasivu}

\begin{tehtava}
    Onko luku
    \begin{alakohdat}
        \alakohta{$43$}
        \alakohta{$39$}
        \alakohta{$79$}
        \alakohta{$221$}
    \end{alakohdat}
    alkuluku?
    
    \begin{vastaus}
        \begin{alakohdat}
            \alakohta{On.}
            \alakohta{Ei ole.}
            \alakohta{On.}
            \alakohta{Ei ole.}
        \end{alakohdat}
    \end{vastaus}
    
\end{tehtava}

\begin{tehtava}
    Pitääkö väite paikkansa? Perustele.
    
    \begin{alakohdat}
        \alakohta{Jos tulo $mn$ on jaollinen luvulla $171$, niin ainakin
toinen kokonaisluvuista $m$ ja $n$ on jaollinen luvulla $171$.}
        \alakohta{Jos tulo $mn$ on jaollinen luvulla $149$, niin ainakin
toinen kokonaisluvuista $m$ ja $n$ on jaollinen luvulla $149$.}
    \end{alakohdat}

    \begin{vastaus}
    
        \begin{alakohdat}
            \alakohta{Väite ei pidä paikkaansa. Vastaesimerkki: $m = 9$ ja $n = 19$.}
            \alakohta{Väite pitää paikkansa Eukleideen lemman nojalla, sillä $149$ on alkuluku.}
        \end{alakohdat}
    \end{vastaus}
    
\end{tehtava}

\begin{tehtava}
    Määritä luvun alkutekijähajotelma.

    \begin{alakohdat}
        \alakohta{$126$}
        \alakohta{$13200$}
        \alakohta{$1000000$}
    \end{alakohdat}

    \begin{vastaus}
        \begin{alakohdat}
            \alakohta{$2\cdot 3^2\cdot 7$}
            \alakohta{$2^4\cdot 3\cdot 5^2\cdot 11$}
            \alakohta{$2^6\cdot 5^6$}
        \end{alakohdat}
    \end{vastaus}
    
\end{tehtava}

\begin{tehtava}
    Millä luvuilla on luvun $267$ jaollisuus vähintään tutkittava, jotta saadaan selville, onko se alkuluku?
    
    \begin{vastaus}
        Riittää tarkistaa jaollisuus alkuluvuilla $\sqrt{267}$ asti, eli siis alkuluvuilla $2$, $3$, $5$, $7$, $11$, $13$.
    \end{vastaus}
    
\end{tehtava}

\begin{tehtava}
    Onko luku
    \begin{alakohdat}
        \alakohta{$229$}
        \alakohta{$251$}
        \alakohta{$707$}
        \alakohta{$2827$}
    \end{alakohdat}
    alkuluku?
    
    \begin{vastaus}
        \begin{alakohdat}
            \alakohta{On.}
            \alakohta{On.}
            \alakohta{Ei ole.}
            \alakohta{Ei ole.}
        \end{alakohdat}
    \end{vastaus}
    
\end{tehtava}

\begin{tehtava}
    Tutki laskimen {\tt factor}-toiminnolla, onko luku
    \begin{alakohdat}
        \alakohta{$72\,222\,222\,227$}
        \alakohta{$7\,222\,222\,227$}
        \alakohta{$722\,222\,227$}
    \end{alakohdat}
    alkuluku.

    \begin{vastaus}
        \begin{alakohdat}
            \alakohta{Luku ei ole alkuluku.}
            \alakohta{Luku ei ole alkuluku.}
            \alakohta{Luku on alkuluku.}
        \end{alakohdat}
    \end{vastaus}
    
\end{tehtava}

\begin{tehtava}
    Etsi Internetistä, mikä on tällä hetkellä suurin tunnettu alkuluku.
\end{tehtava}

\begin{tehtava}
    Jos luvut $n$ ja $n + 2$ ovat alkulukuja, niin kyseistä lukuparia kutsutaan alkulukukaksosiksi. Se, onko alkulukukaksosia äärettömän monta, on vielä ratkaisematon lukuteorian ongelma. Etsi kymmenen alkulukukaksosta. Käytä laskinta tai tietokonetta apunasi.
    
    \begin{vastaus}
        Esimerkiksi 10 pienintä $n$ siten että $n$ ja $n + 2$ ovat alkulukukaksoset ovat: $3$, $5$, $11$, $17$, $29$, $41$, $59$, $71$, $101$, $107$.
    \end{vastaus}
    
\end{tehtava}

\begin{tehtava}
    Mikä on lukujen suurin yhteinen tekijä?
    \begin{alakohdat}
        \alakohta{$2^2 \cdot 3 \cdot 5 \cdot 7 \cdot 11 \cdot 13$ ja
$2^{11} \cdot 3^8 \cdot 11 \cdot 17^{10}$}
        \alakohta{$5 \cdot 7^3 \cdot 19^2$ ja $3 \cdot 11^2 \cdot 41$}
    \end{alakohdat}

    \begin{vastaus}
        \begin{alakohdat}
            \alakohta{$132 = 2^2\cdot 3\cdot 11$}
            \alakohta{$1$}
        \end{alakohdat}
    \end{vastaus}
    
\end{tehtava}

\begin{tehtava}
    Mikä on lukujen pienin yhteinen jaettava?
    
    \begin{alakohdat}
        \alakohta{$7^3 \cdot 11^2$ ja $2^4 \cdot 7^2 \cdot 11$}
        \alakohta{$41 \cdot 43$ ja $41 \cdot 43$}
        \alakohta{$2 \cdot 7$ ja $5 \cdot 13$}
    \end{alakohdat}

    \begin{vastaus}
    
        \begin{alakohdat}
            \alakohta{$2^4\cdot 7^3\cdot 11^2$}
            \alakohta{$1763 = 41\cdot 43$}
            \alakohta{$910 = 2\cdot 5\cdot 7\cdot 13$}
        \end{alakohdat}
    \end{vastaus}
    
\end{tehtava}

\begin{tehtava}
    Määritä alkutekijähajotelman avulla lukujen
    \begin{alakohdat}
        \alakohta{$70$ ja $385$}
        \alakohta{$240$ ja $4900$}
        \alakohta{$2310$ ja $17199$}
    \end{alakohdat}
    suurin yhteinen tekijä ja pienin yhteinen jaettava.

    \begin{vastaus}
        Suurin yhteinen tekijä:
        \begin{alakohdatrivi}
            \alakohta{$35$}
            \alakohta{$20$}
            \alakohta{$21$.}
        \end{alakohdatrivi}

        Pienin yhteinen jaettava:
        \begin{alakohdat}
            \alakohta{$770 = 2\cdot 5\cdot 7\cdot 11$}
            \alakohta{$58800 = 2^4\cdot 3\cdot 5^2\cdot 7^2$}
            \alakohta{$2\cdot 3^3\cdot 5\cdot 7^2\cdot 11\cdot 13$.}
        \end{alakohdat}
    \end{vastaus}
    
\end{tehtava}

\begin{tehtava}
    Määritä $\syt(117325, 16625)$ ja $\pyj(117325, 16625)$.
    
    \begin{vastaus}
        \begin{align*}
        \syt(117325, 16625) &= 5^2\cdot 19 = 475 \\
        \pyj(117325, 16625) &= 5^3\cdot 7\cdot 13\cdot 19^2
        \end{align*}
    \end{vastaus}
    
\end{tehtava}

\begin{tehtava}
    Pulsarit ovat nopeasti pyöriviä ja rytmisiä radiopulsseja säteileviä tähtiä. Pulsari X lähettää radiopulssin $510$ millisekunnin välein, pulsari Y taas $357$ millisekunnin välein.

    \begin{alakohdat}
        \alakohta{Eräänä ajanhetkenä molempien pulsarien lähettämä pulssi
havaitaan yhtä aikaa. Kuinka pitkän ajan kuluttua molempien
pulsarien pulssit havaitaan yhtä aikaa seuraavan kerran?}
        \alakohta{Kuinka monta kierrosta pulsarit pyörähtävät kyseisenä
aikana?} % FIXME: ei voi ratkoa, poista kun yhteensopivuuden voi rikkoa
    \end{alakohdat}

    \begin{vastaus}
        \begin{alakohdat}
            \alakohta{$3570$ millisekunnin kuluttua.}
            \alakohta{Virhe kirjassa: tehtävää ei voi ratkaista annetuilla tiedoilla, mikäli ei tunneta pulsarien pyörimisen yhteyttä radiopulsseihin.}
        \end{alakohdat}
    \end{vastaus}
    
\end{tehtava}

\begin{tehtava}
    Perustele väite todeksi tai epätodeksi.
    \begin{alakohdat}
        \alakohta{Jos luku on jaollinen luvuilla $15$ ja $12$, niin se on
jaollinen myös luvulla $180$.}
        \alakohta{Jos luku on jaollinen luvuilla $36$ ja $5$, niin se on
jaollinen myös luvulla $180$}
    \end{alakohdat}

    \begin{vastaus}
        \begin{alakohdat}
            \alakohta{Väite ei päde. Vastaesimerkki: luku $60$ on jaollinen luvuilla $15$ ja $12$ mutta ei luvulla $180$.}
            \alakohta{Väite pätee sillä $\syt(36, 5) = 1$ ja $180 = 36\cdot 5$.}
        \end{alakohdat}
    \end{vastaus}
    
\end{tehtava}

\begin{tehtava}
    % Tarkistettu (Topi Talvitie, 10.11.2013)
    Osoita, että luku $5n^3 - 5n$ on jaollinen luvulla $30$, kun $n$ on luonnollinen luku.
\end{tehtava}

\begin{tehtava}
    % Tarkistettu (Topi Talvitie, 10.11.2013)
    Osoita, että muotoa $p^2 - 1$ oleva luku on jaollinen luvulla $12$, kun $p$ on alkuluku ja suurempi kuin $3$. [YO kevät 2010 tehtävä 12]
\end{tehtava}

\begin{tehtava}
    Erään maan olympiajoukkueessa oli vain yleisurheilijoita ja
    purjehtijoita. Yleisurheilijoita oli neljä kertaa niin paljon kuin
    purjehtijoita. Naisia oli kaksi kertaa niin paljon kuin miehiä.
    Kun joukkue matkusti 100-paikkaisella lentokoneella kisoihin,
    oli koneen matkustajista noin puolet joukkueen urheilijoita ja
    loput valmentajia, huoltajia ja median edustajia. Kuinka monta
    urheilijaa joukkueessa oli? [Tarkennus tehtävään: tässä "noin puolet" tarkoittaa että erotus 50:een on alle 7.] % FIXME: Korjaa tehtävä kun yhteensopivuus voidaan rikkoa.
    
    \begin{vastaus}
        45
    \end{vastaus}
    
\end{tehtava}

\begin{tehtava}
    % Tarkistettu (Topi Talvitie, 10.11.2013)
    Olkoon $n$ kokonaisluku. Osoita, että luku $n^5+10n^4+35n^3+50n^2+24n$ on jaollinen luvulla $120$. Vihje: Käytä symbolisen laskimen {\tt factor}-toimintoa.
\end{tehtava}

% ---

\begin{tehtava}
    Määritä jakojäännös, kun luku $51\cdot 31^{94}+102$ jaetaan
    luvulla $47$. Voit käyttää Fermat'n pientä lausetta apunasi.
\end{tehtava}

\begin{tehtava}
    Osoita, että $2^{341} = 2^{11 \cdot 31} \equiv 2 (\mod 341)
    $. Tulos osoittaa, että kiinalaiseen alkulukutestiin liittyvä
    otaksuma on epätosi eikä testi siten toimi kaikilla luvuilla.
\end{tehtava}

\begin{tehtava}
    Alkulukujen lukumääräfunktio $\pi(x)$ kertoo välillä $[0, x]$
    olevien alkulukujen lukumäärän. Alkulukujen tiheys välillä $[0, x]
    $ saadaan jakamalla $\pi(x)$ välin pituudella $x$.
    \begin{enumerate}[a)]
    \item Laske $\pi(10^n)$, kun $n=1, 2, 3, 4, 5$.
    \item Mitä alkulukujen määrälle $\pi(x)$ näyttää tapahtuvan,
    kun $x$ kasvaa? Miksi?
    \item Laske alkulukujen tiheys väleillä $[0, 10^n]$, kun
    $n=1, 2, 3, 4, 5$.
    \item Mitä alkulukujen tiheydelle
    \[
    \frac{\pi(x)}{x}
    \]

    näyttää tapahtuvan, kun $x$ kasvaa?
    \end{enumerate}
    Käytä tehtävässä laskinta tai matemaattisia ohjelmistoja, kuten
    verkosta ilmaiseksi ladattavissa olevaa Maxima-ohjelmistoa.
    Esimerkiksi Texas Instrumentsin TI-Nspire CX CAS -laskin sisältää
    toiminnon \\{\tt numtheory$\backslash$primecount(a,b)}, joka antaa
    alkulukujen määrän välillä $[a, b]$.
\end{tehtava}

\begin{tehtava}
    \termi{Mersennen alkuluvut}{Mersennen alkulukuja} ovat alkuluvut, jotka ovat
    muotoa $2^p - 1$, missä $p$ on alkuluku.
    \begin{enumerate}[a)]
    \item Etsi neljä pienintä Mersennen alkulukua.
    \item Ovatko kaikki muotoa $2^p - 1$ olevat luvut alkulukuja?
    \item Osoita, että jos $2^p - 1$ on alkuluku, myös $p$ on
    alkuluku.

    Vihje: Käytä epäsuoraa todistusta ja sovella kaavaa
    \[
    x^n-1 = (x-1)(x^{n-1}+\ldots+x+1).
    \]
    \end{enumerate}
\end{tehtava}

\begin{tehtava}
    \begin{enumerate}[a)]
    \item Selvitä kokeilemalla pienillä luvuilla, mitä on
    $\syt(a, b)\cdot \pyj(a, b)$.
    \item Todista saamasi tulos.
    \item Miten Eukleideen algoritmia ja saamaasi tulosta voidaan
    hyödyntää pienimmän yhteisen jaettavan määrittämisessä?
    \item Määritä Eukleideen algoritmin ja saamasi tuloksen
    avulla\\ $\pyj(496125, 9450)$.
    \item Texas Instrumentsin TI-Nspire CX CAS -laskimesta löytyy
    ohjelma \\{\tt numtheory$\backslash$gcdstep(a,b)}, joka määrittää
    lukujen $a$ ja $b$ suurimman yhteisen tekijän Eukleideen
    algoritmilla välivaiheet näyttäen. Tutustu ohjelman toimintaan ja
    määritä sen avulla $\pyj(7856, 678)$. Kirjoita ylös Eukleideen
    algoritmin välivaiheet käyttäen tämän oppikirjan mukaisia
    merkintöjä.
    \end{enumerate}
\end{tehtava}

\begin{tehtava}
    * Salakirjoita RSA-algoritmia käyttäen
    viesti $66$, joka vastaa ASCII-kooda-\\usjärjestelmässä kirjainta
    B. Käytä esimerkin 7 avainta ja esim. \href{http://
    www.wolframalpha.com}{Wolfram Alphaa}. Tarkasta tuloksesi
    purkamalla viesti.
    Vihje: Voit antaa syötteen Wolfram Alphalle muodossa {\tt
    Mod[a\^{}p,n]}, esim. {\tt Mod[660\^{}821,2773]}.
\end{tehtava}


\begin{tehtava}
    * Osoita, että Eulerin $\varphi$-funktio on multiplikatiivinen
    eli $\varphi(mn) = \varphi(m) \varphi(n)$, kun $\syt(m, n)=1$.
    \begin{enumerate}[a)]
    \item Käyttäen Eulerin tulokaavaa
    \[
    \varphi(n)=n \prod_{p|n} \bigg(1-\frac{1}{p}\bigg),
    \]
    missä $p$ on alkuluku. 
    \item
    Suoraan $\varphi$-funktion määritelmästä.
    %Vihje: Tutki Eulerin tulokaavaa. Mitä tapahtuu, kun $\syt(m, n) =1$?
    \end{enumerate}
\end{tehtava}

\begin{tehtava}
    *  Osoita seuraavat $\varphi$-funktion ominaisuudet:
    \begin{enumerate}[a)]
    \item
    Jos $n$ on pariton, niin $\varphi(2n)=\varphi(n)$.
    \item
    Jos $n$ on parillinen, niin $\varphi(2n)=2\varphi(n)$
    \end{enumerate}
\end{tehtava}

\begin{tehtava}
    * Laske $\varphi(1001)$, $\varphi(5040)$ ja $\varphi(36\,000)$.
\end{tehtava}

\begin{tehtava}
    * Osoita, että $\varphi(n)$ on parillinen, kun $n>2$. Vihje: Oleta ensin, että $n$ on muotoa $n=2^k$, $k\ge 2$.
\end{tehtava}

\begin{tehtava}
    * Osoita edellisen tehtävän tulosta käyttäen, että alkulukuja on äärettömän monta.
\end{tehtava}

\begin{tehtava}
    * Osoita Eulerin tulokaava käyttäen hyväksi  $\varphi$-funktion multiplikatiivisuutta.
\end{tehtava}

\end{kotitehtavasivu}
