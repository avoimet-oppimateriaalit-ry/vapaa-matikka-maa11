%Luvun 2.3 harjoitustehtävien vastauksia, vastauksien tekijä Valtteri Vistiaho 10.11.2013
\begin{tehtava}
     Tutki, onko lause tautologia.
    \begin{alakohdat}
        \alakohta{$A\lor \lnot A$,}
        \alakohta{$A \land \lnot A$,}
        \alakohta{$A \to B \lor A$,}
        \alakohta{$A \to B \land A$,}
    \end{alakohdat}

    \begin{vastaus}
    
        \begin{alakohdat}
            \alakohta{\begin{center}
		    \begin{tabular}{|c|c|c|c|c|}\hline
		    $A$ & $\lnot A$ & $A\lor \lnot A$ \\ \hline
		    $1$ & $0$ & $1$ \\ %\hline
		    $0$ & $1$ & $1$ \\ \hline
\end{tabular}
\end{center}
Lause on tautologia.}

            \alakohta{\begin{center}
		    \begin{tabular}{|c|c|c|c|c|}\hline
		    $A$ & $\lnot A$ & $A\land \lnot A$ \\ \hline
		    $1$ & $0$ & $0$ \\ %\hline
		    $0$ & $1$ & $0$ \\ \hline
\end{tabular}
\end{center}
Lause ei ole tautologia.}

            \alakohta{\begin{center}
		    \begin{tabular}{|c|c|c|c|c|c|c|}\hline
		    $A$ & $B$ & $B\lor A$ & $A\to B\lor A$\\ \hline
		    $1$ & $1$ & $1$ & $1$ \\ %\hline
		    $1$ & $0$ & $1$ & $1$ \\
		    $0$ & $1$ & $1$ & $1$ \\
		    $0$ & $0$ & $0$ & $1$ \\ \hline
\end{tabular}
\end{center}
Lause on tautologia.}
            \alakohta{\begin{center}
		    \begin{tabular}{|c|c|c|c|c|c|c|}\hline
		    $A$ & $B$ & $B\land A$ & $A\to B\land A$\\ \hline
		    $1$ & $1$ & $1$ & $1$ \\ %\hline
		    $1$ & $0$ & $0$ & $0$ \\
		    $0$ & $1$ & $0$ & $1$ \\
		    $0$ & $0$ & $0$ & $1$ \\ \hline
\end{tabular}
\end{center}
Lause ei ole tautologia.}
        \end{alakohdat}
    \end{vastaus}
    
\end{tehtava}

\begin{tehtava}
     Osoita lause tautologiaksi.
    \begin{alakohdat}
        \alakohta{$A\to B\land \lnot B \lequiv \lnot A$.}
        \alakohta{$(A\to B) \land (B\to C)\to (A\to C)$.}
    \end{alakohdat}

    \begin{vastaus}
    
        \begin{alakohdat}
            \alakohta{\begin{center}
		    \begin{tabular}{|c|c|c|c|c|c|c|}\hline
		    $A$ & $B$ & $\lnot A$ & $\lnot B$ & $B\land \lnot B$ & $A\to B\land \lnot B$ & $A\to B\land \lnot B \lequiv \lnot A$ \\ \hline
		    $1$ & $1$ & $0$ & $0$ & $0$ & $0$ & $1$ \\ %\hline
		    $1$ & $0$ & $0$ & $1$ & $0$ & $0$ & $1$ \\
		    $0$ & $1$ & $1$ & $0$ & $0$ & $1$ & $1$ \\
		    $0$ & $0$ & $1$ & $1$ & $0$ & $1$ & $1$ \\ \hline
\end{tabular}
\end{center}
Koska viimeisen sarakkeen kaikki totuusarvot ovat 1, lause on tautologia.}
            \alakohta{\begin{center}
		    \begin{tabular}{|c|c|c|c|c|c|c|c|}\hline
		    $A$ & $B$ & $C$ & $A\to B$ & $B\to C$ & $A\to C$ & $(A\to B)\land(B\to C)$ & $(A\to B)\land(B\to C)\to(A\to C)$ \\ \hline
		    $1$ & $1$ & $1$ & $1$ & $1$ & $1$ & $1$ & $1$ \\ %\hline
		    $1$ & $1$ & $0$ & $1$ & $0$ & $0$ & $0$ & $1$ \\
		    $1$ & $0$ & $1$ & $0$ & $1$ & $1$ & $0$ & $1$ \\
		    $1$ & $0$ & $0$ & $0$ & $1$ & $0$ & $0$ & $1$ \\
		    $0$ & $1$ & $1$ & $1$ & $1$ & $1$ & $1$ & $1$ \\
		    $0$ & $0$ & $1$ & $1$ & $1$ & $1$ & $1$ & $1$ \\
		    $0$ & $1$ & $0$ & $1$ & $0$ & $1$ & $0$ & $1$ \\
		    $0$ & $0$ & $0$ & $1$ & $1$ & $1$ & $1$ & $1$ \\ \hline
\end{tabular}
\end{center}
Koska viimeisen sarakkeen kaikki totuusarvot ovat 1, lause on tautologia.}
        \end{alakohdat}
    \end{vastaus}
    
\end{tehtava}

\begin{tehtava}
     Olkoot $A$: ''kello soi'' ja $B$: ''tunti loppuu''.
Kirjoita luonnollisella kielellä lauseet $A\to B$ ja
$\lnot(A \land \lnot B )$. Tarkoittavatko ne samaa?
    
    \begin{vastaus}
    
       Jos kello soi, tunti loppuu. \newline
       Ei ole totta, että kello soi ja tunti ei lopu. \newline
       Totuusarvoiltaan lauseet tarkoittavat samaa.
    \end{vastaus}
    
\end{tehtava}

\begin{tehtava}
     Osoita lauseet loogisesti ekvivalenteiksi
totuustaulujen avulla.
    \begin{alakohdat}
        \alakohta{$A\land B$ ja $\lnot(\lnot A\lor \lnot B )$.}
        \alakohta{$A\lequiv B$ ja $(A\to B)\land (B \to A)$.}
        \alakohta{$A\lequiv B$ ja $(A\land B) \lor (\lnot A\land
\lnot B )$.}
    \end{alakohdat}

    \begin{vastaus}
    
        \begin{alakohdat}
            \alakohta{\begin{center}
		    \begin{tabular}{|c|c|c|c|c|c|c|c|}\hline
		    $A$ & $B$ & $\lnot A$ & $\lnot B$ & $A\land B$ & $\lnot A\lor \lnot B$ & $\lnot(\lnot A\lor \lnot B)$ & $A\land B\lequiv \lnot(\lnot A\lor \lnot B)$ \\ \hline
		    $1$ & $1$ & $0$ & $0$ & $1$ & $0$ & $1$ & $1$ \\ %\hline
		    $1$ & $0$ & $0$ & $1$ & $0$ & $1$ & $0$ & $1$ \\
		    $0$ & $1$ & $1$ & $0$ & $0$ & $1$ & $0$ & $1$ \\
		    $0$ & $0$ & $1$ & $1$ & $0$ & $1$ & $0$ & $1$ \\ \hline
\end{tabular}
\end{center}}
            \alakohta{\begin{center}
		    \begin{tabular}{|c|c|c|c|c|c|c|}\hline
		    $A$ & $B$ & $A\lequiv B$ & $A\to B$ & $B\to A$ & $(A\to B)\land (B\to A)$ & $(A\lequiv B)\lequiv((A\to B)\land (B\to A))$ \\ \hline
		    $1$ & $1$ & $1$ & $1$ & $1$ & $1$ & $1$ \\ %\hline
		    $1$ & $0$ & $0$ & $0$ & $1$ & $0$ & $1$ \\
		    $0$ & $1$ & $0$ & $1$ & $0$ & $0$ & $1$ \\
		    $0$ & $0$ & $1$ & $1$ & $1$ & $1$ & $1$ \\ \hline
\end{tabular}
\end{center}}
            \alakohta{\begin{center}
		    \begin{tabular}{|c|c|c|c|c|c|c|c|c|}\hline
		    $A$ & $B$ & $\lnot A$ & $\lnot B$ & $A\lequiv B$ & $A\land B$ & $\lnot A\land \lnot B$ & $(A\land B)\lor(\lnot A\land \lnot B)$ & $(A\lequiv B)\lequiv((A\land B)\lor(\lnot A\land \lnot B))$ \\ \hline
		    $1$ & $1$ & $0$ & $0$ & $1$ & $1$ & $0$ & $1$ & $1$ \\ %\hline
		    $1$ & $0$ & $0$ & $1$ & $0$ & $0$ & $0$ & $0$ & $1$ \\
		    $0$ & $1$ & $1$ & $0$ & $0$ & $0$ & $0$ & $0$ & $1$ \\
		    $0$ & $0$ & $1$ & $1$ & $1$ & $0$ & $1$ & $1$ & $1$ \\ \hline
\end{tabular}
\end{center}}
        \end{alakohdat}
    \end{vastaus}
    
\end{tehtava}

\begin{tehtava}
     Osoita vaihdantalait
    \begin{alakohdat}
        \alakohta{\[
A \land B \lequiv B \land A,
\]}
        \alakohta{\[
A \lor B \lequiv B \lor A
\] oikeaksi totuustaulujen avulla.}

    \end{alakohdat}

    \begin{vastaus}
    
        \begin{alakohdat}
            \alakohta{\begin{center}
		    \begin{tabular}{|c|c|c|c|c|}\hline
		    $A$ & $B$ & $A\land B$ & $B\land A$ & $(A\land B)\lequiv(B\land A)$\\ \hline
		    $1$ & $1$ & $1$ & $1$ & $1$ \\ %\hline
		    $1$ & $0$ & $0$ & $0$ & $1$ \\
		    $0$ & $1$ & $0$ & $0$ & $1$ \\
		    $0$ & $0$ & $0$ & $0$ & $1$ \\ \hline
\end{tabular}
\end{center}}
            \alakohta{\begin{center}
		    \begin{tabular}{|c|c|c|c|c|}\hline
		    $A$ & $B$ & $A\lor B$ & $B\lor A$ & $(A\land B)\lequiv(B\land A)$\\ \hline
		    $1$ & $1$ & $1$ & $1$ & $1$ \\ %\hline
		    $1$ & $0$ & $1$ & $1$ & $1$ \\
		    $0$ & $1$ & $1$ & $1$ & $1$ \\
		    $0$ & $0$ & $0$ & $0$ & $1$ \\ \hline
\end{tabular}
\end{center}}
        \end{alakohdat}
    \end{vastaus}
    
\end{tehtava}

\begin{tehtava}
     Muodosta atomilauseista $A$ ja $B$ lause, joka on
loogisesti ekvivalentti lauseen $X$ kanssa.

\begin{center}
\begin{tabular}{|c|c|c|}\hline
$A$ & $B$ & $X$\\ \hline
$1$ & $1$ & $0$\\
$1$ & $0$ & $1$\\
$0$ & $1$ & $1$\\
$0$ & $0$ & $1$\\ \hline
\end{tabular}
\end{center}

    \begin{vastaus}
    \begin{center}
\begin{tabular}{|c|c|c|c|}\hline
$A$ & $B$ & $X$ & $\lnot(A\land B)$\\ \hline
$1$ & $1$ & $0$ & $0$\\
$1$ & $0$ & $1$ & $1$\\
$0$ & $1$ & $1$ & $1$\\
$0$ & $0$ & $1$ & $1$\\ \hline
\end{tabular}
\end{center}

    \end{vastaus}
    
\end{tehtava}

\begin{tehtava}
     Tutki, onko lause loogisesti ristiriitainen.
    \begin{alakohdat}
        \alakohta{$\lnot (A \lor B) \land \lnot (A\land B)$,}
        \alakohta{$(\lnot A \land B) \land (A \lequiv B)$,}
        \alakohta{$(\lnot (A \to B) \land C) \land B$.}
    \end{alakohdat}

    \begin{vastaus}
    
        \begin{alakohdat}
            \alakohta{\begin{center}
		    \begin{tabular}{|c|c|c|c|c|c|c|}\hline
		    $A$ & $B$ & $A\lor B$ & $A\land B$ & $\lnot(A\lor B)$ & $\lnot(A\land B)$ & $\lnot(A\lor B)\land \lnot(A\land B)$\\ \hline
		    $1$ & $1$ & $1$ & $1$ & $0$ & $0$ & $0$ \\ %\hline
		    $1$ & $0$ & $1$ & $0$ & $0$ & $1$ & $0$ \\
		    $0$ & $1$ & $1$ & $0$ & $0$ & $1$ & $0$ \\
		    $0$ & $0$ & $0$ & $0$ & $1$ & $1$ & $1$ \\ \hline
\end{tabular}
\end{center}
Lause ei ole loogisesti ristiriitainen.}
            \alakohta{\begin{center}
		    \begin{tabular}{|c|c|c|c|c|c|}\hline
		    $A$ & $B$ & $\lnot A$ & $\lnot A\land B$ & $A\lequiv B$ & $(\lnot A\land B)\land(A\lequiv B)$\\ \hline
		    $1$ & $1$ & $0$ & $0$ & $1$ & $0$ \\ %\hline
		    $1$ & $0$ & $0$ & $0$ & $0$ & $0$ \\
		    $0$ & $1$ & $1$ & $1$ & $0$ & $0$ \\
		    $0$ & $0$ & $1$ & $0$ & $1$ & $0$ \\ \hline
\end{tabular}
\end{center}
Lause on loogisesti ristiriitainen.}
            \alakohta{\begin{center}
		    \begin{tabular}{|c|c|c|c|c|c|c|}\hline
		    $A$ & $B$ & $C$ & $A\to B$ & $\lnot(A\to B)$ & $\lnot(A\to B)\land C$ & $(\lnot(A\to B)\land C)\land B$\\ \hline
		    $1$ & $1$ & $1$ & $1$ & $0$ & $0$ & $0$ \\ %\hline
		    $1$ & $1$ & $0$ & $1$ & $0$ & $0$ & $0$ \\
		    $1$ & $0$ & $1$ & $0$ & $1$ & $1$ & $0$ \\
		    $1$ & $0$ & $0$ & $0$ & $1$ & $0$ & $0$ \\
		    $0$ & $1$ & $1$ & $1$ & $0$ & $0$ & $0$ \\
		    $0$ & $0$ & $1$ & $1$ & $0$ & $0$ & $0$ \\
		    $0$ & $1$ & $0$ & $1$ & $0$ & $0$ & $0$ \\
		    $0$ & $0$ & $0$ & $1$ & $0$ & $0$ & $0$ \\ \hline
\end{tabular}
\end{center}
Lause on loogisesti ristiriitainen.}
        \end{alakohdat}
    \end{vastaus}
    
\end{tehtava}

\begin{tehtava}
     Tulkitse sanallisesti
    \begin{alakohdat}
        \alakohta{De Morganin 2. laki
\[
\lnot (A\lor B) \lequiv \lnot A\land \lnot B
\]}
        \alakohta{modus ponens -päättelysääntö
\[
(A \land (A\to B))\to B
\]}
        \alakohta{reductio ad absurdum -päättelysääntö.
\[
(\lnot A \to (B\land \lnot B))\to A
\]}
    \end{alakohdat}

    \begin{vastaus}
    
        \begin{alakohdat}
            \alakohta{Lauseet ''Ei ole totta, että A tai B'' ja ''Ei A eikä B'' ovat loogisesti ekvivalentit.}
            \alakohta{?????????????????????????}
            \alakohta{?????????????????????????}%En keksinyt järkevää tapaa sanoa näitä.
        \end{alakohdat}
    \end{vastaus}
    
\end{tehtava}

\begin{tehtava}
     Osoita
    \begin{alakohdat}
        \alakohta{kaksoisnegaation laki,}
        \alakohta{De Morganin 1. laki,}
        \alakohta{modus ponens -päättelysääntö}
    \end{alakohdat}
        oikeaksi totuustaulujen avulla.

    \begin{vastaus}
    
        \begin{alakohdat}
            \alakohta{\begin{center}
		    \begin{tabular}{|c|c|c|c|}\hline
		    $A$ & $\lnot A$ & $\lnot \lnot A$ & $\lnot \lnot A\lequiv A$\\ \hline
		    $1$ & $0$ & $1$ & $1$ \\ %\hline
		    $0$ & $1$ & $0$ & $1$ \\ \hline
\end{tabular}
\end{center}}
            \alakohta{\begin{center}
		    \begin{tabular}{|c|c|c|c|c|c|c|c|}\hline
		    $A$ & $B$ & $\lnot A$ & $\lnot B$ & $A\land B$ & $\lnot(A\land B)$ & $\lnot A\lor \lnot B$ & $\lnot(A\land B)\lequiv \lnot A\lor \lnot B$\\ \hline
		    $1$ & $1$ & $0$ & $0$ & $1$ & $0$ & $0$ & $1$ \\ %\hline
		    $1$ & $0$ & $0$ & $1$ & $0$ & $1$ & $1$ & $1$ \\
		    $0$ & $1$ & $1$ & $0$ & $0$ & $1$ & $1$ & $1$ \\
		    $0$ & $0$ & $1$ & $1$ & $0$ & $1$ & $1$ & $1$ \\ \hline
\end{tabular}
\end{center}}
            \alakohta{\begin{center}
		    \begin{tabular}{|c|c|c|c|c|}\hline
		    $A$ & $B$ & $A\to B$ & $A\land(A\to B)$ & $A\land(A\to B)\to B$\\ \hline
		    $1$ & $1$ & $1$ & $1$ & $1$ \\ %\hline
		    $1$ & $0$ & $0$ & $0$ & $1$ \\
		    $0$ & $1$ & $1$ & $0$ & $1$ \\
		    $0$ & $0$ & $1$ & $0$ & $1$ \\ \hline
\end{tabular}
\end{center}}
        \end{alakohdat}
    \end{vastaus}
    
\end{tehtava}
%Loppuu Valtteri Vistiahon tekemät vastaukset 10.11.2013

\end{tehtavasivu}

% LOPUT:

%\item Osoita kontraposition laki oikeaksi ilman
%totuustauluja.
%Vihje: Voit korvata implikaation $A\to B$ loogisesti
%ekvivalentilla lauseella $\lnot A \lor B$.
%
%\item Esitä lauseelle ''jos Jaakko saa logiikan kurssista
%arvosanan 10, hän tarjoaa ystävilleen kahvit'' kolme
%loogisesti ekvivalenttia lausetta.
%
%\item Osoita lauseet loogisesti ekvivalenteiksi
%päättelysääntöjen avulla ilman totuustauluja.
%\begin{enumerate}[a)]
%\item $A\land B$ ja $\lnot(\lnot A \lor \lnot B)$,
%\item $C\to (\lnot A \land B)$ ja $(A\lor \lnot B)\to
%\lnot C$.
%\item $A \land (A\to \lnot B)\to \lnot B$ ja $B\to
%\lnot A\lor \lnot (B\to \lnot A)$.
%\end{enumerate}
%
%\item * Kolmiarvologiikassa on kolme totuusarvoa: $1$
%tosi, $0$ epätosi ja $u$ epävarma. Kleenen totuustaulut
%perustuvat ajatukseen, että epävarma voi myöhemmin
%osoittautua todeksi tai epätodeksi. Alla on esitetty
%negaation ja konjunktion totuustaulut. Täydennä
%oheinen disjunktion totuustaulu. Laadi implikaation ja
%ekvivalenssin totuustaulut.
%
%\begin{center}
%\begin{tabular}{|c|c|c|}\hline
%$A$ & $\lnot A$ \\ \hline
%$1$ & $0$ \\
%$0$ & $1$ \\
%$u$ & $u$ \\ \hline
%\end{tabular}
%%\end{center}
%\qquad
%%\begin{center}
%\begin{tabular}{|c|c|c|}\hline
%$A$ & $B$ & $A\land B$\\ \hline
%$1$ & $1$ & $1$\\
%$1$ & $0$ & $0$\\
%$0$ & $1$ & $0$\\
%$0$ & $0$ & $0$\\
%$1$ & $u$ & $u$\\
%$u$ & $1$ & $u$\\
%$0$ & $u$ & $0$\\
%$u$ & $0$ & $0$\\
%$u$ & $u$ & $u$\\ \hline
%\end{tabular}
%%\end{center}
%\qquad
%%\begin{center}
%\begin{tabular}{|c|c|c|}\hline
%$A$ & $B$ & $A\lor B$\\ \hline
%$1$ & $1$ & \\
%$1$ & $0$ & \\
%$0$ & $1$ & \\
%$0$ & $0$ & \\
%$1$ & $u$ & \\
%$u$ & $1$ & \\
%$0$ & $u$ & \\
%$u$ & $0$ & \\
%$u$ & $u$ & \\ \hline
%\end{tabular}
%\end{center}
%
%\end{enumerate}
%
%\subsection*{Kotitehtäviä}
%
%\begin{enumerate}
%
%\item Osoita lause tautologiaksi.
%\begin{enumerate}[a)]
%\item $A\to A$.
%\item $A\land B \to A$.
%\item $(A\lequiv B) \to (B\to A)$.
%
%\end{enumerate}
%
%\item Tutki, onko lause tautologia.
%\begin{enumerate}[a)]
%\item $(A\land B \to \lnot C)\lequiv (A\to(B\to C))$.
%\item $(\lnot A \lequiv (B \land C)) \lequiv \lnot (A
%\lequiv B \land C)$.
%\end{enumerate}
%
%\item Olkoot $A$: ''tunti jatkuu'' ja $B$: ''kello soi''.
%Kirjoita luonnollisella kielellä lauseet $A\to \lnot B$
%ja $B\to \lnot A$. Osoita totuustaulujen avulla, että
%lauseet ovat loogisesti ekvivalentit.
%
%\item Osoita, että lauseet ovat loogisesti ekvivalentit.
%\begin{enumerate}[a)]
%\item ''Tero soittaa kitaraa, mutta Suvi ei
%laskettele'' ja ''ei ole niin, että jos Tero soittaa
%kitaraa, niin Suvi laskettelee''.
%\item ''Jos Tero soittaa kitaraa tai Suvi laskettelee,
%niin Anni kirjoittaa runoja'' ja ''jos Tero soittaa
%kitaraa, niin Anni kirjoittaa runoja, ja jos Suvi
%laskettelee, niin Anni kirjoittaa runoja''.
%\end{enumerate}
%
%\item Sievennä lause eli muodosta mahdollisimman
%yksinkertainen lause, joka on loogisesti ekvivalentti
%alkuperäisen lauseen kanssa.
%\begin{enumerate}[a)]
%\item $(A\land B) \lor A$.
%\item $(A\lor B) \land A$.
%\item $(A\lor B) \land (A\lor \lnot B)$
%\end{enumerate}
%
%\item Osoita osittelulait
%\begin{enumerate}[a)]
%\item
%\[
%A \land (B \lor C) \lequiv (A \land B) \lor (A\land C),
%\]
%\item
%\[
%A \lor (B \land C) \lequiv (A \lor B) \land (A\lor C)
%\]
%\end{enumerate}
%oikeaksi totuustaulujen avulla.
%
%\item Muodosta atomilauseista $A$ ja $B$ lause, joka on
%loogisesti ekvivalentti lauseen $Y$ kanssa.
%\begin{center}
%\begin{tabular}{|c|c|c|}\hline
%$A$ & $B$ & $Y$\\ \hline
%$1$ & $1$ & $1$\\
%$1$ & $0$ & $1$\\
%$0$ & $1$ & $1$\\
%
%$0$ & $0$ & $1$\\ \hline
%\end{tabular}
%\end{center}
%
%\item Tarkastele seuraavaa ennustetta: maapallon öljyvarat ehtyvät, jos ja vain jos länsimaiset demokratiat romahtavat, %mutta ei pidä paikkaansa, että jos maapallon
%öljyvarat ehtyvät, niin länsimaiset demokratiat romahtavat. Miten %ennusteeseen pitäisi suhtautua?
%
%\item Mainitse esimerkki tilanteesta, jossa olet käyttänyt
%\begin{enumerate}[a)]
%\item modus ponens -päättelysääntöä
%\item modus tollens -päättelysääntöä.
%\end{enumerate}
%
%\item Osoita
%\begin{enumerate}[a)]
%\item De Morganin 2. laki,
%\item modus tollens -päättelysääntö,
%\item reductio ad absurdum -päättelysääntö
%\end{enumerate}
%oikeaksi totuustaulujen avulla.
%
%\item Osoita lauseet loogisesti ekvivalenteiksi
%päättelysääntöjen avulla ilman totuustauluja.
%\begin{enumerate}[a)]
%\item $\lnot \lnot \lnot \lnot \lnot A$ ja $\lnot A$,
%\item $A \to (B \to C)$ ja $\lnot (\lnot C \to \lnot
%B) \to \lnot A$.
%\item $\lnot (A \land B \land \lnot C)$ ja $\lnot A
%\lor \lnot B \lor C$.
%\end{enumerate}
%
%\item Shefferin viivaan ja Peircen nuoleen on tutustuttu
%edellisen kappaleen kotitehtävässä 12. Esitä negaatio,
%konjunktio ja disjunktio a) Shefferin viivan avulla b)
%Peircen nuolen avulla.
%
%\item *
%Sumea logiikka on kaksiarvoisen logiikan
%laajennus, jossa lauseella on diskreetin totuusarvon
%(tosi tai epätosi) sijasta reaalinen totuusarvo, joka
%kuuluu välille $[0, 1]$. Konnektiivit voidaan määritellä
%esimerkiksi seuraavasti:
%\[
%\begin{array}{rcl}
%\lnot A &=& 1-A,\\
%A\land B &=& \min(A, B),\\
%A\lor B &=& \max(A, B),\\
%A\to B
%&=& \min(1, 1-A+B),
%\end{array}
%\]
%missä $\min$ tarkoittaa luvuista pienempää ja $\max$
%suurempaa.
%
%\begin{enumerate}[a)]
%\item Olkoot lauseen $A$ totuusarvo $0,3$ ja lauseen
%$B$ totuusarvo $0,5$. Laske lauseiden $\lnot A$, $A\land
%B$, $A\lor B$ ja $A \to B$ totuusarvot.
%\item Osoita, että jos lauseet $A$ ja $B$ saavat
%vain arvoja $0$ ja $1$, niin edellä mainitut määritelmät johtavat
%klassisen kaksiarvoisen logiikan totuustauluihin.
%\item Osoita, että sumeassa logiikassa $\lnot(A\land B)
%\lequiv \lnot A \lor \lnot B$.
%\item Etsi Internetistä sumean logiikan
%käyttökohteita.
%\end{enumerate}
%
%\item *
%Tutustu Wolfram Alphan
%logiikkatoimintoihin\\
%\href{http://www.wolframalpha.com/examples/BooleanAlgebra.html}{{\%tt http://www.wolframalpha.com/examples/BooleanAlgebra.html}}\\
%ja yritä laskea sen avulla joitakin kirjan tehtäviä.
%
%%Ratkaisu: Lauseet $A$: '' Jaska on syyllinen'',
%%$B$ ''Jaskalla on rikostoveri'', $\lnot (A\to B)=\lnot
%%(\lnot A \lor B)=A \land \lnot B$.
