\begin{tehtavasivu}

\begin{enumerate}

\item Onko luku a) $50$ b) $48$ c) $-72$ d) $-34$ jaollinen luvulla $6$?

\item Jakaako luku $13$ luvun a) $117$ b) $-65$ c) $160$ d) $-81$?

\item Osoita, että a) $3|66$ b) $7\nmid 120$ c) $15|(-330)$ d) $11\nmid (-619)$.

\item Onko väite tosi?
\begin{enumerate}[a)]
\item $3|53$
\item $-9|108$
\item $12 \nmid (-158)$
\item $-7|175$
\item $-17 \nmid (-646)$
\end{enumerate}

\item Kirjoita jakoyhtälö, kun
\begin{enumerate}[a)]
\item luku $7$ jaetaan luvulla $3$
\item luku $51$ jaetaan luvulla $4$
\item luku $1000$ jaetaan luvulla $125$
\item luku $3858$ jaetaan luvulla $97$.
\end{enumerate}

\item Kirjoita jakoyhtälö, kun
\begin{enumerate}[a)]
\item luku $-22$ jaetaan luvulla $7$
\item luku $-2844$ jaetaan luvulla $36$
\item luku $-3858$ jaetaan luvulla $97$.
\end{enumerate}

\item Kirjoita jakoyhtälö jakolaskulle
\begin{enumerate}[a)]
\item $9/25$
\item $-9/25$
\item $124/120$
\item $-124/120$.
\end{enumerate}

\item Päättele, mikä on jakojäännös, kun
\begin{enumerate}[a)]
\item luku $1967$ jaetaan luvulla $5$
\item luku $-426$ jaetaan luvulla $5$
\item luku $67876$ jaetaan luvulla $50$
\item luku $-30509$ jaetaan luvulla $50$.
\end{enumerate}

\item Leipomossa on pakattavana $260$ sämpylää. Kuinka monta täyttä pussia saadaan ja kuinka monta sämpylää jää yli, jos käytetään vain a) $24$ b) $10$ c) $6$ d) $4$ sämpylän pusseja?

\item 
Jos kello on nyt 13.05, niin mitä kello oli 2012 tuntia ja 45 minuuttia sitten?

\item 
\begin{enumerate}[a)]
\item Mikä luku jaettuna luvulla $17$ antaa osamääräksi $98$ ja jakojäännökseksi $5$?
\item Mikä luku jaettuna luvulla $12$ antaa osamääräksi $-91$ ja jakojäännökseksi $0$?
\item Millä positiivisella kokonaisluvulla luku $146$ on jaettava, jotta osamäärä olisi $3$ ja jakojäännös $2$?
\item Millä positiivisella kokonaisluvulla luku $72$ on jaettava, jotta jakojäännös olisi 7?
\end{enumerate}

\item Määritä osamäärä ja jakojäännös sekä kirjoita jakoyhtälö, kun a) luku $2^{18} + 10$ jaetaan luvulla $2^{15} + 1$ b) luku $3^{100} + 100$ jaetaan luvulla $3^{98} + 10$.

\item Olkoot $a$, $b$, $c$, $r$ ja $s$ kokonaislukuja ja $a \neq 0$. Osoita, että jos $a|b$ ja $a|c$, niin $a|(rb + sc)$.

\item Olkoot $a$, $b$, $c$ ja $d$ kokonaislukuja ja $a, b \neq 0$. Osoita, että jos $a|c$ ja $b|d$, niin $(ab)|(cd)$.

\item Osoita, että jos $a$ ja $b$ ovat parittomia kokonaislukuja, niin $4 | (a^2 - b^2)$.

\item Olkoot $a$ ja $b$ nollasta eroavia kokonaislukuja. Mitä voidaan päätellä, jos $a|b$ ja $b|a$?

\item Olkoon $n$ positiivinen kokonaisluku. Määritä osamäärä ja jakojäännös sekä kirjoita jakoyhtälö, kun
\begin{enumerate}[a)]
\item luku $5n + 3$ jaetaan luvulla $5$
\item luku $n^2 + 2n + 2$ jaetaan luvulla $n + 1$
\item luku $n^3 + 3n^2 - n - 3$ jaetaan luvulla $n + 3$
\item luku $2n^3 + 3n^2 + 4n + 9$ jaetaan luvulla $2n + 3$.
\end{enumerate}

%%%%%%%%%%%%%% FIX ME   Mikä on oikea tapa laittaa välit \ldots komentoa ennen ja jälkee? Nyt ennen on väli, jälkeen ei, kannattaa korjata kaikki esiintymät samanlaisiksi

\item
\begin{enumerate}[a)]
\item Muodosta jakoyhtälö luvuille $0, 1, 2, \ldots, 7$, kun jakajana on luku $3$. Mitä arvoja jakojäännös voi saada?
\item Osoita, että jos kahden kokonaisluvun tulo on jaollinen luvulla $3$, niin ainakin toinen luvuista on jaollinen luvulla $3$. Vihje: Käytä epäsuoraa todistusta.
\end{enumerate}

\end{enumerate}

\end{tehtavasivu}
\begin{kotitehtavasivu}

\begin{enumerate}

\item Onko luku a) $42$ b) $-75$ c) $102$ d) $-98$ jaollinen luvulla $7$?

\item Jakaako luku $11$ luvun a) $-165$ b) $21$ c) $-101$ d) $209$?

\item Osoita, että a) $2|234$ b) $-17|408$ c) $14 \nmid 223$ d) $6 \nmid (-472)$.

\item Onko väite tosi? a) $8 \nmid 168$ b) $13 \nmid (-95)$ c) $-5|777$ d) $29\nmid 2583$ e) $-4|(-924)$

\item Kirjoita jakoyhtälö, kun
\begin{enumerate}[a)]
\item luku $9$ jaetaan luvulla $6$
\item luku $576$ jaetaan luvulla $19$
\item luku $3712$ jaetaan luvulla $32$.
\end{enumerate}

\item Kirjoita jakoyhtälö, kun
\begin{enumerate}[a)]
\item luku $-12$ jaetaan luvulla $5$
\item luku $-147$ jaetaan luvulla $6$
\item luku $-875$ jaetaan luvulla $35$.
\end{enumerate}

\item Kirjoita jakoyhtälö jakolaskulle a) $3/17$ b) $-3/17$ c) $88/80$ d) $-88/80$.

\item Päättele, mikä on jakojäännös, kun
\begin{enumerate}[a)]
\item luku $5555$ jaetaan luvulla $4$
\item luku $-555$ jaetaan luvulla $4$
\item luku $123456$ jaetaan luvulla $100$
\item luku $-654321$ jaetaan luvulla $100$.
\end{enumerate}

\item Tänään on keskiviikko. Mikä viikonpäivä a) on 1000 päivän kuluttua b) oli 500 päivää sitten?

\item Kello on $17.28$ ja on tiistai. Mitä kello on 2073 tunnin kuluttua? Mikä viikonpäivä silloin on?

\item Tarkastellaan kirjainjonoa ABCDEFGABCDEFGABCDEFG... a) Mikä on jonon 12742. kirjain? b) Kuinka monta A-kirjainta on jonossa ennen sitä?

\item 
\begin{enumerate}[a)]
\item Mikä luku jaettuna luvulla $19$ antaa osamääräksi $6$ ja jakojäännökseksi $13$?
\item Mikä luku jaettuna luvulla $7$ antaa osamääräksi $-23$ ja jakojäännökseksi $5$?
\item Millä positiivisella kokonaisluvulla luku $1263$ on jaettava, jotta osamäärä olisi $114$ ja jakojäännös $9$?
\item  Millä positiivisella kokonaisluvulla luku $-140$ on jaettava, jotta jakojäännös olisi $3$?
\end{enumerate}

\item Olkoon $n$ kokonaisluku. Osoita, että luku $(3n+1)^4 - (3n+1)^3$ on jaollinen luvulla 3. Vihje: Käytä symbolisen laskimen {\tt expand}-toimintoa.

\item Olkoon $n$ kokonaisluku. Osoita, että luku $(2n+1)^5 - (2n-1)^4-2n$ on jaollinen luvulla $16$.

\item Määritä osamäärä ja jakojäännös sekä kirjoita jakoyhtälö, kun a) luku $7^{50} + 1$ jaetaan luvulla $7^{48} - 1$ b) luku $7^{50} - 1$ jaetaan luvulla $7^{25} + 1$.

\item Olkoot $a$, $b$ ja $c$ kokonaislukuja ja $a \neq 0$. Osoita, että jos $a|b$ ja $a|(b + c)$, niin $a|c$.

\item Olkoot $p$, $q$ ja $r$ positiivisia kokonaislukuja. Osoita, että jos $p$ on luvun $q$ tekijä ja $q$ on luvun $r$ tekijä, niin $p$ on luvun $r$ tekijä.

\item Olkoot $a$, $b$ ja $c$ kokonaislukuja ja $a, c \neq 0$. Osoita, että jos $(ac)|(bc)$, niin $a|b$.

\item Olkoot $a$ ja $b$ kokonaislukuja. Osoita, että luku $a + b$ on jaollinen luvulla $3$ silloin ja vain silloin, kun luku $a - 2b$ on jaollinen luvulla $3$.

\item Olkoon $n$ positiivinen kokonaisluku. Määritä osamäärä ja jakojäännös sekä kirjoita jakoyhtälö, kun
\begin{enumerate}[a)]
\item luku $2n^2 - n - 2$ jaetaan luvulla $n + 1$
\item luku $n^3 + n^2 + 6$ jaetaan luvulla $n + 2$
\item luku $n^2 + 2n - 1$ jaetaan luvulla $n$.
\end{enumerate}
Vihje: Voit laskea polynomien jakolaskun symbolisen laskimen avulla.

\item * %(Lisämateriaalia.)
Todista jakoyhtälö, kun $a<0$.

\item *
%(Lisämateriaalia.) 
\termi{lukujärjestelmä}{Lukujärjestelmä} tarkoittaa tapaa, jolla luvut kirjoitetaan numeroiden avulla. \termi{kantaluku}{Kantaluku} kertoo, kuinka monta eri numeroa lukujärjestelmän luvuissa voi esiintyä. Esimerkiksi \termi{kymmenjärjestelmä}{kymmenjärjestelmässä} kantaluku on $10$ ja käytössä ovat numerot $0, 1, 2, 3, 4, 5, 6, 7, 8$ ja $9$. Kymmenjärjestelmän luku $3258$ muodostuu numeroista $3, 2, 5$ ja $8$ kantaluvun $10$ potenssien avulla seuraavasti:
\begin{eqnarray*}
3258 &=&3\cdot 1000+2\cdot 100+5\cdot 10+8\\
&=& 3\cdot 10^3+2\cdot 10^2+5\cdot 10^1+8\cdot 10^0.
\end{eqnarray*}
Kymmenjärjestelmässä siis luvun viimeinen numero on luvun $10^0$ kerroin, toiseksi viimeinen luvun $10^1$ kerroin, kolmanneksi viimeinen luvun $10^2$ kerroin jne.

\termi{kaksijärjestelmä}{Kaksijärjestelmän} eli \termi{binäärijärjestelmä}{binäärijärjestelmän} luvut taas muodostuvat numeroista $0$ ja $1$. Esimerkiksi binääriluku $101101$ voidaan ilmaista kymmenjärjestelmässä kirjoittamalla luku kantaluvun $2$ potenssien avulla seuraavasti:
\begin{eqnarray*}
101101&=&1\cdot2^5+0\cdot2^4+1\cdot2^3+1\cdot2^2+0\cdot2^1+1\cdot2^0 \\
&=& 1\cdot32+0\cdot16+1\cdot8+1\cdot4+0\cdot2+1=45.
\end{eqnarray*}

Kaksijärjestelmässä siis luvun viimeinen numero on luvun $2^0$ kerroin, toiseksi viimeinen luvun $2^1$ kerroin, kolmanneksi viimeinen luvun $2^2$ kerroin jne.

Kymmenjärjestelmän lukuja voidaan muuntaa binääriluvuiksi jakoyhtälöiden avulla. Muunnetaan luku 18 binääriluvuksi. Jaetaan ensin kymmenjärjestelmän luku 18 kaksijärjestelmään kantaluvulla 2 ja kirjataan ylös jakojäännös. Tämän jälkeen jaetaan edellisen jakolaskun osamäärä kantaluvulla 2 ja kirjataan taas jakojäännös muistiin. Näin jatketaan, kunnes osamääräksi jää luku 0. 
\begin{eqnarray*}
18&=&9\cdot 2+0\\
9&=&4\cdot 2+1\\
4&=&2\cdot 2+0\\
2&=&1\cdot 2+0\\
1&=&0\cdot2+1 \to \textrm{lopetetaan}
\end{eqnarray*}
Kirjoittamalla nyt jakojäännökset lopusta alkuun saadaan luku $10010$. Se on kymmenjärjestelmän luvun $18$ binääriesitys.
\begin{enumerate}[a)]
\item Muunna binäärijärjestelmän luvut $1001$ ja $110110100$ kymmenjärjestelmään.
\item Muunna kymmenjärjestelmän luvut $25$ ja $520$ binäärijärjestelmään.
\item Etsi laskimesi ohjekirjasta, miten muunnokset voidaan toteuttaa laskimella.
\item Ohjelmoi laskimesi ohjelmointikielellä ohjelma, joka suorittaa muunnoksen kymmenjärjestelmästä binäärijärjestelmään.
\end{enumerate}

\end{enumerate}

\end{kotitehtavasivu}
