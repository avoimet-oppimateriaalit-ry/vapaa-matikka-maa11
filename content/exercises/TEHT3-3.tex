\begin{tehtavasivu}

\begin{enumerate}%
\item Olkoon lause
+$T(x)$: ''opiskelija $x$ on täysi-ikäinen'' ja
+perusjoukko kaikki koulun opiskelijat. Suomenna lause.
\begin{enumerate}[a)]
\item $\forall x T(x)$
\item $\exists x T(x)$
\item $\exists x \lnot T(x)$
\item $\forall x \lnot T(x)$
\end{enumerate}%
\item Suomenna lause. Onko lause tosi? Perustele.
\begin{enumerate}[a)]
\item $\forall x\in\rr (|x| > 0)$
\item $\forall x\in\rr (|x| \ge 0)$
\item $\exists x\in\nn (x^2 < 2)$
\item $\exists x\in\nn (x^2 = 2)$
\end{enumerate}%
\item
Formalisoi lause. Onko lause tosi? Perustele.
\begin{enumerate}[a)]
\item Jokainen kokonaisluku on joko positiivinen tai
negatiivinen.
\item On olemassa sellainen kokonaisluku, jonka neliöjuuri on
yhtä suuri kuin luku itse.
\item Minkään kokonaisluvun neliö ei ole $7$.
\end{enumerate}%
\item Osoita, että yhtälö $\sqrt{x^2} = x$ ei pidä paikkaansa
kaikilla reaaliluvuilla.%
\item Jalkapallojoukkue on lähdössä turnausmatkalle. Olkoon
$P(x, y)$ avoin lause ''pelaajalla $x$ on pelaajan $y$
puhelinnumero''. Suomenna lause.
\begin{enumerate}[a)]
\item $\forall x \exists y P(x, y)$
\item $\exists x \forall y P(x, y)$
\item $\exists y \forall x P(x, y)$
\end{enumerate}%
\item
Formalisoi lause. Onko lause tosi?
\begin{enumerate}[a)]
\item Positiivisten kokonaislukujen joukossa on pienin alkio.
\item Negatiivisten kokonaislukujen joukossa on pienin alkio.
\end{enumerate}%
\item Onko lause tosi? Perustele.
\begin{enumerate}[a)]
\item $\forall x\in \rr\exists y\in \rr (xy=1)$
\item $\exists x\in \rr\forall y\in \rr (xy=y)$
\end{enumerate}%
\item Onko lause tosi? Perustele. % NVI
\begin{enumerate}[a)]
\item $\exists x\in \rr\exists y\in \rr (xy=x+y)$
\item $\forall x\in \rr\exists y\in \rr (xy=x+y)$ 
\end{enumerate}%
\item Olkoon $M(x)$: ''$x$ on matemaatikko''. Formalisoi lause.%
\begin{enumerate}[a)]
\item Kaikki eivät ole matemaatikkoja.
\item Joku ei ole matemaatikko.
\item Ei ole olemassa matemaatikkoa.
\item Kukaan ei ole matemaatikko.
\end{enumerate}%
\item
Muodosta lauseen negaatio.
\begin{enumerate}[a)]
\item $\forall x\in \zz (x < 12)$
\item $\exists x\in \zz (x^2 = 12)$
\item $\exists x\in \nn ((x^2=9) \land (x<5))$
\item $\forall x\in \nn ((x=0)\lor (x\ge 1))$
\end{enumerate}%
\item Osoita, että lause $\exists a, b, c \in \zz_{+} (a^2 + b^2 =
c^2)$ on tosi.%
\item *
Olkoot $M(x)$: ''$x$ on matemaatikko'' ja
$I(x)$: ''$x$ on iloinen''. Formalisoi lause.
\begin{enumerate}[a)]
\item Kaikki ovat iloisia matemaatikkoja.
\item Matemaatikot ovat iloisia.
\item Kukaan matemaatikko ei ole iloinen.
\item Kaikki matemaatikot eivät ole iloisia.
\end{enumerate}%
\item *
Olkoot $K(x)$: ''$x$ on kampaaja'' ja
$T(x, y)$: ''$x$ tekee $y$:lle kampauksen''. Formalisoi lause.
\begin{enumerate}[a)]
\item Kukaan kampaaja ei tee kampausta itselleen.
\item Joku kampaaja tekee kampauksen niille kampaajille,
jotka eivät tee kampausta itselleen.
\end{enumerate}%
\end{enumerate}%

\end{tehtavasivu}
\begin{kotitehtavasivu}

\begin{enumerate}%
\item Olkoon perusjoukko $xy$-tason suorien joukko. Olkoot
lauseet $N(x)$: ''suora on nouseva'' ja $L(x)$: ''suora on laskeva''.
Suomenna lause. Onko lause tosi?
\begin{enumerate}[a)]
\item $\exists x L(x)$
\item $\forall x (N(x) \lor L(x))$
\item $\exists x (\lnot N(x) \land \lnot L(x))$
\item $\forall x \lnot N(x)$
\end{enumerate}%
\item Onko lause tosi? Perustele.
\begin{enumerate}[a)]
\item $\forall x\in\rr (x^2 \ge 0)$
\item $\exists x\in\nn (x^2 = -9)$
\item $\forall x\in\rr ((x-1)(x-3) \ge 0)$
\item $\exists x\in\zz (-x^2 - 2 \le 0)$
\end{enumerate}%
\item Onko lause tosi? Perustele.
\begin{enumerate}[a)]
\item $\exists x\in \rr (x^2 - 3x + 3 = 0)$
\item $\forall x\in \rr (x^2 - 3x + 3 \ge 0)$
\end{enumerate}%
\item
\begin{enumerate}[a)]
\item Osoita, että yhtälö $\sqrt{xy} = \sqrt{x}\sqrt{y}$ ei
pidä paikkaansa kaikilla reaaliluvuilla $x$ ja $y$.
\item Osoita, että on olemassa sellaiset reaaliluvut $x$ ja
$y$, että yhtälö $\sqrt{xy} = \sqrt{x}\sqrt{y}$ pitää paikkansa.
\item Millä ehdolla yhtälö $\sqrt{xy} = \sqrt{x}\sqrt{y}$
pitää yleisesti paikkansa?
\end{enumerate}%
\item Olkoon $S(x, y)$: ''$x$ on suorittanut kurssin $y$''.
Formalisoi lause.
\begin{enumerate}[a)]
\item Joku on suorittanut kaikki kurssit.
\item Jokainen on suorittanut ainakin yhden kurssin.
\item Kukaan ei ole suorittanut kaikkia kursseja.
\item On olemassa kurssi, jota kukaan ei ole suorittanut.
\end{enumerate}%
\item Onko lause tosi? Perustele.
\begin{enumerate}[a)]
\item $\forall x\in \zz_{+} \exists y\in \zz_{+} (x = \sqrt{y})
$
\item $\exists y\in \rr \forall x\in \rr (x^2 - 4 > y)$
\end{enumerate}%
\item Ilmaise suomen kielellä lauseen negaatio kahdella eri
tavalla soveltamalla kvanttorien negaatioiden loogisesti
ekvivalentteja muotoja.
\begin{enumerate}[a)]
\item Jokainen opiskelija saa tästä kurssista arvosanan 10.
\item On olemassa opiskelija, joka saa tästä kurssista
arvosanan 10.
\end{enumerate}%
\item Kirjoita lause toisin.
\begin{enumerate}[a)]
\item $\lnot \forall x \lnot P(x)$
\item $\lnot \exists x (P(x) \lor Q(x))$
\end{enumerate}%
\item
Muodosta lauseen negaatio. Onko negaatio tosi?
\begin{enumerate}[a)]
\item $\forall x\in ]1, \infty [ (\sqrt{x} < x)$
\item $\exists x\in \qq (x^3 = 5)$
\item $\forall n \in \zz \exists m \in \zz ((n = 2m) \lor (n =
2m+1))$
\end{enumerate}%
\item Funktiota $f\colon X\to Y$ voidaan ajatella kahden
muuttujan avoimena lauseena $P(x, y)$: ''$f(x) = y$'', missä
$x$ kuuluu määrittelyjoukkoon $X$ ja $y$ maalijoukkoon $Y$.
Funktiolta edellytetään lisäksi, että
\begin{itemize}
\item $\forall x \exists y P(x, y)$, ja
\item $\lnot (\exists x \exists y \exists z (P(x, y) \land P(x, z)
\land (y \neq z)))$.
\end{itemize}
Tulkitse sanallisesti tai kuvaa käyttäen, mitä tämä määritelmä
tarkoittaa.%
\item *
 Olkoot $S(x, y)$: ''$x$ on suorittanut
kurssin $y$'' ja $L(x)$: ''$x$ on lukiolainen''. Formalisoi lause.
\begin{enumerate}[a)]
\item Joku lukiolainen on suorittanut kaikki kurssit.
\item Kukaan lukiolainen ei ole suorittanut kaikkia kursseja.
\end{enumerate}%
\item *
Olkoot $S(x, y)$: ''$x$ on suorittanut
kurssin $y$'', $L(x)$: ''$x$ on lukiolainen'' ja $M(y)$: ''$y$ on
pitkän matematiikan kurssi''. Formalisoi lause.
\begin{enumerate}[a)]
\item Joku lukiolainen ei ole suorittanut yhtään pitkän
matematiikan kurssia.
\item Joku lukiolainen on suorittanut pitkän matematiikan
kurssin.
\item Jokaisella lukiolaisella on pitkän matematiikan kurssi
suoritettuna.
\item On olemassa pitkän matematiikan kurssi, jota kukaan
lukiolainen ei ole suorittanut.
\end{enumerate}%
\item *
Määritä kaikki funktiot $f\colon X\to Y$,
kun $X=\{a, b, c\}$ ja $Y=\{1, 2\}$.%
\item *
Määritä kaikki funktiot $f\colon X\to Y$,
kun $X=\emptyset$ ja $Y\neq \emptyset$.%
\item *
Määritä kaikki funktiot $f\colon X\to Y$,
kun $X\neq \emptyset$ ja $Y= \emptyset$.%
%%%%%%%%%%%%%%%%%%%%%%%%%%%%%%%FIX ME, linkki toimimaton  %%%%%%%
%\item *
%Tutustu logiikkapohjaiseen Prolog-ohjelmointikieleen\\
%\href{http://www.cs.helsinki.fi/u/wikla/OKP/OppaatK07/prolog.html}
%{{\tt http://www.cs.helsinki.fi/u/wikla/OKP/OppaatK07/prolog.html}}
%
%Lataa koneellesi Prolog-tulkki \href{http://www.gprolog.org/}
%{{\tt http://www.gprolog.org/}}
%ja kokeile Prolog-ohjelmointia.%
\end{enumerate}

\end{kotitehtavasivu}
