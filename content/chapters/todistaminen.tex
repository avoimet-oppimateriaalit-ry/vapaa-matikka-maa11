\chapter{Matemaattisen väitteen todistaminen}

Matematiikassa pyritään \termi{todistaminen}{todistamaan} matemaattisia tuloksia.
Todistamisen ideana on, että tulos johdetaan loogisesti sitovalla päättelyllä,
jossa voidaan vedota aikaisemmin todistettuihin tuloksiin,
käsiteltävän teorian \termi{aksiooma}{aksioomiin} eli perusoletuksiin sekä loogisiin päättelysääntöihin.
Matematiikassa tärkeimpiä tuloksia sanotaan yleensä \termi{lause}{lauseiksi} eli
\termi{teoreema}{teoreemoiksi} (esim. Pythagoraan lause) ja pienempiä aputuloksia kutsutaan
\termi{apulause}{apulauseiksi} eli \termi{lemma}{lemmoiksi}.

Kun tulos on todistettu matemaattisesti, se on kumoamaton. Matemaattiset tulokset eivät siksi vanhene.
Toisaalta matemaattisten tulosten soveltamista rajoittavat niiden johtamisessa käytetyt oletukset,
koska tulos ei sano mitään tilanteesta, jossa nämä oletukset eivät ole voimassa.

\subsection*{Tutkimustehtävä} % Tutkitaan lukujen parillisuutta ja parittomuutta.

\begin{enumerate}
	\item Mitä lukuja tarkoittaa merkintä $2n$, kun $n$ käy läpi kaikki luonnolliset luvut? Entä mitä lukuja 
		tarkoittaa merkintä $2n + 1$, kun $n\in \nn$?
	\item %Kaikki parilliset luvut voidaan esittää muodossa $2n$, missä $n$ on kokonaisluku. Esimerkiksi $12=2\cdot 6$.
		Esitä luvut $6$, $10$ ja $26$ muodossa $2n$.
	\item %Kaikki parittomat luvut voidaan esittää muodossa $2n + 1$, missä $n$ on kokonaisluku. 
		Esitä luvut $7$, $11$ ja $35$ muodossa $2n+1$.
	%\item Laske yhteen parittomia lukupareja. Esimerkiksi $5 + 7 =12$. Mitä havaitset summasta? 
		%Kahden parittoman luvun summa on aina \underline{\phantom{xxxxxxxxxx}}.
	%\item Todista edellisessä kohdassa muotoilemasi väite käyttäen kohdissa (1) ja (2) esitettyjä parillisten ja 
		%parittomien lukujen esitysmuotoja.
	%\item Todista väite matemaattisesti. Voit käyttää merkintöjä $2n$ ja $2n + 1$, $n\in\nn$.
	\item Laske yhteen parittomia lukuja. Esimerkiksi $5 + 7 = 12$. Mitä havaitset summasta?
	\item Todista havainto matemaattisesti. Kahdelle eri suurelle parittomalle luvulle voidaan käyttää yleisiä 
		merkintöjä $2n + 1$, $n \in\nn$, ja $2m + 1$, $m \in\nn$.
\end{enumerate}

\subsection*{Oletus, väite ja deduktio} % formaaleja väittämiä, suora todistus, vastaesimerkkitodistus
%tehtäviä oletuksista ja väitteistä

Matemaattinen tulos muodostuu kolmesta osasta. Tulokseen liittyy \termi{oletus}{oletus}.
Oletusten täytyy olla tosia, jotta tulosta voitaisiin käyttää. Tuloksen toinen osa on \termi{väite}{väite}.
Matematiikan lause sanoo, että oletusten vallitessa lauseen väite on myös tosi.
Tuloksen kolmas osa on \termi{todistus}{todistus}. Matemaattinen todistus on deduktiivinen päättelyketju,
jossa oletuksista johdetaan lauseen väite. Matemaattisen todistuksen loppuun merkitään yleensä
pieni neliö merkiksi todistuksen päättymisestä.

Yksinkertaisin matemaattisen todistuksen tyyppi on \termi{suora todistus}{suora todistus}.
Suorassa todistuksessa tulos saadaan suoralla päättelyllä, jossa voidaan vedota lauseen oletuksiin,
aksioomiin, määritelmiin ja aikaisemmin todistettuihin tuloksiin, esimerkiksi laskusääntöihin.
Suoralla todistuksella voidaan todistaa seuraavan esimerkin tulos.

\laatikko{\textbf{Lause 1.} Olkoot $m$ ja $n$ parillisia kokonaislukuja. Tällöin luku $m+n$ on parillinen.}

\begin{esimerkki}
	Todista Lause 1.
	
	\begin{todistus}
		Koska $m$ ja $n$ ovat parillisia, voidaan kirjoittaa $m=2a$ ja $n=2b$, missä $a$ ja $b$ ovat kokonaislukuja.
		
		Siten
		\[
			m+n =2a+2b = 2(a+b).
		\]
		Luku $2(a+b)$ on parillinen, joten väite on todistettu.
	\end{todistus}
\end{esimerkki}

\laatikko{\textbf{Lause 2.} Kahden rationaaliluvun summa on rationaalinen.}

\begin{esimerkki}
	Todista Lause 2.
	
	\begin{todistus}
		Olkoot $q_1$ ja $q_2$ kaksi rationaalilukua. Ne voidaan kirjoittaa muodossa
		\[
			q_1=\frac{m_1}{n_1},\qquad 
			q_2=\frac{m_2}{n_2},
		\]
		missä $m_1, m_2, n_1, n_2$ ovat kokonaislukuja ja $n_1, n_2\neq 0$.

		Edelleen voidaan kirjoittaa
		\[
			q_1+q_2 = \frac{m_1}{n_1}+ \frac{m_2}{n_2}
		\]
		\[
			= \frac{m_1 n_2}{n_1 n_2}+ \frac{m_2 n_1}{n_2 n_1} = \frac{m_1n_2 + m_2 n_1}{n_1 n_2}.
		\]

		Nyt $m_1n_2 + m_2 n_1$ sekä $n_1 n_2$ ovat kokonaislukuja ja $n_1 n_2\neq 0$, joten summa $q_1+q_2$ kuuluu rationaalilukujen joukkoon $\qq$.
	\end{todistus}
\end{esimerkki}

\laatikko{\textbf{Lause 3.} Olkoot $a$ ja $b$ reaalilukuja.
	Tulo $ab<0$, jos ja vain jos joko $a>0$ ja $b<0$ tai $a<0$ ja $b>0$.}

\begin{esimerkki}
	Todista Lause 3.
	
	\begin{todistus}
		Ekvivalenssilauseen väite on muotoa ''jos ja vain jos''. Tällainen lause todistetaan usein kahdessa osassa.

		Todistetaan aluksi lauseen ''jos''-osa: $ab<0$, jos joko $a>0$ ja $b<0$ tai $a<0$ ja $b>0$.

		Oletetaan aluksi, että $a>0$ ja $b<0$. Kertomalla epäyhtälö $b<0$ luvulla $a$ saadaan $ab<0$.

		Vastaavasti jos $a<0$ ja $b>0$, niin kertomalla epäyhtälö $a<0$ luvulla $b$ saadaan $ab<0$. Siten lauseen 
		ensimmäinen osa on todistettu.

		Osoitetaan seuraavaksi lauseen toinen osa: $ab<0$ vain, jos $a>0$ ja $b<0$ tai $a<0$ ja $b>0$.

		Huomataan aluksi, että jos $ab<0$, niin $a\neq 0$ ja $b\neq 0$. Jos $a>0$, niin jakamalla epäyhtälö $ab<0$ 
		luvulla $a$ saadaan $b<0$. Toisaalta, jos $a<0$, niin jaettaessa epäyhtälön $ab<0$ suunta vaihtuu ja saadaan 
		$b>0$. Siis myös lauseen toinen osa on todistettu.
	\end{todistus}
\end{esimerkki}

\subsection*{Epäsuora todistus} % epäsuora todistus, käänteinen todistus, ristiriitatodistus

Tärkeä esimerkki matemaattisesta todistusmenetelmästä on \termi{epäsuora todistus}{epäsuora todistus}.
Epäsuora todistus perustuu jo aikaisemmin tässä kurssissa esiintyneeseen kolmannen poissulkevan lakiin.
Tämä laki sanoo, että kaikki väitteet ovat joko tosia tai epätosia.

\termi{käänteinen todistus}{Käänteisessä todistuksessa} ajatuksena on, että väitteen $A\to B$
todistamiseksi riittää osoittaa, että väitteen $B$ negaatio $\lnot B$ eli niin sanottu
\termi{vastaoletus}{vastaoletus} johtaa oletuksen $A$ negaatioon $\lnot A$.
Käänteinen todistus perustuu kappaleessa 2.3 esiteltyyn kontraposition lakiin,
jonka mukaan lauseet $A\to B$ ja $\lnot B \to \lnot A$ ovat loogisesti ekvivalentit.

Käänteisessä todistuksessa täytyy siis olettaa väitteen negaatio ja päätellä siitä,
että oletus on epätosi. Tällä tavoin voidaan osoittaa esimerkiksi seuraava tulos:

\laatikko{\textbf{Lause 4.} Olkoon $m\in \nn$ siten, että $m^2$ on parillinen. Tällöin $m$ on parillinen.}

\begin{esimerkki}
	Todista Lause 4.

	\begin{todistus}
		Lauseen todistamiseksi tehdään vastaoletus, joka on lauseen väitteen ''$m$ on parillinen'' negaatio.

		Vastaoletus: $m$ on pariton.

		Tästä seuraa, että on olemassa $k\in \nn$, jolle $m=2k+1$.

		Nyt
		\[
			m^2 = (2k+1)^2 = 4k^2+4k+1 = 2(2k^2+2k)+1.
		\]

		Siten $m^2$ on pariton. On päädytty ristiriitaan lauseen oletuksen kanssa. Siten lauseen väite on tosi.
	\end{todistus}
\end{esimerkki}

\termi{ristiriitatodistus}{Ristiriitatodistuksessa} pyritään todistamaan muotoa $A\to B$ oleva
väite tekemällä ensin vastaoletus $\lnot B$ ja päätymällä johonkin ristiriitaan, ei kuitenkaan
välttämättä oletuksen $A$ negaatioon.

Ehkä kuuluisin esimerkki epäsuorasta todistuksesta on todistus sille, että luku $\sqrt{2}$ on irrationaalinen. 
Todistuksessa tarvitaan seuraavaa lisätietoa rationaaliluvuista. Sanotaan, että kokonaisluku $a$ on kokonaisluvun $b$ 
\termi{tekijä}{tekijä}, jos on olemassa sellainen kokonaisluku $c$, että $b=a\cdot c$. Rationaaliluku $q\in\qq$ voidaan 
aina esittää \termi{supistettu muoto}{supistetussa muodossa} $q=m/n$, missä $m$ ja $n$ ovat kokonaislukuja, joilla ei 
ole yhteisiä tekijöitä, ja $n\neq 0$. Esimerkiksi murtoluku $2/5$ on supistetussa muodossa, koska luvuilla $2$ ja $5$ 
ei ole yhteisiä tekijöitä. Sen sijaan murtoluku $6/9$ ei ole supistetussa muodossa, koska luku $3$ on sekä osoittajan 
että nimittäjän tekijä. Murtoluvun $6/9$ esitys supistetussa muodossa on $2/3$.

\laatikko{\textbf{Lause 5.} Luku $\sqrt{2}$ on irrationaalinen.}

\begin{esimerkki}
	Todista Lause 5.

	\begin{todistus}
		Vastaoletus: Luku $\sqrt{2}$ on rationaalinen.

		On siis olemassa luku $q\in \qq$ siten, että $q^2=2$.

		Kirjoitetaan luku $q$ supistetussa muodossa $q=m/n$, missä kokonaisluvuilla
		$m$ ja $n$ ei ole yhteisiä tekijöitä. 
		Nyt $q^2=2$, jos ja vain jos
		\[
			\bigg(\frac{m}{n}\bigg)^2=2.
		\]

		Kertomalla puolittain luvulla $n^2$ saadaan yhtälö
		\[
			m^2 = 2n^2.
		\]

		Koska luku $2n^2$ on parillinen, on $m^2$ parillinen. Lauseen 4 perusteella myös $m$ on parillinen.

		Siten on olemassa $k\in \zz$, jolle $m=2k$.

		Koska
		\[
			n^2=m^2/2=(2k)^2/2= 2k^2,
		\]
		luku $n^2$ on parillinen. Lauseen 4 nojalla myös $n$ on parillinen.

		Näin ollen luku $2$ on lukujen $m$ ja $n$ yhteinen tekijä. Tämä on ristiriita,
		koska oletettiin, ettei näillä luvuilla ole yhteisiä tekijöitä.
	\end{todistus}
\end{esimerkki}

% Tekemättä % TEHTÄVÄT
