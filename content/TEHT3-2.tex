\setcounter{tehtava}{0}

\begin{tehtavasivu}

\begin{tehtava}
	Olkoot $A(x)$ avoin lause ''paikkakunnalla $x$ paistaa
	aurinko'' ja $T(x)$ avoin lause ''paikkakunnalla $x$
	tuulee''. Suomenna lause.
	\begin{alakohdat}
	\alakohta{$\lnot T(x)$,}
	\alakohta{$A(x) \land T(y)$,}
	\alakohta{$\lnot A(\textrm{Lempäälä})$,}
	\alakohta{$\lnot (A(\textrm{Turku}) \to T(\textrm{Kuopio}))$.}
	\end{alakohdat}%
	\begin{vastaus}
		\begin{alakohdat}
			\alakohta{Paikkakunnalla $x$ ei tuule.}
			\alakohta{Paikkakunnalla $x$ paistaa aurinko ja paikkakunnalla $y$ tuulee.}
			\alakohta{Lempäälässä ei paista aurinko.}
			\alakohta{Turussa paistaa aurinko ja Kuopiossa ei tuule.}
		\end{alakohdat}
	\end{vastaus}
\end{tehtava}

\begin{tehtava}
	Olkoot $S(x)$ avoin lause ''$x$ on suomalainen
	formulakuljettaja'' ja $E(x)$ avoin lause ''$x$ on
	eurooppalainen formulakuljettaja, joka ei ole
	suomalainen''. Kuljettaja voi olla joko nykyinen tai
	entinen. Ratkaise avoin lause, kun perusjoukko on
	$\{\textrm{Räikkönen}, \textrm{Vettel}, \textrm{Senna}\}$.
	\begin{alakohdat}
	\alakohta{$S(x)$,}
	\alakohta{$\lnot S(x)$,}
	\alakohta{$E(x)$,}
	\alakohta{$\lnot (S(x) \lor E(x))$.}
	\end{alakohdat}%
	\begin{vastaus}
		\begin{alakohdat}
			\alakohta{$\{\textrm{Räikkönen}\}$}
			\alakohta{$\{\textrm{Vettel}, \textrm{Senna}\}$}
			\alakohta{$\{\textrm{Vettel}\}$}
			\alakohta{$\{\textrm{Senna}\}$}
		\end{alakohdat}
	\end{vastaus}
\end{tehtava}

\begin{tehtava}
	Olkoon $Q(x)$ avoin lause ''$x^2 = 64$''. Onko lause a)
	$Q(-8)$ b) $Q(6)$ tosi?%
	\begin{vastaus}
		\begin{alakohdat}
			\alakohta{On.}
			\alakohta{Ei ole.}
		\end{alakohdat}
	\end{vastaus}
\end{tehtava}

\begin{tehtava}
	Olkoon $P(x, y)$ avoin lause ''$x^2 + y \le 0$''. Onko
	lause a) $P(0, 0)$ b) $P(-1, 1)$ c) $P(5, -100)$ tosi?%
	\begin{vastaus}
		\begin{alakohdat}
			\alakohta{On.}
			\alakohta{Ei ole.}
			\alakohta{On.}
		\end{alakohdat}
	\end{vastaus}
\end{tehtava}

\begin{tehtava}
	Olkoot $S(x)$ avoin lause ''kuvio $x$ on symmetrinen
	jonkin suoran suhteen'' ja $P(x)$ avoin lause ''kuvio $x$
	on symmetrinen jonkin pisteen suhteen''. Perusjoukon
	muodostavat oheiset kuviot. Ratkaise avoin lause.
	\begin{alakohdat}
	\alakohta{$S(x)$,}
	\alakohta{$P(x)$,}
	\alakohta{$\lnot (S(x) \lor P(x))$,}
	\alakohta{$S(x) \lequiv P(x)$.}
	\end{alakohdat}%
	\begin{center}
	\includegraphics[width=10cm]{pictures/kpl3_2_teht7}
	\end{center}%
	\begin{vastaus}
		\begin{alakohdat}
			\alakohta{$\{A, D, E\}$}
			\alakohta{$\{A, B, D\}$}
			\alakohta{$\{C, F\}$}
			\alakohta{$\{A, D\}$}
		\end{alakohdat}
	\end{vastaus}
\end{tehtava}

\begin{tehtava}
	Ratkaise avoin lause $4x^2 + 7x - 2 = 0$, kun perusjoukko
	on a) reaalilukujen joukko b) kokonaislukujen joukko.%
	\begin{vastaus}
		\begin{alakohdat}
			\alakohta{$\{-2, \frac{1}{4}\}$}
			\alakohta{$\{-2\}$}
		\end{alakohdat}
	\end{vastaus}
\end{tehtava}

\begin{tehtava}
	Olkoot perusjoukko $\{ 0, 1, 2, \ldots , 10\}$, $A(x)$
	avoin lause ''$x \le 1$'' ja $B(x)$ avoin lause ''$x > 5$''.
	Ratkaise avoin lause.
	\begin{alakohdat}
	\alakohta{$A(x)$,}
	\alakohta{$\lnot B(x)$,}
	\alakohta{$A(x) \land B(x)$,}
	\alakohta{$\lnot A(x) \to B(x)$.}
	\end{alakohdat}%
	\begin{vastaus}
		\begin{alakohdat}
			\alakohta{$\{0, 1\}$}
			\alakohta{$\{0, 1, 2, 3, 4 ,5\}$}
			\alakohta{$\emptyset$}
			\alakohta{$\{6, 7, 8, 9, 10\}$}
		\end{alakohdat}
	\end{vastaus}
\end{tehtava}

\begin{tehtava}
	Arvotaan kaksi lukua, joista molemmat voivat olla 1, 2 tai 3. Arvonnan tuloksena saadaan lukupari $(x, y)$, missä $x$ on ensimmäisen arvonnan tulos ja $y$ toisen arvonnan tulos. Olkoot avoimet lauseet $S(x, y)$: ''$x \le y$'' ja $T(x, y)$: ''tulo $xy$ on jaollinen luvulla 3''. Ratkaise avoin lause, kun perusjoukon muodostavat arvonnan tuloksena saatavat mahdolliset lukuparit
	\[
	(1, 1),\ (1, 2),\ (1, 3),\ (2, 1),\ (2, 2),\ (2, 3),\ (3, 1),\ (3, 2),\ (3, 3).
	\]
	\begin{alakohdat}
	\alakohta{$S(x, y)$,}
	\alakohta{$\lnot T(x, y)$,}
	\alakohta{$\lnot (S(x, y) \lor T(x, y))$,}
	\alakohta{$S(x, y) \to T(x, y)$.}
	\end{alakohdat}%
	\begin{vastaus}
		\begin{alakohdat}
			\alakohta{$\{(1, 1), (1, 2), (1, 3), (2, 2), (2, 3), (3, 3)\}$}
			\alakohta{$\{(1, 1), (1, 2), (2, 1), (2, 2)\}$}
			\alakohta{$\{(2, 1)\}$}
			\alakohta{$\{(1, 3), (2, 1), (2, 3), (3, 1), (3, 2), (3, 3)\}$}
		\end{alakohdat}
	\end{vastaus}
\end{tehtava}

\begin{tehtava}
	Ratkaise reaalilukujen joukossa
	\begin{alakohdat}
	\alakohta{epäyhtälö $x^2 \le 4$,}
	\alakohta{epäyhtälöpari}
	\[
	\left\{
	\begin{array}{rcl}
	x^2 & \le & 4 \\
	x & > & -1.
	\end{array}\right.
	\]
	\end{alakohdat}%
	\begin{vastaus}
		\begin{alakohdat}
			\alakohta{$-2 \le x \le 2$}
			\alakohta{$-1 < x \le 2$}
		\end{alakohdat}
	\end{vastaus}
\end{tehtava}

\begin{tehtava}
	Ratkaise reaalilukujen joukossa epäyhtälö
	\begin{alakohdat}
	\alakohta{$|x| > 5$,}
	\alakohta{$|x| \le 1$,}
	\alakohta{$|x - 4| > 5$,}
	\alakohta{$|2x - 6| \le 1$.}
	\end{alakohdat}%
	\begin{vastaus}
		\begin{alakohdat}
			\alakohta{$(x < -5) \lor (x > 5)$}
			\alakohta{$-1 \le x \le 1$}
			\alakohta{$(-1 < x) \lor (x > 9)$}
			\alakohta{$2,5 \le x \le 3,5$}
		\end{alakohdat}
	\end{vastaus}
\end{tehtava}

\begin{tehtava}
	Ratkaise avoin lause reaalilukujen joukossa. Ilmaise ratkaisujoukko myös välimerkintää käyttäen.
	\begin{alakohdat}
	\alakohta{$(x^2 > 1) \lor (0 \le x \le 2)$,}
	\alakohta{$(x^2 > 1) \land (0 \le x \le 2)$,}
	\alakohta{$(x^2 > 1) \to (0 \le x \le 2)$.}
	\end{alakohdat}%
	\begin{vastaus}
		\begin{alakohdat}
			\alakohta{$(x < -1) \lor (x \ge 0)$}
			\alakohta{$1 < x \le 2$}
			\alakohta{$-1 < x \le 2$}
		\end{alakohdat}
	\end{vastaus}
\end{tehtava}

\begin{tehtava}
	Olkoot $P(x)$ avoin lause $(x < 10) \land (x^2 = 100)$ ja
	$Q(x)$ avoin lause $(x \ge 10) \lequiv (x^2 = 100)$. Ratkaise avoin lause reaalilukujen joukossa.
	\begin{alakohdat}
	\alakohta{$P(x)$,}
	\alakohta{$Q(x)$,}
	\alakohta{$P(x) \lor Q(x)$,}
	\alakohta{$P(x) \land Q(x)$.}
	\end{alakohdat}%
	\begin{vastaus}
		\begin{alakohdat}
			\alakohta{$x = -10$}
			\alakohta{$(x \le 10) \land (x \ne -10)$}
			\alakohta{$x \le 10$}
			\alakohta{$\emptyset$}
		\end{alakohdat}
	\end{vastaus}
\end{tehtava}

\end{tehtavasivu}

% -----
\setcounter{tehtava}{0}

\begin{kotitehtavasivu}

\begin{tehtava}
	Olkoon $T(x, y)$ avoin lause ''$x$ lähettää tekstiviestin
	$y$:lle''. Suomenna lause.
	\begin{alakohdat}
	\alakohta{$T(\textrm{Tiina}, \textrm{Elias})$,}
	\alakohta{$\lnot T(\textrm{Elias}, \textrm{Tiina}) \land T(\textrm{Elias}, \textrm{Vilma})$,}
	\alakohta{$T(x, x)$,}
	\alakohta{$T(y, \textrm{Rasmus}) \lequiv T(\textrm{Rasmus}, y)$.}
	\end{alakohdat}%
\end{tehtava}

\begin{tehtava}
	Olkoot $E(x)$ avoin lause ''$x$ on eurooppalainen
	pääkaupunki'' ja $S(x)$ avoin lause ''$x$ on suomalainen
	kaupunki''. Ratkaise avoin lause, kun perusjoukko on
	$\{\textrm{Helsinki}, \textrm{Rovaniemi}, \textrm{Praha}, \textrm{Milano}\}$.
	\begin{alakohdat}
	\alakohta{$E(x)$,}
	\alakohta{$\lnot S(x)$,}
	\alakohta{$S(x) \land \lnot E(x)$,}
	\alakohta{$\lnot (E(x) \land S(x))$,}
	\alakohta{$E(x) \lequiv S(x)$.}
	\end{alakohdat}%
\end{tehtava}

\begin{tehtava}
	Olkoon $R(x)$ avoin lause ''$x^2 - 100 > 0$''. Onko lause
	a) $R(10)$ b) $R(-15)$ tosi?%
\end{tehtava}

\begin{tehtava}
	Olkoon $S(x, y)$ avoin lause ''$x^2 + y^2 = 5$''. Onko
	lause a) $S(2, 3)$ b) $S(2, -1)$ c) $S(-\sqrt{5}, 0)$ d)
	$S(-1, -1)$ tosi?%
\end{tehtava}

\begin{tehtava}
	Ratkaise avoin lause $9x^2 + 30x + 25 \le 0$, kun
	määrittelyjoukko on a) reaalilukujen joukko b)
	kokonaislukujen joukko.%
\end{tehtava}

\begin{tehtava}
	Olkoot perusjoukko $\{ 1, 2, 3, \ldots , 16\}$, $C(x)$
	avoin lause ''luku 12 on jaollinen luvulla $x$'' ja $D(x)$ avoin lause ''luvun $x$ neliöjuuri on kokonaisluku''.
	Ratkaise avoin lause.
	\begin{alakohdat}
	\alakohta{$C(x)$,}
	\alakohta{$D(x)$,}
	\alakohta{$\lnot C(x) \land D(x)$,}
	\alakohta{$C(x) \lequiv D(x)$.}
	\end{alakohdat}%
\end{tehtava}

\begin{tehtava}
	Arvotaan kolme lukua, joista jokainen voi olla 1 tai
	2. Arvonnan tuloksena saadaan lukukolmikko $(x, y, z)$, missä $x$ on ensimmäisen arvonnan tulos, $y$ toisen arvonnan tulos ja $z$ kolmannen arvonnan tulos. Olkoot
	avoimet lauseet $S(x, y, z)$: ''$x + y + z \ge 5$'', $T(x,
	y, z)$: ''tulo $xyz$ on pariton'' ja $U(x, y, z)$: “summa
	$x + y + z$ on jaollinen luvulla 3''. Ratkaise avoin
	lause, kun perusjoukon muodostavat arvonnan tuloksena
	saatavat mahdolliset lukukolmikot
	\[
	(1, 1, 1),\ (1, 1, 2),\ (1, 2, 1),\ (1, 2, 2),\ (2, 1, 1),\ (2, 1, 2),\ (2, 2, 1),\ (2, 2, 2).
	\]
	\begin{alakohdat}
	\alakohta{$S(x, y, z)$,}
	\alakohta{$T(x, y, z)$,}
	\alakohta{$U(x, y, z)$,}
	\alakohta{$S(x, y, z) \land T(x, y, z)$,}
	\alakohta{$S(x, y, z) \lor \lnot U(x, y, z)$,}
	\alakohta{$\lnot (S(x, y, z) \lor T(x, y, z) \lor U(x, y, z))$,}
	\alakohta{$S(x, y, z) \land (T(x, y, z) \lequiv U(x, y, z))$.}
	\end{alakohdat}%
\end{tehtava}

\begin{tehtava}
	Ratkaise reaalilukujen joukossa.
	\begin{alakohdat}
	\alakohta{$x^2 = 1$ tai $x(x + 5)(x - 1) = 0$}
	\alakohta{$x^2 = 1$ ja $x(x + 5)(x - 1) = 0$}
	\end{alakohdat}%
\end{tehtava}

\begin{tehtava}
	Ratkaise reaalilukujen joukossa
	\begin{alakohdat}
	\alakohta{epäyhtälö $x^2 - 11x + 30 > 0$,}
	\alakohta{epäyhtälöpari}
	\[
	\left\{
	\begin{array}{rcl}
	x^2 - 11x + 30 & > & 0 \\
	2x & > & -6 - x.
	\end{array}\right.
	\]
	\end{alakohdat}%
\end{tehtava}

\begin{tehtava}
	Ratkaise avoin lause reaalilukujen joukossa. Ilmaise ratkaisujoukko myös välimerkintää käyttäen.
	\begin{alakohdat}
	\alakohta{$(-4 < x) \lor (x < 8)$,}
	\alakohta{$(-4 < x) \land (x < 8)$,}
	\alakohta{$((-4 < x) \land (x < 8)) \land (-5 \le x \le -3)$,}
	\alakohta{$(-4 < x) \lequiv (-5 \le x \le -3)$.}
	\end{alakohdat}%
\end{tehtava}

\begin{tehtava}
	Olkoot $A(x)$ avoin lause $(x \ge 0) \to (x \ge 20)$
	ja $B(x)$ avoin lause $\lnot ((x \le 0) \lor (x \ge 20))$. Ratkaise avoin lause reaalilukujen joukossa.
	\begin{alakohdat}
	\alakohta{$A(x)$,}
	\alakohta{$B(x)$,}
	\alakohta{$A(x) \lor B(x)$.}
	\end{alakohdat}
\end{tehtava}

\end{kotitehtavasivu}
