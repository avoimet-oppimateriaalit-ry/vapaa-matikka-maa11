% Niko

\subsection*{Luvun 7.1 harjoitustehtävät}

\setcounter{tehtava}{0}

\begin{tehtava}
	Määritä lauseen a) $\lnot(B\land A)\lor (A\land\lnot B)$, b) $(B\lor \lnot A)\land (A\lor \lnot B)$ totuusarvo, kun $A$ on tosi ja $B$ on epätosi.
	\begin{alakohdat}
		\alakohta{tosi}
		\alakohta{epätosi}
	\end{alakohdat}
\end{tehtava}

\begin{tehtava}
	Esitä a) kaksoisnegaation laki $\lnot \lnot A \lequiv A$ b) De Morganin laki $\lnot(A\land B)\lequiv \lnot A\lor \lnot B$ loogisen algebran merkinnöin. Laadi totuustaulu laskemalla loogisen algebran laskusäännöillä ja osoita laki päteväksi.
	\begin{alakohdat}
		\alakohta{$-(-A)=A$}
		\alakohta{$-(A\cdot B)=(-A)+(-B)$}
	\end{alakohdat}
\end{tehtava}

\begin{tehtava}
	Esitä a) implikaatio $A\to B$, b) ekvivalenssi $A\lequiv B$ käyttäen loogisen algebran merkintöjä. Laske lauseen totuusarvot atomilauseiden $A$ ja $B$ eri totuusarvoilla käyttäen loogisen algebran laskusääntöjä.
	\begin{alakohdat}
		\alakohta{$(-A)+(A\cdot B)$}
		\alakohta{$((-A)\cdot (-B))+(A\cdot B)$}
	\end{alakohdat}
\end{tehtava}
