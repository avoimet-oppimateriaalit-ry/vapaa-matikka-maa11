%Täänne on kerätty tehtävät muiden kirjojen käyttämässä tehtävä-ympäristössä
%
%ESIMERKKI
%
%
%
%\begin{tehtava}
%    
%   \begin{vastaus}
%   \end{vastaus}
%    
%\end{tehtava}
%
%
%\begin{tehtava}
%    
%    \begin{alakohdat}
%        \alakohta{}
%        \alakohta{}
%        \alakohta{}
%    \end{alakohdat}
%
%    \begin{vastaus}
%        \begin{alakohdat}
%            \alakohta{}
%            \alakohta{}
%            \alakohta{}
%        \end{alakohdat}
%    \end{vastaus}
%    
%\end{tehtava}

%luku logiikka ja päättely


%Harj.Tehtävät Luku 1
\begin{tehtava}
    Onko päättely loogisesti pätevä? Perustele.
    \begin{alakohdat}
        \alakohta{Kaikki ihmiset ovat kuolevaisia. Lasse-kissa on kuolevainen. Siis Lasse-kissa on ihminen.}
        \alakohta{Kaikki koirat osaavat haukkua. Halli on koira. Siis Halli osaa haukkua.}
    \end{alakohdat}

    \begin{vastaus}
        \begin{alakohdat}
            \alakohta{Ei}
            \alakohta{On}
        \end{alakohdat}
    \end{vastaus}
    
\end{tehtava}

\begin{tehtava}
    Ovatko seuraavat päättelyt loogisesti päteviä? Perustele.
    \begin{alakohdat}
        \alakohta{\kolmepaattely{Kaikki tetraedrit ovat pyramideja.}{Jotkut kartiot ovat tetraedrejä.}
                {Jotkut kartiot ovat pyramideja.}}
        \alakohta{\kolmepaattely{Kaikki sylinterit ovat lieriöitä.}{Mikään lieriö ei ole kartio.}
                {Jotkut sylinterit ovat kartioita.}}
        \alakohta{\kolmepaattely{Luku $345$ päättyy numeroon $5$.}{Nollaan päättyvä luku on viidellä jaollinen.}
                {Luku $345$ on viidellä jaollinen.}}	        
    \end{alakohdat}

    \begin{vastaus}
        \begin{alakohdat}
            \alakohta{On}
            \alakohta{Ei}
            \alakohta{Ei}
        \end{alakohdat}
    \end{vastaus}
    
\end{tehtava}

\begin{tehtava}
    Onko seuraava päättely loogisesti pätevä? Perustele.
    \begin{alakohdat}
        \alakohta{Jos ulkona on pakkanen, menen hiihtämään. Ulkona ei ole pakkanen. Siis en mene hiihtämään.}
        \alakohta{Jos ulkona on pakkanen, menen hiihtämään. En mene hiihtämään. Siis ulkona ei ole pakkanen.}
        \alakohta{Jos tiedän nukkuvani, niin nukun. Jos tiedän nukkuvani, niin en nuku. Siis en tiedä nukkuvani.}
    \end{alakohdat}

    \begin{vastaus}
        \begin{alakohdat}
            \alakohta{On}
            \alakohta{Ei}
            \alakohta{Ei, ???????????????} %En ole täysin varma vastauksesta
        \end{alakohdat}
    \end{vastaus}
    
\end{tehtava}

\begin{tehtava}
    Tutkitaan polynomia
        \[ P(x) = x^5 -10x^4+35x^3 -50 x^2 +25x. \]
    \begin{alakohdat}
        \alakohta{Laske $P(0)$, $P(1)$, $P(2)$, $P(3)$ ja $P(4)$.}
        \alakohta{Mitä voit sanoa luvuista $P(n)$, kun $n$ on luonnollinen luku?}
        \alakohta{Testaa päätelmääsi kokeilemalla myös muilla luonnollisilla luvuilla esimerkiksi laskinta käyttäen.}
    \end{alakohdat}

    \begin{vastaus}
        \begin{alakohdat}
            \alakohta{$P(0)=0$, $P(1)=1$, $P(2)=2$, $P(3)=3$, $P(4)=4$}
            \alakohta{$P(n)=n$}
            \alakohta{$P(5)=125$, päätelmä ei päde.}
        \end{alakohdat}
    \end{vastaus}
    
\end{tehtava}

\begin{tehtava}
    Arkiajattelussa käytetään usein ajattelumalleja, jotka eivät ole loogisesti perusteltavissa.
        Mikä virhe on seuraavissa päätelmissä?
    \begin{alakohdat}
        \alakohta{Ilta-Sanomien kyselyssä $66~\%$ vastaajista uskoo maan ulkopuoliseen elämään.
                Maan ulkopuolista elämää on olemassa.}
        \alakohta{The Sunday Times -lehden haastattelussa kuuluisa tiedemies Stephen Hawking
                totesi pitävänsä lähes varmana, että avaruudessa on maan ulkopuolista älykästä elämää.
                Maan ulkopuolista elämää on olemassa.}
        \alakohta{Tiedemiehistä 90~\% väittää, että nykyinen ilmastonmuutos on ihmisen aiheuttamaa
                eikä johdu maapallon lämpötilan luontaisesta jaksollisuudesta.
                Siis nykyinen ilmastonmuutos on ihmisen aiheuttamaa.}
        \alakohta{Televisiouutisissa kerrottiin, että toisen maailmansodan aikainen holokausti oli
                vain liittoutuneiden propagandaa.
                Siis holokaustia ei tapahtunut toisen maailmansodan aikana.}
    \end{alakohdat}

    \begin{vastaus}
        \begin{alakohdat}
            \alakohta{Johtopäätös vedetty vastaajien uskomuksesta.}
            \alakohta{Johtopäätös vedetty Stephen Hawkingin uskomuksesta.}
            \alakohta{Johtopäätös perustuu tiedemiehien väitteisiin, joista ei ole varmaa totuutta.}
            \alakohta{Johtopäätös perustuu televisiouutisen väitteeseen, josta ei ole selviä todisteita.}
        \end{alakohdat}
    \end{vastaus}
    
\end{tehtava}

\begin{tehtava}
    Ovatko seuraavat päättelyt loogisesti päteviä? Perustele.
    \begin{alakohdat}
        \alakohta{\kolmepaattely{Kaikilla $x$ toteutuu $y$.}{Joillakin $z$ toteutuu $x$.}{Joillakin $z$ toteutuu $y$.}}
        \alakohta{\kolmepaattely{Kaikilla $A$ toteutuu $B$.}{$C$ toteuttaa $B$:n.}{$C$ toteuttaa $A$:n.}}
        \alakohta{\kolmepaattely{Kaikilla $A$ toteutuu $B$.}{Millään $C$ ei toteudu $B$.}{Millään $C$ ei toteudu $A$.}}
    \end{alakohdat}

    \begin{vastaus}
        \begin{alakohdat}
            \alakohta{On}
            \alakohta{Ei. Jos A toteuttaa B:n, B ei välttämättä toteuta A:ta.}
            \alakohta{Ei, ?????????????} %En ole täysin varma vastauksesta
        \end{alakohdat}
    \end{vastaus}
    
\end{tehtava}
%Kotitehtävät Luku 1
\begin{tehtava}
    Ovatko seuraavat päättelyt päteviä? Perustele.
    \begin{alakohdat}
        \alakohta{\kolmepaattely{Kaikki kissat osaavat kehrätä.}{Tämä eläin osaa kehrätä.}{Tämä eläin on kissa.}}
        \alakohta{\kolmepaattely{Kukaan laiska opiskelija ei selviä kokeesta.}
                {On opiskelijoita, jotka selviävät kokeesta.}{On opiskelijoita, jotka eivät ole laiskoja.}}
    \end{alakohdat}

    \begin{vastaus}
        \begin{alakohdat}
            \alakohta{Ei}
            \alakohta{On}
        \end{alakohdat}
    \end{vastaus}
    
\end{tehtava}

\begin{tehtava}
    Ovatko seuraavat päättelyt päteviä? Perustele.
    \begin{alakohdat}
        \alakohta{\kolmepaattely{Neljäkkään lävistäjät ovat kohtisuorassa toisiaan vastaan.}
                {Neljäkäs on suunnikas.}{Suunnikkaan lävistäjät ovat kohtisuorassa toisiaan vastaan.}}
        \alakohta{\kolmepaattely{Kaikki suorakulmiot ovat suunnikkaita.}{Jotkut nelikulmiot ovat suorakulmioita.}
                {Jotkut nelikulmiot ovat suunnikkaita.}}
        \alakohta{\kolmepaattely{Kolmio $ABC$ on tasakylkinen.}{Tasasivuiset kolmiot ovat tasakylkisiä.}
                {Kolmio $ABC$ on tasasivuinen.}}
    \end{alakohdat}

    \begin{vastaus}
        \begin{alakohdat}
            \alakohta{Ei}
            \alakohta{On}
            \alakohta{Ei}
        \end{alakohdat}
    \end{vastaus}
    
\end{tehtava}

\begin{tehtava}
    Ovatko seuraavat päättelyt päteviä? Perustele.
    \begin{alakohdat}
        \alakohta{\kaksipaattely{Kaikki lammasfarmin lampaat ovat joko mustia tai valkoisia.}
                {Kaikki lampaat ovat mustia tai valkoisia.}}
        \alakohta{\neljapaattely{Tavallisessa korttipakassa kortti on aina joko pata,\\& risti, hertta tai ruutu.}
                {Pata- ja risti-kortit ovat mustia.}{Hertta- ja ruutu-kortit ovat punaisia.}
                {Kaikki tavallisten korttipakkojen kortit ovat\\ & joko mustia tai punaisia.}}
    \end{alakohdat}

    \begin{vastaus}
        \begin{alakohdat}
            \alakohta{Ei}
            \alakohta{On}
        \end{alakohdat}
    \end{vastaus}
    
\end{tehtava}

\begin{tehtava}
    Jäämaa on kokonaan Merimaan itäpuolella.
        Kummallakin on etelärajaa Aurinkomaan kanssa.
        Merimaalla ja Aurinkomaalla on länsiraja Lumimaan kanssa.
        Kukkamaa on kokonaan Jäämaan ja Aurinkomaan itäpuolella.
    \begin{alakohdat}
        \alakohta{Onko Merimaalla ja Kukkamaalla yhteistä rajaa?}
        \alakohta{Voiko Lumimaalla ja Kukkamaalla olla yhteistä rajaa?}
    \end{alakohdat}

    \begin{vastaus}
        \begin{alakohdat}
            \alakohta{Ei}
            \alakohta{Ei, ?????????????} %En ole täysin varma vastauksesta
        \end{alakohdat}
    \end{vastaus}
    
\end{tehtava}

\begin{tehtava}
    Ovatko seuraavat päättelyt päteviä? Perustele.
    \begin{alakohdat}
        \alakohta{\kolmepaattely{Millään $x$ ei toteudu $y$.}{Kaikilla $z$ toteutuu $x$.}{Millään $z$ ei toteudu $y$.}}
        \alakohta{\kolmepaattely{Kaikilla $A$ toteutuu $B$.}{Joillakin $C$ toteutuu $B$.}{Joillakin $A$ toteutuu $C$.}}
    \end{alakohdat}

    \begin{vastaus}
        \begin{alakohdat}
            \alakohta{On}
            \alakohta{Ei}
        \end{alakohdat}
    \end{vastaus}
    
\end{tehtava}
%Harj. Tehtävät Luku 2
\begin{tehtava}
    
    \begin{alakohdat}
        \alakohta{}
        \alakohta{}
        \alakohta{}
    \end{alakohdat}

    \begin{vastaus}
        \begin{alakohdat}
            \alakohta{}
            \alakohta{}
            \alakohta{}
        \end{alakohdat}
    \end{vastaus}
    
\end{tehtava}

\begin{tehtava}
    
    \begin{alakohdat}
        \alakohta{}
        \alakohta{}
        \alakohta{}
    \end{alakohdat}

    \begin{vastaus}
        \begin{alakohdat}
            \alakohta{}
            \alakohta{}
            \alakohta{}
        \end{alakohdat}
    \end{vastaus}
    
\end{tehtava}

\begin{tehtava}
    
    \begin{alakohdat}
        \alakohta{}
        \alakohta{}
        \alakohta{}
    \end{alakohdat}

    \begin{vastaus}
        \begin{alakohdat}
            \alakohta{}
            \alakohta{}
            \alakohta{}
        \end{alakohdat}
    \end{vastaus}
    
\end{tehtava}

\begin{tehtava}
    
    \begin{alakohdat}
        \alakohta{}
        \alakohta{}
        \alakohta{}
    \end{alakohdat}

    \begin{vastaus}
        \begin{alakohdat}
            \alakohta{}
            \alakohta{}
            \alakohta{}
        \end{alakohdat}
    \end{vastaus}
    
\end{tehtava}

\begin{tehtava}
    
    \begin{alakohdat}
        \alakohta{}
        \alakohta{}
        \alakohta{}
    \end{alakohdat}

    \begin{vastaus}
        \begin{alakohdat}
            \alakohta{}
            \alakohta{}
            \alakohta{}
        \end{alakohdat}
    \end{vastaus}
    
\end{tehtava}

\begin{tehtava}
    
    \begin{alakohdat}
        \alakohta{}
        \alakohta{}
        \alakohta{}
    \end{alakohdat}

    \begin{vastaus}
        \begin{alakohdat}
            \alakohta{}
            \alakohta{}
            \alakohta{}
        \end{alakohdat}
    \end{vastaus}
    
\end{tehtava}

\begin{tehtava}
    
    \begin{alakohdat}
        \alakohta{}
        \alakohta{}
        \alakohta{}
    \end{alakohdat}

    \begin{vastaus}
        \begin{alakohdat}
            \alakohta{}
            \alakohta{}
            \alakohta{}
        \end{alakohdat}
    \end{vastaus}
    
\end{tehtava}

\begin{tehtava}
    
    \begin{alakohdat}
        \alakohta{}
        \alakohta{}
        \alakohta{}
    \end{alakohdat}

    \begin{vastaus}
        \begin{alakohdat}
            \alakohta{}
            \alakohta{}
            \alakohta{}
        \end{alakohdat}
    \end{vastaus}
    
\end{tehtava}

\begin{tehtava}
    
    \begin{alakohdat}
        \alakohta{}
        \alakohta{}
        \alakohta{}
    \end{alakohdat}

    \begin{vastaus}
        \begin{alakohdat}
            \alakohta{}
            \alakohta{}
            \alakohta{}
        \end{alakohdat}
    \end{vastaus}
    
\end{tehtava}

\begin{tehtava}
    
    \begin{alakohdat}
        \alakohta{}
        \alakohta{}
        \alakohta{}
    \end{alakohdat}

    \begin{vastaus}
        \begin{alakohdat}
            \alakohta{}
            \alakohta{}
            \alakohta{}
        \end{alakohdat}
    \end{vastaus}
    
\end{tehtava}

\begin{tehtava}
    
    \begin{alakohdat}
        \alakohta{}
        \alakohta{}
        \alakohta{}
    \end{alakohdat}

    \begin{vastaus}
        \begin{alakohdat}
            \alakohta{}
            \alakohta{}
            \alakohta{}
        \end{alakohdat}
    \end{vastaus}
    
\end{tehtava}

\begin{tehtava}
    
    \begin{alakohdat}
        \alakohta{}
        \alakohta{}
        \alakohta{}
    \end{alakohdat}

    \begin{vastaus}
        \begin{alakohdat}
            \alakohta{}
            \alakohta{}
            \alakohta{}
        \end{alakohdat}
    \end{vastaus}
    
\end{tehtava}

\begin{tehtava}
    
    \begin{alakohdat}
        \alakohta{}
        \alakohta{}
        \alakohta{}
    \end{alakohdat}

    \begin{vastaus}
        \begin{alakohdat}
            \alakohta{}
            \alakohta{}
            \alakohta{}
        \end{alakohdat}
    \end{vastaus}
    
\end{tehtava}

\begin{tehtava}
    
    \begin{alakohdat}
        \alakohta{}
        \alakohta{}
        \alakohta{}
    \end{alakohdat}

    \begin{vastaus}
        \begin{alakohdat}
            \alakohta{}
            \alakohta{}
            \alakohta{}
        \end{alakohdat}
    \end{vastaus}
    
\end{tehtava}

\begin{tehtava}
    
    \begin{alakohdat}
        \alakohta{}
        \alakohta{}
        \alakohta{}
    \end{alakohdat}

    \begin{vastaus}
        \begin{alakohdat}
            \alakohta{}
            \alakohta{}
            \alakohta{}
        \end{alakohdat}
    \end{vastaus}
    
\end{tehtava}

\begin{tehtava}
    
    \begin{alakohdat}
        \alakohta{}
        \alakohta{}
        \alakohta{}
    \end{alakohdat}

    \begin{vastaus}
        \begin{alakohdat}
            \alakohta{}
            \alakohta{}
            \alakohta{}
        \end{alakohdat}
    \end{vastaus}
    
\end{tehtava}

%Harj.Tehtävät Luku 6

% \section{Suurin yhteinen tekijä ja Eukleideen algoritmi}
\begin{tehtava}
    Määritä lukujen suurin yhteinen tekijä.
    
    \begin{alakohdat}
        \alakohta{$15$ ja $20$}
        \alakohta{$9$ ja $36$}
        \alakohta{$4$ ja $7$}
    \end{alakohdat}

    \begin{vastaus}
        \begin{alakohdat}
            \alakohta{$5$}
            \alakohta{$9$}
            \alakohta{$1$}
        \end{alakohdat}
    \end{vastaus}
    
\end{tehtava}

\begin{tehtava}
    Määritä Eukleideen algoritmia käyttäen
    
    \begin{alakohdat}
        \alakohta{$\syt(184, 152)$}
        \alakohta{$\syt(227, 143)$.}
    \end{alakohdat}

    \begin{vastaus}
        \begin{alakohdat}
            \alakohta{$8$}
            \alakohta{$1$}
        \end{alakohdat}
    \end{vastaus}
    
\end{tehtava}

\begin{tehtava}
    Määritä Eukleideen algoritmia käyttäen
    
    \begin{alakohdat}
        \alakohta{$\syt(272, 1479)$}
        \alakohta{$\syt(4719, 18207)$.}
    \end{alakohdat}

    \begin{vastaus}
        \begin{alakohdat}
            \alakohta{$17$}
            \alakohta{$3$}
        \end{alakohdat}
    \end{vastaus}
    
\end{tehtava}

\begin{tehtava}
    Esitä murtoluku
    \begin{alakohdat}
        \alakohta{$\frac{143}{605}$}
        \alakohta{$\frac{5989}{30899}$}
    \end{alakohdat}
    supistetussa muodossa. Vihje: Määritä osoittajan ja nimittäjän suurin yhteinen tekijä.

    \begin{vastaus}
        \begin{alakohdat}
            \alakohta{$\frac{13}{55}$}
            \alakohta{$\frac{113}{583}$}
        \end{alakohdat}
    \end{vastaus}
    
\end{tehtava}

\begin{tehtava}
    % Tarkistettu (Topi Talvitie, 9.11.2013)
    Osoita, että murtoluku $\frac{8788}{13475}$ ei supistu.
\end{tehtava}

\begin{tehtava}
    Leirille osallistui 780 tyttöä ja 612 poikaa. Osallistujat jaettiin keskenään yhtä suuriin ryhmiin siten, että kussakin ryhmässä oli vain tyttöjä tai poikia. Mikä oli suurin mahdollinen ryhmäkoko?
    
    \begin{vastaus}
        12
    \end{vastaus}
    
\end{tehtava}

\begin{tehtava}
    Määritä lukujen $188000100$ ja $188$ suurin yhteinen tekijä.

    \begin{vastaus}
        $4$
    \end{vastaus}
    
\end{tehtava}

\begin{tehtava}
    Olkoon $a$ positiivinen kokonaisluku. Määritä
    \begin{alakohdat}
        \alakohta{$\syt(a, a)$}
        \alakohta{$\syt(a, 1)$}
        \alakohta{$\syt(a^2, a)$}
        \alakohta{$\syt((a+1)!, a!)$.}
    \end{alakohdat}
    Merkintä $a!$ tarkoittaa luvun $a$ \termi{kertoma}{kertomaa}. Se on tulo $a! = a \cdot (a-1) \cdot (a-2) \cdot \ldots \cdot 3 \cdot 2 \cdot 1$.

    \begin{vastaus}
        \begin{alakohdat}
            \alakohta{$a$}
            \alakohta{$1$}
            \alakohta{$a$}
            \alakohta{$a!$}
        \end{alakohdat}
    \end{vastaus}
    
\end{tehtava}

\begin{tehtava}
    % Tarkistettu (Topi Talvitie, 9.11.2013)
    Olkoon $n$ positiivinen kokonaisluku. Osoita Eukleideen algoritmia käyttäen, että $\syt(n+1, n)=1$.
\end{tehtava}

\begin{tehtava}
    Olkoon $n$ positiivinen kokonaisluku. Määritä lukujen $n^2 + 2n$ ja $n + 1$ suurin yhteinen tekijä.
    
    \begin{vastaus}
        1
    \end{vastaus}
    
\end{tehtava}

\begin{tehtava}
    % Tarkistettu (Topi Talvitie, 9.11.2013)
    Olkoon $n$ positiivinen kokonaisluku. Osoita, että
    \[\syt(3^{n+1} + 10, 3^n + 2)=1.\]
\end{tehtava}

% \subsection*{Kotitehtäviä}
\begin{tehtava}
    Määritä lukujen suurin yhteinen tekijä.
    
    \begin{alakohdat}
        \alakohta{$63$ ja $7$}
        \alakohta{$64$ ja $33$}
        \alakohta{$45$ ja $60$}
    \end{alakohdat}

    \begin{vastaus}
        \begin{alakohdat}
            \alakohta{$7$}
            \alakohta{$1$}
            \alakohta{$15$}
        \end{alakohdat}
    \end{vastaus}
    
\end{tehtava}

\begin{tehtava}
    Määritä Eukleideen algoritmia käyttäen
    
    \begin{alakohdat}
        \alakohta{$\syt(657, 306)$}
        \alakohta{$\syt(2197, 4641)$}
        \alakohta{$\syt(15787, 4111)$.}
    \end{alakohdat}

    \begin{vastaus}
        \begin{alakohdat}
            \alakohta{$9$}
            \alakohta{$13$}
            \alakohta{$1$}
        \end{alakohdat}
    \end{vastaus}
    
\end{tehtava}

\begin{tehtava}
    Esitä murtoluku
    \begin{alakohdat}
        \alakohta{$\frac{182}{299}$}
        \alakohta{$\frac{7697}{32041}$}
    \end{alakohdat}
    supistetussa muodossa.

    \begin{vastaus}
        \begin{alakohdat}
            \alakohta{$\frac{14}{23}$}
            \alakohta{$\frac{43}{179}$}
        \end{alakohdat}
    \end{vastaus}
    
\end{tehtava}

\begin{tehtava}
    Leipomo suljettiin remontin ajaksi. Ennen sulkemista laskettiin, että leipomon varastossa oli 4896 vehnäsämpylää ja 1408 grahamsämpylää. Sämpylät pakattiin kuljetusta varten keskenään samankokoisiin pusseihin siten, että kuhunkin pussiin tuli vain vehnä- tai grahamsämpylöitä. Mikä oli suurin mahdollinen pussikoko? Oletetaan, että vehnäsämpylä oli samankokoinen kuin grahamsämpylä ja että yksikään pussi ei jäänyt vajaaksi.

    \begin{vastaus}
        $32$
    \end{vastaus}
    
\end{tehtava}

\begin{tehtava}
    Määritä lukujen $468468468108$ ja $234$ suurin yhteinen tekijä.
    
    \begin{vastaus}
        $18$
    \end{vastaus}
    
\end{tehtava}

\begin{tehtava}
    Olkoot $a$, $b$ ja $c$ positiivisia kokonaislukuja. Lukujen $a$, $b$ ja $c$ suurin yhteinen tekijä eli $\syt(a, b, c)$ voidaan määrittää siten, että ensin määritetään kahden luvun suurin yhteinen tekijä ja sitten tämän ja kolmannen luvun suurin yhteinen tekijä. Määritä
    
    \begin{alakohdat}
        \alakohta{$\syt(15, 30, 40)$}
        \alakohta{$\syt(6, 9, 11)$}
        \alakohta{$\syt(171, 456, 665)$.}
    \end{alakohdat}

    \begin{vastaus}
        \begin{alakohdat}
            \alakohta{$5$}
            \alakohta{$1$}
            \alakohta{$19$}
        \end{alakohdat}
    \end{vastaus}
    
\end{tehtava}

\begin{tehtava}
    Olkoot $a$ ja $b$ positiivisia kokonaislukuja ja $\syt(a, b)=9$. Voiko tällöin yhtälö $a + b = 186$ olla tosi?
    
    \begin{vastaus}
        Ei voi.
    \end{vastaus}
    
\end{tehtava}

\begin{tehtava}
    Olkoon $n$ positiivinen kokonaisluku. Tutki, mitä arvoja $\syt(n+4, n)$ voi saada.
    
    \begin{vastaus}
        $1$, $2$ ja $4$.
    \end{vastaus}
    
\end{tehtava}

\begin{tehtava}
    Olkoon $n$ positiivinen kokonaisluku. Määritä lukujen $n^2 + 3n$ ja $n + 2$ suurin yhteinen tekijä.
    
    \begin{vastaus}
        Jos $n$ on parillinen, $\syt(n^2 + 3n, n + 2) = 2$, muuten $\syt(n^2 + 3n, n + 2) = 1$.
    \end{vastaus}
    
\end{tehtava}

\begin{tehtava}
    % Tarkistettu (Topi Talvitie, 9.11.2013)
    Olkoot $a$ ja $b$ positiivisia kokonaislukuja. Osoita, että $\syt(a, b)$ on jaollinen kaikilla lukujen $a$ ja $b$ yhteisillä tekijöillä.
\end{tehtava}

% \section{Diofantoksen yhtälöt}
\begin{tehtava}
    Tutki, onko Diofantoksen yhtälöllä ratkaisua.
    
    \begin{alakohdat}
        \alakohta{$7x + 5y = 3$}
        \alakohta{$5x + 85y = 42$}
        \alakohta{$6x + 51y = 100$}
    \end{alakohdat}

    \begin{vastaus}
        \begin{alakohdat}
            \alakohta{On ratkaisu.}
            \alakohta{Ei ole ratkaisua.}
            \alakohta{Ei ole ratkaisua.}
        \end{alakohdat}
    \end{vastaus}
    
\end{tehtava}

\begin{tehtava}
    Leirille osallistui $364$ nuorta. Oliko mahdollista majoittaa osallistujat $24$ ja $16$ hengen parakkeihin siten, että yksikään parakki ei jäänyt vajaaksi?
    
    \begin{vastaus}
        Ei ole mahdollista.
    \end{vastaus}
    
\end{tehtava}

\begin{tehtava}
    Määritä Diofantoksen yhtälön jokin ratkaisu.
    
    \begin{alakohdat}
        \alakohta{$14x + 49y = \syt(14, 49)$}
        \alakohta{$56x + 72y = \syt(56, 72)$}
    \end{alakohdat}

    \begin{vastaus}
        \begin{alakohdat}
            \alakohta{Esimerkiksi $x = -3$ ja $y = 1$.}
            \alakohta{Esimerkiksi $x = -5$ ja $y = 4$.}
        \end{alakohdat}
    \end{vastaus}
    
\end{tehtava}

\begin{tehtava}
    Määritä Diofantoksen yhtälön jokin ratkaisu.
    
    \begin{alakohdat}
        \alakohta{$56x + 72y = 40$}
        \alakohta{$24x + 138y = -24$}
    \end{alakohdat}

    \begin{vastaus}
        \begin{alakohdat}
            \alakohta{Esimerkiksi $x = -7$ ja $y = 6$.}
            \alakohta{Esimerkiksi $x = -1$ ja $y = 0$.}
        \end{alakohdat}
    \end{vastaus}
    
\end{tehtava}

\begin{tehtava}
    Tutki, onko suoralla
    \begin{alakohdat}
        \alakohta{$26x + 91y + 10 = 0$}
        \alakohta{$529x + 621y - 92 = 0$}
    \end{alakohdat}
    pisteitä, joiden molemmat koordinaatit ovat kokonaislukuja.

    \begin{vastaus}
        \begin{alakohdat}
            \alakohta{Ei ole.}
            \alakohta{On, esimerkiksi piste $(-1, 1)$.}
        \end{alakohdat}
    \end{vastaus}
    
\end{tehtava}

\begin{tehtava}
    Määritä Diofantoksen yhtälön kaikki ratkaisut.
    
    \begin{alakohdat}
        \alakohta{$2x + 3y = 1$}
        \alakohta{$2x + 3y = 7$}
    \end{alakohdat}

    \begin{vastaus}
        \begin{alakohdat}
            \alakohta{$x = -1 + 3n$ ja $y = 1 - 2n$, $n\in\zz$.}
            \alakohta{$x = 2 + 3n$ ja $y = 1 - 2n$, $n\in\zz$.}
        \end{alakohdat}
    \end{vastaus}
    
\end{tehtava}

\begin{tehtava}
    Määritä Diofantoksen yhtälön $45x + 21y = -6$ kaikki ratkaisut.

    \begin{vastaus}
        $x = -2 + 7n$ ja $y = 4 - 15n$, $n\in\zz$.
    \end{vastaus}
    
\end{tehtava}

\begin{tehtava}
    Määritä Diofantoksen yhtälön $13509x + 10203y = 228$ kaikki ratkaisut.
    
    \begin{vastaus}
        $x = 105 + 179n$ ja $y = -139 - 237n$, $n\in\zz$.
    \end{vastaus}
    
\end{tehtava}

\begin{tehtava}
    Keksi Diofantoksen yhtälö, jolla
    \begin{alakohdat}
        \alakohta{ei ole ratkaisua}
        \alakohta{on äärettömän monta ratkaisua.}
    \end{alakohdat}

    \begin{vastaus}
        \begin{alakohdat}
            \alakohta{Esimerkiksi $6x + 8y = 1$}
            \alakohta{Esimerkiksi $13x + 24y = 47$}
        \end{alakohdat}
    \end{vastaus}
    
\end{tehtava}

\begin{tehtava}
    Määritä Diofantoksen yhtälön $63x + 279y = 450$ kaikki ratkaisut. Mitkä niistä toteuttavat ehdon $|x| + |y| < 25$?
    
    \begin{vastaus}
        Ratkaisut ovat $x = 16 + 31n$ ja $y = -2 - 7n$, $n\in\zz$. Näistä ehdon $|x| + |y| < 25$ toteuttaa ratkaisu $x = -15$ ja $y = 5$ sekä ratkaisut $x = 16$ ja $y = -2$.
    \end{vastaus}
    
\end{tehtava}

\begin{tehtava}
    Käytettävissä on 8 gramman ja 12 gramman punnuksia. Kuinka monta kummankinlaista punnusta tarvitaan, jotta punnusten kokonaismassaksi tulisi 100 grammaa? Selvitä kaikki vaihtoehdot.
    
    \begin{vastaus}
        Vaihtoehdot kun 8 gramman punnuksien määrä on $x$ ja 12 gramman punnuksien määrä on $y$:
        \begin{itemize}
            \item $x = 2$ ja $y = 7$
            \item $x = 5$ ja $y = 5$
            \item $x = 8$ ja $y = 3$
            \item $x = 11$ ja $y = 1$.
        \end{itemize}
    \end{vastaus}
    
\end{tehtava}

\begin{tehtava}
    Ruhtinas jakoi $63$ yhtä suurta kekoa hedelmiä sekä $7$ erillistä hedelmää tasan $23$ matkalaiselle. Kuinka monta hedelmää kussakin keossa oli? Vihje: Tutki yhtälöä $63x + 7 = 23y$. (Mahavira, v. 850)
    
    \begin{vastaus}
        Mikä tahansa hedelmien määrä muotoa $5 + 23n$, missä $n\in\zz$ ja $n\geq 0$, on mahdollinen.
    \end{vastaus}
    
\end{tehtava}

\begin{tehtava}
    % Tarkistettu (Topi Talvitie, 9.11.2013)
    Olkoot $a$, $b$, $c$ ja $d$ positiivisia kokonaislukuja. Osoita, että yhtälöllä $ax+by+cz=d$ on kokonaislukuratkaisu, jos ja vain jos luku $d$ on jaollinen lukujen $a, b$ ja $c$ suurimmalla yhteisellä tekijällä.
\end{tehtava}


% \subsection*{Kotitehtäviä}

\begin{tehtava}
    Tutki, onko Diofantoksen yhtälöllä ratkaisua.
    
    \begin{alakohdat}
        \alakohta{$9x + 6y = 72$}
        \alakohta{$12x + 10y = 323$}
        \alakohta{$14x + 35y = -91$}
    \end{alakohdat}

    \begin{vastaus}
        \begin{alakohdat}
            \alakohta{On ratkaisu.}
            \alakohta{Ei ole ratkaisua.}
            \alakohta{On ratkaisu.}
        \end{alakohdat}
    \end{vastaus}
    
\end{tehtava}
