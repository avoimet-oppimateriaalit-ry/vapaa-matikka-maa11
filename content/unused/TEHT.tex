%Täänne on kerätty tehtävät muiden kirjojen käyttämässä tehtävä-ympäristössä
%
%ESIMERKKI
%
%
%
%\begin{tehtava}
%    
%   \begin{vastaus}
%   \end{vastaus}
%    
%\end{tehtava}
%
%
%\begin{tehtava}
%    
%    \begin{alakohdat}
%        \alakohta{}
%        \alakohta{}
%        \alakohta{}
%    \end{alakohdat}
%
%    \begin{vastaus}
%        \begin{alakohdat}
%            \alakohta{}
%            \alakohta{}
%            \alakohta{}
%        \end{alakohdat}
%    \end{vastaus}
%    
%\end{tehtava}

%luku logiikka ja päättely

\begin{tehtava}
    Onko päättely loogisesti pätevä? Perustele.
    \begin{alakohdat}
        \alakohta{Kaikki ihmiset ovat kuolevaisia. Lasse-kissa on kuolevainen. Siis Lasse-kissa on ihminen.}
        \alakohta{Kaikki koirat osaavat haukkua. Halli on koira. Siis Halli osaa haukkua.}
    \end{alakohdat}

    \begin{vastaus}
        \begin{alakohdat}
            \alakohta{Ei}
            \alakohta{On}
        \end{alakohdat}
    \end{vastaus}
    
\end{tehtava}
