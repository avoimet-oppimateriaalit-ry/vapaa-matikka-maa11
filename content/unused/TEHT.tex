%Täänne on kerätty tehtävät muiden kirjojen käyttämässä tehtävä-ympäristössä
%
%ESIMERKKI
%
%
%
%\begin{tehtava}
%    
%   \begin{vastaus}
%   \end{vastaus}
%    
%\end{tehtava}
%
%
%\begin{tehtava}
%    
%    \begin{alakohdat}
%        \alakohta{}
%        \alakohta{}
%        \alakohta{}
%    \end{alakohdat}
%
%    \begin{vastaus}
%        \begin{alakohdat}
%            \alakohta{}
%            \alakohta{}
%            \alakohta{}
%        \end{alakohdat}
%    \end{vastaus}
%    
%\end{tehtava}

%luku logiikka ja päättely


%Harj.Tehtävät Luku 1
\begin{tehtava}
    Onko päättely loogisesti pätevä? Perustele.
    \begin{alakohdat}
        \alakohta{Kaikki ihmiset ovat kuolevaisia. Lasse-kissa on kuolevainen. Siis Lasse-kissa on ihminen.}
        \alakohta{Kaikki koirat osaavat haukkua. Halli on koira. Siis Halli osaa haukkua.}
    \end{alakohdat}

    \begin{vastaus}
        \begin{alakohdat}
            \alakohta{Ei}
            \alakohta{On}
        \end{alakohdat}
    \end{vastaus}
    
\end{tehtava}

\begin{tehtava}
    Ovatko seuraavat päättelyt loogisesti päteviä? Perustele.
    \begin{alakohdat}
        \alakohta{\kolmepaattely{Kaikki tetraedrit ovat pyramideja.}{Jotkut kartiot ovat tetraedrejä.}
                {Jotkut kartiot ovat pyramideja.}}
        \alakohta{\kolmepaattely{Kaikki sylinterit ovat lieriöitä.}{Mikään lieriö ei ole kartio.}
                {Jotkut sylinterit ovat kartioita.}}
        \alakohta{\kolmepaattely{Luku $345$ päättyy numeroon $5$.}{Nollaan päättyvä luku on viidellä jaollinen.}
                {Luku $345$ on viidellä jaollinen.}}	        
    \end{alakohdat}

    \begin{vastaus}
        \begin{alakohdat}
            \alakohta{On}
            \alakohta{Ei}
            \alakohta{Ei}
        \end{alakohdat}
    \end{vastaus}
    
\end{tehtava}

\begin{tehtava}
    Onko seuraava päättely loogisesti pätevä? Perustele.
    \begin{alakohdat}
        \alakohta{Jos ulkona on pakkanen, menen hiihtämään. Ulkona ei ole pakkanen. Siis en mene hiihtämään.}
        \alakohta{Jos ulkona on pakkanen, menen hiihtämään. En mene hiihtämään. Siis ulkona ei ole pakkanen.}
        \alakohta{Jos tiedän nukkuvani, niin nukun. Jos tiedän nukkuvani, niin en nuku. Siis en tiedä nukkuvani.}
    \end{alakohdat}

    \begin{vastaus}
        \begin{alakohdat}
            \alakohta{On}
            \alakohta{Ei}
            \alakohta{Ei, ???????????????} %En ole täysin varma vastauksesta
        \end{alakohdat}
    \end{vastaus}
    
\end{tehtava}

\begin{tehtava}
    Tutkitaan polynomia
        \[ P(x) = x^5 -10x^4+35x^3 -50 x^2 +25x. \]
    \begin{alakohdat}
        \alakohta{Laske $P(0)$, $P(1)$, $P(2)$, $P(3)$ ja $P(4)$.}
        \alakohta{Mitä voit sanoa luvuista $P(n)$, kun $n$ on luonnollinen luku?}
        \alakohta{Testaa päätelmääsi kokeilemalla myös muilla luonnollisilla luvuilla esimerkiksi laskinta käyttäen.}
    \end{alakohdat}

    \begin{vastaus}
        \begin{alakohdat}
            \alakohta{$P(0)=0$, $P(1)=1$, $P(2)=2$, $P(3)=3$, $P(4)=4$}
            \alakohta{$P(n)=n$}
            \alakohta{$P(5)=125$, päätelmä ei päde.}
        \end{alakohdat}
    \end{vastaus}
    
\end{tehtava}

\begin{tehtava}
    Arkiajattelussa käytetään usein ajattelumalleja, jotka eivät ole loogisesti perusteltavissa.
        Mikä virhe on seuraavissa päätelmissä?
    \begin{alakohdat}
        \alakohta{Ilta-Sanomien kyselyssä $66~\%$ vastaajista uskoo maan ulkopuoliseen elämään.
                Maan ulkopuolista elämää on olemassa.}
        \alakohta{The Sunday Times -lehden haastattelussa kuuluisa tiedemies Stephen Hawking
                totesi pitävänsä lähes varmana, että avaruudessa on maan ulkopuolista älykästä elämää.
                Maan ulkopuolista elämää on olemassa.}
        \alakohta{Tiedemiehistä 90~\% väittää, että nykyinen ilmastonmuutos on ihmisen aiheuttamaa
                eikä johdu maapallon lämpötilan luontaisesta jaksollisuudesta.
                Siis nykyinen ilmastonmuutos on ihmisen aiheuttamaa.}
        \alakohta{Televisiouutisissa kerrottiin, että toisen maailmansodan aikainen holokausti oli
                vain liittoutuneiden propagandaa.
                Siis holokaustia ei tapahtunut toisen maailmansodan aikana.}
    \end{alakohdat}

    \begin{vastaus}
        \begin{alakohdat}
            \alakohta{Johtopäätös vedetty vastaajien uskomuksesta.}
            \alakohta{Johtopäätös vedetty Stephen Hawkingin uskomuksesta.}
            \alakohta{Johtopäätös perustuu tiedemiehien väitteisiin, joista ei ole varmaa totuutta.}
            \alakohta{Johtopäätös perustuu televisiouutisen väitteeseen, josta ei ole selviä todisteita.}
        \end{alakohdat}
    \end{vastaus}
    
\end{tehtava}

\begin{tehtava}
    Ovatko seuraavat päättelyt loogisesti päteviä? Perustele.
    \begin{alakohdat}
        \alakohta{\kolmepaattely{Kaikilla $x$ toteutuu $y$.}{Joillakin $z$ toteutuu $x$.}{Joillakin $z$ toteutuu $y$.}}
        \alakohta{\kolmepaattely{Kaikilla $A$ toteutuu $B$.}{$C$ toteuttaa $B$:n.}{$C$ toteuttaa $A$:n.}}
        \alakohta{\kolmepaattely{Kaikilla $A$ toteutuu $B$.}{Millään $C$ ei toteudu $B$.}{Millään $C$ ei toteudu $A$.}}
    \end{alakohdat}

    \begin{vastaus}
        \begin{alakohdat}
            \alakohta{On}
            \alakohta{Ei. Jos A toteuttaa B:n, B ei välttämättä toteuta A:ta.}
            \alakohta{Ei, ?????????????} %En ole täysin varma vastauksesta
        \end{alakohdat}
    \end{vastaus}
    
\end{tehtava}
%Kotitehtävät Luku 1
\begin{tehtava}
    Ovatko seuraavat päättelyt päteviä? Perustele.
    \begin{alakohdat}
        \alakohta{\kolmepaattely{Kaikki kissat osaavat kehrätä.}{Tämä eläin osaa kehrätä.}{Tämä eläin on kissa.}}
        \alakohta{\kolmepaattely{Kukaan laiska opiskelija ei selviä kokeesta.}
                {On opiskelijoita, jotka selviävät kokeesta.}{On opiskelijoita, jotka eivät ole laiskoja.}}
    \end{alakohdat}

    \begin{vastaus}
        \begin{alakohdat}
            \alakohta{Ei}
            \alakohta{On}
        \end{alakohdat}
    \end{vastaus}
    
\end{tehtava}

\begin{tehtava}
    Ovatko seuraavat päättelyt päteviä? Perustele.
    \begin{alakohdat}
        \alakohta{\kolmepaattely{Neljäkkään lävistäjät ovat kohtisuorassa toisiaan vastaan.}
                {Neljäkäs on suunnikas.}{Suunnikkaan lävistäjät ovat kohtisuorassa toisiaan vastaan.}}
        \alakohta{\kolmepaattely{Kaikki suorakulmiot ovat suunnikkaita.}{Jotkut nelikulmiot ovat suorakulmioita.}
                {Jotkut nelikulmiot ovat suunnikkaita.}}
        \alakohta{\kolmepaattely{Kolmio $ABC$ on tasakylkinen.}{Tasasivuiset kolmiot ovat tasakylkisiä.}
                {Kolmio $ABC$ on tasasivuinen.}}
    \end{alakohdat}

    \begin{vastaus}
        \begin{alakohdat}
            \alakohta{Ei}
            \alakohta{On}
            \alakohta{Ei}
        \end{alakohdat}
    \end{vastaus}
    
\end{tehtava}

\begin{tehtava}
    Ovatko seuraavat päättelyt päteviä? Perustele.
    \begin{alakohdat}
        \alakohta{\kaksipaattely{Kaikki lammasfarmin lampaat ovat joko mustia tai valkoisia.}
                {Kaikki lampaat ovat mustia tai valkoisia.}}
        \alakohta{\neljapaattely{Tavallisessa korttipakassa kortti on aina joko pata,\\& risti, hertta tai ruutu.}
                {Pata- ja risti-kortit ovat mustia.}{Hertta- ja ruutu-kortit ovat punaisia.}
                {Kaikki tavallisten korttipakkojen kortit ovat\\ & joko mustia tai punaisia.}}
    \end{alakohdat}

    \begin{vastaus}
        \begin{alakohdat}
            \alakohta{Ei}
            \alakohta{On}
        \end{alakohdat}
    \end{vastaus}
    
\end{tehtava}

\begin{tehtava}
    Jäämaa on kokonaan Merimaan itäpuolella.
        Kummallakin on etelärajaa Aurinkomaan kanssa.
        Merimaalla ja Aurinkomaalla on länsiraja Lumimaan kanssa.
        Kukkamaa on kokonaan Jäämaan ja Aurinkomaan itäpuolella.
    \begin{alakohdat}
        \alakohta{Onko Merimaalla ja Kukkamaalla yhteistä rajaa?}
        \alakohta{Voiko Lumimaalla ja Kukkamaalla olla yhteistä rajaa?}
    \end{alakohdat}

    \begin{vastaus}
        \begin{alakohdat}
            \alakohta{Ei}
            \alakohta{Ei, ?????????????} %En ole täysin varma vastauksesta
        \end{alakohdat}
    \end{vastaus}
    
\end{tehtava}

\begin{tehtava}
    Ovatko seuraavat päättelyt päteviä? Perustele.
    \begin{alakohdat}
        \alakohta{\kolmepaattely{Millään $x$ ei toteudu $y$.}{Kaikilla $z$ toteutuu $x$.}{Millään $z$ ei toteudu $y$.}}
        \alakohta{\kolmepaattely{Kaikilla $A$ toteutuu $B$.}{Joillakin $C$ toteutuu $B$.}{Joillakin $A$ toteutuu $C$.}}
    \end{alakohdat}

    \begin{vastaus}
        \begin{alakohdat}
            \alakohta{On}
            \alakohta{Ei}
        \end{alakohdat}
    \end{vastaus}
    
\end{tehtava}
%Harj. Tehtävät Luku 2
\begin{tehtava}
    
    \begin{alakohdat}
        \alakohta{}
        \alakohta{}
        \alakohta{}
    \end{alakohdat}

    \begin{vastaus}
        \begin{alakohdat}
            \alakohta{}
            \alakohta{}
            \alakohta{}
        \end{alakohdat}
    \end{vastaus}
    
\end{tehtava}

\begin{tehtava}
    
    \begin{alakohdat}
        \alakohta{}
        \alakohta{}
        \alakohta{}
    \end{alakohdat}

    \begin{vastaus}
        \begin{alakohdat}
            \alakohta{}
            \alakohta{}
            \alakohta{}
        \end{alakohdat}
    \end{vastaus}
    
\end{tehtava}

\begin{tehtava}
    
    \begin{alakohdat}
        \alakohta{}
        \alakohta{}
        \alakohta{}
    \end{alakohdat}

    \begin{vastaus}
        \begin{alakohdat}
            \alakohta{}
            \alakohta{}
            \alakohta{}
        \end{alakohdat}
    \end{vastaus}
    
\end{tehtava}

\begin{tehtava}
    
    \begin{alakohdat}
        \alakohta{}
        \alakohta{}
        \alakohta{}
    \end{alakohdat}

    \begin{vastaus}
        \begin{alakohdat}
            \alakohta{}
            \alakohta{}
            \alakohta{}
        \end{alakohdat}
    \end{vastaus}
    
\end{tehtava}

\begin{tehtava}
    
    \begin{alakohdat}
        \alakohta{}
        \alakohta{}
        \alakohta{}
    \end{alakohdat}

    \begin{vastaus}
        \begin{alakohdat}
            \alakohta{}
            \alakohta{}
            \alakohta{}
        \end{alakohdat}
    \end{vastaus}
    
\end{tehtava}

\begin{tehtava}
    
    \begin{alakohdat}
        \alakohta{}
        \alakohta{}
        \alakohta{}
    \end{alakohdat}

    \begin{vastaus}
        \begin{alakohdat}
            \alakohta{}
            \alakohta{}
            \alakohta{}
        \end{alakohdat}
    \end{vastaus}
    
\end{tehtava}

\begin{tehtava}
    
    \begin{alakohdat}
        \alakohta{}
        \alakohta{}
        \alakohta{}
    \end{alakohdat}

    \begin{vastaus}
        \begin{alakohdat}
            \alakohta{}
            \alakohta{}
            \alakohta{}
        \end{alakohdat}
    \end{vastaus}
    
\end{tehtava}

\begin{tehtava}
    
    \begin{alakohdat}
        \alakohta{}
        \alakohta{}
        \alakohta{}
    \end{alakohdat}

    \begin{vastaus}
        \begin{alakohdat}
            \alakohta{}
            \alakohta{}
            \alakohta{}
        \end{alakohdat}
    \end{vastaus}
    
\end{tehtava}

\begin{tehtava}
    
    \begin{alakohdat}
        \alakohta{}
        \alakohta{}
        \alakohta{}
    \end{alakohdat}

    \begin{vastaus}
        \begin{alakohdat}
            \alakohta{}
            \alakohta{}
            \alakohta{}
        \end{alakohdat}
    \end{vastaus}
    
\end{tehtava}

\begin{tehtava}
    
    \begin{alakohdat}
        \alakohta{}
        \alakohta{}
        \alakohta{}
    \end{alakohdat}

    \begin{vastaus}
        \begin{alakohdat}
            \alakohta{}
            \alakohta{}
            \alakohta{}
        \end{alakohdat}
    \end{vastaus}
    
\end{tehtava}

\begin{tehtava}
    
    \begin{alakohdat}
        \alakohta{}
        \alakohta{}
        \alakohta{}
    \end{alakohdat}

    \begin{vastaus}
        \begin{alakohdat}
            \alakohta{}
            \alakohta{}
            \alakohta{}
        \end{alakohdat}
    \end{vastaus}
    
\end{tehtava}

\begin{tehtava}
    
    \begin{alakohdat}
        \alakohta{}
        \alakohta{}
        \alakohta{}
    \end{alakohdat}

    \begin{vastaus}
        \begin{alakohdat}
            \alakohta{}
            \alakohta{}
            \alakohta{}
        \end{alakohdat}
    \end{vastaus}
    
\end{tehtava}

\begin{tehtava}
    
    \begin{alakohdat}
        \alakohta{}
        \alakohta{}
        \alakohta{}
    \end{alakohdat}

    \begin{vastaus}
        \begin{alakohdat}
            \alakohta{}
            \alakohta{}
            \alakohta{}
        \end{alakohdat}
    \end{vastaus}
    
\end{tehtava}

\begin{tehtava}
    
    \begin{alakohdat}
        \alakohta{}
        \alakohta{}
        \alakohta{}
    \end{alakohdat}

    \begin{vastaus}
        \begin{alakohdat}
            \alakohta{}
            \alakohta{}
            \alakohta{}
        \end{alakohdat}
    \end{vastaus}
    
\end{tehtava}

\begin{tehtava}
    
    \begin{alakohdat}
        \alakohta{}
        \alakohta{}
        \alakohta{}
    \end{alakohdat}

    \begin{vastaus}
        \begin{alakohdat}
            \alakohta{}
            \alakohta{}
            \alakohta{}
        \end{alakohdat}
    \end{vastaus}
    
\end{tehtava}

\begin{tehtava}
    
    \begin{alakohdat}
        \alakohta{}
        \alakohta{}
        \alakohta{}
    \end{alakohdat}

    \begin{vastaus}
        \begin{alakohdat}
            \alakohta{}
            \alakohta{}
            \alakohta{}
        \end{alakohdat}
    \end{vastaus}
    
\end{tehtava}



