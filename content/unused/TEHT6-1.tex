%Harj.Tehtävät Luku 6

% \section{Suurin yhteinen tekijä ja Eukleideen algoritmi}
\begin{tehtava}
    Määritä lukujen suurin yhteinen tekijä.
    
    \begin{alakohdat}
        \alakohta{$15$ ja $20$}
        \alakohta{$9$ ja $36$}
        \alakohta{$4$ ja $7$}
    \end{alakohdat}

    \begin{vastaus}
        \begin{alakohdat}
            \alakohta{$5$}
            \alakohta{$9$}
            \alakohta{$1$}
        \end{alakohdat}
    \end{vastaus}
    
\end{tehtava}

\begin{tehtava}
    Määritä Eukleideen algoritmia käyttäen
    
    \begin{alakohdat}
        \alakohta{$\syt(184, 152)$}
        \alakohta{$\syt(227, 143)$.}
    \end{alakohdat}

    \begin{vastaus}
        \begin{alakohdat}
            \alakohta{$8$}
            \alakohta{$1$}
        \end{alakohdat}
    \end{vastaus}
    
\end{tehtava}

\begin{tehtava}
    Määritä Eukleideen algoritmia käyttäen
    
    \begin{alakohdat}
        \alakohta{$\syt(272, 1479)$}
        \alakohta{$\syt(4719, 18207)$.}
    \end{alakohdat}

    \begin{vastaus}
        \begin{alakohdat}
            \alakohta{$17$}
            \alakohta{$3$}
        \end{alakohdat}
    \end{vastaus}
    
\end{tehtava}

\begin{tehtava}
    Esitä murtoluku
    \begin{alakohdat}
        \alakohta{$\frac{143}{605}$}
        \alakohta{$\frac{5989}{30899}$}
    \end{alakohdat}
    supistetussa muodossa. Vihje: Määritä osoittajan ja nimittäjän suurin yhteinen tekijä.

    \begin{vastaus}
        \begin{alakohdat}
            \alakohta{$\frac{13}{55}$}
            \alakohta{$\frac{113}{583}$}
        \end{alakohdat}
    \end{vastaus}
    
\end{tehtava}

\begin{tehtava}
    % Tarkistettu (Topi Talvitie, 9.11.2013)
    Osoita, että murtoluku $\frac{8788}{13475}$ ei supistu.
\end{tehtava}

\begin{tehtava}
    Leirille osallistui 780 tyttöä ja 612 poikaa. Osallistujat jaettiin keskenään yhtä suuriin ryhmiin siten, että kussakin ryhmässä oli vain tyttöjä tai poikia. Mikä oli suurin mahdollinen ryhmäkoko?
    
    \begin{vastaus}
        12
    \end{vastaus}
    
\end{tehtava}

\begin{tehtava}
    Määritä lukujen $188000100$ ja $188$ suurin yhteinen tekijä.

    \begin{vastaus}
        $4$
    \end{vastaus}
    
\end{tehtava}

\begin{tehtava}
    Olkoon $a$ positiivinen kokonaisluku. Määritä
    \begin{alakohdat}
        \alakohta{$\syt(a, a)$}
        \alakohta{$\syt(a, 1)$}
        \alakohta{$\syt(a^2, a)$}
        \alakohta{$\syt((a+1)!, a!)$.}
    \end{alakohdat}
    Merkintä $a!$ tarkoittaa luvun $a$ \termi{kertoma}{kertomaa}. Se on tulo $a! = a \cdot (a-1) \cdot (a-2) \cdot \ldots \cdot 3 \cdot 2 \cdot 1$.

    \begin{vastaus}
        \begin{alakohdat}
            \alakohta{$a$}
            \alakohta{$1$}
            \alakohta{$a$}
            \alakohta{$a!$}
        \end{alakohdat}
    \end{vastaus}
    
\end{tehtava}

\begin{tehtava}
    % Tarkistettu (Topi Talvitie, 9.11.2013)
    Olkoon $n$ positiivinen kokonaisluku. Osoita Eukleideen algoritmia käyttäen, että $\syt(n+1, n)=1$.
\end{tehtava}

\begin{tehtava}
    Olkoon $n$ positiivinen kokonaisluku. Määritä lukujen $n^2 + 2n$ ja $n + 1$ suurin yhteinen tekijä.
    
    \begin{vastaus}
        1
    \end{vastaus}
    
\end{tehtava}

\begin{tehtava}
    % Tarkistettu (Topi Talvitie, 9.11.2013)
    Olkoon $n$ positiivinen kokonaisluku. Osoita, että
    \[\syt(3^{n+1} + 10, 3^n + 2)=1.\]
\end{tehtava}

\subsubsection*{6.1 Kotitehtäviä}
\begin{tehtava}
    Määritä lukujen suurin yhteinen tekijä.
    
    \begin{alakohdat}
        \alakohta{$63$ ja $7$}
        \alakohta{$64$ ja $33$}
        \alakohta{$45$ ja $60$}
    \end{alakohdat}

    \begin{vastaus}
        \begin{alakohdat}
            \alakohta{$7$}
            \alakohta{$1$}
            \alakohta{$15$}
        \end{alakohdat}
    \end{vastaus}
    
\end{tehtava}

\begin{tehtava}
    Määritä Eukleideen algoritmia käyttäen
    
    \begin{alakohdat}
        \alakohta{$\syt(657, 306)$}
        \alakohta{$\syt(2197, 4641)$}
        \alakohta{$\syt(15787, 4111)$.}
    \end{alakohdat}

    \begin{vastaus}
        \begin{alakohdat}
            \alakohta{$9$}
            \alakohta{$13$}
            \alakohta{$1$}
        \end{alakohdat}
    \end{vastaus}
    
\end{tehtava}

\begin{tehtava}
    Esitä murtoluku
    \begin{alakohdat}
        \alakohta{$\frac{182}{299}$}
        \alakohta{$\frac{7697}{32041}$}
    \end{alakohdat}
    supistetussa muodossa.

    \begin{vastaus}
        \begin{alakohdat}
            \alakohta{$\frac{14}{23}$}
            \alakohta{$\frac{43}{179}$}
        \end{alakohdat}
    \end{vastaus}
    
\end{tehtava}

\begin{tehtava}
    Leipomo suljettiin remontin ajaksi. Ennen sulkemista laskettiin, että leipomon varastossa oli 4896 vehnäsämpylää ja 1408 grahamsämpylää. Sämpylät pakattiin kuljetusta varten keskenään samankokoisiin pusseihin siten, että kuhunkin pussiin tuli vain vehnä- tai grahamsämpylöitä. Mikä oli suurin mahdollinen pussikoko? Oletetaan, että vehnäsämpylä oli samankokoinen kuin grahamsämpylä ja että yksikään pussi ei jäänyt vajaaksi.

    \begin{vastaus}
        $32$
    \end{vastaus}
    
\end{tehtava}

\begin{tehtava}
    Määritä lukujen $468468468108$ ja $234$ suurin yhteinen tekijä.
    
    \begin{vastaus}
        $18$
    \end{vastaus}
    
\end{tehtava}

\begin{tehtava}
    Olkoot $a$, $b$ ja $c$ positiivisia kokonaislukuja. Lukujen $a$, $b$ ja $c$ suurin yhteinen tekijä eli $\syt(a, b, c)$ voidaan määrittää siten, että ensin määritetään kahden luvun suurin yhteinen tekijä ja sitten tämän ja kolmannen luvun suurin yhteinen tekijä. Määritä
    
    \begin{alakohdat}
        \alakohta{$\syt(15, 30, 40)$}
        \alakohta{$\syt(6, 9, 11)$}
        \alakohta{$\syt(171, 456, 665)$.}
    \end{alakohdat}

    \begin{vastaus}
        \begin{alakohdat}
            \alakohta{$5$}
            \alakohta{$1$}
            \alakohta{$19$}
        \end{alakohdat}
    \end{vastaus}
    
\end{tehtava}

\begin{tehtava}
    Olkoot $a$ ja $b$ positiivisia kokonaislukuja ja $\syt(a, b)=9$. Voiko tällöin yhtälö $a + b = 186$ olla tosi?
    
    \begin{vastaus}
        Ei voi.
    \end{vastaus}
    
\end{tehtava}

\begin{tehtava}
    Olkoon $n$ positiivinen kokonaisluku. Tutki, mitä arvoja $\syt(n+4, n)$ voi saada.
    
    \begin{vastaus}
        $1$, $2$ ja $4$.
    \end{vastaus}
    
\end{tehtava}

\begin{tehtava}
    Olkoon $n$ positiivinen kokonaisluku. Määritä lukujen $n^2 + 3n$ ja $n + 2$ suurin yhteinen tekijä.
    
    \begin{vastaus}
        Jos $n$ on parillinen, $\syt(n^2 + 3n, n + 2) = 2$, muuten $\syt(n^2 + 3n, n + 2) = 1$.
    \end{vastaus}
    
\end{tehtava}

\begin{tehtava}
    % Tarkistettu (Topi Talvitie, 9.11.2013)
    Olkoot $a$ ja $b$ positiivisia kokonaislukuja. Osoita, että $\syt(a, b)$ on jaollinen kaikilla lukujen $a$ ja $b$ yhteisillä tekijöillä.
\end{tehtava}

