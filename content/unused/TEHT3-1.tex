% Harj. Tehtävät luku 3.1

\subsection*{Luvun 3.1 harjoitustehtävät}

\begin{tehtava}
     Onko lause tosi? 
    \begin{alakohdat}
        \alakohta{$K \subset A$}
        \alakohta{$A \subset V$}
        \alakohta{$V \subset P$}
        \alakohta{$V \subset T$}
    \end{alakohdat}
Merkinnät ovat samat kuin esimerkissä 1. %FIXME Viittaa oikeaan esimerkkiin!
    \begin{vastaus}
    
        \begin{alakohdat}
            \alakohta{On tosi.}
            \alakohta{Ei ole tosi.}
            \alakohta{On tosi.}
            \alakohta{Ei ole tosi.}
        \end{alakohdat}
    \end{vastaus}
\end{tehtava}


\begin{tehtava}
 Olkoot $A = \{7, 8, 9\}$ ja $B=\{5, 6, 7\}$. Määritä joukot
    \begin{alakohdat}
        \alakohta{$A \cup B$,}
        \alakohta{$A \cap B$,}
        \alakohta{$A \setminus B$,}
        \alakohta{$B \setminus A$.}
    \end{alakohdat}

    \begin{vastaus}
    
        \begin{alakohdat}
            \alakohta{$\{5, 6, 7, 8, 9 \}$}
            \alakohta{$\{7\}$}
            \alakohta{$\{8, 9\}$}
            \alakohta{$\{5,6\}$}
        \end{alakohdat}
    \end{vastaus}
\end{tehtava}

\begin{tehtava}
     Olkoot perusjoukko $X$ aakkoset, $A$ vokaalit ja $B=\{x, y, z\}$.
Määritä joukot
    \begin{alakohdat}
        \alakohta{$B\setminus A$,}
        \alakohta{$A\cap B$,}
        \alakohta{$\complement A$,}
        \alakohta{$A \setminus X$.}
    \end{alakohdat}

    \begin{vastaus}
    
        \begin{alakohdat}
            \alakohta{$\{x,z\}$}
            \alakohta{$\{y\}$}
            \alakohta{$\complement A$ on konsonanttien joukko, eli
            $\{b,c,d,f,g,h,j,k,l,m,n,p,q,r,s,t,v,w,x,z \}$}
            \alakohta{$\emptyset$}
        \end{alakohdat}
    \end{vastaus}
\end{tehtava}

\begin{tehtava}
     Olkoot $A$ koulussa opiskelevien täysi-ikäisten opiskelijoiden joukko ja $B$ niiden opiskelijoiden joukko, jotka asuvat vanhempiensa luona. Millaiset opiskelijat kuuluvat joukkoon
    \begin{alakohdat}
        \alakohta{$A \cap B$,}
        \alakohta{$\compl A$,}
        \alakohta{$A \cup \compl B$,}
        \alakohta{$A\setminus B$,}
        \alakohta{$B \setminus A$,}
        \alakohta{$\compl (A \cup B)$?}
    \end{alakohdat}

    \begin{vastaus}
    
        \begin{alakohdat}
    Perusjoukko on koulun opiskelijat.
            \alakohta{Vanhempiensa luona asuvat täysi-ikäiset.}
            \alakohta{Alle täysi-ikäiset.}
            \alakohta{Opiskelijat, jotka ovat täysi-ikäisiä tai eivät asu vanhempiensa luona.}
            \alakohta{Täysi-ikäiset, jotka eivät asu vanhempiensa luona.}
            \alakohta{Vanhempiensa luona asuvat, jotka eivät ole täysi-ikäisiä.}

            \alakohta{Opiskelijat, jotka eivät ole täysi-ikäisiä eivätkä asu vanhempiensa luona.}
        \end{alakohdat}
    \end{vastaus}
    
\end{tehtava}


%\item
%Ilmaise joukkomerkintöjä käyttäen kuvan
%\begin{enumerate}[a)]
%\item keltainen alue,
%\item sininen alue,
%\item violetti alue,
%\item vihreä alue.
%\end{enumerate}
%
%\begin{center}
%
%%% Kuva s. 62
%
%\begin{tikzpicture}[outline/.style={draw=#1,thick}]
%\draw[fill=none] (0,0) rectangle (6.8,4.2);
%\fill[fill=none,outline=black] (2.55,2.1) circle (2);
%\begin{scope}
%\pgfsetfillopacity{0.5}
%\fill[fill=none,outline=black] (4.25,2.1) circle (2);
%\end{scope}
%\draw (.5,3.65) node {\LARGE $X$};
%\draw (2.1,1.0) node {\textcolor{black}{\LARGE $A$}};
%\draw (4.7,1.0) node {\textcolor{black}{\LARGE $B$}};
%\end{tikzpicture}
%
%\end{center}
%%PICXX2

\begin{tehtava}
     Olkoot $A \cap B$, $A \setminus B$ ja $B \setminus A$ epätyhjiä joukkoja. Esitä Venn-diagrammissa varjostettuna alue, joka kuvaa joukkoa
    \begin{alakohdat}
        \alakohta{$\compl (A \cup B)$,}
        \alakohta{$\compl (A \cap B)$,}
        \alakohta{$\compl B \cap A$.}
    \end{alakohdat}

    \begin{vastaus}
    
        \begin{alakohdat}
            \alakohta{}
            \alakohta{}
            \alakohta{}
        \end{alakohdat}
    \end{vastaus}
    
\end{tehtava}



\begin{tehtava}
     Luokalla on 30 opiskelijaa. Heistä 14 harrastaa jääkiekkoa, 12 jalkapalloa ja 3 molempia. 
    \begin{alakohdat}
        \alakohta{Kuinka moni opiskelija harrastaa pelkkää jääkiekkoa?}
        \alakohta{Kuinka moni opiskelija harrastaa pelkkää jalkapalloa?}
        \alakohta{Kuinka moni opiskelija ei harrasta kumpaakaan?}
    \end{alakohdat}
Vihje: Tilanteesta kannattaa piirtää Venn-diagrammi.
    \begin{vastaus}
    
        \begin{alakohdat}
            \alakohta{}
            \alakohta{}
            \alakohta{}
        \end{alakohdat}
    \end{vastaus}
    
\end{tehtava}


\begin{tehtava}
     Olkoot perusjoukko $\rr$ sekä joukot $A$ ja $B$ reaalilukuvälit $A=[1, 5]$ ja $B=[3, 6]$. Ilmaise välimerkintää käyttäen joukot
    \begin{alakohdat}
        \alakohta{$A \cap B$,}
        \alakohta{$A \cup B$,}
        \alakohta{$A \setminus B$,}
        \alakohta{$\compl A$.}
    \end{alakohdat}

    \begin{vastaus}
    
        \begin{alakohdat}
            \alakohta{}
            \alakohta{}
            \alakohta{}
            \alakohta{}
        \end{alakohdat}
    \end{vastaus}
    
\end{tehtava}

\item 

\begin{tehtava}
     Yhdistä samaa tarkoittavat lauseet.
\[
\begin{array}{llll}
A & (x\in A)\land (x\in B) & 1 & x\in \complement A \\
B & x\notin A & 2 & x \in A\setminus B \\
C & (x\in A)\lor (x\in B) & 3 & x\in (A\cap B) \\
D & (x\in A)\land (x\notin B) & 4 & x\in (A\cup B)
\end{array}
\]

    \begin{vastaus}
    
    \end{vastaus}
    
\end{tehtava}

\item 
\begin{enumerate}[a)]
\item 

\item 

\end{enumerate}

\begin{tehtava}
     Osoita Venn-diagrammin avulla, että
    \begin{alakohdat}
        \alakohta{\[
A\cup (B \cap C) = (A\cup B)\cap(A\cup C),
\]}
        \alakohta{\[
A\cap (B \cup C) = (A\cap B)\cup(A\cap C).
\]}
    \end{alakohdat}

    \begin{vastaus}
    
        \begin{alakohdat}
            \alakohta{}
            \alakohta{}
        \end{alakohdat}
    \end{vastaus}
    
\end{tehtava}

\begin{tehtava}
     Sievennä Venn-diagrammin avulla.
    \begin{alakohdat}
        \alakohta{$(B\cap A) \cup (B \cap \compl A)$}
        \alakohta{$\compl (A\cup B) \cup (A \setminus B) \cup (B \setminus A)$}
    \end{alakohdat}

    \begin{vastaus}
    
        \begin{alakohdat}
            \alakohta{}
            \alakohta{}
        \end{alakohdat}
    \end{vastaus}
    
\end{tehtava}

\begin{tehtava}
  Olkoot perusjoukko kokonaislukujen joukko $\zz$, $W$ parillisten kokonaislukujen joukko ja $Y$ luvulla $3$ jaollisten kokonaislukujen joukko. Näitä joukkoja voidaan merkitä $W = \{2n\,|\, n\in\zz\}$ ja $Y = \{3n \,|\, n \in \zz\}$. Määritä joukot   
    \begin{alakohdat}
        \alakohta{$W \cap Y$,}
        \alakohta{$W \setminus Y$,}
        \alakohta{$W \cup Y$.}
    \end{alakohdat}

    \begin{vastaus}
    
        \begin{alakohdat}
            \alakohta{}
            \alakohta{}
            \alakohta{}
        \end{alakohdat}
    \end{vastaus}
    
\end{tehtava}

\begin{tehtava}
     Onko lause tosi?
    \begin{alakohdat}
        \alakohta{$\sqrt{2} \in \rr$}
        \alakohta{$-3 \in \nn$}
        \alakohta{$\pi \in \rr \setminus \qq$}
        \alakohta{$\frac{4}{2} \in \qq \setminus \zz$}
        \alakohta{$\emptyset \subset \zz$}
    \end{alakohdat}

    \begin{vastaus}
    
        \begin{alakohdat}
            \alakohta{}
            \alakohta{}
            \alakohta{}
            \alakohta{}
            \alakohta{}
        \end{alakohdat}
    \end{vastaus}
    
\end{tehtava}

\begin{tehtava}
     Määritä joukon
    \begin{alakohdat}
        \alakohta{$\{1, 2\}$}
        \alakohta{$\{1, 2, 3\}$}
    \end{alakohdat}
kaikki osajoukot.
    \begin{vastaus}
    
        \begin{alakohdat}
            \alakohta{}
            \alakohta{}
            \alakohta{}
            \alakohta{}
        \end{alakohdat}
    \end{vastaus}
    
\end{tehtava}

\begin{tehtava}
*
Joukon $A$ \termi{potenssijoukko}{potenssijoukoksi} $\mathcal{P}(A)$ kutsutaan kaikkien joukon $A$ osajoukkojen muodostamaa joukkoa:
\[
\mathcal{P}(A)=\{ B \, | \, B\subset A\}.
\]
    \begin{alakohdat}
        \alakohta{Määritä joukon $\{1, 2, 3\}$ potenssijoukko.}
        \alakohta{Määritä tyhjän joukon $\emptyset$ potenssijoukko.}
        \alakohta{Joukko $\{\emptyset\}$ on joukko, jonka ainut alkio on tyhjä joukko.}
        \alakohta{Määritä joukon $\{\emptyset\}$ potenssijoukko.}
    \end{alakohdat}

    \begin{vastaus}
    
        \begin{alakohdat}
            \alakohta{}
            \alakohta{}
            \alakohta{}
            \alakohta{}
        \end{alakohdat}
    \end{vastaus}
    
\end{tehtava}

\item 
\begin{enumerate}[a)]
\item 
\item 
\item  Miksi?
\end{enumerate}

\begin{tehtava}
    *
Luonnolliset luvut voidaan tulkita joukko-opillisesti käyttämällä edellisessä tehtävässä esitettyä potenssijoukon käsitettä. Ajatuksena on, että lukua $0$ vastaa tyhjä joukko $\emptyset$ ja kutakin luonnollista lukua $n + 1$ vastaava joukko on luonnollista lukua $n$ vastaavan joukon potenssijoukko. Siis esimerkiksi lukua $1$ vastaa tyhjän joukon $\emptyset$ potenssijoukko $\mathcal{P}(\emptyset)$, lukua $2$ potenssijoukko $\mathcal{P}(\mathcal{P}(\emptyset))$ ja niin edelleen. 
    \begin{alakohdat}
        \alakohta{Määritä lukuja $3$ ja $4$ vastaavat joukot.}
        \alakohta{Mitä luonnollista lukua vastaa joukko $\{\emptyset,\{\emptyset\}\}$?}
        \alakohta{Vastaako joukko $\{\{\emptyset\}\}$ jotakin luonnollista lukua?}
    \end{alakohdat}

    \begin{vastaus}
    
        \begin{alakohdat}
            \alakohta{}
            \alakohta{}
            \alakohta{}
        \end{alakohdat}
    \end{vastaus}
    
\end{tehtava}

% Kotitehtävät luku 3.1