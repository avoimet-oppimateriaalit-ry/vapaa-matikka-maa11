\begin{tehtavasivu}

\begin{tehtava}
	Osoita, että a) $23 \equiv 8 \quad (\mod 5)$ b) $1326 \equiv -546\quad (\mod 8)$ c) $403 \equiv 0 \quad (\mod 13)$.
\end{tehtava}

\begin{tehtava}
	Määritä pienin epänegatiivinen luku, jonka kanssa luku a) $56$ b) $-72$ c) $3857$ on kongruentti modulo $6$.
\end{tehtava}

\begin{tehtava}
	Määritä kaikki kokonaisluvut $x$, joille $x \equiv 7 \quad (\mod 12)$ ja $18 < x < 130$.
\end{tehtava}

\begin{tehtava}
	Pitkäkestoiset kynttilät sytytettiin samaan aikaan. Toinen paloi 128 tuntia ja toinen 181 tuntia. Sammuivatko kynttilät samaan kellonaikaan?
\end{tehtava}

\begin{tehtava}
	Aaron syntymäpäiväjuhlat alkavat 912 tunnin kuluttua ja Mirkun syntymäpäiväjuhlat 1419 tunnin kuluttua.
	\begin{enumerate}[a)]
	\item Mihin kellonaikaan Mirkun juhlat alkavat, kun Aaron juhlat alkavat kello 12.00?
	\item Aaron juhlat ovat lauantaina. Mikä viikonpäivä nyt on?
	\end{enumerate}
\end{tehtava}

\begin{tehtava}
	Määritä pienin epänegatiivinen luku, jonka kanssa luku
	\begin{enumerate}[a)]
	\item $234 - 15$
	\item $79 \cdot 650$
	\item $19^{12} + 772$
	\end{enumerate}
	on kongruentti modulo 4.
\end{tehtava}

\begin{tehtava}
	Tiedetään, että jakojäännös on $2$, kun kokonaisluku $n$ jaetaan luvulla $5$. Mikä on jakojäännös, kun luku $3n^2 + 8n + 7$ jaetaan luvulla $5$?
\end{tehtava}

\begin{tehtava}
	Kokonaisluvut voidaan jakaa osajoukkoihin esimerkiksi niin, että kussakin osajoukossa ovat ne luvut, jotka ovat kongruentteja keskenään jaettaessa luvulla 3. Näihin osajoukkoihin kuuluvia lukuja voidaan merkitä $3q$, $3q + 1$ ja $3q + 2$, missä $q \in \zz$. Miten voidaan merkitä lukuja osajoukoissa, jotka syntyvät, kun kokonaisluvut jaetaan luvulla a) $4$ b) $5$ c) $2$?
\end{tehtava}

\begin{tehtava}
	Olkoon $n$ kokonaisluku. Osoita, että $n^2 - n \equiv 0\quad (\mod 2)$.
\end{tehtava}

\begin{tehtava}
	Olkoon $n$ kokonaisluku. Osoita, että luku $n(n + 5)$ on parillinen.
\end{tehtava}

\begin{tehtava}
	Osoita, että luku $n(n + 1)(n + 8)$ on jaollinen luvulla 3, kun $n$ on kokonaisluku.
\end{tehtava}

\begin{tehtava}
	Osoita, että luku $n^5 - n$ on jaollinen luvulla 5, kun $n$ on kokonaisluku.
\end{tehtava}

\begin{tehtava}
	Osoita, että luku $n^3 + 2$ ei ole jaollinen luvulla 4 millään kokonaisluvun $n$ arvolla.
\end{tehtava}

\begin{tehtava}
	Jokainen kokonaisluku on joko parillinen tai pariton. Osoita erikseen näissä osajoukoissa, että luku $n^4 + 2n^3 + n^2$ on jaollinen luvulla $4$, kun $n$ on kokonaisluku. (Vrt. esimerkki 5.42.)
\end{tehtava}

\begin{tehtava}
	Etsi kaikki kokonaisluvut $x$, jotka toteuttavat kongruenssin $5x\equiv 1 \quad (\mod 3)$.
\end{tehtava}

\begin{tehtava}
	Olkoon $k$ positiivinen kokonaisluku. Osoita, että kokonaisluvut $a$ ja $b$ ovat kongruentteja modulo $k$ eli erotus $a-b$ on jaollinen luvulla $k$, jos ja vain jos luvuilla $a$ ja $b$ on sama jakojäännös, kun jaetaan luvulla $k$.
\end{tehtava}

\begin{tehtava}
	Julius Caesar käytti erästä vanhimmista tunnetuista salakirjoitusmenetelmistä. Hän muutti viestin salaiseksi korvaamalla jokaisen kirjaimen toisella kirjaimella, joka sijaitsi aakkosissa kolme kirjainta myöhemmin. Viimeisten kirjainten jälkeen palattiin taas aakkosten alkuun. Suomen kielessä on 28 kirjainta, jos W-kirjainta ei lasketa mukaan. Siten esimerkiksi viesti AVAIN ON ÖLJYRUUKUSSA kirjoitettaisiin Caesarin menetelmällä DZDLQ RQ COMÄUYYNYVVD. Suomenna viestit 
	\begin{enumerate}[a)]
	\item KÄCNNBÄV NHVNLÄCOOB
	\item BOB VÄC UÄSBOHLXB.
	\end{enumerate}
\end{tehtava}

\begin{tehtava}
	Caesarin salakirjoitusmenetelmää voidaan kuvata funktiolla $f(n) = n + 3 \ (\mod 28)$, missä $n$ on kunkin kirjaimen järjestysnumero aakkosissa. Koodattavan kirjaimen järjestysnumeroon lisätään luku $3$ ja näin saatu funktion arvo tulkitaan takaisin kirjaimeksi.

	Kirjoita viesti AVAIN ON ÖLJYRUUKUSSA käyttäen salakirjoitusta, jota kuvaa funktio 
	\begin{enumerate}[a)]
	\item $f(n) = n + 13 \ (\mod 28)$
	\item $f(n) = 3n + 7 \ (\mod 28)$.
	\item Määritä funktiot, joiden avulla a- ja b-kohtien salakirjoitus voidaan purkaa.
	\end{enumerate}
\end{tehtava}

\end{tehtavasivu}

% -----

\begin{kotitehtavasivu}

\begin{tehtava}
	Osoita, että a) $34 \equiv 10\ (\mod 3)$ b) $-128 \equiv 274\ (\mod 6)$ c) $1053 \equiv 0\ (\mod 27)$.
\end{tehtava}

\begin{tehtava}
	Määritä pienin epänegatiivinen luku, jonka kanssa luku a) $25$ b) $-121$ c) $5777$ on kongruentti modulo $5$.
\end{tehtava}

\begin{tehtava}
	Määritä kaikki kokonaisluvut $x$, joille $x \equiv 3\quad (\mod 8)$ ja $-50 < x < 51$.
\end{tehtava}

\begin{tehtava}
	Neptunus-planeetan pyörähdysaika on noin 16 tuntia. Onko Neptunuksella sama kellonaika 223 tunnin kuluttua kuin 49 tuntia sitten?
\end{tehtava}

\begin{tehtava}
	Maijan isoäiti on syntynyt 7.3. eräänä karkausvuonna. Maijan äiti on syntynyt 9525 päivää isoäitiä myöhemmin ja Maija itse 10229 päivää äitiään myöhemmin. Tutki, ovatko jotkut henkilöistä syntyneet samana viikonpäivänä.
	% Eikö esim. vuosi 1900 voi aiheuttaa tässä ongelmia? -Niko
\end{tehtava}

\begin{tehtava}
	Nyt on tammikuu. Mikä kuukausi
	\begin{enumerate}[a)]
	\item on 76 kuukauden kuluttua
	\item oli 555 kuukautta sitten?
	\end{enumerate}
\end{tehtava}

\begin{tehtava}
	Määritä pienin epänegatiivinen luku, jonka kanssa luku
	\begin{enumerate}[a)]
	\item $17 - 930 + 452$
	\item $19 \cdot 30 + 183 \cdot 11$
	\item $5 \cdot 26^{15} + 493$
	\end{enumerate}
	on kongruentti modulo 9.
\end{tehtava}

\begin{tehtava}
	Tiedetään, että jakojäännös on $5$, kun kokonaisluku $n$ jaetaan luvulla $11$. Mikä on jakojäännös, kun luku $2n^3 - 6n - 9$ jaetaan luvulla $11$?
\end{tehtava}

\begin{tehtava}
	Osoita kongruenssin laskusääntöjä koskevan lauseen kohta 2: Jos $a\equiv c$ ja $b\equiv d\quad (\mod k)$, niin $a-b\equiv c-d \quad(\mod k)$.
\end{tehtava}

\begin{tehtava}
	Olkoon $n$ kokonaisluku. Osoita, että luku $n^2 + n$ on jaollinen kahdella.
\end{tehtava}

\begin{tehtava}
	Osoita, että luku $n(n + 1)(4n - 1)$ on jaollinen luvulla 6, kun $n$ on kokonaisluku.
\end{tehtava}

\begin{tehtava}
	Osoita, että luku $n(4n^2 - 1)$ on jaollinen luvulla 3, kun $n$ on kokonaisluku.
\end{tehtava}

\begin{tehtava}
	Osoita, että luku $n^2 - 3$ ei ole jaollinen luvulla 5 millään kokonaisluvun $n$ arvolla.
\end{tehtava}

\begin{tehtava}
	Etsi kaikki kokonaisluvut $x$, jotka toteuttavat kongruenssin $4x+1\equiv 3\quad (\mod 7)$.
\end{tehtava}

\begin{tehtava}
	Olkoot $a$ ja $b$ kokonaislukuja. Osoita väite todeksi tai epätodeksi: jos $ab \equiv 0 \quad (\mod m)$, niin $a \equiv 0$ tai $b\equiv 0 \quad (\mod m)$.
\end{tehtava}

\begin{tehtava}
	Olkoot $a$ ja $b$ kokonaislukuja. Osoita väite todeksi tai epätodeksi: jos $a^2 \equiv b^2 \quad (\mod m)$, niin $a \equiv b \quad (\mod m)$.
\end{tehtava}

\begin{tehtava}
	* %(Lisämateriaalia)
	Luvun $a$ määräämä \termi{jäännösluokka modulo $m$}{jäännösluokka modulo} $m$ on luvun $a$ kanssa kongruenttien lukujen joukko. Luvun $a$ jäännösluokkaa merkitään $\underline{a}$. Siis $\underline{a} = \{a + km\, |\, k\in \zz\}$. Esimerkiksi jäännösluokat modulo $4$ ovat $\underline{0}=\{4k\, |\, k\in \zz\}$, $\underline{1}=\{1+4k\, |\, k\in \zz\}$, $\underline{2}=\{2+4k\, |\, k\in \zz\}$ ja $\underline{3}=\{3+4k\, |\, k\in \zz\}$.
	\begin{enumerate}[a)]
	\item Määritä jäännösluokat modulo $5$.
	\item Jäännösluokkien yhteenlasku ja kertolasku määritellään seuraavasti: $\underline{a} + \underline{b} = \underline{a + b}$ ja $\underline{a} \cdot \underline{b} = \underline{a\cdot b}$. Täydennä seuraavat jäännösluokkien yhteen- ja kertolaskutaulut modulo $5$.
	\[
	\begin{array}{|l|l|l|l|l|l|}
	\hline
	+& \underline{0} & \underline{1} & \underline{2} & \underline{3} & \underline{4} \\ \hline
	\underline{0} & \underline{0} & \underline{1} & \underline{2} &  \underline{3} & \underline{4}  
	\\ \hline
	 \underline{1} &&&&& \\ \hline
	 \underline{2} &&&&& \\ \hline
	 \underline{3} &&&&& \\ \hline
	\underline{4} & \underline{4} & \underline{0} & \underline{1} &  \underline{2} & \underline{3}  
	\\ \hline
	\end{array}
	\qquad
	\begin{array}{|l|l|l|l|l|l|}
	\hline
	\cdot& \underline{0} & \underline{1} & \underline{2} & \underline{3} & \underline{4} \\ \hline
	\underline{0} & \underline{0} & \underline{0} & \underline{0} &  \underline{0} & \underline{0}  
	\\ \hline
	 \underline{1} &&&&& \\ \hline
	 \underline{2} &&&&& \\ \hline
	 \underline{3} &&&&& \\ \hline
	\underline{4} & \underline{0} & \underline{4} & \underline{3} &  \underline{2} & \underline{1}  
	\\ \hline
	\end{array}
	\]
	\item Jäännösluokan vasta-alkio määritellään seuraavasti: $\underline{b}$ on alkion $\underline{a}$ vasta-alkio, jos $\underline{a} + \underline{b} = \underline{0}$. Määritä b-kohdan  taulukoiden avulla kunkin jäännösluokan vasta-alkiot modulo $5$.
	\item Jäännösluokan käänteisalkio määritellään seuraavasti: $\underline{c}$ on alkion $\underline{a}$ käänteisalkio, jos $\underline{a} \cdot \underline{c} = \underline{1}$. Määritä b-kohdan taulukoiden avulla käänteisalkiot niille jäännösluokkien modulo $5$ alkioille, joilla sellainen on olemassa.
	\item Ratkaise jäännösluokissa modulo $5$ yhtälö $\underline{x}+\underline{x}=\underline{1}$.
	\item Ratkaise jäännösluokissa modulo $5$ yhtälö $\underline{x}\cdot \underline{x}=\underline{4}$.
	\end{enumerate}
\end{tehtava}

\end{kotitehtavasivu}
