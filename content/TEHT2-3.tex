\begin{tehtavasivu}

%Luvun 2.3 harjoitustehtävien vastauksia, vastauksien tekijä Valtteri Vistiaho 10.11.2013
\begin{tehtava}
     Tutki, onko lause tautologia.
    \alakohdat{
        § $A\lor \lnot A$,
        § $A \land \lnot A$,
        § $A \to B \lor A$,
        § $A \to B \land A$,
    }

    \begin{vastaus}
    
        \alakohdat{
            § \begin{center}
		    \begin{tabular}{|c|c|c|c|c|}\hline
		    $A$ & $\lnot A$ & $A\lor \lnot A$ \\ \hline
		    $1$ & $0$ & $1$ \\ %\hline
		    $0$ & $1$ & $1$ \\ \hline
\end{tabular}
\end{center}
Lause on tautologia.

            § \begin{center}
		    \begin{tabular}{|c|c|c|c|c|}\hline
		    $A$ & $\lnot A$ & $A\land \lnot A$ \\ \hline
		    $1$ & $0$ & $0$ \\ %\hline
		    $0$ & $1$ & $0$ \\ \hline
\end{tabular}
\end{center}
Lause ei ole tautologia.

            § \begin{center}
		    \begin{tabular}{|c|c|c|c|c|c|c|}\hline
		    $A$ & $B$ & $B\lor A$ & $A\to B\lor A$\\ \hline
		    $1$ & $1$ & $1$ & $1$ \\ %\hline
		    $1$ & $0$ & $1$ & $1$ \\
		    $0$ & $1$ & $1$ & $1$ \\
		    $0$ & $0$ & $0$ & $1$ \\ \hline
\end{tabular}
\end{center}
Lause on tautologia.
            § \begin{center}
		    \begin{tabular}{|c|c|c|c|c|c|c|}\hline
		    $A$ & $B$ & $B\land A$ & $A\to B\land A$\\ \hline
		    $1$ & $1$ & $1$ & $1$ \\ %\hline
		    $1$ & $0$ & $0$ & $0$ \\
		    $0$ & $1$ & $0$ & $1$ \\
		    $0$ & $0$ & $0$ & $1$ \\ \hline
\end{tabular}
\end{center}
Lause ei ole tautologia.
        }
    \end{vastaus}
    
\end{tehtava}

\begin{tehtava}
     Osoita lause tautologiaksi.
    \alakohdat{
        § $A\to B\land \lnot B \lequiv \lnot A$.
        § $(A\to B) \land (B\to C)\to (A\to C)$.
    }

    \begin{vastaus}
    \begin{footnotesize}
        \alakohdat{
            § \begin{center}
		    \begin{tabular}{|c|c|c|c|c|c|c|}\hline
		    $A$ & $B$ & $\lnot A$ & $\lnot B$ & $B\land \lnot B$ & $A\to B\land \lnot B$ & $A\to B\land \lnot B \lequiv \lnot A$ \\ \hline
		    $1$ & $1$ & $0$ & $0$ & $0$ & $0$ & $1$ \\ %\hline
		    $1$ & $0$ & $0$ & $1$ & $0$ & $0$ & $1$ \\
		    $0$ & $1$ & $1$ & $0$ & $0$ & $1$ & $1$ \\
		    $0$ & $0$ & $1$ & $1$ & $0$ & $1$ & $1$ \\ \hline
\end{tabular}
\end{center}
{\normalsize Koska viimeisen sarakkeen kaikki totuusarvot ovat 1, lause on tautologia.}
            § \begin{center}
		    \begin{tabular}{|c|c|c|c|c|c|c|c|}\hline
		    $A$ & $B$ & $C$ & $A\to B$ & $B\to C$ & $A\to C$ & $(A\to B)\land(B\to C)$ & $(A\to B)\land(B\to C)\to(A\to C)$ \\ \hline
		    $1$ & $1$ & $1$ & $1$ & $1$ & $1$ & $1$ & $1$ \\ %\hline
		    $1$ & $1$ & $0$ & $1$ & $0$ & $0$ & $0$ & $1$ \\
		    $1$ & $0$ & $1$ & $0$ & $1$ & $1$ & $0$ & $1$ \\
		    $1$ & $0$ & $0$ & $0$ & $1$ & $0$ & $0$ & $1$ \\
		    $0$ & $1$ & $1$ & $1$ & $1$ & $1$ & $1$ & $1$ \\
		    $0$ & $0$ & $1$ & $1$ & $1$ & $1$ & $1$ & $1$ \\
		    $0$ & $1$ & $0$ & $1$ & $0$ & $1$ & $0$ & $1$ \\
		    $0$ & $0$ & $0$ & $1$ & $1$ & $1$ & $1$ & $1$ \\ \hline
\end{tabular}
\end{center}
{\normalsize Koska viimeisen sarakkeen kaikki totuusarvot ovat 1, lause on tautologia.}
        }
    \end{footnotesize}
    \end{vastaus}
    
\end{tehtava}

\begin{tehtava}
     Olkoot $A$: ''kello soi'' ja $B$: ''tunti loppuu''.
Kirjoita luonnollisella kielellä lauseet $A\to B$ ja
$\lnot(A \land \lnot B )$. Tarkoittavatko ne samaa?
    
    \begin{vastaus}
    
       Jos kello soi, tunti loppuu. \newline
       Ei ole totta, että kello soi ja tunti ei lopu. \newline
       Totuusarvoiltaan lauseet tarkoittavat samaa.
    \end{vastaus}
    
\end{tehtava}

\begin{tehtava}
     Osoita lauseet loogisesti ekvivalenteiksi
totuustaulujen avulla.
    \alakohdat{
        § $A\land B$ ja $\lnot(\lnot A\lor \lnot B )$.
        § $A\lequiv B$ ja $(A\to B)\land (B \to A)$.
        § $A\lequiv B$ ja $(A\land B) \lor (\lnot A\land
\lnot B )$.
    }

    \begin{vastaus}
    \begin{scriptsize}
        \alakohdat{
            § \begin{center}
		    \begin{tabular}{|c|c|c|c|c|c|c|c|}\hline
		    $A$ & $B$ & $\lnot A$ & $\lnot B$ & $A\land B$ & $\lnot A\lor \lnot B$ & $\lnot(\lnot A\lor \lnot B)$ & $A\land B\lequiv \lnot(\lnot A\lor \lnot B)$ \\ \hline
		    $1$ & $1$ & $0$ & $0$ & $1$ & $0$ & $1$ & $1$ \\ %\hline
		    $1$ & $0$ & $0$ & $1$ & $0$ & $1$ & $0$ & $1$ \\
		    $0$ & $1$ & $1$ & $0$ & $0$ & $1$ & $0$ & $1$ \\
		    $0$ & $0$ & $1$ & $1$ & $0$ & $1$ & $0$ & $1$ \\ \hline
\end{tabular}
\end{center}
            § \begin{center}
		    \begin{tabular}{|c|c|c|c|c|c|c|}\hline
		    $A$ & $B$ & $A\lequiv B$ & $A\to B$ & $B\to A$ & $(A\to B)\land (B\to A)$ & $(A\lequiv B)\lequiv((A\to B)\land (B\to A))$ \\ \hline
		    $1$ & $1$ & $1$ & $1$ & $1$ & $1$ & $1$ \\ %\hline
		    $1$ & $0$ & $0$ & $0$ & $1$ & $0$ & $1$ \\
		    $0$ & $1$ & $0$ & $1$ & $0$ & $0$ & $1$ \\
		    $0$ & $0$ & $1$ & $1$ & $1$ & $1$ & $1$ \\ \hline
\end{tabular}
\end{center}
            § \begin{center}
		    \begin{tabular}{|c|c|c|c|c|c|c|c|c|}\hline
		    $A$ & $B$ & $\lnot A$ & $\lnot B$ & $A\lequiv B$ & $A\land B$ & $\lnot A\land \lnot B$ & $(A\land B)\lor(\lnot A\land \lnot B)$ & $(A\lequiv B)\lequiv((A\land B)\lor(\lnot A\land \lnot B))$ \\ \hline
		    $1$ & $1$ & $0$ & $0$ & $1$ & $1$ & $0$ & $1$ & $1$ \\ %\hline
		    $1$ & $0$ & $0$ & $1$ & $0$ & $0$ & $0$ & $0$ & $1$ \\
		    $0$ & $1$ & $1$ & $0$ & $0$ & $0$ & $0$ & $0$ & $1$ \\
		    $0$ & $0$ & $1$ & $1$ & $1$ & $0$ & $1$ & $1$ & $1$ \\ \hline
\end{tabular}
\end{center}
        }
    \end{scriptsize}
    \end{vastaus}
    
\end{tehtava}

\begin{tehtava}
     Osoita vaihdantalait
    \alakohdat{
        § \[
A \land B \lequiv B \land A,
\]
        § \[
A \lor B \lequiv B \lor A
\] oikeaksi totuustaulujen avulla.

    }

    \begin{vastaus}
    
        \alakohdat{
            § \begin{center}
		    \begin{tabular}{|c|c|c|c|c|}\hline
		    $A$ & $B$ & $A\land B$ & $B\land A$ & $(A\land B)\lequiv(B\land A)$\\ \hline
		    $1$ & $1$ & $1$ & $1$ & $1$ \\ %\hline
		    $1$ & $0$ & $0$ & $0$ & $1$ \\
		    $0$ & $1$ & $0$ & $0$ & $1$ \\
		    $0$ & $0$ & $0$ & $0$ & $1$ \\ \hline
\end{tabular}
\end{center}
            § \begin{center}
		    \begin{tabular}{|c|c|c|c|c|}\hline
		    $A$ & $B$ & $A\lor B$ & $B\lor A$ & $(A\land B)\lequiv(B\land A)$\\ \hline
		    $1$ & $1$ & $1$ & $1$ & $1$ \\ %\hline
		    $1$ & $0$ & $1$ & $1$ & $1$ \\
		    $0$ & $1$ & $1$ & $1$ & $1$ \\
		    $0$ & $0$ & $0$ & $0$ & $1$ \\ \hline
\end{tabular}
\end{center}
        }
    \end{vastaus}
    
\end{tehtava}

\begin{tehtava}
     Muodosta atomilauseista $A$ ja $B$ lause, joka on
loogisesti ekvivalentti lauseen $X$ kanssa.

\begin{center}
\begin{tabular}{|c|c|c|}\hline
$A$ & $B$ & $X$\\ \hline
$1$ & $1$ & $0$\\
$1$ & $0$ & $1$\\
$0$ & $1$ & $1$\\
$0$ & $0$ & $1$\\ \hline
\end{tabular}
\end{center}

    \begin{vastaus}
    \begin{center}
\begin{tabular}{|c|c|c|c|}\hline
$A$ & $B$ & $X$ & $\lnot(A\land B)$\\ \hline
$1$ & $1$ & $0$ & $0$\\
$1$ & $0$ & $1$ & $1$\\
$0$ & $1$ & $1$ & $1$\\
$0$ & $0$ & $1$ & $1$\\ \hline
\end{tabular}
\end{center}

    \end{vastaus}
    
\end{tehtava}

\begin{tehtava}
     Tutki, onko lause loogisesti ristiriitainen.
    \alakohdat{
        § $\lnot (A \lor B) \land \lnot (A\land B)$,
        § $(\lnot A \land B) \land (A \lequiv B)$,
        § $(\lnot (A \to B) \land C) \land B$.
    }

    \begin{vastaus}
    
        \alakohdat{
            § \begin{center}
		    \begin{tabular}{|c|c|c|c|c|c|c|}\hline
		    $A$ & $B$ & $A\lor B$ & $A\land B$ & $\lnot(A\lor B)$ & $\lnot(A\land B)$ & $\lnot(A\lor B)\land \lnot(A\land B)$\\ \hline
		    $1$ & $1$ & $1$ & $1$ & $0$ & $0$ & $0$ \\ %\hline
		    $1$ & $0$ & $1$ & $0$ & $0$ & $1$ & $0$ \\
		    $0$ & $1$ & $1$ & $0$ & $0$ & $1$ & $0$ \\
		    $0$ & $0$ & $0$ & $0$ & $1$ & $1$ & $1$ \\ \hline
\end{tabular}
\end{center}
Lause ei ole loogisesti ristiriitainen.
            § \begin{center}
		    \begin{tabular}{|c|c|c|c|c|c|}\hline
		    $A$ & $B$ & $\lnot A$ & $\lnot A\land B$ & $A\lequiv B$ & $(\lnot A\land B)\land(A\lequiv B)$\\ \hline
		    $1$ & $1$ & $0$ & $0$ & $1$ & $0$ \\ %\hline
		    $1$ & $0$ & $0$ & $0$ & $0$ & $0$ \\
		    $0$ & $1$ & $1$ & $1$ & $0$ & $0$ \\
		    $0$ & $0$ & $1$ & $0$ & $1$ & $0$ \\ \hline
\end{tabular}
\end{center}
Lause on loogisesti ristiriitainen.
            § \begin{center}
		    \begin{tabular}{|c|c|c|c|c|c|c|}\hline
		    $A$ & $B$ & $C$ & $A\to B$ & $\lnot(A\to B)$ & $\lnot(A\to B)\land C$ & $(\lnot(A\to B)\land C)\land B$\\ \hline
		    $1$ & $1$ & $1$ & $1$ & $0$ & $0$ & $0$ \\ %\hline
		    $1$ & $1$ & $0$ & $1$ & $0$ & $0$ & $0$ \\
		    $1$ & $0$ & $1$ & $0$ & $1$ & $1$ & $0$ \\
		    $1$ & $0$ & $0$ & $0$ & $1$ & $0$ & $0$ \\
		    $0$ & $1$ & $1$ & $1$ & $0$ & $0$ & $0$ \\
		    $0$ & $0$ & $1$ & $1$ & $0$ & $0$ & $0$ \\
		    $0$ & $1$ & $0$ & $1$ & $0$ & $0$ & $0$ \\
		    $0$ & $0$ & $0$ & $1$ & $0$ & $0$ & $0$ \\ \hline
\end{tabular}
\end{center}
Lause on loogisesti ristiriitainen.
        }
    \end{vastaus}
    
\end{tehtava}

\begin{tehtava}
     Tulkitse sanallisesti
    \alakohdat{
        § De Morganin 2. laki
\[
\lnot (A\lor B) \lequiv \lnot A\land \lnot B
\]
        § modus ponens -päättelysääntö
\[
(A \land (A\to B))\to B
\]
        § reductio ad absurdum -päättelysääntö
\[
(\lnot A \to (B\land \lnot B))\to A
\]
    }

    \begin{vastaus}
    
        \alakohdat{
            § Lauseet ''Ei ole totta, että A tai B'' ja ''Ei A eikä B'' ovat loogisesti ekvivalentit.
            § Jos A tapahtuu ja A:sta seuraa B, B tapahtuu % /Niko
            § Ristiriidasta seuraa mitä vain. % /Niko
        }
    \end{vastaus}
    
\end{tehtava}

\begin{tehtava}
     Osoita
    \alakohdat{
        § kaksoisnegaation laki,
        § De Morganin 1. laki,
        § modus ponens -päättelysääntö
    }
        oikeaksi totuustaulujen avulla.

    \begin{vastaus}
    
        \alakohdat{
            § \begin{center}
		    \begin{tabular}{|c|c|c|c|}\hline
		    $A$ & $\lnot A$ & $\lnot \lnot A$ & $\lnot \lnot A\lequiv A$\\ \hline
		    $1$ & $0$ & $1$ & $1$ \\ %\hline
		    $0$ & $1$ & $0$ & $1$ \\ \hline
\end{tabular}
\end{center}
            § \begin{center}
		    \begin{tabular}{|c|c|c|c|c|c|c|c|}\hline
		    $A$ & $B$ & $\lnot A$ & $\lnot B$ & $A\land B$ & $\lnot(A\land B)$ & $\lnot A\lor \lnot B$ & $\lnot(A\land B)\lequiv \lnot A\lor \lnot B$\\ \hline
		    $1$ & $1$ & $0$ & $0$ & $1$ & $0$ & $0$ & $1$ \\ %\hline
		    $1$ & $0$ & $0$ & $1$ & $0$ & $1$ & $1$ & $1$ \\
		    $0$ & $1$ & $1$ & $0$ & $0$ & $1$ & $1$ & $1$ \\
		    $0$ & $0$ & $1$ & $1$ & $0$ & $1$ & $1$ & $1$ \\ \hline
\end{tabular}
\end{center}
            § \begin{center}
		    \begin{tabular}{|c|c|c|c|c|}\hline
		    $A$ & $B$ & $A\to B$ & $A\land(A\to B)$ & $A\land(A\to B)\to B$\\ \hline
		    $1$ & $1$ & $1$ & $1$ & $1$ \\ %\hline
		    $1$ & $0$ & $0$ & $0$ & $1$ \\
		    $0$ & $1$ & $1$ & $0$ & $1$ \\
		    $0$ & $0$ & $1$ & $0$ & $1$ \\ \hline
\end{tabular}
\end{center}
        }
    \end{vastaus}
    
\end{tehtava}
%Loppuu Valtteri Vistiahon tekemät vastaukset 10.11.2013

% LOPUT:

\begin{tehtava}
	Osoita kontraposition laki oikeaksi ilman
	totuustauluja.
	Vihje: Voit korvata implikaation $A\to B$ loogisesti
	ekvivalentilla lauseella $\lnot A \lor B$.
\end{tehtava}

\begin{tehtava}
	Esitä lauseelle ''jos Jaakko saa logiikan kurssista
	arvosanan 10, hän tarjoaa ystävilleen kahvit'' kolme
	loogisesti ekvivalenttia lausetta.
\end{tehtava}

\begin{tehtava}
	Osoita lauseet loogisesti ekvivalenteiksi
	päättelysääntöjen avulla ilman totuustauluja.
	\alakohdat{
	§ $A\land B$ ja $\lnot(\lnot A \lor \lnot B)$,
	§ $C\to (\lnot A \land B)$ ja $(A\lor \lnot B)\to
	\lnot C$.
	§ $A \land (A\to \lnot B)\to \lnot B$ ja $B\to
	\lnot A\lor \lnot (B\to \lnot A)$.
	}
\end{tehtava}

\begin{tehtava}
	* Kolmiarvologiikassa on kolme totuusarvoa: $1$
	tosi, $0$ epätosi ja $u$ epävarma. Kleenen totuustaulut
	perustuvat ajatukseen, että epävarma voi myöhemmin
	osoittautua todeksi tai epätodeksi. Alla on esitetty
	negaation ja konjunktion totuustaulut. Täydennä
	oheinen disjunktion totuustaulu. Laadi implikaation ja
	ekvivalenssin totuustaulut.

	\begin{center}
	\begin{tabular}{|c|c|c|}\hline
	$A$ & $\lnot A$ \\ \hline
	$1$ & $0$ \\
	$0$ & $1$ \\
	$u$ & $u$ \\ \hline
	\end{tabular}
	%\end{center}
	\qquad
	%\begin{center}
	\begin{tabular}{|c|c|c|}\hline
	$A$ & $B$ & $A\land B$\\ \hline
	$1$ & $1$ & $1$\\
	$1$ & $0$ & $0$\\
	$0$ & $1$ & $0$\\
	$0$ & $0$ & $0$\\
	$1$ & $u$ & $u$\\
	$u$ & $1$ & $u$\\
	$0$ & $u$ & $0$\\
	$u$ & $0$ & $0$\\
	$u$ & $u$ & $u$\\ \hline
	\end{tabular}
	%\end{center}
	\qquad
	%\begin{center}
	\begin{tabular}{|c|c|c|}\hline
	$A$ & $B$ & $A\lor B$\\ \hline
	$1$ & $1$ & \\
	$1$ & $0$ & \\
	$0$ & $1$ & \\
	$0$ & $0$ & \\
	$1$ & $u$ & \\
	$u$ & $1$ & \\
	$0$ & $u$ & \\
	$u$ & $0$ & \\
	$u$ & $u$ & \\ \hline
	\end{tabular}
	\end{center}
	
	\begin{vastaus}
		\begin{center}
		\begin{tabular}{|c|c|c|c|c|}\hline
		$A$ & $B$ & $A\lor B$ & $A\to B$ & $A\lequiv B$ \\ \hline
		$1$ & $1$ & $1$ & $1$ & $1$ \\
		$1$ & $0$ & $1$ & $0$ & $0$ \\
		$0$ & $1$ & $1$ & $1$ & $0$ \\
		$0$ & $0$ & $0$ & $1$ & $1$ \\
		$1$ & $u$ & $1$ & $u$ & $u$ \\
		$u$ & $1$ & $1$ & $1$ & $u$ \\
		$0$ & $u$ & $u$ & $1$ & $u$ \\
		$u$ & $0$ & $u$ & $u$ & $u$ \\
		$u$ & $u$ & $u$ & $u$ & $u$ \\ \hline
		\end{tabular}
		\end{center}
	\end{vastaus}
\end{tehtava}

\end{tehtavasivu}

% -----


\begin{kotitehtavasivu}

\begin{tehtava}
	Osoita lause tautologiaksi.
	\alakohdat{
	§ $A\to A$.
	§ $A\land B \to A$.
	§ $(A\lequiv B) \to (B\to A)$.
	}
\end{tehtava}

\begin{tehtava}
	Tutki, onko lause tautologia.
	\alakohdat{
	§ $(A\land B \to \lnot C)\lequiv (A\to(B\to C))$.
	§ $(\lnot A \lequiv (B \land C)) \lequiv \lnot (A
	\lequiv B \land C)$.
	}
	
	\begin{vastaus}
	\alakohdat{
	§ Ei ole.
	§ On.
	}
	\end{vastaus}
\end{tehtava}

\begin{tehtava}
	Olkoot $A$: ''tunti jatkuu'' ja $B$: ''kello soi''.
	Kirjoita luonnollisella kielellä lauseet $A\to \lnot B$
	ja $B\to \lnot A$. Osoita totuustaulujen avulla, että
	lauseet ovat loogisesti ekvivalentit.
	
	\begin{vastaus}
	\begin{description}
	\item[$A\to \lnot B$:] ''jos tunti jatkuu, kello ei soi''
	\item[$B\to \lnot B$:] ''jos kello soi, tunti ei jatku''
	\end{description}
	\end{vastaus}
\end{tehtava}

\begin{tehtava}
	Osoita, että lauseet ovat loogisesti ekvivalentit.
	\alakohdat{
	§ ''Tero soittaa kitaraa, mutta Suvi ei
	laskettele'' ja ''ei ole niin, että jos Tero soittaa
	kitaraa, niin Suvi laskettelee''.
	§ ''Jos Tero soittaa kitaraa tai Suvi laskettelee,
	niin Anni kirjoittaa runoja'' ja ''jos Tero soittaa
	kitaraa, niin Anni kirjoittaa runoja, ja jos Suvi
	laskettelee, niin Anni kirjoittaa runoja''.
	}
\end{tehtava}

\begin{tehtava}
	Sievennä lause eli muodosta mahdollisimman
	yksinkertainen lause, joka on loogisesti ekvivalentti
	alkuperäisen lauseen kanssa.
	\alakohdat{
	§ $(A\land B) \lor A$.
	§ $(A\lor B) \land A$.
	§ $(A\lor B) \land (A\lor \lnot B)$
	}
	
	\begin{vastaus}
	\alakohdat{
	§ $A$
	§ $A$
	§ $A$
	}
	\end{vastaus}
\end{tehtava}

\begin{tehtava}
	Osoita osittelulait
	\alakohdat{
	§ 
	\[
	A \land (B \lor C) \lequiv (A \land B) \lor (A\land C),
	\]
	§ 
	\[
	A \lor (B \land C) \lequiv (A \lor B) \land (A\lor C)
	\]
	}
	oikeaksi totuustaulujen avulla.
\end{tehtava}

\begin{tehtava}
	Muodosta atomilauseista $A$ ja $B$ lause, joka on
	loogisesti ekvivalentti lauseen $Y$ kanssa.
	\begin{center}
	\begin{tabular}{|c|c|c|}\hline
	$A$ & $B$ & $Y$\\ \hline
	$1$ & $1$ & $1$\\
	$1$ & $0$ & $1$\\
	$0$ & $1$ & $1$\\

	$0$ & $0$ & $1$\\ \hline
	\end{tabular}
	\end{center}
	
	\begin{vastaus}
	Esimerkiksi $A \lor \lnot A$
	\end{vastaus}
\end{tehtava}

\begin{tehtava}
	Tarkastele seuraavaa ennustetta: maapallon öljyvarat ehtyvät, jos ja vain jos länsimaiset demokratiat romahtavat, mutta ei pidä paikkaansa, että jos maapallon
	öljyvarat ehtyvät, niin länsimaiset demokratiat romahtavat. Miten ennusteeseen pitäisi suhtautua?
	
	\begin{vastaus}
	Maapallon öljyvarat eivät ehdy eivätkä länsimaiset demokratiat romahda.
	\end{vastaus}
\end{tehtava}

\begin{tehtava}
	Mainitse esimerkki tilanteesta, jossa olet käyttänyt
	\alakohdat{
	§ modus ponens -päättelysääntöä
	§ modus tollens -päättelysääntöä.
	}
\end{tehtava}

\begin{tehtava}
	Osoita
	\alakohdat{
	§ De Morganin 2. laki,
	§ modus tollens -päättelysääntö,
	§ reductio ad absurdum -päättelysääntö
	}
	oikeaksi totuustaulujen avulla.
\end{tehtava}

\begin{tehtava}
	Osoita lauseet loogisesti ekvivalenteiksi
	päättelysääntöjen avulla ilman totuustauluja.
	\alakohdat{
	§ $\lnot \lnot \lnot \lnot \lnot A$ ja $\lnot A$,
	§ $A \to (B \to C)$ ja $\lnot (\lnot C \to \lnot
	B) \to \lnot A$.
	§ $\lnot (A \land B \land \lnot C)$ ja $\lnot A
	\lor \lnot B \lor C$.
	}
\end{tehtava}

\begin{tehtava}
	Shefferin viivaan ja Peircen nuoleen on tutustuttu
	edellisen kappaleen \ref{shefferpeirce}. Esitä negaatio,
	konjunktio ja disjunktio a) Shefferin viivan avulla b)
	Peircen nuolen avulla.
	
	\begin{vastaus}
	\alakohdat{
	§ Jos Shefferin viivaa merkitään merkillä $|$, seuraavat lauseet ovat aina tosia:
	\begin{align*}
	\lnot A &\lequiv A | A, \\
	A \land B &\lequiv (A | B) | (A | B), \\
	A \lor B &\lequiv (A | A) | (B | B), \\
	\end{align*}
	§ Jos Peircen nuolta merkitään merkillä $\downarrow$, seuraavat lauseet ovat aina tosia:
	\begin{align*}
	\lnot A &\lequiv A \downarrow A, \\
	A \land B &\lequiv (A \downarrow A) \downarrow (B \downarrow B), \\
	A \lor B &\lequiv (A \downarrow B) \downarrow (A \downarrow B), \\
	\end{align*}
	}
	\end{vastaus}
\end{tehtava}

\begin{tehtava}
	* Sumea logiikka on kaksiarvoisen logiikan
	laajennus, jossa lauseella on diskreetin totuusarvon
	(tosi tai epätosi) sijasta reaalinen totuusarvo, joka
	kuuluu välille $[0, 1]$. Konnektiivit voidaan määritellä
	esimerkiksi seuraavasti:
	\[
	\begin{array}{rcl}
	\lnot A &=& 1-A,\\
	A\land B &=& \min(A, B),\\
	A\lor B &=& \max(A, B),\\
	A\to B
	&=& \min(1, 1-A+B),
	\end{array}
	\]
	missä $\min$ tarkoittaa luvuista pienempää ja $\max$
	suurempaa.

	\alakohdat{
	§ Olkoot lauseen $A$ totuusarvo $0,3$ ja lauseen
	$B$ totuusarvo $0,5$. Laske lauseiden $\lnot A$, $A\land
	B$, $A\lor B$ ja $A \to B$ totuusarvot.
	§ Osoita, että jos lauseet $A$ ja $B$ saavat
	vain arvoja $0$ ja $1$, niin edellä mainitut määritelmät johtavat
	klassisen kaksiarvoisen logiikan totuustauluihin.
	§ Osoita, että sumeassa logiikassa $\lnot(A\land B)$ ja $\lnot A \lor \lnot B$ ovat loogisesti ekvivalentit.
	§ Etsi Internetistä sumean logiikan
	käyttökohteita.
	}
	
	\begin{vastaus}
	\alakohdat{
	§ $\lnot A = 0,7$, $A\land B = 0,3$, $A\lor B = 0,5$, $A\to B = 1$
	}
	\end{vastaus}
\end{tehtava}

\begin{tehtava}
	* Tutustu Wolfram Alphan
	logiikkatoimintoihin\\
	\href{http://www.wolframalpha.com/examples/BooleanAlgebra.html}{{\tt http://www.wolframalpha.com/examples/BooleanAlgebra.html}}\\
	ja yritä laskea sen avulla joitakin kirjan tehtäviä.
\end{tehtava}

%Ratkaisu: Lauseet $A$: '' Jaska on syyllinen'',
%$B$ ''Jaskalla on rikostoveri'', $\lnot (A\to B)=\lnot
%(\lnot A \lor B)=A \land \lnot B$.

\end{kotitehtavasivu}
