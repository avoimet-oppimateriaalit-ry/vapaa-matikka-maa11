\begin{tehtavasivu}

\begin{tehtava}
    Onko luku
    \alakohdatm{
    § $50$
    § $48$
    § $-72$
    § $-34$
    }
    jaollinen luvulla $6$?
    \begin{vastaus}
        \alakohdatm{
        § Ei
        § On
        § On
        § Ei
        }
    \end{vastaus}

\end{tehtava}

\begin{tehtava}
    Jakaako luku $13$ luvun a) $117$ b) $-65$ c) $160$ d) $-81$?
    \begin{vastaus}
        a) Kyllä b) Kyllä c) Ei d) Ei 
    \end{vastaus}
\end{tehtava}

\begin{tehtava}
    Osoita, että a) $3|66$ b) $7\nmid 120$ c) $15|(-330)$ d) $11\nmid (-619)$.
    \begin{vastaus}
        a) $66 = 22\cdot 3$ b) $120 = 17\cdot 7 + 1$ c) $-330 = -22\cdot 15$ d) $-619 = -57\cdot 11 + 8$ 
    \end{vastaus}

\end{tehtava}

\begin{tehtava}
    Onko väite tosi?
    \alakohdat{
    § $3|53$
    § $-9|108$
    § $12 \nmid (-158)$
    § $-7|175$
    § $-17 \nmid (-646)$
    }
    
    \begin{vastaus}
        \alakohdat{
        § Ei
        § On
        § On
        § On
        § Ei
        }
    \end{vastaus}
\end{tehtava}

\begin{tehtava}
    Kirjoita jakoyhtälö, kun
    \alakohdat{
    § luku $7$ jaetaan luvulla $3$
    § luku $51$ jaetaan luvulla $4$
    § luku $1\,000$ jaetaan luvulla $125$
    § luku $3\,858$ jaetaan luvulla $97$.
    }
    
    \begin{vastaus}
        \alakohdat{
        § $7 = 2\cdot 3 + 1$
        § $51 = 12\cdot 4 + 3$
        § $1\,000 = 8 \cdot 125$
        § $3\,858 = 39\cdot 97 + 75$
        }
    \end{vastaus}
\end{tehtava}

\begin{tehtava}
    Kirjoita jakoyhtälö, kun
    \alakohdat{
    § luku $-22$ jaetaan luvulla $7$
    § luku $-2\,844$ jaetaan luvulla $36$
    § luku $-3\,858$ jaetaan luvulla $97$.
    }
    
    \begin{vastaus}
        \alakohdat{
        § $-22 = -4\cdot 7 + 6$
        § $-2\,844 = -79\cdot 36$
        § $-3\,858 = -40\cdot 97 + 22$
        }
    \end{vastaus}
\end{tehtava}

\begin{tehtava}
    Kirjoita jakoyhtälö jakolaskulle
    \alakohdat{
    § $9/25$
    § $-9/25$
    § $124/120$
    § $-124/120$.
    }
    
    \begin{vastaus}
        \alakohdat{
        § $9 = 0\cdot 25 + 9$
        § $-9 = -1\cdot 25 + 16$
        § $124 = 1\cdot 120 + 4$
        § $-124 = -2\cdot 120 + 116$
        }
    \end{vastaus}
\end{tehtava}

\begin{tehtava}
    Päättele, mikä on jakojäännös, kun
    \alakohdat{
    § luku $1\,967$ jaetaan luvulla $5$
    § luku $-426$ jaetaan luvulla $5$
    § luku $67\,876$ jaetaan luvulla $50$
    § luku $-30\,509$ jaetaan luvulla $50$.
    }
    
    \begin{vastaus}
        \alakohdat{
        § $2$
        § $4$
        § $26$
        § $41$
        }
    \end{vastaus}
\end{tehtava}

\begin{tehtava}
    Leipomossa on pakattavana $260$ sämpylää. Kuinka monta täyttä pussia saadaan ja kuinka monta sämpylää jää yli, jos käytetään vain a) $24$ b) $10$ c) $6$ d) $4$ sämpylän pusseja?
    \begin{vastaus}
        \alakohdat{
        § 10 täyttä pussia, 20 sämpylää jää yli.
        § 26 täyttä pussia.
        § 43 täyttä pussia, 2 sämpylää jää yli.
        § 65 täyttä pussia.
        }     
    \end{vastaus}

\end{tehtava}

\begin{tehtava}
    Jos kello on nyt 13.05, niin mitä kello oli 2012 tuntia ja 45 minuuttia sitten?
    \begin{vastaus}
%        $13\, t + 5\, m - 2012\, t - 45\, m \equiv 13\, t - 20(\mod 24)\, t + 5\, m - 45\, m \equiv -7\, t -40\, m \equiv 24 -7\, t -40\, m \equiv 16(\mod 24)\, t + 20\, m $ \\
%        Vastaus: $16.20$
         16.20
    \end{vastaus}
\end{tehtava}

\begin{tehtava}
	\alakohdat{
	§ Mikä luku jaettuna luvulla $17$ antaa osamääräksi $98$ ja jakojäännökseksi $5$?
	§ Mikä luku jaettuna luvulla $12$ antaa osamääräksi $-91$ ja jakojäännökseksi $0$?
	§ Millä positiivisella kokonaisluvulla luku $146$ on jaettava, jotta osamäärä olisi $3$ ja jakojäännös $2$?
	§ Millä positiivisella kokonaisluvulla luku $72$ on jaettava, jotta jakojäännös olisi 7?
	}
    \begin{vastaus}
        \alakohdat{
        § $1671 = 98\cdot 17 + 5$
        § $-1092 = -91\cdot 12 + 0$
        § $48 (146 = 3\cdot 48 + 2)$
        § $13 (72 = 5\cdot 13 + 7$
        }     
    \end{vastaus}
\end{tehtava}

\begin{tehtava}
	Määritä osamäärä ja jakojäännös sekä kirjoita jakoyhtälö, kun a) luku $2^{18} + 10$ jaetaan luvulla $2^{15} + 1$ b) luku $3^{100} + 100$ jaetaan luvulla $3^{98} + 10$.
    \begin{vastaus}
        \alakohdat{
        § $2^{18} + 10 = 8(2^{15} +1) + 2$ \\
        Vaillinainen osamäärä on 8, jakojäännös 2.
        § $3^{100} + 100 = 9(3^{98} + 10) + 10$ \\
        Vaillinainen osamäärä on 9, jakojäännös 10.
        }     
    \end{vastaus}
\end{tehtava}

\begin{tehtava}
	Olkoot $a$, $b$, $c$, $r$ ja $s$ kokonaislukuja ja $a \neq 0$. Osoita, että jos $a|b$ ja $a|c$, niin $a|(rb + sc)$.
    \begin{vastaus}
        Koska $a\mid b$ ja $a\mid c$ niin $b = ak$ ja $c = al$ joillain $k, l \in \mathbb{Z}$. Nyt $rb = rak$ ja $sc = sal$. Siis $rb +sc = rak + sal = (rk + sl)a$. $(rk +sl) \in \mathbb{Z}$ ja $a\in \mathbb{Z}$. Siten $a\mid(rb+sc)$.
    \end{vastaus}
\end{tehtava}

\begin{tehtava}
	Olkoot $a$, $b$, $c$ ja $d$ kokonaislukuja ja $a, b \neq 0$. Osoita, että jos $a|c$ ja $b|d$, niin $(ab)|(cd)$.
\end{tehtava}

\begin{tehtava}
	Osoita, että jos $a$ ja $b$ ovat parittomia kokonaislukuja, niin $4 | (a^2 - b^2)$.
\end{tehtava}

\begin{tehtava}
	Olkoot $a$ ja $b$ nollasta eroavia kokonaislukuja. Mitä voidaan päätellä, jos $a|b$ ja $b|a$?
    \begin{vastaus}
        Joko $a = b$ tai $a = -b$.
    \end{vastaus}
\end{tehtava}

\begin{tehtava}
	Olkoon $n$ positiivinen kokonaisluku. Määritä osamäärä ja jakojäännös sekä kirjoita jakoyhtälö, kun
	\alakohdat{
	§ luku $5n + 3$ jaetaan luvulla $5$
	§ luku $n^2 + 2n + 2$ jaetaan luvulla $n + 1$
	§ luku $n^3 + 3n^2 - n - 3$ jaetaan luvulla $n + 3$
	§ luku $2n^3 + 3n^2 + 4n + 9$ jaetaan luvulla $2n + 3$.
	}
    \begin{vastaus}
        \alakohdat{
        § $5n + 3 = n\cdot 5 + 3$ \\
        Vaillinainen osamäärä on $n$, jakojäännös $3$.
        § $n^2 + 2n + 2 = (n + 1) (n + 1)$ \\
        Vaillinainen osamäärä on $n + 1$, jakojäännös $0$.
        § $n^3 + 3n^2 - n - 3 = (n^2 - 1) (n+3)$ \\
        Vaillinainen osamäärä on $n^2 - 1$, jakojäännös $0$.
        § $2n^3 + 3n^2 + 4n + 9 = (n^2 + 2) (2n + 3) + 3$ \\
        Vaillinainen osamäärä on $n^2 + 2$, jakojäännös $3$.
        }
    \end{vastaus}
\end{tehtava}

%%%%%%%%%%%%%% FIX ME   Mikä on oikea tapa laittaa välit \ldots komentoa ennen ja jälkee? Nyt ennen on väli, jälkeen ei, kannattaa korjata kaikki esiintymät samanlaisiksi

\begin{tehtava}
	\alakohdat{
	§ Muodosta jakoyhtälö luvuille $0, 1, 2, \ldots, 7$, kun jakajana on luku $3$. Mitä arvoja jakojäännös voi saada?
	§ Osoita, että jos kahden kokonaisluvun tulo on jaollinen luvulla $3$, niin ainakin toinen luvuista on jaollinen luvulla $3$. Vihje: Käytä epäsuoraa todistusta.
	}
    \begin{vastaus}
        \alakohdat{
        § $0 = 0\cdot 3$ \\
          $1 = 0\cdot 3 + 1$ \\
          $2 = 0\cdot 3 + 2$ \\
          $3 = 1\cdot 3$ \\
          $4 = 1\cdot 3 + 1$ \\
          $5 = 1\cdot 3 + 2$ \\
          $6 = 2\cdot 3$ \\
          $7 = 2\cdot 3 + 1$ \\
          Jakojäännös saa arvoja $0$, $1$ ja $2$.
        }
    \end{vastaus}
\end{tehtava}

\end{tehtavasivu}

% -----


\begin{kotitehtavasivu}

\begin{tehtava}
	Onko luku a) $42$ b) $-75$ c) $102$ d) $-98$ jaollinen luvulla $7$?
\end{tehtava}

\begin{tehtava}
	Jakaako luku $11$ luvun a) $-165$ b) $21$ c) $-101$ d) $209$?
\end{tehtava}

\begin{tehtava}
	Osoita, että a) $2|234$ b) $-17|408$ c) $14 \nmid 223$ d) $6 \nmid (-472)$.
\end{tehtava}

\begin{tehtava}
	Onko väite tosi? a) $8 \nmid 168$ b) $13 \nmid (-95)$ c) $-5|777$ d) $29\nmid 2583$ e) $-4|(-924)$
\end{tehtava}

\begin{tehtava}
	Kirjoita jakoyhtälö, kun
	\alakohdat{
	§ luku $9$ jaetaan luvulla $6$
	§ luku $576$ jaetaan luvulla $19$
	§ luku $3712$ jaetaan luvulla $32$.
	}
\end{tehtava}

\begin{tehtava}
	Kirjoita jakoyhtälö, kun
	\alakohdat{
	§ luku $-12$ jaetaan luvulla $5$
	§ luku $-147$ jaetaan luvulla $6$
	§ luku $-875$ jaetaan luvulla $35$.
	}
\end{tehtava}

\begin{tehtava}
	Kirjoita jakoyhtälö jakolaskulle a) $3/17$ b) $-3/17$ c) $88/80$ d) $-88/80$.
\end{tehtava}

\begin{tehtava}
	Päättele, mikä on jakojäännös, kun
	\alakohdat{
	§ luku $5555$ jaetaan luvulla $4$
	§ luku $-555$ jaetaan luvulla $4$
	§ luku $123456$ jaetaan luvulla $100$
	§ luku $-654321$ jaetaan luvulla $100$.
	}
\end{tehtava}

\begin{tehtava}
	Tänään on keskiviikko. Mikä viikonpäivä a) on 1000 päivän kuluttua b) oli 500 päivää sitten?
\end{tehtava}

\begin{tehtava}
	Kello on $17.28$ ja on tiistai. Mitä kello on 2073 tunnin kuluttua? Mikä viikonpäivä silloin on?
\end{tehtava}

\begin{tehtava}
	Tarkastellaan kirjainjonoa ABCDEFGABCDEFGABCDEFG... a) Mikä on jonon 12742. kirjain? b) Kuinka monta A-kirjainta on jonossa ennen sitä?
\end{tehtava}

\begin{tehtava}
	\alakohdat{
	§ Mikä luku jaettuna luvulla $19$ antaa osamääräksi $6$ ja jakojäännökseksi $13$?
	§ Mikä luku jaettuna luvulla $7$ antaa osamääräksi $-23$ ja jakojäännökseksi $5$?
	§ Millä positiivisella kokonaisluvulla luku $1263$ on jaettava, jotta osamäärä olisi $114$ ja jakojäännös $9$?
	§ Millä positiivisella kokonaisluvulla luku $-140$ on jaettava, jotta jakojäännös olisi $3$?
	}
\end{tehtava}

\begin{tehtava}
	Olkoon $n$ kokonaisluku. Osoita, että luku $(3n+1)^4 - (3n+1)^3$ on jaollinen luvulla 3. Vihje: Käytä symbolisen laskimen {\tt expand}-toimintoa.
\end{tehtava}

\begin{tehtava}
	Olkoon $n$ kokonaisluku. Osoita, että luku $(2n+1)^5 - (2n-1)^4-2n$ on jaollinen luvulla $16$.
\end{tehtava}

\begin{tehtava}
	Määritä osamäärä ja jakojäännös sekä kirjoita jakoyhtälö, kun a) luku $7^{50} + 1$ jaetaan luvulla $7^{48} - 1$ b) luku $7^{50} - 1$ jaetaan luvulla $7^{25} + 1$.
\end{tehtava}

\begin{tehtava}
	Olkoot $a$, $b$ ja $c$ kokonaislukuja ja $a \neq 0$. Osoita, että jos $a|b$ ja $a|(b + c)$, niin $a|c$.
\end{tehtava}

\begin{tehtava}
	Olkoot $p$, $q$ ja $r$ positiivisia kokonaislukuja. Osoita, että jos $p$ on luvun $q$ tekijä ja $q$ on luvun $r$ tekijä, niin $p$ on luvun $r$ tekijä.
\end{tehtava}

\begin{tehtava}
	Olkoot $a$, $b$ ja $c$ kokonaislukuja ja $a, c \neq 0$. Osoita, että jos $(ac)|(bc)$, niin $a|b$.
\end{tehtava}

\begin{tehtava}
	Olkoot $a$ ja $b$ kokonaislukuja. Osoita, että luku $a + b$ on jaollinen luvulla $3$ silloin ja vain silloin, kun luku $a - 2b$ on jaollinen luvulla $3$.
\end{tehtava}

\begin{tehtava}
	Olkoon $n$ positiivinen kokonaisluku. Määritä osamäärä ja jakojäännös sekä kirjoita jakoyhtälö, kun
	\alakohdat{
	§ luku $2n^2 - n - 2$ jaetaan luvulla $n + 1$
	§ luku $n^3 + n^2 + 6$ jaetaan luvulla $n + 2$
	§ luku $n^2 + 2n - 1$ jaetaan luvulla $n$.
	}
	Vihje: Voit laskea polynomien jakolaskun symbolisen laskimen avulla.
\end{tehtava}

\begin{tehtava}
	* Todista jakoyhtälö, kun $a<0$.
\end{tehtava}

\begin{tehtava}
	* \termi{lukujärjestelmä}{Lukujärjestelmä} tarkoittaa tapaa, jolla luvut kirjoitetaan numeroiden avulla. \termi{kantaluku}{Kantaluku} kertoo, kuinka monta eri numeroa lukujärjestelmän luvuissa voi esiintyä. Esimerkiksi \termi{kymmenjärjestelmä}{kymmenjärjestelmässä} kantaluku on $10$ ja käytössä ovat numerot $0, 1, 2, 3, 4, 5, 6, 7, 8$ ja $9$. Kymmenjärjestelmän luku $3258$ muodostuu numeroista $3, 2, 5$ ja $8$ kantaluvun $10$ potenssien avulla seuraavasti:
	\begin{eqnarray*}
	3258 &=&3\cdot 1000+2\cdot 100+5\cdot 10+8\\
	&=& 3\cdot 10^3+2\cdot 10^2+5\cdot 10^1+8\cdot 10^0.
	\end{eqnarray*}
	Kymmenjärjestelmässä siis luvun viimeinen numero on luvun $10^0$ kerroin, toiseksi viimeinen luvun $10^1$ kerroin, kolmanneksi viimeinen luvun $10^2$ kerroin jne.

	\termi{kaksijärjestelmä}{Kaksijärjestelmän} eli \termi{binäärijärjestelmä}{binäärijärjestelmän} luvut taas muodostuvat numeroista $0$ ja $1$. Esimerkiksi binääriluku $101101$ voidaan ilmaista kymmenjärjestelmässä kirjoittamalla luku kantaluvun $2$ potenssien avulla seuraavasti:
	\begin{eqnarray*}
	101101&=&1\cdot2^5+0\cdot2^4+1\cdot2^3+1\cdot2^2+0\cdot2^1+1\cdot2^0 \\
	&=& 1\cdot32+0\cdot16+1\cdot8+1\cdot4+0\cdot2+1=45.
	\end{eqnarray*}

	Kaksijärjestelmässä siis luvun viimeinen numero on luvun $2^0$ kerroin, toiseksi viimeinen luvun $2^1$ kerroin, kolmanneksi viimeinen luvun $2^2$ kerroin jne.

	Kymmenjärjestelmän lukuja voidaan muuntaa binääriluvuiksi jakoyhtälöiden avulla. Muunnetaan luku 18 binääriluvuksi. Jaetaan ensin kymmenjärjestelmän luku 18 kaksijärjestelmään kantaluvulla 2 ja kirjataan ylös jakojäännös. Tämän jälkeen jaetaan edellisen jakolaskun osamäärä kantaluvulla 2 ja kirjataan taas jakojäännös muistiin. Näin jatketaan, kunnes osamääräksi jää luku 0. 
	\begin{eqnarray*}
	18&=&9\cdot 2+0\\
	9&=&4\cdot 2+1\\
	4&=&2\cdot 2+0\\
	2&=&1\cdot 2+0\\
	1&=&0\cdot2+1 \to \textrm{lopetetaan}
	\end{eqnarray*}
	Kirjoittamalla nyt jakojäännökset lopusta alkuun saadaan luku $10010$. Se on kymmenjärjestelmän luvun $18$ binääriesitys.
	\alakohdat{
	§ Muunna binäärijärjestelmän luvut $1001$ ja $110110100$ kymmenjärjestelmään.
	§ Muunna kymmenjärjestelmän luvut $25$ ja $520$ binäärijärjestelmään.
	§ Etsi laskimesi ohjekirjasta, miten muunnokset voidaan toteuttaa laskimella.
	§ Ohjelmoi laskimesi ohjelmointikielellä ohjelma, joka suorittaa muunnoksen kymmenjärjestelmästä binäärijärjestelmään.
	}
\end{tehtava}

\end{kotitehtavasivu}
