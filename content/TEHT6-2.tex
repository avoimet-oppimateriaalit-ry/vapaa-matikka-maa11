\begin{tehtavasivu}

\begin{tehtava}
    Tutki, onko Diofantoksen yhtälöllä ratkaisua.
    
    \alakohdat{
        § $7x + 5y = 3$
        § $5x + 85y = 42$
        § $6x + 51y = 100$
    }

    \begin{vastaus}
        \alakohdat{
            § On ratkaisu.
            § Ei ole ratkaisua.
            § Ei ole ratkaisua.
        }
    \end{vastaus}
    
\end{tehtava}

\begin{tehtava}
    Leirille osallistui $364$ nuorta. Oliko mahdollista majoittaa osallistujat $24$ ja $16$ hengen parakkeihin siten, että yksikään parakki ei jäänyt vajaaksi?
    
    \begin{vastaus}
        Ei ole mahdollista.
    \end{vastaus}
    
\end{tehtava}

\begin{tehtava}
    Määritä Diofantoksen yhtälön jokin ratkaisu.
    
    \alakohdat{
        § $14x + 49y = \syt(14, 49)$
        § $56x + 72y = \syt(56, 72)$
    }

    \begin{vastaus}
        \alakohdat{
            § Esimerkiksi $x = -3$ ja $y = 1$.
            § Esimerkiksi $x = -5$ ja $y = 4$.
        }
    \end{vastaus}
    
\end{tehtava}

\begin{tehtava}
    Määritä Diofantoksen yhtälön jokin ratkaisu.
    
    \alakohdat{
        § $56x + 72y = 40$
        § $24x + 138y = -24$
    }

    \begin{vastaus}
        \alakohdat{
            § Esimerkiksi $x = -7$ ja $y = 6$.
            § Esimerkiksi $x = -1$ ja $y = 0$.
        }
    \end{vastaus}
    
\end{tehtava}

\begin{tehtava}
    Tutki, onko suoralla
    \alakohdat{
        § $26x + 91y + 10 = 0$
        § $529x + 621y - 92 = 0$
    }
    pisteitä, joiden molemmat koordinaatit ovat kokonaislukuja.

    \begin{vastaus}
        \alakohdat{
            § Ei ole.
            § On, esimerkiksi piste $(-1, 1)$.
        }
    \end{vastaus}
    
\end{tehtava}

\begin{tehtava}
    Määritä Diofantoksen yhtälön kaikki ratkaisut.
    
    \alakohdat{
        § $2x + 3y = 1$
        § $2x + 3y = 7$
    }

    \begin{vastaus}
        \alakohdat{
            § $x = -1 + 3n$ ja $y = 1 - 2n$, $n\in\zz$.
            § $x = 2 + 3n$ ja $y = 1 - 2n$, $n\in\zz$.
        }
    \end{vastaus}
    
\end{tehtava}

\begin{tehtava}
    Määritä Diofantoksen yhtälön $45x + 21y = -6$ kaikki ratkaisut.

    \begin{vastaus}
        $x = -2 + 7n$ ja $y = 4 - 15n$, $n\in\zz$.
    \end{vastaus}
    
\end{tehtava}

\begin{tehtava}
    Määritä Diofantoksen yhtälön $13509x + 10203y = 228$ kaikki ratkaisut.
    
    \begin{vastaus}
        $x = 105 + 179n$ ja $y = -139 - 237n$, $n\in\zz$.
    \end{vastaus}
    
\end{tehtava}

\begin{tehtava}
    Keksi Diofantoksen yhtälö, jolla
    \alakohdat{
        § ei ole ratkaisua
        § on äärettömän monta ratkaisua.
    }

    \begin{vastaus}
        \alakohdat{
            § Esimerkiksi $6x + 8y = 1$
            § Esimerkiksi $13x + 24y = 47$
        }
    \end{vastaus}
    
\end{tehtava}

\begin{tehtava}
    Määritä Diofantoksen yhtälön $63x + 279y = 450$ kaikki ratkaisut. Mitkä niistä toteuttavat ehdon $|x| + |y| < 25$?
    
    \begin{vastaus}
        Ratkaisut ovat $x = 16 + 31n$ ja $y = -2 - 7n$, $n\in\zz$. Näistä ehdon $|x| + |y| < 25$ toteuttaa ratkaisu $x = -15$ ja $y = 5$ sekä ratkaisut $x = 16$ ja $y = -2$.
    \end{vastaus}
    
\end{tehtava}

\begin{tehtava}
    Käytettävissä on 8 gramman ja 12 gramman punnuksia. Kuinka monta kummankinlaista punnusta tarvitaan, jotta punnusten kokonaismassaksi tulisi 100 grammaa? Selvitä kaikki vaihtoehdot.
    
    \begin{vastaus}
        Vaihtoehdot kun 8 gramman punnuksien määrä on $x$ ja 12 gramman punnuksien määrä on $y$:
        \luettelo{
            § $x = 2$ ja $y = 7$
            § $x = 5$ ja $y = 5$
            § $x = 8$ ja $y = 3$
            § $x = 11$ ja $y = 1$.
        }
    \end{vastaus}
    
\end{tehtava}

\begin{tehtava}
    Ruhtinas jakoi $63$ yhtä suurta kekoa hedelmiä sekä $7$ erillistä hedelmää tasan $23$ matkalaiselle. Kuinka monta hedelmää kussakin keossa oli? Vihje: Tutki yhtälöä $63x + 7 = 23y$. (Mahavira, v. 850)
    
    \begin{vastaus}
        Mikä tahansa hedelmien määrä muotoa $5 + 23n$, missä $n\in\zz$ ja $n\geq 0$, on mahdollinen.
    \end{vastaus}
    
\end{tehtava}

\begin{tehtava}
    % Tarkistettu (Topi Talvitie, 9.11.2013)
    Olkoot $a$, $b$, $c$ ja $d$ positiivisia kokonaislukuja. Osoita, että yhtälöllä $ax+by+cz=d$ on kokonaislukuratkaisu, jos ja vain jos luku $d$ on jaollinen lukujen $a, b$ ja $c$ suurimmalla yhteisellä tekijällä.
\end{tehtava}

\end{tehtavasivu}

% -----


\begin{kotitehtavasivu}

\begin{tehtava}
    Tutki, onko Diofantoksen yhtälöllä ratkaisua.
    
    \alakohdat{
        § $9x + 6y = 72$
        § $12x + 10y = 323$
        § $14x + 35y = -91$
    }

    \begin{vastaus}
        \alakohdat{
            § On ratkaisu.
            § Ei ole ratkaisua.
            § On ratkaisu.
        }
    \end{vastaus}
    
\end{tehtava}

\begin{tehtava}
    Emilia sai valmistujaislahjaksi $200$ euron lahjakortin erääseen keramiikkapajaan. Hän osti pajasta $27$ euron hintaisia kynttilänjalkoja ja $15$ euron hintaisia jälkiruokalautasia. Hän maksoi ostoksensa lahjakortilla ja sai rahaa takaisin $12$ euroa. Laskiko myyjä oikein?
    
    \begin{vastaus}
        Ei laskenut.
    \end{vastaus}
    
\end{tehtava}

\begin{tehtava}
    Määritä Diofantoksen yhtälön jokin ratkaisu.

    \alakohdat{
        § $59x + 12y = \syt(59, 12)$
        § $119x + 272y = \syt(119, 272)$
    }
    
    \begin{vastaus}
        \alakohdat{
            § Esimerkiksi $x = 1$ ja $y = -4$.
            § Esimerkiksi $x = -9$ ja $y = 4$.
        }
    \end{vastaus}
    
\end{tehtava}

\begin{tehtava}
    Määritä Diofantoksen yhtälön jokin ratkaisu.

    \alakohdat{
        § $36x + 16y = 28$
        § $221x + 35y = 2$
    }
    
    \begin{vastaus}
        \alakohdat{
            § Esimerkiksi $x = -1$ ja $y = 4$.
            § Esimerkiksi $x = -3$ ja $y = 19$.
        }
    \end{vastaus}
    
\end{tehtava}

\begin{tehtava}
    Määritä Diofantoksen yhtälön kaikki ratkaisut.
    \alakohdat{
        § $2x + 6y = 2$
        § $2x + 6y = -10$
    }

    \begin{vastaus}
        \alakohdat{
            § $x = 1 + 3n$ ja $y = -n$, $n\in\zz$
            § $x = -5 + 3n$ ja $y = -n$, $n\in\zz$
        }
    \end{vastaus}
    
\end{tehtava}

\begin{tehtava}
    Määritä Diofantoksen yhtälön $35x + 84y = 14$ kaikki ratkaisut.
    
    \begin{vastaus}
        $x = -2 + 12n$ ja $y = 1 - 5n$, $n\in\zz$
    \end{vastaus}
    
\end{tehtava}

\begin{tehtava}
    Määritä Diofantoksen yhtälön $11925x + 3843y = -117$ kaikki ratkaisut.
    
    \begin{vastaus}
        $x = -10 + 427n$ ja $y = 31 - 1325n$, $n\in\zz$
    \end{vastaus}
    
\end{tehtava}

\begin{tehtava}
    Määritä Diofantoksen yhtälön $168x + 204y = 24$ kaikki ratkaisut. Mille ratkaisuille pätee $-50 \le x \le 0$ ja $y > 10$?
    
    \begin{vastaus}
        Kaikki ratkaisut ovat $x = 5 + 17n$ ja $y = -4 - 14n$, missä $n\in\zz$. Niistä ehdot $-50 \le x \le 0$ ja $y > 10$ toteuttaa ratkaisut $x = -29$ ja $y = 24$ sekä $x = -46$ ja $y = 38$.
    \end{vastaus}
    
\end{tehtava}

\begin{tehtava}
    Esitä luku $100$ kahden positiivisen kokonaisluvun summana niin, että toinen luvuista on jaollinen luvulla $7$ ja toinen luvulla $11$. (Euler, v. 1770)
    
    \begin{vastaus}
        $100 = 56 + 44$, missä $56 = 8 \cdot 7$ ja $44 = 4 \cdot 11$.
    \end{vastaus}
    
\end{tehtava}

\begin{tehtava}
    % Miksi tämä on lineaaristen Diofantoksen yhtälöiden luvussa?
    Yhtiön kauppavoitto 150 mk on jaettava tasan osakkaille. Jos osakkaita olisi ollut 5 enemmän, olisi jokainen saanut 5 mk vähemmän. Montako osakasta oli yhtiössä?  [YO 1874 tehtävä 6]
    
    \begin{vastaus}
        10 osakasta
    \end{vastaus}
    
\end{tehtava}

\begin{tehtava}
    Sata lyhdettä viljaa jaetaan sadalle henkilölle niin, että kukin mies saa $3$ lyhdettä, nainen $2$ lyhdettä ja lapsi puoli lyhdettä. Kuinka monta miestä, naista ja lasta on? (Alcuin Yorkilainen, v. 775)
    
    \begin{vastaus}
        Merkitään miesten lukumäärää kirjaimella $a$, naisten lukumäärää kirjaimella $b$ ja lasten lukumäärää kirjaimella $c$. Mahdolliset vastaukset ovat
        \luettelo{
            § $a = 20$, $b = 0$ ja $c = 80$
            § $a = 17$, $b = 5$ ja $c = 78$
            § $a = 14$, $b = 10$ ja $c = 76$
            § $a = 11$, $b = 15$ ja $c = 74$
            § $a = 8$, $b = 20$ ja $c = 72$
            § $a = 5$, $b = 25$ ja $c = 70$
            § $a = 2$, $b = 30$ ja $c = 68$.
        }
    \end{vastaus}
    
\end{tehtava}

\begin{tehtava} %(Lisämateriaalia.)
    % Tarkistettu (Topi Talvitie, 10.11.2013)
    * Osoita suoralla sijoituksella, että $a=2$, $b=4$, $c=3$ ja $d=1$ on yksi Kexleruksen viiniongelman ratkaisuista.
\end{tehtava}

\begin{tehtava} %(Lisämateriaalia.)
    * Ratkaise Kexleruksen viiniongelma, kun asetetaan $b=0$ ja $d=0$.

    \begin{vastaus}
        $a = 4$ ja $c = 6$
    \end{vastaus}
    
\end{tehtava}

\begin{tehtava} %(Lisämateriaalia.)
    % Tarkistettu (Topi Talvitie, 10.11.2013)
    * Osoita, että Kexleruksen viiniongelmalla ei ole ratkaisuja, jos asetetaan $a=0$ ja $d=0$.
\end{tehtava}

\end{kotitehtavasivu}
