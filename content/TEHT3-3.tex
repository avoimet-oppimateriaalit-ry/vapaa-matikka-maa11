\begin{tehtavasivu}

\begin{tehtava}
	Olkoon lause
	+$T(x)$: ''opiskelija $x$ on täysi-ikäinen'' ja
	+perusjoukko kaikki koulun opiskelijat. Suomenna lause.
	\alakohdat{
	§ $\forall x T(x)$
	§ $\exists x T(x)$
	§ $\exists x \lnot T(x)$
	§ $\forall x \lnot T(x)$
	}%
	\begin{vastaus}
		\alakohdat{
			§ Kaikki koulun opiskelijat ovat täysi-ikäisiä.
			§ Ainakin yksi koulun opiskelija on täysi-ikäinen.
			§ Ainakin yksi koulun opiskelija ei ole täysi-ikäinen.
			§ Yksikään koulun opiskelija ei ole täysi-ikäinen.
		}
	\end{vastaus}
\end{tehtava}

\begin{tehtava}
	Suomenna lause. Onko lause tosi? Perustele.
	\alakohdat{
	§ $\forall x\in\rr (|x| > 0)$
	§ $\forall x\in\rr (|x| \ge 0)$
	§ $\exists x\in\nn (x^2 < 2)$
	§ $\exists x\in\nn (x^2 = 2)$
	}%
	\begin{vastaus}
		\alakohdat{
			§ Lause on epätosi, sillä $0 \in \mathbb{R}$ ja $|0| = 0$.
			§ Lause on tosi, sillä reaaliluvun itseisarvo on aina suurempaa tai yhtä suurta kuin $0$.
			§ Lause on tosi, sillä $1 \in \mathbb{N}$ ja $1^2 < 2$.
			§ Lause on epätosi, sillä yhtälön $x^2 = 2$ ratkaisut $\sqrt{2}$ ja $-\sqrt{2}$ eivät ole luonnollisia lukuja.
		}
	\end{vastaus}
\end{tehtava}

\begin{tehtava}
	Formalisoi lause. Onko lause tosi? Perustele.
	\alakohdat{
	§ Jokainen kokonaisluku on joko positiivinen tai
	negatiivinen.
	§ On olemassa sellainen kokonaisluku, jonka neliöjuuri on
	yhtä suuri kuin luku itse.
	§ Minkään kokonaisluvun neliö ei ole $7$.
	}%
	\begin{vastaus}
		\alakohdat{
			§ $\forall x \in \mathbb{R}(x \ne 0)$ Lause on epätosi, sillä $0 \in \mathbb{R}$.
			§ $\exists x \in \mathbb{Z}(\sqrt{x} = x)$ Lause on tosi, sillä $1 \in \mathbb{Z}$ ja $\sqrt{1} = 1$.
			§ $\forall x \in \mathbb{Z}(x^2 \ne 7)$ Lause on tosi, sillä yhtälön $x^2 = 7$ ratkaisut $\sqrt{7}$ ja $-\sqrt{7}$ eivät ole kokonaislukuja.
		}
	\end{vastaus}
\end{tehtava}

\begin{tehtava}
	Osoita, että yhtälö $\sqrt{x^2} = x$ ei pidä paikkaansa
	kaikilla reaaliluvuilla.%
	\begin{vastaus}
		Koska $-1 \in \mathbb{R}$ ja $\sqrt{(-1)^2} = \sqrt{1} = 1 \ne -1$, niin yhtälö $\forall x \in \mathbb{R}(\sqrt{x^2} = x)$ on epätosi.
	\end{vastaus}
\end{tehtava}

\begin{tehtava}
	Jalkapallojoukkue on lähdössä turnausmatkalle. Olkoon
	$P(x, y)$ avoin lause ''pelaajalla $x$ on pelaajan $y$
	puhelinnumero''. Suomenna lause.
	\alakohdat{
	§ $\forall x \exists y P(x, y)$
	§ $\exists x \forall y P(x, y)$
	§ $\exists y \forall x P(x, y)$
	}%
	\begin{vastaus}
		\alakohdat{
			§ Jokaisella on jonkun puhelinnumero.
			§ Ainakin yhdellä pelaajalla on kaikkien puhelinnumerot.
			§ Ainakin yhden pelaajan puhelinnumero on kaikilla pelaajilla.
		}
	\end{vastaus}
\end{tehtava}

\begin{tehtava}
	Formalisoi lause. Onko lause tosi?
	\alakohdat{
	§ Positiivisten kokonaislukujen joukossa on pienin alkio.
	§ Negatiivisten kokonaislukujen joukossa on pienin alkio.
	}%
	\begin{vastaus}
		\alakohdat{
			§ $\exists x \in \mathbb{Z}_{+} \forall y \in \mathbb{Z}_{+}(x \le y)$ Lause on tosi, sillä tämä pätee, kun $x = 1$.
			§ $\exists x \in \mathbb{Z}_{-} \forall y \in \mathbb{Z}_{-}(x \le y)$ Lause on epätosi, sillä jos $x \in \mathbb{Z}_{-}$, niin $x - 1 \in \mathbb{Z}_{-}$ ja $x - 1 < x$ $\forall x \in \mathbb{Z}_{-}$.
		}
	\end{vastaus}
\end{tehtava}

\begin{tehtava}
	Onko lause tosi? Perustele.
	\alakohdat{
	§ $\forall x\in \rr\exists y\in \rr (xy=1)$
	§ $\exists x\in \rr\forall y\in \rr (xy=y)$
	}%
	\begin{vastaus}
		\alakohdat{
			§ Lause on epätosi, sillä $0 \in \mathbb{R}$ ja $\neg \exists x \in \mathbb{R}(x \cdot 0 = 1)$.
			§ Lause on tosi, sillä luku $1$ on tällainen luku.
		}
	\end{vastaus}
\end{tehtava}

\begin{tehtava}
	Onko lause tosi? Perustele. % NVI
	\alakohdat{
	§ $\exists x\in \rr\exists y\in \rr (xy=x+y)$
	§ $\forall x\in \rr\exists y\in \rr (xy=x+y)$
	}%
	\begin{vastaus}
		\alakohdat{
			§ Lause on tosi, sillä $2 \in \mathbb{R}$ ja $2 \cdot 2 = 4 = 2 + 2$.
			§ Lause on epätosi, sillä jos $x = 1 \in \mathbb{R}$, niin täytyisi olla $1 \cdot y = 1 + y \Leftrightarrow y = 1 + y \Leftrightarrow 0 = 1$, mikä on ristiriita.
		}
	\end{vastaus}
\end{tehtava}

\begin{tehtava}
	Olkoon $M(x)$: ''$x$ on matemaatikko''. Formalisoi lause.%
	\alakohdat{
	§ Kaikki eivät ole matemaatikkoja.
	§ Joku ei ole matemaatikko.
	§ Ei ole olemassa matemaatikkoa.
	§ Kukaan ei ole matemaatikko.
	}%
	\begin{vastaus}
		\alakohdat{
			§ $\neg \forall x M(x)$
			§ $\exists x \neg M(x)$
			§ $\neg \exists x M(x)$
			§ $\forall x \neg M(x)$
		}
	\end{vastaus}
\end{tehtava}

\begin{tehtava}
	Muodosta lauseen negaatio.
	\alakohdat{
	§ $\forall x\in \zz (x < 12)$
	§ $\exists x\in \zz (x^2 = 12)$
	§ $\exists x\in \nn ((x^2=9) \land (x<5))$
	§ $\forall x\in \nn ((x=0)\lor (x\ge 1))$
	}%
	\begin{vastaus}
		\alakohdat{
			§ $\exists x \in \mathbb{Z}(x \ge 12)$
			§ $\forall x \in \mathbb{Z}(x^2 \ne 12)$
			§ $\forall x \in \mathbb{N}((x^2 \ne 9) \lor (x > 5))$
			§ $\exists x \in \mathbb{N}((x \ne 0) \land (x < 1))$
		}
	\end{vastaus}
\end{tehtava}

\begin{tehtava}
	Osoita, että lause $\exists a, b, c \in \zz_{+} (a^2 + b^2 = c^2)$ on tosi.%
	\begin{vastaus}
		Koska $3, 4, 5 \in \mathbb{Z}_{+}$ ja $3^2 + 4^2 = 9 + 16 = 25 = 5^2$, niin väite on tosi.
	\end{vastaus}
\end{tehtava}

\begin{tehtava}
	* Olkoot $M(x)$: ''$x$ on matemaatikko'' ja
	$I(x)$: ''$x$ on iloinen''. Formalisoi lause.
	\alakohdat{
	§ Kaikki ovat iloisia matemaatikkoja.
	§ Matemaatikot ovat iloisia.
	§ Kukaan matemaatikko ei ole iloinen.
	§ Kaikki matemaatikot eivät ole iloisia.
	}%
	\begin{vastaus}
		\alakohdat{
			§ $\forall x(M(x) \land I(x))$
			§ $\forall x(M(x) \Rightarrow I(x))$
			§ $\forall x(M(x) \Rightarrow \neg I(x))$
			§ $\exists x(M(x) \Rightarrow \neg I(x))$
		}
	\end{vastaus}
\end{tehtava}

\begin{tehtava}
	* Olkoot $K(x)$: ''$x$ on kampaaja'' ja
	$T(x, y)$: ''$x$ tekee $y$:lle kampauksen''. Formalisoi lause.
	\alakohdat{
	§ Kukaan kampaaja ei tee kampausta itselleen.
	§ Joku kampaaja tekee kampauksen niille kampaajille,
	jotka eivät tee kampausta itselleen.
	}%
	\begin{vastaus}
		\alakohdat{
			§ $\forall x(\neg T(x, x))$
			§ $\exists x \forall y(\neg T(y, y) \Rightarrow T(x, y))$
		}
	\end{vastaus}
\end{tehtava}


\end{tehtavasivu}

% -----


\begin{kotitehtavasivu}

\begin{tehtava}
	Olkoon perusjoukko $xy$-tason suorien joukko. Olkoot
	lauseet $N(x)$: ''suora on nouseva'' ja $L(x)$: ''suora on laskeva''.
	Suomenna lause. Onko lause tosi?
	\alakohdat{
	§ $\exists x L(x)$
	§ $\forall x (N(x) \lor L(x))$
	§ $\exists x (\lnot N(x) \land \lnot L(x))$
	§ $\forall x \lnot N(x)$
	}%
	\begin{vastaus}
		\alakohdat{
			§ On olemassa suora, joka on laskeva. Lause on tosi.
			§ Kaikki suorat ovat joko nousevia tai laskevia. Lause on epätosi.
			§ On olemassa suora, joka ei ole nouseva eikä laskeva. Lause on tosi.
			§ Mikään suora ei ole nouseva.
		}
	\end{vastaus}
\end{tehtava}

\begin{tehtava}
	Onko lause tosi? Perustele.
	\alakohdat{
	§ $\forall x\in\rr (x^2 \ge 0)$
	§ $\exists x\in\nn (x^2 = -9)$
	§ $\forall x\in\rr ((x-1)(x-3) \ge 0)$
	§ $\exists x\in\zz (-x^2 - 2 \le 0)$
	}%
	\begin{vastaus}
		\alakohdat{
			§ Lause on tosi, sillä reaaliluvun neliö ei ole koskaan negatiivinen.
			§ Lause on epätosi, sillä $\forall x \in \mathbb{N}(x^2 \ge 0)$.
			§ Lause on epätosi, sillä $2 \in \mathbb{R}$ ja $(2 - 1)(2 - 3) = -1 \cdot 1 = -1 < 0$.
			§ Lause on tosi, sillä $0 \in \mathbb{R}$ ja $-0^2 - 2 = -2 \le 0$.
		}
	\end{vastaus}
\end{tehtava}

\begin{tehtava}
	Onko lause tosi? Perustele.
	\alakohdat{
	§ $\exists x\in \rr (x^2 - 3x + 3 = 0)$
	§ $\forall x\in \rr (x^2 - 3x + 3 \ge 0)$
	}%
	\begin{vastaus}
		\alakohdat{
			§ Ratkaistaan funktion $x^2 - 3x + 3 = 0$ nollakohdat:
			\[x = \frac{-(-3) \pm \sqrt{(-3)^2 - 4 \cdot 1 \cdot 3}}{2 \cdot 1} = \frac{3 \pm \sqrt{-3}}{2}\]
			Koska diskriminantti $-3$ on negatiivinen, niin yhtälöllä ei ole ratkaisuja. Lisäksi koska yhtälön $x^2 - 3x + 3$ kuvaaja on ylöspäin aukeava paraabeli, niin pätee $x^2 - 3x + 3 > 0$ $\forall x \in \mathbb{R}$. Siis lause on epätosi.
			§ Koska $x^2 - 3x + 3 > 0$ $\forall x \in \mathbb{R}$, niin $\forall x \in \mathbb{R}(x^2 - 3x + 3 \ge 0)$. Siis lause on tosi.
		}
	\end{vastaus}
\end{tehtava}

\begin{tehtava}
	\alakohdat{
	§ Osoita, että yhtälö $\sqrt{xy} = \sqrt{x}\sqrt{y}$ ei
	pidä paikkaansa kaikilla reaaliluvuilla $x$ ja $y$.
	§ Osoita, että on olemassa sellaiset reaaliluvut $x$ ja
	$y$, että yhtälö $\sqrt{xy} = \sqrt{x}\sqrt{y}$ pitää paikkansa.
	§ Millä ehdolla yhtälö $\sqrt{xy} = \sqrt{x}\sqrt{y}$
	pitää yleisesti paikkansa?
	}%
	\begin{vastaus}
		\alakohdat{
			§ Jos $x = y = -1 \in \mathbb{R}$, niin $\sqrt{xy} = \sqrt{(-1) \cdot (-1)} = \sqrt{1} = 1$, mutta $\sqrt{x} \cdot \sqrt{y} = \sqrt{-1} \cdot \sqrt{-1}$, jota ei ole määritelty reaalilukujen joukossa.
			§ Kun $x = y = 1 \in \mathbb{R}$, niin $\sqrt{xy} = \sqrt{1 \cdot 1} = \sqrt{1} = 1 = 1 \cdot 1 = \sqrt{1} \cdot \sqrt{1} = \sqrt{x} \cdot \sqrt{y}$.
			§ $x \ge 0 \land y \ge 0$
		}
	\end{vastaus}
\end{tehtava}

\begin{tehtava}
	Olkoon $S(x, y)$: ''$x$ on suorittanut kurssin $y$''.
	Formalisoi lause.
	\alakohdat{
	§ Joku on suorittanut kaikki kurssit.
	§ Jokainen on suorittanut ainakin yhden kurssin.
	§ Kukaan ei ole suorittanut kaikkia kursseja.
	§ On olemassa kurssi, jota kukaan ei ole suorittanut.
	}%
	\begin{vastaus}
		\alakohdat{
			§ $\exists x \forall y S(x, y)$
			§ $\forall x \exists y S(x, y)$
			§ $\forall x \exists y \neg S(x, y)$
			§ $\exists y \forall x \neg S(x, y)$
		}
	\end{vastaus}
\end{tehtava}

\begin{tehtava}
	Onko lause tosi? Perustele.
	\alakohdat{
	§ $\forall x\in \zz_{+} \exists y\in \zz_{+} (x = \sqrt{y})$
	§ $\exists y\in \rr \forall x\in \rr (x^2 - 4 > y)$
	}%
	\begin{vastaus}
		\alakohdat{
			§ Lause on tosi, sillä jos $x \in \mathbb{Z}_{+}$ ja $y = x^2 \in \mathbb{Z}_{+}$, niin $\sqrt{y} = \sqrt{x^2} = |x| = x$.
			§ Koska $\forall x \in \mathbb{R}(x \ge 0 > -1)$, niin $\forall x \in \mathbb{R}(x > -1) \Leftrightarrow \forall x \in \mathbb{R}(x - 4 > -5)$. Valitaan $y = -5$. Nyt pätee $\exists y \in \mathbb{R} \forall x \in \mathbb{R}(x - 4 > y)$ eli lause on tosi.
		}
	\end{vastaus}
\end{tehtava}

\begin{tehtava}
	Ilmaise suomen kielellä lauseen negaatio kahdella eri
	tavalla soveltamalla kvanttorien negaatioiden loogisesti
	ekvivalentteja muotoja.
	\alakohdat{
	§ Jokainen opiskelija saa tästä kurssista arvosanan 10.
	§ On olemassa opiskelija, joka saa tästä kurssista
	arvosanan 10.
	}%
	\begin{vastaus}
		\alakohdat{
			§ On olemassa ainakin yksi opiskelija, joka ei saa tästä kurssista arvosanaa 10. ($\exists x \neg K(x)$) Kaikki opiskelijat eivät saa tästä kurssista arvosanaa 10. ($\neg \forall x K(x)$)
			§ Ei ole olemassa opiskelijaa, joka saisi tästä kurssista arvosanan 10. ($\neg \exists x K(x)$) Kukaan opiskelija ei saa tästä kurssista arvosanaa 10. ($\forall x \neg K(x)$)
		}
	\end{vastaus}
\end{tehtava}

\begin{tehtava}
	Kirjoita lause toisin.
	\alakohdat{
	§ $\lnot \forall x \lnot P(x)$
	§ $\lnot \exists x (P(x) \lor Q(x))$
	}%
	\begin{vastaus}
		\alakohdat{
			§ $\exists x P(x)$
			§ $\forall x (\neg P(x) \land \neg Q(x))$
		}
	\end{vastaus}
\end{tehtava}

\begin{tehtava}
	Muodosta lauseen negaatio. Onko negaatio tosi?
	\alakohdat{
	§ $\forall x\in ]1, \infty [ (\sqrt{x < x})$
	§ $\exists x\in \qq (x^3 = 5)$
	§ $\forall n \in \zz \exists m \in \zz ((n = 2m) \lor (n =
	2m+1))$
	}%
	\begin{vastaus}
		\alakohdat{
			§ $\exists x \in {]-1, \infty[}(\sqrt{x} \ge x)$ Negaatio on epätosi, koska
			\begin{align*}
			& \sqrt{x} > 1 && \parallel \cdot \sqrt{x} > 0 \\
			\Leftrightarrow & \sqrt{x} \cdot \sqrt{x} > \sqrt{x} && \\
			\Leftrightarrow & x > \sqrt{x} \forall x \in {]1, \infty[} &&
			\end{align*}
			§ $\forall x \in \mathbb{Q}(x^3 \ne 5)$ Negaatio on tosi, sillä yhtälön $x^3 = 5$ ainoa ratkaisu on $x = \sqrt[3]{5} \notin \mathbb{Q}$.
			§ $\forall n \in \mathbb{Z} \forall m \in \mathbb{Z} ((n \ne 2m) \land (n \ne 2m + 1))$ Valitaan $n = 2 \in \mathbb{Z}$ ja $m = 1 \in \mathbb{Z}$. Tällöin $n = 2 = 2 \cdot 1 = 2m$, joten negaatio on epätosi.
		}
	\end{vastaus}
\end{tehtava}

\begin{tehtava}
	Funktiota $f\colon X\to Y$ voidaan ajatella kahden
	muuttujan avoimena lauseena $P(x, y)$: ''$f(x) = y$'', missä
	$x$ kuuluu määrittelyjoukkoon $X$ ja $y$ maalijoukkoon $Y$.
	Funktiolta edellytetään lisäksi, että
	\luettelo{
	§ $\forall x \exists y P(x, y)$, ja
	§ $\lnot (\exists x \exists y \exists z (P(x, y) \land P(x, z)
	\land (y \neq z)))$.
	}
	Tulkitse sanallisesti tai kuvaa käyttäen, mitä tämä määritelmä
	tarkoittaa.%
	\begin{vastaus}
		\alakohdat{
			§ Funktio $f : X \to Y$ saa jokaisessa pisteessä $x \in X$ jonkin arvon.
			§ Funktio $f : X \to Y$ ei saa kahta eri arvoa missään pisteessä $x \in X$.
		}
	\end{vastaus}
\end{tehtava}

\begin{tehtava}
	* Olkoot $S(x, y)$: ''$x$ on suorittanut
	kurssin $y$'' ja $L(x)$: ''$x$ on lukiolainen''. Formalisoi lause.
	\alakohdat{
	§ Joku lukiolainen on suorittanut kaikki kurssit.
	§ Kukaan lukiolainen ei ole suorittanut kaikkia kursseja.
	}%
\end{tehtava}

\begin{tehtava}
	* Olkoot $S(x, y)$: ''$x$ on suorittanut
	kurssin $y$'', $L(x)$: ''$x$ on lukiolainen'' ja $M(y)$: ''$y$ on
	pitkän matematiikan kurssi''. Formalisoi lause.
	\alakohdat{
	§ Joku lukiolainen ei ole suorittanut yhtään pitkän
	matematiikan kurssia.
	§ Joku lukiolainen on suorittanut pitkän matematiikan
	kurssin.
	§ Jokaisella lukiolaisella on pitkän matematiikan kurssi
	suoritettuna.
	§ On olemassa pitkän matematiikan kurssi, jota kukaan
	lukiolainen ei ole suorittanut.
	}%
\end{tehtava}

\begin{tehtava}
	* Määritä kaikki funktiot $f\colon X\to Y$,
	kun $X=\{a, b, c\}$ ja $Y=\{1, 2\}$.%
\end{tehtava}

\begin{tehtava}
	* Määritä kaikki funktiot $f\colon X\to Y$,
	kun $X=\emptyset$ ja $Y\neq \emptyset$.%
\end{tehtava}

\begin{tehtava}
	* Määritä kaikki funktiot $f\colon X\to Y$,
	kun $X\neq \emptyset$ ja $Y= \emptyset$.%
\end{tehtava}

\end{kotitehtavasivu}


%%%%%%%%%%%%%%%%%%%%%%%%%%%%%%%FIX ME, linkki toimimaton  %%%%%%%
%\begin{tehtava}
%	Tutustu logiikkapohjaiseen Prolog-ohjelmointikieleen\\
%	\href{http://www.cs.helsinki.fi/u/wikla/OKP/OppaatK07/prolog.html}
%	{{\tt http://www.cs.helsinki.fi/u/wikla/OKP/OppaatK07/prolog.html}}
%	
%	Lataa koneellesi Prolog-tulkki \href{http://www.gprolog.org/}
%	{{\tt http://www.gprolog.org/}}
%	ja kokeile Prolog-ohjelmointia.%
%\end{tehtava}
