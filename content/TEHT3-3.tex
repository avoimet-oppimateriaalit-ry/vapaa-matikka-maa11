\begin{tehtavasivu}

\begin{tehtava}
	Olkoon lause
	+$T(x)$: ''opiskelija $x$ on täysi-ikäinen'' ja
	+perusjoukko kaikki koulun opiskelijat. Suomenna lause.
	\begin{alakohdat}
	\alakohta{$\forall x T(x)$}
	\alakohta{$\exists x T(x)$}
	\alakohta{$\exists x \lnot T(x)$}
	\alakohta{$\forall x \lnot T(x)$}
	\end{alakohdat}%
\end{tehtava}

\begin{tehtava}
	Suomenna lause. Onko lause tosi? Perustele.
	\begin{alakohdat}
	\alakohta{$\forall x\in\rr (|x| > 0)$}
	\alakohta{$\forall x\in\rr (|x| \ge 0)$}
	\alakohta{$\exists x\in\nn (x^2 < 2)$}
	\alakohta{$\exists x\in\nn (x^2 = 2)$}
	\end{alakohdat}%
\end{tehtava}

\begin{tehtava}
	Formalisoi lause. Onko lause tosi? Perustele.
	\begin{alakohdat}
	\alakohta{Jokainen kokonaisluku on joko positiivinen tai
	negatiivinen.}
	\alakohta{On olemassa sellainen kokonaisluku, jonka neliöjuuri on
	yhtä suuri kuin luku itse.}
	\alakohta{Minkään kokonaisluvun neliö ei ole $7$.}
	\end{alakohdat}%
\end{tehtava}

\begin{tehtava}
	Osoita, että yhtälö $\sqrt{x^2} = x$ ei pidä paikkaansa
	kaikilla reaaliluvuilla.%
\end{tehtava}

\begin{tehtava}
	Jalkapallojoukkue on lähdössä turnausmatkalle. Olkoon
	$P(x, y)$ avoin lause ''pelaajalla $x$ on pelaajan $y$
	puhelinnumero''. Suomenna lause.
	\begin{alakohdat}
	\alakohta{$\forall x \exists y P(x, y)$}
	\alakohta{$\exists x \forall y P(x, y)$}
	\alakohta{$\exists y \forall x P(x, y)$}
	\end{alakohdat}%
\end{tehtava}

\begin{tehtava}
	Formalisoi lause. Onko lause tosi?
	\begin{alakohdat}
	\alakohta{Positiivisten kokonaislukujen joukossa on pienin alkio.}
	\alakohta{Negatiivisten kokonaislukujen joukossa on pienin alkio.}
	\end{alakohdat}%
\end{tehtava}

\begin{tehtava}
	Onko lause tosi? Perustele.
	\begin{alakohdat}
	\alakohta{$\forall x\in \rr\exists y\in \rr (xy=1)$}
	\alakohta{$\exists x\in \rr\forall y\in \rr (xy=y)$}
	\end{alakohdat}%
\end{tehtava}

\begin{tehtava}
	Onko lause tosi? Perustele. % NVI
	\begin{alakohdat}
	\alakohta{$\exists x\in \rr\exists y\in \rr (xy=x+y)$}
	\alakohta{$\forall x\in \rr\exists y\in \rr (xy=x+y)$}
	\end{alakohdat}%
\end{tehtava}

\begin{tehtava}
	Olkoon $M(x)$: ''$x$ on matemaatikko''. Formalisoi lause.%
	\begin{alakohdat}
	\alakohta{Kaikki eivät ole matemaatikkoja.}
	\alakohta{Joku ei ole matemaatikko.}
	\alakohta{Ei ole olemassa matemaatikkoa.}
	\alakohta{Kukaan ei ole matemaatikko.}
	\end{alakohdat}%
\end{tehtava}

\begin{tehtava}
	Muodosta lauseen negaatio.
	\begin{alakohdat}
	\alakohta{$\forall x\in \zz (x < 12)$}
	\alakohta{$\exists x\in \zz (x^2 = 12)$}
	\alakohta{$\exists x\in \nn ((x^2=9) \land (x<5))$}
	\alakohta{$\forall x\in \nn ((x=0)\lor (x\ge 1))$}
	\end{alakohdat}%
\end{tehtava}

\begin{tehtava}
	Osoita, että lause $\exists a, b, c \in \zz_{+} (a^2 + b^2 = c^2)$ on tosi.%
\end{tehtava}

\begin{tehtava}
	* Olkoot $M(x)$: ''$x$ on matemaatikko'' ja
	$I(x)$: ''$x$ on iloinen''. Formalisoi lause.
	\begin{alakohdat}
	\alakohta{Kaikki ovat iloisia matemaatikkoja.}
	\alakohta{Matemaatikot ovat iloisia.}
	\alakohta{Kukaan matemaatikko ei ole iloinen.}
	\alakohta{Kaikki matemaatikot eivät ole iloisia.}
	\end{alakohdat}%
\end{tehtava}

\begin{tehtava}
	* Olkoot $K(x)$: ''$x$ on kampaaja'' ja
	$T(x, y)$: ''$x$ tekee $y$:lle kampauksen''. Formalisoi lause.
	\begin{alakohdat}
	\alakohta{Kukaan kampaaja ei tee kampausta itselleen.}
	\alakohta{Joku kampaaja tekee kampauksen niille kampaajille,
	jotka eivät tee kampausta itselleen.}
	\end{alakohdat}%
\end{tehtava}


\end{tehtavasivu}

% -----
\setcounter{tehtava}{0}

\begin{kotitehtavasivu}

\begin{tehtava}
	Olkoon perusjoukko $xy$-tason suorien joukko. Olkoot
	lauseet $N(x)$: ''suora on nouseva'' ja $L(x)$: ''suora on laskeva''.
	Suomenna lause. Onko lause tosi?
	\begin{alakohdat}
	\alakohta{$\exists x L(x)$}
	\alakohta{$\forall x (N(x) \lor L(x))$}
	\alakohta{$\exists x (\lnot N(x) \land \lnot L(x))$}
	\alakohta{$\forall x \lnot N(x)$}
	\end{alakohdat}%
\end{tehtava}

\begin{tehtava}
	Onko lause tosi? Perustele.
	\begin{alakohdat}
	\alakohta{$\forall x\in\rr (x^2 \ge 0)$}
	\alakohta{$\exists x\in\nn (x^2 = -9)$}
	\alakohta{$\forall x\in\rr ((x-1)(x-3) \ge 0)$}
	\alakohta{$\exists x\in\zz (-x^2 - 2 \le 0)$}
	\end{alakohdat}%
\end{tehtava}

\begin{tehtava}
	Onko lause tosi? Perustele.
	\begin{alakohdat}
	\alakohta{$\exists x\in \rr (x^2 - 3x + 3 = 0)$}
	\alakohta{$\forall x\in \rr (x^2 - 3x + 3 \ge 0)$}
	\end{alakohdat}%
\end{tehtava}

\begin{tehtava}
	\begin{alakohdat}
	\alakohta{Osoita, että yhtälö $\sqrt{xy} = \sqrt{x}\sqrt{y}$ ei
	pidä paikkaansa kaikilla reaaliluvuilla $x$ ja $y$.}
	\alakohta{Osoita, että on olemassa sellaiset reaaliluvut $x$ ja
	$y$, että yhtälö $\sqrt{xy} = \sqrt{x}\sqrt{y}$ pitää paikkansa.}
	\alakohta{Millä ehdolla yhtälö $\sqrt{xy} = \sqrt{x}\sqrt{y}$
	pitää yleisesti paikkansa?}
	\end{alakohdat}%
\end{tehtava}

\begin{tehtava}
	Olkoon $S(x, y)$: ''$x$ on suorittanut kurssin $y$''.
	Formalisoi lause.
	\begin{alakohdat}
	\alakohta{Joku on suorittanut kaikki kurssit.}
	\alakohta{Jokainen on suorittanut ainakin yhden kurssin.}
	\alakohta{Kukaan ei ole suorittanut kaikkia kursseja.}
	\alakohta{On olemassa kurssi, jota kukaan ei ole suorittanut.}
	\end{alakohdat}%
\end{tehtava}

\begin{tehtava}
	Onko lause tosi? Perustele.
	\begin{alakohdat}
	\alakohta{$\forall x\in \zz_{+} \exists y\in \zz_{+} (x = \sqrt{y})$}
	\alakohta{$\exists y\in \rr \forall x\in \rr (x^2 - 4 > y)$}
	\end{alakohdat}%
\end{tehtava}

\begin{tehtava}
	Ilmaise suomen kielellä lauseen negaatio kahdella eri
	tavalla soveltamalla kvanttorien negaatioiden loogisesti
	ekvivalentteja muotoja.
	\begin{alakohdat}
	\alakohta{Jokainen opiskelija saa tästä kurssista arvosanan 10.}
	\alakohta{On olemassa opiskelija, joka saa tästä kurssista
	arvosanan 10.}
	\end{alakohdat}%
\end{tehtava}

\begin{tehtava}
	Kirjoita lause toisin.
	\begin{alakohdat}
	\alakohta{$\lnot \forall x \lnot P(x)$}
	\alakohta{$\lnot \exists x (P(x) \lor Q(x))$}
	\end{alakohdat}%
\end{tehtava}

\begin{tehtava}
	Muodosta lauseen negaatio. Onko negaatio tosi?
	\begin{alakohdat}
	\alakohta{$\forall x\in ]1, \infty [ (\sqrt{x} < x)$}
	\alakohta{$\exists x\in \qq (x^3 = 5)$}
	\alakohta{$\forall n \in \zz \exists m \in \zz ((n = 2m) \lor (n =
	2m+1))$}
	\end{alakohdat}%
\end{tehtava}

\begin{tehtava}
	Funktiota $f\colon X\to Y$ voidaan ajatella kahden
	muuttujan avoimena lauseena $P(x, y)$: ''$f(x) = y$'', missä
	$x$ kuuluu määrittelyjoukkoon $X$ ja $y$ maalijoukkoon $Y$.
	Funktiolta edellytetään lisäksi, että
	\begin{itemize}
	\item $\forall x \exists y P(x, y)$, ja
	\item $\lnot (\exists x \exists y \exists z (P(x, y) \land P(x, z)
	\land (y \neq z)))$.
	\end{itemize}
	Tulkitse sanallisesti tai kuvaa käyttäen, mitä tämä määritelmä
	tarkoittaa.%
\end{tehtava}

\begin{tehtava}
	* Olkoot $S(x, y)$: ''$x$ on suorittanut
	kurssin $y$'' ja $L(x)$: ''$x$ on lukiolainen''. Formalisoi lause.
	\begin{alakohdat}
	\alakohta{Joku lukiolainen on suorittanut kaikki kurssit.}
	\alakohta{Kukaan lukiolainen ei ole suorittanut kaikkia kursseja.}
	\end{alakohdat}%
\end{tehtava}

\begin{tehtava}
	* Olkoot $S(x, y)$: ''$x$ on suorittanut
	kurssin $y$'', $L(x)$: ''$x$ on lukiolainen'' ja $M(y)$: ''$y$ on
	pitkän matematiikan kurssi''. Formalisoi lause.
	\begin{alakohdat}
	\alakohta{Joku lukiolainen ei ole suorittanut yhtään pitkän
	matematiikan kurssia.}
	\alakohta{Joku lukiolainen on suorittanut pitkän matematiikan
	kurssin.}
	\alakohta{Jokaisella lukiolaisella on pitkän matematiikan kurssi
	suoritettuna.}
	\alakohta{On olemassa pitkän matematiikan kurssi, jota kukaan
	lukiolainen ei ole suorittanut.}
	\end{alakohdat}%
\end{tehtava}

\begin{tehtava}
	* Määritä kaikki funktiot $f\colon X\to Y$,
	kun $X=\{a, b, c\}$ ja $Y=\{1, 2\}$.%
\end{tehtava}

\begin{tehtava}
	* Määritä kaikki funktiot $f\colon X\to Y$,
	kun $X=\emptyset$ ja $Y\neq \emptyset$.%
\end{tehtava}

\begin{tehtava}
	* Määritä kaikki funktiot $f\colon X\to Y$,
	kun $X\neq \emptyset$ ja $Y= \emptyset$.%
\end{tehtava}

\end{kotitehtavasivu}


%%%%%%%%%%%%%%%%%%%%%%%%%%%%%%%FIX ME, linkki toimimaton  %%%%%%%
%\begin{tehtava}
%	Tutustu logiikkapohjaiseen Prolog-ohjelmointikieleen\\
%	\href{http://www.cs.helsinki.fi/u/wikla/OKP/OppaatK07/prolog.html}
%	{{\tt http://www.cs.helsinki.fi/u/wikla/OKP/OppaatK07/prolog.html}}
%	
%	Lataa koneellesi Prolog-tulkki \href{http://www.gprolog.org/}
%	{{\tt http://www.gprolog.org/}}
%	ja kokeile Prolog-ohjelmointia.%
%\end{tehtava}
