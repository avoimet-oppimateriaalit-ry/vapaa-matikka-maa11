\begin{tehtavasivu}

	\begin{tehtava}
		\alakohdat{
			§ Laske neljän peräkkäisen kokonaisluvun summa.
			Toista lasku useilla neljän peräkkäisen kokonaisluvun
			joukoilla. Mitä havaitset summasta?
			§ Jos neljän peräkkäisen kokonaisluvun joukon pienin
			luku on $m$, niin millainen esitysmuoto on kolmella
			muulla luvulla?
			§ Todista a-kohdan tulos matemaattisesti käyttäen
			hyväksi b-kohdan esitysmuotoja.
		}
		\begin{vastaus}
			\alakohdat{
				§ $1+2+3+4=10$, $2+3+4+5=14$, $3+4+5+6=18$
				§ $\{m, m+1, m+2, m+3\}$
				§ $m+(m+1)+(m+2)+(m+3) = m+m+m+m+1+2+3 = 4m+6 = 2(2m+3)$
			}
		\end{vastaus}
	\end{tehtava}

	\begin{tehtava}
		Todista, että parillisen ja parittoman kokonaisluvun summa on pariton.
		\begin{vastaus}
			Parillinen luku voidaan esittää muodossa $2k$ ja pariton luku muodossa $2l+1$, missä $k, l \in \zz$.
			Tällöin summaksi saadaan $2k+2l+1 = 2(k+l)+1$, joka on pariton.
		\end{vastaus}
	\end{tehtava}

	\begin{tehtava}
		Todista, että kahden parillisen kokonaisluvun tulo on parillinen.
		\begin{vastaus}
			Parilliset luvut voidaan esittää muodoissa $2k$ ja $2l$, missä $k, l \in \zz$
			Tällöin tuloksi saadaan $2k \cdot 2l = 2(2kl)$, joka on parillinen.
		\end{vastaus}
	\end{tehtava}

	\begin{tehtava}
		Todista, että kahden rationaaliluvun tulo on rationaalinen.
		\begin{vastaus}
		Merkitään rationaalilukuja $q_1 = \frac{m_1}{n_1}$ ja $q_2 = \frac{m_2}{n_2}$. Nyt sekä luvut $m_1$, $n_1$, $m_2$ ja $n_2$ ovat kaikki kokonaislukuja. Lukujen $q_1$ ja $q_2$ tulo on $q_3 = q_1q_2 =  \frac{m_1n_1}{m_2n_2}$, ja koska kahden kokonaisluvun tulo on aina kokonaisluku, sekä luvun $q_3$ osoittaja että nimittäjä ovat kokonaislukuja, ja siten $q_3$ on rationaaliluku.
		
		\end{vastaus}
	\end{tehtava}

	\begin{tehtava}
		Olkoot $a$ ja $b$ kokonaislukuja. Todista, että luku
		$a + b$ on parillinen jos ja vain jos luku
		$a - b$ on parillinen.
		\begin{vastaus}
		Todistetaan ensin väite ''$a + b$ on parillinen'' $\Rightarrow$ ''$a - b$ on parillinen'': Olkoon $a+b = 2c$, jossa $c$ on kokonaisluku. Nyt
		\begin{align*}
		 a+b-2b &= 2c-2b \\
		 a-b &= 2(c-b) 
		\end{align*}
		josta nähdään että $a-b$ on parillinen. Vastaavasti todistetaan väite ''$a - b$ on parillinen'' $\Rightarrow$ ''$a + b$ on parillinen''. Oletetaan että $a-b$ on parillinen; tällöin
		\begin{align*}
		a - b &= 2c \\
		a-b+2b &= 2c + 2b \\
		a+b &= 2(c+b) 
		\end{align*}
		josta nähdään että $a+b$ on parillinen. Molemminsuuntaisista implikaatioista seuraa, että koko väite on tosi. %kirjoitin ekstensiivisen todistuksen koska tämän tyyppinen ''jos ja vain jos'' näyttää esiintyvän tässä ensimmäistä kertaa -Jouni

		\end{vastaus}
	\end{tehtava}

	\begin{tehtava}
		Olkoon $m$ sellainen kokonaisluku, että $m^2$ on
		pariton. Todista, että tällöin $m$ on pariton.
		\begin{vastaus}
		Todistetaan käänteisesti olettamalla, että $m$ on parillinen, jolloin $m = 2c$ jossa $c \in \mathbb{Z}$. Tällöin $m^2 = (2c)^2 = 4c^2$, joka on parillinen. Siis väite pätee.
		\end{vastaus}
	\end{tehtava}

	\begin{tehtava}
		Todista väite todeksi tai epätodeksi: Kahden
		irrationaaliluvun summa on aina irrationaaliluku.
		\begin{vastaus}
			(Epätosi.)
		\end{vastaus}
	\end{tehtava}

	\begin{tehtava}
		Luku $\sqrt{3}$ on irrationaaliluku. Todista,
		että $\sqrt{3}+1/2$ on irrationaaliluku. Vihje: Käytä
		epäsuoraa todistusta.
	\end{tehtava}

	\begin{tehtava}
		Todista, että irrationaaliluvun ja nollasta poikkeavan rationaaliluvun tulo on irrationaaliluku.
	\end{tehtava}

	\begin{tehtava}
		Todista, että luku $\sqrt{6}$ on irrationaaliluku.
	\end{tehtava}

	\begin{tehtava}
		Todista: Jos luku $a$ on irrationaaliluku, niin
		myös luku
		\[
			\frac{2a-5}{3a-11}
		\]
		on irrationaaliluku. Vihje: Käytä symbolisen laskimen
		\texttt{solve}-toimintoa.
	\end{tehtava}

	\begin{tehtava}
		Tarkastellaan suorakulmaista kolmiota, jonka kateettien
		pituudet ovat $a$ ja $b$ ja hypotenuusan pituus $c$.
		Todista väite todeksi tai epätodeksi.
		\alakohdat{
			§ On olemassa sellainen suorakulmainen kolmio,
				jonka sivujen pituudet $a$, $b$ ja $c$ ovat kaikki parillisia kokonaislukuja.
			§ On olemassa sellainen suorakulmainen kolmio,
				jonka sivujen pituudet $a$, $b$ ja $c$ ovat kaikki 
				parittomia kokonaislukuja.
		}
		\begin{vastaus}
			\alakohdat{
				§ (Tosi.)
				§ (Tosi.)
			}
		\end{vastaus}
	\end{tehtava}

	\begin{tehtava}
		Kokonaisluvut $0, 1, 2, \ldots, 12$ asetetaan
		mielivaltaiseen järjestykseen ympyrän kehälle. Todista,
		että joidenkin neljän peräkkäisen luvun summa on
		vähintään 26.
		\begin{vastaus}
			Vihje: tutki tilannetta, jossa olisi vain luvut $1, 2, \ldots, 12$ (ilman nollaa).
		\end{vastaus}
	\end{tehtava}

\end{tehtavasivu}

% -----


\begin{kotitehtavasivu}

	\begin{tehtava}
		\alakohdat{
		§ Laske luonnollisten lukujen $1, 2, 3,\ldots, 10$ neliöt.
		§ Esitä väite luonnollisten lukujen neliöiden parillisuudesta tai parittomuudesta.
		§ Todista väitteesi matemaattisesti.
		}
		\begin{vastaus}
			\alakohdat{
				§ $1, 4, 9, 16, 25, 36, 49, 64, 81, 100$
				§ Luonnollisen luvun neliön parillisuus on sama kuin luvun itsensä.
			}
		\end{vastaus}
	\end{tehtava}

	\begin{tehtava}
		Todista, että kolmen parittoman kokonaisluvun summa on pariton.
	\end{tehtava}

	\begin{tehtava}
		Todista, että kolmen peräkkäisen kokonaisluvun summa on jaollinen luvulla 3.
	\end{tehtava}

	\begin{tehtava}
		Todista, että kahden parittoman kokonaisluvun tulo on pariton.
	\end{tehtava}

	\begin{tehtava}
		Olkoot $a$ ja $b$ reaalilukuja. Todista, että tulo $ab = 0$, jos ja vain jos $a=0$ tai $b=0$.
	\end{tehtava}

	\begin{tehtava}
		Olkoon $n$ kokonaisluku. Todista, että luku $n^{2} + 3n + 1$ on aina pariton. Käytä a) suoraa b) epäsuoraa
		todistusta.
	\end{tehtava}

	\begin{tehtava}
		Todista, että ei ole olemassa sellaisia positiivisia kokonaislukuja $x$ ja $y$, jotka toteuttavat
		yhtälön $4^{x} = 7^{y}$.
	\end{tehtava}

	\begin{tehtava}
		Todista väite todeksi tai epätodeksi: Kahden
		irrationaaliluvun tulo voi olla rationaaliluku.
		\begin{vastaus}
			(Tosi.)
		\end{vastaus}
	\end{tehtava}

	\begin{tehtava}
		Todista, että luku $\sqrt{5}$ on irrationaaliluku.
		Tarvitset seuraavaa lisätietoa: jos kahden kokonaisluvun
		tulo on jaollinen luvulla 5, niin ainakin toinen tulon
		tekijöistä on jaollinen luvulla 5.
	\end{tehtava}

	\begin{tehtava}
		Tarkastellaan paritonta määrää parittomia kokonaislukuja.
		Todista, että lukujen keskiarvo ei voi olla nolla.
	\end{tehtava}

	\begin{tehtava}
		Tarkastellaan kolmiota, jonka sivujen pituudet ovat
		$a$, $b$ ja $c$. Niin sanotun \termi{Heronin kaava}{Heronin kaavan} mukaan
		kolmion pinta-alalle pätee \[A = \sqrt{p(p-a)(p-b)(p-c)},\]
		missä $p = \frac{1}{2}(a+b+c)$. Todista, että jos kolmion
		sivujen pituudet ovat luvulla 4 jaollisia kokonaislukuja,
		niin kolmion pinta-alan neliö on jaollinen luvulla 16.
	\end{tehtava}

	\begin{tehtava}
		Niin sanotun \termi{Fermat'n suuri lause}{Fermat'n suuren lauseen} mukaan ei ole
		olemassa sellaisia positiivisia kokonaislukuja $x$, $y$
		ja $z$, jotka toteuttaisivat yhtälön $x^{n} + y^{n} = z^{n}$,
		missä $n$ on lukua 2 suurempi kokonaisluku.
		Erityisesti siis yhtälöllä $x^{3} + y^{3} = z^{3}$ ei
		ole ratkaisua, jos $x$, $y$ ja $z$ ovat positiivisia
		kokonaislukuja. Käytä tätä tulosta hyväksesi a- ja b-kohtien ratkaisemisessa.
            \alakohdat{
			§ Todista, että ei ole olemassa sellaisia
			positiivisia rationaalilukuja $x$, $y$ ja $z$, jotka
			toteuttavat yhtälön $x^{3} + y^{3} = z^{3}$.
			§ Todista väite todeksi tai epätodeksi: ei ole
			olemassa sellaisia keskenään eri suuria kokonaislukuja
			$x$, $y$ ja $z$, jotka toteuttavat yhtälön $x^{3} + y^{3} = z^{3}$.
		}
		\begin{vastaus}
			\alakohdat{
				§ 
				§ (Epätosi eli on olemassa.)
			}
		\end{vastaus}
	\end{tehtava}

	\begin{tehtava}
		Olkoot $a$ ja $b$ reaalilukuja, joille pätee $0 \le a \le 1$ ja $0 \le b \le 1$.
		Todista, että tällöin $0 \le \frac{a + b}{1 + ab} \le 1$.
	\end{tehtava}

\end{kotitehtavasivu}
