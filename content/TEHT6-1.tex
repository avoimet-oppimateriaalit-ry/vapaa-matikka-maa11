\begin{tehtavasivu}

\begin{tehtava}
    Määritä lukujen suurin yhteinen tekijä.
    
    \alakohdat{
        § $15$ ja $20$
        § $9$ ja $36$
        § $4$ ja $7$
    }

    \begin{vastaus}
        \alakohdat{
            § $5$
            § $9$
            § $1$
        }
    \end{vastaus}
    
\end{tehtava}

\begin{tehtava}
    Määritä Eukleideen algoritmia käyttäen
    
    \alakohdat{
        § $\syt(184, 152)$
        § $\syt(227, 143)$.
    }

    \begin{vastaus}
        \alakohdat{
            § $8$
            § $1$
        }
    \end{vastaus}
    
\end{tehtava}

\begin{tehtava}
    Määritä Eukleideen algoritmia käyttäen
    
    \alakohdat{
        § $\syt(272, 1479)$
        § $\syt(4719, 18207)$.
    }

    \begin{vastaus}
        \alakohdat{
            § $17$
            § $3$
        }
    \end{vastaus}
    
\end{tehtava}

\begin{tehtava}
    Esitä murtoluku
    \alakohdat{
        § $\frac{143}{605}$
        § $\frac{5989}{30899}$
    }
    supistetussa muodossa. Vihje: Määritä osoittajan ja nimittäjän suurin yhteinen tekijä.

    \begin{vastaus}
        \alakohdat{
            § $\frac{13}{55}$
            § $\frac{113}{583}$
        }
    \end{vastaus}
    
\end{tehtava}

\begin{tehtava}
    % Tarkistettu (Topi Talvitie, 9.11.2013)
    Osoita, että murtoluku $\frac{8788}{13475}$ ei supistu.
\end{tehtava}

\begin{tehtava}
    Leirille osallistui 780 tyttöä ja 612 poikaa. Osallistujat jaettiin keskenään yhtä suuriin ryhmiin siten, että kussakin ryhmässä oli vain tyttöjä tai poikia. Mikä oli suurin mahdollinen ryhmäkoko?
    
    \begin{vastaus}
        12
    \end{vastaus}
    
\end{tehtava}

\begin{tehtava}
    Määritä lukujen $188000100$ ja $188$ suurin yhteinen tekijä.

    \begin{vastaus}
        $4$
    \end{vastaus}
    
\end{tehtava}

\begin{tehtava}
    Olkoon $a$ positiivinen kokonaisluku. Määritä
    \alakohdat{
        § $\syt(a, a)$
        § $\syt(a, 1)$
        § $\syt(a^2, a)$
        § $\syt((a+1)!, a!)$.
    }
    Merkintä $a!$ tarkoittaa luvun $a$ \termi{kertoma}{kertomaa}. Se on tulo $a! = a \cdot (a-1) \cdot (a-2) \cdot \ldots \cdot 3 \cdot 2 \cdot 1$.

    \begin{vastaus}
        \alakohdat{
            § $a$
            § $1$
            § $a$
            § $a!$
        }
    \end{vastaus}
    
\end{tehtava}

\begin{tehtava}
    % Tarkistettu (Topi Talvitie, 9.11.2013)
    Olkoon $n$ positiivinen kokonaisluku. Osoita Eukleideen algoritmia käyttäen, että $\syt(n+1, n)=1$.
\end{tehtava}

\begin{tehtava}
    Olkoon $n$ positiivinen kokonaisluku. Määritä lukujen $n^2 + 2n$ ja $n + 1$ suurin yhteinen tekijä.
    
    \begin{vastaus}
        1
    \end{vastaus}
    
\end{tehtava}

\begin{tehtava}
    % Tarkistettu (Topi Talvitie, 9.11.2013)
    Olkoon $n$ positiivinen kokonaisluku. Osoita, että
    \[\syt(3^{n+1} + 10, 3^n + 2)=1.\]
\end{tehtava}

\end{tehtavasivu}

% -----


\begin{kotitehtavasivu}

\begin{tehtava}
    Määritä lukujen suurin yhteinen tekijä.
    
    \alakohdat{
        § $63$ ja $7$
        § $64$ ja $33$
        § $45$ ja $60$
    }

    \begin{vastaus}
        \alakohdat{
            § $7$
            § $1$
            § $15$
        }
    \end{vastaus}
    
\end{tehtava}

\begin{tehtava}
    Määritä Eukleideen algoritmia käyttäen
    
    \alakohdat{
        § $\syt(657, 306)$
        § $\syt(2197, 4641)$
        § $\syt(15787, 4111)$.
    }

    \begin{vastaus}
        \alakohdat{
            § $9$
            § $13$
            § $1$
        }
    \end{vastaus}
    
\end{tehtava}

\begin{tehtava}
    Esitä murtoluku
    \alakohdat{
        § $\frac{182}{299}$
        § $\frac{7697}{32041}$
    }
    supistetussa muodossa.

    \begin{vastaus}
        \alakohdat{
            § $\frac{14}{23}$
            § $\frac{43}{179}$
        }
    \end{vastaus}
    
\end{tehtava}

\begin{tehtava}
    Leipomo suljettiin remontin ajaksi. Ennen sulkemista laskettiin, että leipomon varastossa oli 4896 vehnäsämpylää ja 1408 grahamsämpylää. Sämpylät pakattiin kuljetusta varten keskenään samankokoisiin pusseihin siten, että kuhunkin pussiin tuli vain vehnä- tai grahamsämpylöitä. Mikä oli suurin mahdollinen pussikoko? Oletetaan, että vehnäsämpylä oli samankokoinen kuin grahamsämpylä ja että yksikään pussi ei jäänyt vajaaksi.

    \begin{vastaus}
        $32$
    \end{vastaus}
    
\end{tehtava}

\begin{tehtava}
    Määritä lukujen $468468468108$ ja $234$ suurin yhteinen tekijä.
    
    \begin{vastaus}
        $18$
    \end{vastaus}
    
\end{tehtava}

\begin{tehtava}
    Olkoot $a$, $b$ ja $c$ positiivisia kokonaislukuja. Lukujen $a$, $b$ ja $c$ suurin yhteinen tekijä eli $\syt(a, b, c)$ voidaan määrittää siten, että ensin määritetään kahden luvun suurin yhteinen tekijä ja sitten tämän ja kolmannen luvun suurin yhteinen tekijä. Määritä
    
    \alakohdat{
        § $\syt(15, 30, 40)$
        § $\syt(6, 9, 11)$
        § $\syt(171, 456, 665)$.
    }

    \begin{vastaus}
        \alakohdat{
            § $5$
            § $1$
            § $19$
        }
    \end{vastaus}
    
\end{tehtava}

\begin{tehtava}
    Olkoot $a$ ja $b$ positiivisia kokonaislukuja ja $\syt(a, b)=9$. Voiko tällöin yhtälö $a + b = 186$ olla tosi?
    
    \begin{vastaus}
        Ei voi.
    \end{vastaus}
    
\end{tehtava}

\begin{tehtava}
    Olkoon $n$ positiivinen kokonaisluku. Tutki, mitä arvoja $\syt(n+4, n)$ voi saada.
    
    \begin{vastaus}
        $1$, $2$ ja $4$.
    \end{vastaus}
    
\end{tehtava}

\begin{tehtava}
    Olkoon $n$ positiivinen kokonaisluku. Määritä lukujen $n^2 + 3n$ ja $n + 2$ suurin yhteinen tekijä.
    
    \begin{vastaus}
        Jos $n$ on parillinen, $\syt(n^2 + 3n, n + 2) = 2$, muuten $\syt(n^2 + 3n, n + 2) = 1$.
    \end{vastaus}
    
\end{tehtava}

\begin{tehtava}
    % Tarkistettu (Topi Talvitie, 9.11.2013)
    Olkoot $a$ ja $b$ positiivisia kokonaislukuja. Osoita, että $\syt(a, b)$ on jaollinen kaikilla lukujen $a$ ja $b$ yhteisillä tekijöillä.
\end{tehtava}

\end{kotitehtavasivu}
