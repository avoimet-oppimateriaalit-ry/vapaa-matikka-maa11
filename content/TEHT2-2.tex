%Luku 2.2 Harjoitustehtäviä, vastauksien tekijä Valtteri Vistiaho 10.11.2013
\begin{tehtavasivu}

\begin{tehtava}
    Olkoot $A$: ''Matti on insinööri'' ja $B$: ''Matilla on hyvä työpaikka''.
Suomenna lause.
    \alakohdat{
        § $A\land B$,
        § $\lnot A \lor B$,
        § $A\to B$,
        § $\lnot B\to A$,
        § $\lnot( B \to A )$ ja
        § $ B \lequiv A$.
    }

    \begin{vastaus}
        \alakohdat{
            § Matti on insinööri, jolla on hyvä työpaikka.
            § Matti ei ole insinööri tai hänellä on hyvä työpaikka.
            § Jos Matti on insinööri, hänellä on hyvä työpaikka.
            § Jos Matilla ei ole hyvää työpaikkaa, hän on insinööri.
            § Ei ole totta, että jos Matilla on hyvä työpaikka, hän olisi insinööri.
            § Matilla on hyvä työpaikka, jos ja vain jos hän on insinööri.
        }
    \end{vastaus}
    
\end{tehtava}

\begin{tehtava}
    Olkoot $A$: ''Liisa on lomalla'' ja $B$: ''Liisa on iloinen''. Formalisoi lauseet:
    \alakohdat{
        § Jos Liisa on lomalla, hän on iloinen.
        § Liisa on lomalla, mutta hän ei ole iloinen.
        § Liisa on iloinen silloin ja vain silloin, kun hän on lomalla.
        § Jos Liisa ei ole iloinen, hän ei ole lomalla.
    }

    \begin{vastaus}
        \alakohdat{
            § $A\to B$
            § $A\land B$
            § $B\lequiv A$
            § $\lnot B\to \lnot A$
        }
    \end{vastaus}
    
\end{tehtava}

\begin{tehtava}
    Onko lause tosi?
    \alakohdat{
        § Jos Tallinna on Norjan pääkaupunki, niin Tukholma on Ruotsin pääkaupunki.
        § Jos Tallinna on Norjan pääkaupunki, niin Budapest on Ruotsin pääkaupunki.
        § Jos Leonardo da Vinci on kuollut, niin Aleksis Kivi on saksalainen ralliautoilija.
        § Jos Leonardo da Vinci on kuollut, niin Leo Tolstoi on venäläinen kirjailija.
    }

    \begin{vastaus}
        \alakohdat{
            § Epätosi
            § Tosi
            § Epätosi
            § Tosi
        }
    \end{vastaus}
    
\end{tehtava}

\begin{tehtava}
    Laadi lauseen totuustaulu.
    \alakohdat{
        § $\lnot A \to B$
        § $A\lequiv \lnot B$
        § $\lnot( A\land B )\to A$
    }

    \begin{vastaus}
        \alakohdat{
            § \begin{center}
		    \begin{tabular}{|c|c|c|c|}\hline
		    $A$ & $B$ & $\lnot A$ & $\lnot A \to B$ \\ \hline
		    $1$ & $1$ & $0$ & $1$ \\ %\hline
		    $1$ & $0$ & $0$ & $1$ \\
		    $0$ & $1$ & $1$ & $1$ \\
		    $0$ & $0$ & $1$ & $0$ \\ \hline
\end{tabular}
\end{center}
            § \begin{center}
		    \begin{tabular}{|c|c|c|c|}\hline
		    $A$ & $B$ & $\lnot B$ & $A\lequiv \lnot B$ \\ \hline
		    $1$ & $1$ & $0$ & $0$ \\ %\hline
		    $1$ & $0$ & $1$ & $1$ \\
		    $0$ & $1$ & $0$ & $1$ \\
		    $0$ & $0$ & $1$ & $0$ \\ \hline
\end{tabular}
\end{center}
            § \begin{center}
		    \begin{tabular}{|c|c|c|c|c|}\hline
		    $A$ & $B$ & $A\land B$ & $\lnot(A\land B)$ & $\lnot(A\land B)\to A$ \\ \hline
		    $1$ & $1$ & $1$ & $0$ & $1$ \\ %\hline
		    $1$ & $0$ & $0$ & $1$ & $1$ \\
		    $0$ & $1$ & $0$ & $1$ & $0$ \\
		    $0$ & $0$ & $0$ & $1$ & $0$ \\ \hline
\end{tabular}
\end{center}
        }
    \end{vastaus}
    
\end{tehtava}

\begin{tehtava}
    Laadi lauseen totuustaulu.
    \alakohdat{
        § $\lnot A \lor B \to C \land A$
        § $( A\to B ) \lequiv (C\to B)$
    }

    \begin{vastaus}
        \alakohdat{
            § 
            \begin{center}
		    \begin{tabular}{|c|c|c|c|c|c|c|}\hline
		    $A$ & $B$ & $C$ & $\lnot A$ & $\lnot A \lor B$ & $C\land A$ & $\lnot A \lor B \to C \land A$\\ \hline
		    $1$ & $1$ & $1$ & $0$ & $1$ & $1$ & $1$ \\ %\hline
		    $1$ & $1$ & $0$ & $0$ & $1$ & $0$ & $0$ \\
		    $1$ & $0$ & $1$ & $0$ & $0$ & $1$ & $1$ \\
		    $1$ & $0$ & $0$ & $0$ & $0$ & $0$ & $1$ \\
		    $0$ & $1$ & $1$ & $1$ & $1$ & $0$ & $0$ \\
		    $0$ & $0$ & $1$ & $1$ & $1$ & $0$ & $0$ \\
		    $0$ & $1$ & $0$ & $1$ & $1$ & $0$ & $0$ \\
		    $0$ & $0$ & $0$ & $1$ & $1$ & $0$ & $0$ \\ \hline
\end{tabular}
\end{center}
            § 
            \begin{center}
		    \begin{tabular}{|c|c|c|c|c|c|c|}\hline
		    $A$ & $B$ & $C$ & $A\to B$ & $C\to B$ & $(A\to B)\lequiv(C\to B)$ \\ \hline
		    $1$ & $1$ & $1$ & $1$ & $1$ & $1$ \\ %\hline
		    $1$ & $1$ & $0$ & $1$ & $1$ & $1$ \\
		    $1$ & $0$ & $1$ & $0$ & $0$ & $1$ \\
		    $1$ & $0$ & $0$ & $0$ & $1$ & $0$ \\
		    $0$ & $1$ & $1$ & $1$ & $1$ & $1$ \\
		    $0$ & $0$ & $1$ & $1$ & $0$ & $0$ \\
		    $0$ & $1$ & $0$ & $1$ & $1$ & $1$ \\
		    $0$ & $0$ & $0$ & $1$ & $1$ & $1$ \\ \hline
\end{tabular}
\end{center}
        }
    \end{vastaus}
    
\end{tehtava}

\begin{tehtava}
    Formalisoi lause. Missä tilanteissa lause on tosi?
    \alakohdat{
        § Jos mustikat polun varrella eivät ole kypsiä, niin polulla vaeltaminen on turvallista.
        § Jos mustikat polun varrella ovat kypsiä, niin silloin polulla vaeltaminen on turvallista, jos ja vain jos karhuja ei ole nähty alueella. 
        § Polulla vaeltaminen on turvallista silloin ja vain silloin, kun mustikat eivät ole kypsiä tai alueella ei ole nähty karhuja.
    }

    \begin{vastaus}
    Olkoot $A$: ''Mustikat polun varrella ovat kypsiä.'', $B$: ''Polulla vaeltaminen on turvallista.'' ja $C$: ''Karhuja on nähty alueella.''
        \alakohdat{
            § $\lnot A\to B$. Lause on tosi, kun mustikat ovat kypsiä tai polulla vaeltaminen on turvallista. % /Niko
            § $A\to (B\lequiv \lnot C)$. Lause on tosi, kun joko
            §§ mustikat ovat kypsiä, polulla vaeltaminen ei ole turvallista ja karhuja on nähty alueella,
            §§ mustikat ovat kypsiä, polulla vaeltaminen on turvallista ja karhuja ei ole nähty alueella tai
            §§ mustikat eivät ole kypsiä.
            % /Niko % tässä oli aiemmin $(A\to B)\lequiv \lnot C$
            § $B\lequiv(\lnot A\lor \lnot C)$. Lause on tosi, kun joko
            §§ polulla vaeltaminen ei ole turvallista, mustikat ovat kypsiä ja karhuja on nähty alueella,
            §§ polulla vaeltaminen on turvallista ja mustikat eivät ole kypsiä tai
            §§ polulla vaeltaminen on turvallista ja karhuja ei ole nähty alueella.
            % /Niko
        }
    \end{vastaus}
    
\end{tehtava}

\begin{tehtava}
    Formalisoi lause ja laadi lauseen totuustaulu. Mitä huomaat? Kertooko lause mitään siitä, onko logiikan opiskelu oikeasti hauskaa juuri tällä hetkellä? 
    \alakohdat{
        § Jos sataa ja ei sada, niin logiikan opiskelu on hauskaa.
        § Jos sataa ja ei sada, niin logiikan opiskelu ei ole hauskaa.
    }

    \begin{vastaus}
    Olkoot $A$: ''Sataa.'' ja $B$: ''Logiikan opiskelu on hauskaa.''
        \alakohdat{
            § \begin{center}
		    \begin{tabular}{|c|c|c|c|c|}\hline
		    $A$ & $B$ & $\lnot A$ & $A\land \lnot A$ & $(A\land \lnot A)\to B$ \\ \hline
		    $1$ & $1$ & $0$ & $0$ & $1$ \\ %\hline
		    $1$ & $0$ & $0$ & $0$ & $1$ \\
		    $0$ & $1$ & $1$ & $0$ & $1$ \\
		    $0$ & $0$ & $1$ & $0$ & $1$ \\ \hline
\end{tabular}
\end{center}
	Ei kerro.
            § \begin{center}
		    \begin{tabular}{|c|c|c|c|c|}\hline
		    $A$ & $B$ & $\lnot A$ & $A\land \lnot A$ & $(A\land \lnot A)\to \lnot B$ \\ \hline
		    $1$ & $1$ & $0$ & $0$ & $1$ \\ %\hline
		    $1$ & $0$ & $0$ & $0$ & $1$ \\
		    $0$ & $1$ & $1$ & $0$ & $1$ \\
		    $0$ & $0$ & $1$ & $0$ & $1$ \\ \hline
\end{tabular}
\end{center}
	Ei kerro.
        }
    \end{vastaus}
    
\end{tehtava}

\begin{tehtava}
    Perheen lapsia Annaa ja Markusta epäillään kaappiin piilotetun suklaalevyn katoamisesta. Luotettava todistaja kertoo tiedot:
Anna on syyllinen tai Markus on syyllinen.
Anna on syytön tai Markus on syytön. 
Jos Anna on syyllinen, niin Markus on syyllinen.

Kumpi lapsista on käynyt suklaavarkaissa?
    \begin{vastaus} \newline
        Olkoot $A$: ''Anna on syyllinen.'' ja $B$: ''Markus on syyllinen.
        \begin{center}
		    \begin{tabular}{|c|c|c|c|c|c|c|c|}\hline
		    $A$ & $B$ & $\lnot A$ & $\lnot B$ & $A\lor B$ & $\lnot A\lor \lnot B$ & $A\to B$ & $(A\lor B)\land(\lnot A\lor \lnot B)\land(A\to B)$ \\ \hline
		    $1$ & $1$ & $0$ & $0$ & $1$ & $0$ & $1$ & $0$ \\ %\hline
		    $1$ & $0$ & $0$ & $1$ & $1$ & $1$ & $0$ & $0$ \\
		    $0$ & $1$ & $1$ & $0$ & $1$ & $1$ & $1$ & $1$ \\
		    $0$ & $0$ & $1$ & $1$ & $0$ & $1$ & $1$ & $0$ \\ \hline
\end{tabular}
\end{center}
    Markus on syyllinen.
    \end{vastaus}
    
\end{tehtava}

\begin{tehtava}
     Edwards, Petterson ja Smith ovat syytettyinä omenavarkaudesta. Neiti Marble kuulustelee heitä. Edwards sanoo: ''Jos Petterson on syytön, niin minä olen syyllinen.'' Petterson toteaa: ''Minä olen syyllinen, jos ja vain jos Edwards on syyllinen.'' Smith väittää: ''Olemme kaikki syyttömiä.'' Neiti Marble tietää, että syylliset valehtelevat aina ja syyttömät puhuvat aina totta. Ratkaise totuustaulun avulla, kuka tai ketkä syytetyistä ovat käyneet omenavarkaissa.
    \begin{vastaus}
    Olkoot $E$: ''Edwards on syyllinen.'', $P$: ''Petterson on syyllinen.'' ja $S$: ''Smith on syyllinen.''
            \begin{center}
		    \begin{tabular}{|c|c|c|c|c|c|c|c|c|c|}\hline
		    $E$ & $P$ & $S$ & $\lnot E$ & $\lnot P$ & $\lnot S$ & $\lnot P\to E$ & $P\lequiv E$ & $\lnot E\land \lnot P \land \lnot S$ \\ \hline
		    $1$ & $1$ & $1$ & $0$ & $0$ & $0$ & $1$ & $1$ & $0$ \\ %\hline
		    $1$ & $1$ & $0$ & $0$ & $0$ & $1$ & $1$ & $1$ & $0$ \\
		    $1$ & $0$ & $1$ & $0$ & $1$ & $0$ & $1$ & $0$ & $0$ \\
		    $1$ & $0$ & $0$ & $0$ & $1$ & $1$ & $1$ & $0$ & $0$ \\
		    $0$ & $1$ & $1$ & $1$ & $0$ & $0$ & $1$ & $0$ & $0$ \\
		    $0$ & $0$ & $1$ & $1$ & $1$ & $0$ & $0$ & $1$ & $0$ \\
		    $0$ & $1$ & $0$ & $1$ & $0$ & $1$ & $1$ & $0$ & $0$ \\
		    $0$ & $0$ & $0$ & $1$ & $1$ & $1$ & $0$ & $1$ & $1$ \\ \hline
\end{tabular}
\end{center}
    Petterson ja Smith ovat syyllisiä.  %Tähän voisi lisätä selityksen, miksi näin on.
    \end{vastaus}
    
\end{tehtava}

\begin{tehtava}
     Eräässä maassa kaikki kuuluvat joko hattujen tai myssyjen puolueeseen. Hatut valehtelevat aina ja myssyt puhuvat aina totta. Kolme kansalaista keskusteli keskenään. Heidän joukossaan saattoi olla vieraan puolueen vakoilija. Arthur sanoi, että Claus on myssy. Berit totesi, että Arthur on myssy ja myös hän itse on myssy. Claus väitti, että jos Arthur on hattu, niin myös hän itse on hattu. Tutki totuustaulun avulla, kuka oli mahdollinen vakoilija.
     \begin{vastaus} \newline
    Olkoot $A$: ''Arthur on myssy.'', $B$: ''Berit on myssy.'', $C$: ''Claus on myssy.'' ja niiden negaatiot tarkoittavat heidän kuuluvan hattuihin.
        \begin{center}
		    \begin{tabular}{|c|c|c|c|c|c|c|}\hline
		    $A$ & $B$ & $C$ & $\lnot A$ & $\lnot C$ & $A\land B$ & $\lnot A\to \lnot C$ \\ \hline
		    $1$ & $1$ & $1$ & $0$ & $0$ & $1$ & $1$ \\ %\hline
		    $1$ & $1$ & $0$ & $0$ & $1$ & $1$ & $1$ \\
		    $1$ & $0$ & $1$ & $0$ & $0$ & $0$ & $1$ \\
		    $1$ & $0$ & $0$ & $0$ & $1$ & $0$ & $1$ \\
		    $0$ & $1$ & $1$ & $1$ & $0$ & $0$ & $0$ \\
		    $0$ & $0$ & $1$ & $1$ & $0$ & $0$ & $0$ \\
		    $0$ & $1$ & $0$ & $1$ & $1$ & $0$ & $1$ \\
		    $0$ & $0$ & $0$ & $1$ & $1$ & $0$ & $1$ \\ \hline
\end{tabular}
\end{center}
		Berit on mahdollinen vakoilija. %Tähän voisi lisätä selityksen, miksi näin on.   
    \end{vastaus}
    
\end{tehtava}

\begin{tehtava}
     Olkoot lauseet $A$: ''herään ajoissa'', $B$: ''menen kouluun'' ja $C$: ''opiskelen logiikkaa''.
     Esitä sanoin lause $A \to B \lor C$. Onko lauseen tulkinta $A \to (B \lor C)$ vai $(A \to B) \lor C$? Miten tulkinnat eroavat toisistaan? 
        \begin{vastaus} \newline
	 Jos herään ajoissa, menen kouluun tai opiskelen logiikkaa. Lause tulkitaan $A\to (B\lor C)$, koska konnektiivien suoritusjärjestyksessä disjunktio on ennen implikaatiota. $(A\to B)\lor C$ tarkottaa: ''Jos herään ajoissa, menen kouluun, tai opiskelen logiikkaa.'' Tämän voi tulkita: ''Jos herään ajoissa, menen kouluun ja opiskelen logiikkaa.'' tai ''En herää ajoissa ja opiskelen logiikkaa joko koulussa tai muualla.'' %Voisi tehdä selvemmäksi.
    \end{vastaus}
    
\end{tehtava}

\begin{tehtava}
     Vertaa lauseiden totuusarvoja atomilauseiden $A$, $B$ ja $C$ eri totuusarvoilla. Onko sulkeiden muuttamisella vaikutusta lauseen totuusarvoon?
    \alakohdat{
        § $(A\to B)\to C$ ja $A\to (B\to C)$.
        § $(A\land B)\land C$ ja $A\land (B\land C)$.
    }

    \begin{vastaus}
    
        \alakohdat{
            § \begin{center}
		    \begin{tabular}{|c|c|c|c|c|c|c|}\hline
		    $A$ & $B$ & $C$ & $A\to B$ & $B\to C$ & $(A\to B)\to C$ & $A\to(B\to C)$ \\ \hline
		    $1$ & $1$ & $1$ & $1$ & $1$ & $1$ & $1$ \\ %\hline
		    $1$ & $1$ & $0$ & $1$ & $0$ & $0$ & $0$ \\
		    $1$ & $0$ & $1$ & $0$ & $1$ & $1$ & $1$ \\
		    $1$ & $0$ & $0$ & $0$ & $1$ & $1$ & $1$ \\
		    $0$ & $1$ & $1$ & $1$ & $1$ & $1$ & $1$ \\
		    $0$ & $0$ & $1$ & $1$ & $1$ & $1$ & $1$ \\
		    $0$ & $1$ & $0$ & $1$ & $0$ & $0$ & $1$ \\
		    $0$ & $0$ & $0$ & $1$ & $1$ & $0$ & $1$ \\ \hline
\end{tabular}
\end{center}
Sulkeiden muuttamisella on vaikutusta.
            § \begin{center}
		    \begin{tabular}{|c|c|c|c|c|c|c|}\hline
		    $A$ & $B$ & $C$ & $A\land B$ & $B\land C$ & $(A\land B)\land C$ & $A\land(B\land C)$ \\ \hline
		    $1$ & $1$ & $1$ & $1$ & $1$ & $1$ & $1$ \\ %\hline
		    $1$ & $1$ & $0$ & $1$ & $0$ & $0$ & $0$ \\
		    $1$ & $0$ & $1$ & $0$ & $0$ & $0$ & $0$ \\
		    $1$ & $0$ & $0$ & $0$ & $0$ & $0$ & $0$ \\
		    $0$ & $1$ & $1$ & $0$ & $1$ & $0$ & $0$ \\
		    $0$ & $0$ & $1$ & $0$ & $0$ & $0$ & $0$ \\
		    $0$ & $1$ & $0$ & $0$ & $0$ & $0$ & $0$ \\
		    $0$ & $0$ & $0$ & $0$ & $0$ & $0$ & $0$ \\ \hline
\end{tabular}
\end{center}
Sulkeiden muuttamisella ei ole vaikutusta.
        }
    \end{vastaus}
    
\end{tehtava}

\begin{tehtava}
     * Tietokoneet käsittelevät tietoa käyttäen bittejä. Bitillä on kaksi mahdollista arvoa, $0$ ja $1$.
Bittijono on jono, jossa on yksi tai useampia bittejä. Esimerkiksi $1001\, 0010$ on bittijono, jonka pituus on $8$ bittiä. Samanpituisille bittijonoille määritellään bittikohtaiset tai (or), ja (and) sekä poissulkeva tai (xor). Esimerkiksi bittijonojen $1100$ ja $1010$ bittikohtainen tai on jono $1110$, bittikohtainen ja on jono $1000$ sekä bittikohtainen poissulkeva tai on jono $0110$. Muodosta a) bittikohtainen tai  b) bittikohtainen ja c) bittikohtainen poissulkeva tai jonoille $1111\, 0000$ ja $1010\, 1010$.
    \alakohdat{
        § bittikohtainen tai
        § bittikohtainen ja
        § bittikohtainen poissulkeva tai jonoille $1111\, 0000$ ja $1010\, 1010$.
    }

    \begin{vastaus}
    
        \alakohdat{
            § $1111\, 1010$
            § $1010\, 0000$
            § $0101\, 1010$
        }
    \end{vastaus}
    
\end{tehtava}

\begin{tehtava}
     * Muodosta bittijono 
    \alakohdat{
        § $(0\, 1111 \land 1\, 0101) \lor 0\, 1000$
        § $(0\, 1010 \xor 1\, 1011) \xor 0\, 1001$. 
    }

    \begin{vastaus}
    
        \alakohdat{
            § $0\, 1101$
            § $1\, 1000$
        }
    \end{vastaus}
    
\end{tehtava}

\end{tehtavasivu}

% -----


\begin{kotitehtavasivu}

%Luku 2.2 kotitehtävät, vastauksien tekijä Valtteri Vistiaho 10.11.2013
\begin{tehtava}
     Olkoot $A$: ''Timo opiskelee matematiikkaa'', $B$: ''Timo osallistuu logiikan kurssille'' ja $C$: ''Timo opiskelee filosofiaa''. Suomenna lause.
    \alakohdat{
        § $\lnot A \to \lnot B$.
        § $A\lequiv B$.
        § $A\lor C \to B$.
        § $C\lequiv (B\to \lnot A)$.
    }

    \begin{vastaus}
    
        \alakohdat{
            § Jos Timo ei opiskele matematiikkaa, Timo ei osallistu logiikan kurssille.
            § Timo opiskelee matematiikkaa, jos ja vain jos Timo osallistuu logiikan kurssille.
            § Jos Timo opiskelee matematiikkaa tai filosofiaa, Timo osallistuu logiikan kurssille.
            § Jos Timo opiskelee matematiikkaa ja osallistuu logiikan kurssille, hän ei opiskele filosofiaa. Muussa tapauksessa hän opiskelee filosofiaa.
        }
    \end{vastaus}
    
\end{tehtava}

\begin{tehtava}
     Onko lause tosi?
    \alakohdat{
        § Jos $(-2)^3= -8$, niin $(-2)^3= -8$.
        § Jos $(-2)^3= -8$, niin $(-2)^3=  8$.
        § Jos luku $8$ on pariton, niin luku $8$ on pariton.
        § Jos luku $8$ on pariton, niin luku $8$ on parillinen.
    }

    \begin{vastaus}
    
        \alakohdat{
            § On.
            § Ei.
            § On.
            § On. (Luku $8$ ei ole pariton.) % /Niko
        }
    \end{vastaus}
    
\end{tehtava}

\begin{tehtava}
     Formalisoi lause.
    \alakohdat{
        § Jos opettaja on pirteä, niin televisiosta ei ole tullut illalla jalkapalloa.
        § Suomi voittaa jalkapallon maailmanmestaruuden silloin ja vain silloin, kun lehmät lentävät ja vaaleanpunaiset norsut kävelevät kadulla.
        § En seuraa jalkapalloa eikä Suomen maajoukkue menesty.
        § Jos ottelu ei pääty tasapeliin, niin joko kotijoukkue tai vierasjoukkue voittaa pelin.
    }

    \begin{vastaus}
    
        \alakohdat{
            § Olkoot $A$: ''Opettaja on pirteä.'' ja $B$: ''Televisiosta on tullut illalla jalkapalloa.'' \newline
            $A\to \lnot B$
            § Olkoot $A$: ''Suomi voittaa jalkapallon maailmanmestaruuden.'', $B$: ''Lehmät lentävät.'' ja $C$: ''Vaaleanpunaiset norsut kävelevät kadulla.'' \newline
            $A\lequiv (B\land C)$
            § Olkoot $A$: ''Seuraan jalkapalloa.'' ja $B$: ''Suomen maajoukkue menestyy.'' \newline
            $\lnot A\land \lnot B$
            § Olkoot $A$: ''Ottelu päättyy tasapeliin.'', $B$: ''Kotijoukkue voittaa pelin.'' ja $C$: ''Vierasjoukkue voittaa pelin.'' \newline
            $\lnot A\to (B\lor C)$
        }
    \end{vastaus}
    
\end{tehtava}

\begin{tehtava}
     Laadi lauseen totuustaulu.
    \alakohdat{
        § $A\land B\to \lnot B$
        § $(A\lequiv B)\land B\to C\lor A$
        § $(\lnot A\to B)\lequiv (C\to B\land A)$
    }

    \begin{vastaus}
    
        \alakohdat{
            § \begin{center}
		    \begin{tabular}{|c|c|c|c|c|}\hline
		    $A$ & $B$ & $\lnot B$ & $A\land B$ & $A\land B\to \lnot B$ \\ \hline
		    $1$ & $1$ & $0$ & $1$ & $0$ \\ %\hline
		    $1$ & $0$ & $1$ & $0$ & $1$ \\
		    $0$ & $1$ & $0$ & $0$ & $1$ \\
		    $0$ & $0$ & $1$ & $0$ & $1$ \\ \hline
\end{tabular}
\end{center}
            § \begin{center}
		    \begin{tabular}{|c|c|c|c|c|c|c|}\hline
		    $A$ & $B$ & $C$ & $A\lequiv B$ & $(A\lequiv B)\land B$ & $C\lor A$ & $(A\lequiv B)\land B\to C\lor A$ \\ \hline
		    $1$ & $1$ & $1$ & $1$ & $1$ & $1$ & $1$ \\ %\hline
		    $1$ & $1$ & $0$ & $1$ & $1$ & $1$ & $1$ \\
		    $1$ & $0$ & $1$ & $0$ & $0$ & $1$ & $1$ \\
		    $1$ & $0$ & $0$ & $0$ & $0$ & $1$ & $1$ \\
		    $0$ & $1$ & $1$ & $0$ & $0$ & $1$ & $1$ \\
		    $0$ & $0$ & $1$ & $0$ & $0$ & $1$ & $1$ \\
		    $0$ & $1$ & $0$ & $0$ & $0$ & $0$ & $1$ \\
		    $0$ & $0$ & $0$ & $1$ & $0$ & $0$ & $1$ \\ \hline
\end{tabular}
\end{center}
            § \begin{center}
		    \begin{tabular}{|c|c|c|c|c|c|c|}\hline
		    $A$ & $B$ & $C$ & $\lnot A$ & $\lnot A\to B$ & $C\to B\land A$ & $(\lnot A\to B)\lequiv (C\to B\land A)$ \\ \hline
		    $1$ & $1$ & $1$ & $0$ & $1$ & $1$ & $1$ \\ %\hline
		    $1$ & $1$ & $0$ & $0$ & $1$ & $1$ & $1$ \\
		    $1$ & $0$ & $1$ & $0$ & $1$ & $0$ & $0$ \\
		    $1$ & $0$ & $0$ & $0$ & $1$ & $1$ & $1$ \\
		    $0$ & $1$ & $1$ & $1$ & $1$ & $0$ & $0$ \\
		    $0$ & $0$ & $1$ & $1$ & $0$ & $0$ & $1$ \\
		    $0$ & $1$ & $0$ & $1$ & $1$ & $1$ & $1$ \\
		    $0$ & $0$ & $0$ & $1$ & $0$ & $1$ & $0$ \\ \hline
\end{tabular}
\end{center}
        }
    \end{vastaus}
    
\end{tehtava}

\begin{tehtava}
     Olkoot $A$: ''Timo opiskelee matematiikkaa'', $B$: ''Timo osallistuu logiikan kurssille'' ja $C$: ''Timo opiskelee filosofiaa''. Formalisoi lause. Missä tilanteissa lause on tosi?
    \alakohdat{
        § Timo opiskelee matematiikkaa ja filosofiaa.
        § Jos Timo opiskelee matematiikkaa, hän osallistuu logiikan kurssille.
        § Timo osallistuu logiikan kurssille silloin ja vain silloin, kun hän opiskelee filosofiaa tai matematiikkaa.
        § Jos Timo osallistuu logiikan kurssille, opiskelee hän joko matematiikka tai filosofiaa, muttei molempia.
    }

    \begin{vastaus}
    
        \alakohdat{
            § $A\land C$
            § $A\to B$
            § $B\lequiv (A\lor C)$
            § $B\to A\xor C$ tai $B\to((A\lor C)\land \lnot(A\land C))$
        }
    \end{vastaus}
    
\end{tehtava}

\begin{tehtava}
     Poliisit Mauno ja Martti saavat kiinni rikoksesta epäilemänsä Jaskan. Tiedetään, että Jaska ei koskaan valehtele. Mauno toteaa Martille: ''Jos Jaska on syyllinen, on hänellä ollut rikostoveri.'' Jaska vastaa: ''Tuo ei ole totta!'' Tähän Martti tokaisee: ''Sehän tunnusti helpolla!'' Tutki Maunon lausetta ja osoita Martin johtopäätös oikeaksi.
    \begin{vastaus}
    
        Olkoot $A$: ''Jaska on syyllinen.'' ja $B$: ''Hänellä on ollut rikostoveri.''
        \begin{center}
		    \begin{tabular}{|c|c|c|c|}\hline
		    $A$ & $B$ & $A\to B$ & $\lnot(A\to B)$ \\ \hline
		    $1$ & $1$ & $1$ & $0$ \\ %\hline
		    $1$ & $0$ & $0$ & $1$ \\
		    $0$ & $1$ & $1$ & $0$ \\
		    $0$ & $0$ & $1$ & $0$ \\ \hline
\end{tabular}
\end{center}
Koska Jaska puhuu aina totta ja hän väittää, että ei ole totta, että jos hän olisi syyllinen, olisi hänellä ollut rikostoveri.'', on hän syyllinen eikä hänellä ole rikostoveria.
    \end{vastaus}
    
\end{tehtava}

\begin{tehtava}
     Mattia, Seppoa ja Teppoa epäillään polkupyörävarkaudesta. Konstaapeli Reinikainen tietää, että syytön puhuu aina totta ja syyllinen valehtelee aina. Reinikainen kuulustelee kolmea epäiltyä. Matti sanoo: ''Teppo on syyllinen.'' Seppo väittää: ''Matti on syyllinen tai Teppo valehtelee.'' Teppo toteaa: ''Seppo on syyllinen.'' Ratkaise totuustaulun avulla, kuka on syyllinen. 
    \begin{vastaus}
      Olkoot $M$:''Matti on syyllinen.'', $S$: ''Seppo on syyllinen.'', $T$: ''Teppo on syyllinen.'' ja $V$: ''Teppo valehtelee.''
      \begin{center}
		    \begin{tabular}{|c|c|c|c|c|}\hline
		    $M$ & $S$ & $T$ & $V$ & $M\lor V$ \\ \hline
		    $1$ & $1$ & $1$ & $1$ & $1$ \\ %\hline
		    $1$ & $1$ & $0$ & $1$ & $1$ \\
		    $1$ & $1$ & $1$ & $0$ & $1$ \\
		    $1$ & $1$ & $0$ & $0$ & $1$ \\
		    $1$ & $0$ & $1$ & $1$ & $1$ \\
		    $1$ & $0$ & $0$ & $1$ & $1$ \\
		    $1$ & $0$ & $1$ & $0$ & $1$ \\
		    $1$ & $0$ & $0$ & $0$ & $1$ \\
		    $0$ & $1$ & $1$ & $1$ & $1$ \\
		    $0$ & $1$ & $0$ & $1$ & $1$ \\
		    $0$ & $1$ & $1$ & $0$ & $0$ \\
		    $0$ & $1$ & $0$ & $0$ & $0$ \\
		    $0$ & $0$ & $1$ & $1$ & $1$ \\
		    $0$ & $0$ & $0$ & $1$ & $1$ \\
		    $0$ & $0$ & $1$ & $0$ & $0$ \\
		    $0$ & $0$ & $0$ & $0$ & $0$ \\ \hline
\end{tabular}
\end{center}
Teppo on syyllinen. %Vastausta voisi tarkentaa.
    \end{vastaus}
    
\end{tehtava}

\begin{tehtava}
     Helinä, Heljä ja Helena asuvat samassa talossa. Epäillään, että loton päävoitto on osunut taloon. Luotettavalta taholta on tullut tietoja:

Jos Heljä on voittaja, niin Helinäkin on. 
Heljä on voittaja, jos ja vain jos Helena on voittaja.
Ainakin yksi naisista on voittaja, mutta kaikki eivät ole.

Kuka tai ketkä ovat voittaneet lotossa?

    \begin{vastaus}
    
        Olkoot $A$: ''Helinä on voittaja.'', $B$: ''Heljä on voittaja.'' ja $C$: ''Helena on voittaja.''
        \begin{center}
		    \begin{tabular}{|c|c|c|c|c|c|}\hline
		    $A$ & $B$ & $C$ & $B\to A$ & $B\lequiv C$ & $(A\lor B\lor C)\land \lnot(A\land B\land C)$ \\ \hline
		    $1$ & $1$ & $1$ & $1$ & $1$ & $0$ \\ %\hline
		    $1$ & $1$ & $0$ & $1$ & $0$ & $1$ \\
		    $1$ & $0$ & $1$ & $1$ & $0$ & $1$ \\
		    $1$ & $0$ & $0$ & $1$ & $1$ & $1$ \\
		    $0$ & $1$ & $1$ & $0$ & $1$ & $1$ \\
		    $0$ & $0$ & $1$ & $1$ & $0$ & $1$ \\
		    $0$ & $1$ & $0$ & $0$ & $0$ & $1$ \\
		    $0$ & $0$ & $0$ & $1$ & $1$ & $0$ \\ \hline
\end{tabular}
\end{center}
Helinä on voittaja.
    \end{vastaus}
    
\end{tehtava}

\begin{tehtava}
     Mitä eroa on seuraavissa ehdoissa?
    \alakohdat{
        § Henkilö saa työpaikan matkamuistomyymälässä, jos hänellä on kokemusta kassakoneen käytöstä.
        § Matkamuistomyymälässä työskentelevältä edellytetään kokemusta kassakoneen käytöstä.
        § Henkilö saa työpaikan matkamuistomyymälässä silloin ja vain silloin, kun hänellä on kokemusta kassakoneen käytöstä.
    }

    \begin{vastaus}
        a) on riittävä on ehto, b) on välttämätön ehto ja c) riittävä sekä välttämätön ehto. % /Niko
    \end{vastaus}
    
\end{tehtava}

\begin{tehtava}
     Vertaa lauseiden totuusarvoja atomilauseiden $A$, $B$ ja $C$ eri totuusarvoilla. Onko sulkeiden muuttamisella vaikutusta lauseen totuusarvoon?
    \alakohdat{
        § $(A\lor B)\lor C$ ja $A\lor (B\lor C)$.
        § $(A\lequiv B)\lequiv C$ ja $A\lequiv (B\lequiv C)$.
        § $\lnot A \lequiv (B \land C)$ ja $\lnot (A \lequiv B) \land C$.
    }

    \begin{vastaus}
    
        \alakohdat{
            § \begin{center}
		    \begin{tabular}{|c|c|c|c|c|c|c|}\hline
		    $A$ & $B$ & $C$ & $A\lor B$ & $B\lor C$ & $(A\lor B)\lor C$ & $A\lor(B\lor C)$ \\ \hline
		    $1$ & $1$ & $1$ & $1$ & $1$ & $1$ & $1$ \\ %\hline
		    $1$ & $1$ & $0$ & $1$ & $1$ & $1$ & $1$ \\
		    $1$ & $0$ & $1$ & $1$ & $1$ & $1$ & $1$ \\
		    $1$ & $0$ & $0$ & $1$ & $0$ & $1$ & $1$ \\
		    $0$ & $1$ & $1$ & $1$ & $1$ & $1$ & $1$ \\
		    $0$ & $0$ & $1$ & $0$ & $1$ & $1$ & $1$ \\
		    $0$ & $1$ & $0$ & $1$ & $1$ & $1$ & $1$ \\
		    $0$ & $0$ & $0$ & $0$ & $0$ & $0$ & $0$ \\ \hline
\end{tabular}
\end{center}
Ei vaikutusta.
            § \begin{center}
		    \begin{tabular}{|c|c|c|c|c|c|c|}\hline
		    $A$ & $B$ & $C$ & $A\lequiv B$ & $B\lequiv C$ & $(A\lequiv B)\lequiv C$ & $A\lequiv (B\lequiv C)$ \\ \hline
		    $1$ & $1$ & $1$ & $1$ & $1$ & $1$ & $1$ \\ %\hline
		    $1$ & $1$ & $0$ & $1$ & $0$ & $0$ & $0$ \\
		    $1$ & $0$ & $1$ & $0$ & $0$ & $0$ & $0$ \\
		    $1$ & $0$ & $0$ & $0$ & $1$ & $1$ & $1$ \\
		    $0$ & $1$ & $1$ & $0$ & $1$ & $0$ & $1$ \\
		    $0$ & $0$ & $1$ & $1$ & $0$ & $1$ & $0$ \\
		    $0$ & $1$ & $0$ & $0$ & $0$ & $1$ & $0$ \\
		    $0$ & $0$ & $0$ & $1$ & $1$ & $0$ & $1$ \\ \hline
\end{tabular}
\end{center}
Vaikuttaa.
            § \begin{center}
		    \begin{tabular}{|c|c|c|c|c|c|c|c|c|}\hline
		    $A$ & $B$ & $C$ & $\lnot A$ & $B\land C$ & $\lnot A \lequiv (B \land C)$ & $A\lequiv B$ & $\lnot (A\lequiv B)$ & $\lnot (A \lequiv B) \land C$ \\ \hline
		    $1$ & $1$ & $1$ & $0$ & $1$ & $0$ & $1$ & $0$ & $0$ \\ %\hline
		    $1$ & $1$ & $0$ & $0$ & $0$ & $1$ & $1$ & $0$ & $0$ \\
		    $1$ & $0$ & $1$ & $0$ & $0$ & $1$ & $0$ & $1$ & $1$ \\
		    $1$ & $0$ & $0$ & $0$ & $0$ & $1$ & $0$ & $1$ & $0$ \\
		    $0$ & $1$ & $1$ & $1$ & $1$ & $1$ & $0$ & $1$ & $1$ \\
		    $0$ & $0$ & $1$ & $1$ & $0$ & $0$ & $1$ & $0$ & $0$ \\
		    $0$ & $1$ & $0$ & $1$ & $0$ & $0$ & $0$ & $1$ & $0$ \\
		    $0$ & $0$ & $0$ & $1$ & $0$ & $0$ & $1$ & $0$ & $0$ \\ \hline
\end{tabular}
\end{center}
Vaikuttaa.

        }
    \end{vastaus}
    
\end{tehtava}

\begin{tehtava}
     Olkoon $v$ sellainen funktio, että $v(A) = 1$, jos lause $A$ on tosi, ja $v(A) = 0$, jos lause $A$ on epätosi. Osoita, että yhtälö pätee kaikilla lauseiden $A$ ja $B$ totuusarvojen yhdistelmillä.
    \alakohdat{
        § $v(A\land B)=v(A)v(B)$,
        § $v(A\lor B)=v(A)+v(B)- v(A)v(B)$,
        § $v(A\to B)=1-v(A)(1-v(B))$.
    }

    \begin{vastaus}
    
        \alakohdat{
            § \begin{center}
		    \begin{tabular}{|c|c|c|c|c|c|c|}\hline
		    $A$ & $B$ & $A\land B$ & $v(A\land B)$ & $v(A)$ & $v(B)$ & $v(A)v(B)$ \\ \hline
		    $1$ & $1$ & $1$ & $1$ & $1$ & $1$ & $1$ \\ %\hline
		    $1$ & $0$ & $0$ & $0$ & $1$ & $0$ & $0$ \\
		    $0$ & $1$ & $0$ & $0$ & $0$ & $1$ & $0$ \\
		    $0$ & $0$ & $0$ & $0$ & $0$ & $0$ & $0$ \\ \hline
\end{tabular}
\end{center}
            § \begin{center}
		    \begin{tabular}{|c|c|c|c|c|c|c|c|}\hline
		    $A$ & $B$ & $A\lor B$ & $v(A\lor B)$ & $v(A)$ & $v(B)$ & $v(A)v(B)$ & $v(A)+v(B)- v(A)v(B)$\\ \hline
		    $1$ & $1$ & $1$ & $1$ & $1$ & $1$ & $1$ & $1$ \\ %\hline
		    $1$ & $0$ & $1$ & $1$ & $1$ & $0$ & $0$ & $1$ \\
		    $0$ & $1$ & $1$ & $1$ & $0$ & $1$ & $0$ & $1$ \\
		    $0$ & $0$ & $0$ & $0$ & $0$ & $0$ & $0$ & $0$ \\ \hline
\end{tabular}
\end{center}
            § \begin{center}
		    \begin{tabular}{|c|c|c|c|c|c|c|c|}\hline
		    $A$ & $B$ & $A\to B$ & $v(A\to B)$ & $v(A)$ & $v(B)$ & $1-v(B)$ & $1-v(A)(1-v(B))$\\ \hline
		    $1$ & $1$ & $1$ & $1$ & $1$ & $1$ & $0$ & $1$ \\ %\hline
		    $1$ & $0$ & $0$ & $0$ & $1$ & $0$ & $1$ & $0$ \\
		    $0$ & $1$ & $1$ & $1$ & $0$ & $1$ & $0$ & $1$ \\
		    $0$ & $0$ & $1$ & $1$ & $0$ & $0$ & $1$ & $1$ \\ \hline
\end{tabular}
\end{center}
        }
    \end{vastaus}
    
\end{tehtava}

\begin{tehtava}
     Peircen nuoli on konnektiivi, joka luonnollisessa kielessä tarkoittaa samaa kuin ''ei $A$ eikä $B$''. Shefferin viiva on konnektiivi, joka luonnollisessa kielessä tarkoittaa samaa kuin ''ei molemmat $A$ ja $B$''. Laadi näiden konnektiivien totuustaulut.
    \begin{vastaus}
    \begin{center}
		    \begin{tabular}{|c|c|c|c|c|c|c|}\hline
		    $A$ & $B$ & $\lnot A$ & $\lnot B$ & $\lnot A\land \lnot B$ & $A\land B$ & $\lnot(A\land B)$\\ \hline
		    $1$ & $1$ & $0$ & $0$ & $0$ & $1$ & $0$ \\ %\hline
		    $1$ & $0$ & $0$ & $1$ & $0$ & $0$ & $1$ \\
		    $0$ & $1$ & $1$ & $0$ & $0$ & $0$ & $1$ \\
		    $0$ & $0$ & $1$ & $1$ & $1$ & $0$ & $1$ \\ \hline
\end{tabular}
\end{center}
     
    \end{vastaus}
    
\end{tehtava}

\end{kotitehtavasivu}
